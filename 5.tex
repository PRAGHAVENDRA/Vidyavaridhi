\section{तमसः अभावरूपत्वे मतवैलक्षण्यम्}

'नीलं तमश्चलति' इत्यादिप्रतीतिवशात् तमसि नीलरूपं चलनक्रिया च भासते | क्रियाविशिष्टस्य रूपविशिष्टस्य च द्रव्यत्वात् तमः द्रव्यमिति सिध्यति | किन्तु क्लृप्तेषु नवसु द्रव्येषु तन्नान्तर्भवति | गन्धाभावात् न पृथिवी, नीलरूपवत्त्वाच्च न जलादिकामिति | तस्मादतिरिक्तं द्रव्यं तमः‌ इति प्राप्ते नातिरिक्तं द्रव्यं तदिति वैशेषिकानां नैयायिकानां च सिद्धान्तः | तत्र वैशेषिकेषु एकदेशिनः आरोपितं नीलरूपमेव तमः इति कथयन्ति | अन्ये तावत् भासामभावस्तमः इति वदन्ति | तदत्र निरूप्यते |

\subsection{आरोपितनीलरूपं तमः}

तमसि स्पर्शानुपपत्त्या तस्य स्पर्शवद्द्रव्यानारम्भकस्य द्रव्यत्वानुपपत्तौ तस्य तेजोभावरूपत्वशङ्कायां तन्निराकुर्वन्ति श्रीधराचार्याः | तथा हि - नीलाकारेण प्रतीयमानस्य तमसः अभावरूपत्वं नैव सम्भवति | मध्यन्दिनेऽपि तेजसः सद्भावेऽपि दूरगगनव्यापिनः नीलिम्नः तमसः प्रतिभानात् न तस्य तेजोभावरूपत्वम् | किञ्च यदि तेजोभाव एव तमः स्यात् तस्याप्रत्यक्षत्वप्रसङ्गः | कथम् ? अभावो हि इन्द्रियसंयुक्ततदधिकरणविशेष्यतया विशेषणतया वा प्रतीयते | न स्वतन्त्रः | तमः प्रत्ययस्थले अन्यस्य कस्यापि अप्रतीतेः | एवं तमः तेजोभाव एव इति प्रतीतिरपि न पश्चादुदेति | तस्मान्नाभावेदम् | न च आलोकादर्शनं यदा तदा तम इति प्रतीतिः | तस्मादालोकादर्शनाभाव एव तमः इति वाच्यम् | तमः, छाया इत्यादिना कृष्णाकारस्य प्रतिभानात् | तस्मात् प्रौढप्रकाशकतेजसः अभावे सति प्रतीयमानः  सर्वतः समारोपितः  नीलरूपविशेषः एव तमःपदार्थः इति |

{\fontsize{11.7}{0}\selectfont\s किन्त्वारम्भानुपपत्तेः नीलिममात्रप्रतीतेश्च द्रव्यमिदं न भवतीति ब्रूमः | तर्हि भासामभाव एवायं प्रतीयते ? न, तस्य नीलाकारेण प्रतिभासायोगात् , मध्यन्दिनेऽपि दूरगगनाभोगव्यापिनो नीलिम्नश्च प्रतीतेः | किञ्च गृह्यमाणे प्रतियोगिनि संयुक्तविशेषणतया तदन्यप्रतिषेधमुखेनाभावो गृह्यते, न स्वतन्त्रः | तमसि च गृह्यमाणे नान्यस्य ग्रहणमस्ति | न च प्रतिषेधमुखः प्रत्ययः | तस्मान्नाभावोऽयम् | न चालोकादर्शनमात्रमेवैतत् , बहिर्मुखतया तम इति, छायेति च कृष्णाकारप्रतिभासनात् | तस्माद्रूपविशेषोऽयमत्यन्तं तेजोभावे सति सर्वतः समारोपितस्तम इति प्रतीयते |\footnote{न्या.कं. ३३,३४}}


\subsection{मेयान्तरं तमः}

तमसः नीलरूपवत्त्वात् क्रियादिमत्त्वाच्च मेयान्तरत्वव्यवस्थापयन्ति वल्लभाचार्याः | तथा हि - भगवता सूत्रकारेण कणादेन धर्मविशेषप्रसूतादित्यादिसूत्रे पदार्थानां सर्वेषां निबन्धनं नाकारि | अन्यथा अभावानभिधानात्मकः दोषः स्यात् | अपि तु भावपदार्थानामेव | तमस्तु मेयान्तरमेव | अतः तन्निबन्धनं न कृतम् | ननु आलोकाभाव एव तमः, आलोके सति अननुभूतये, तदभावे चानुभूयते इति अन्वयव्यतिरेकस्य विद्यमानत्वादिति वाच्यम् | नञर्थोल्लेखेन न प्रतीयते तमः | यदि भासामभावः‌ स्यात् तदा 'इदानीं न प्रभा' इति नञर्थोल्लेखेनैव तस्य प्रतीतिः स्यात् | न च आरोपितं नीलरूपमेव तमः इति वाच्यम् | बाधकं विना आरोपानुपपत्तेः | न च आलोकाभावदशायां चाक्षुषत्वमेव बाधकम् , न हि भावस्तदानीं गृह्यते  इति वाच्यम् | विलक्षणमेव मेयान्तरं तमः यद्गृहे चक्षुरालोकं नापेक्षते | अपि तु आलोकाभाव एव अस्य व्यञ्जकः | न च तमसि वास्तवनिलरूपवत्त्वे आलोकाभावे तमसः चाक्षुषत्वं बाधकम् | शुक्लरूपेतररूपवद्द्रव्यस्य प्रत्यक्षं प्रति आलोकसहकृतचक्षुषः अन्वयव्यतिरेकाभ्यां कारणत्वकल्पनात् | 'नीलं तमः' इत्यादिप्रतीतिस्तु भ्रमात्मिका इति वाच्यम् | आलोकाभाववादिमतेऽपि तादृशप्रतीतेः भ्रमत्वसम्पादनार्थं नीलरूपारोपो अभ्युपेयः | स च न सम्भवति | आलोकाभावात् आरोपाधिकरणस्य च अप्रत्यक्षत्वात् | न च तमसः नीलरूपवत्त्वात् दशमद्रव्यत्वमस्तु इति वाच्यम् | नवैव द्रव्याणि इति यदुद्दिष्टं लक्षितं परीक्षितं च शास्त्रे तस्य व्याघातः स्यात् | तर्हि नवस्वेवास्य अन्तर्भावोऽस्तु इत्यपि न चिन्तनीयम् | गन्धस्पर्शशून्यत्वात् नीलरूपवत्त्वाच्च न पृथिव्यादिनवद्रव्येषु अस्यान्तर्भावः | एवं न गुणस्तमः | चतुर्विंशतिगुणानां प्रतिज्ञातत्वात् तद्व्याघातप्रसङ्गः, गुणे गुणानङ्गीकारच्च | तस्मात्तमः षड्भावपदार्थातिरिक्तः भावपदार्थः आलोकनिरपेक्षचक्षुर्ग्राह्यत्वादिति सिद्धम् | एतेन षडतिरिक्तस्य अभावपदार्थस्य सत्त्वान्न सिद्धसाधनमिति |

{\fontsize{11.7}{0}\selectfont\s तमस्तु मेयान्तरं निषेधत्वेनानवभासमानत्वात् | बाधकाभावेन चारोपानुपपत्तेः आलोकाभावे चाक्षुषत्वं नास्तीति बाधकमिति चेन्न, तस्यालोकाभावव्यञ्जनीयत्वात् | अन्यथारोपानुपपत्तेः | भावत्वे यदि द्रव्यान्तरं नवैवेति व्याघातः अद्रव्यान्तरत्वं सर्ववादिनिषिद्धम् | अथ गुणान्तरं चतुर्विंशतित्वव्याघात इति मेयान्तरमेव तमः | अत्रैव सङ्ग्रहः श्लोकः -\\ नाभावोभाववैधर्म्यान्नारोपो बाधहानितः |\\[-5mm] द्रव्यादिषट्कवैधर्म्याज्ज्ञेयं मेयान्तरं तमः ||\footnote{न्या.ली.१८-२०}}


\subsection{भाभावस्तमः}

भासामभाव एव तमः इति किरणावन्यामाचार्याः एवं सेतुटीकाराश्च प्रतिपादयामासुः | तथा हि - तत्र हि द्रव्यगुणकर्मसु अनन्तर्भावं विशदतया प्रतिपाद्य भाभावस्तमः इति न्यरूपि | न च तमसः भाभावरूपत्वे तस्य चाक्षुषत्वोपपादनार्थं प्रतियोगिस्मरणं अभावाधिकरणस्य प्रत्यक्षत्वं चापेक्षितम् | अन्यथा अभावस्यापि अप्रत्यक्षत्वादिति वाच्यम् |‌ यस्य साक्षात्कारे यादृशसामग्रीविशिष्टमिन्द्रियमपेक्ष्यते तदभावग्रहेऽपि तादृशसामग्रीविशिष्टमेव इन्द्रियमपेक्ष्यते इति नियमस्य सत्त्वात् | तथा च आलोकग्रहे आलोकनिरपेक्षचक्षुषः एव कारणत्वात् तदभावग्रहेऽपि आलोकनिरपेक्षं चक्षुरेव कारणम् | आलोकाभावविशिष्टेन्द्रियस्य तद्ग्राहकत्वाभ्युपगमादेव अधिकरणसाक्षात्कारं विना तत्प्रत्यक्षमुपपद्यते | इन्द्रियस्य तादृशसामर्थ्यकल्पनात् | दिवा च गगनादौ नीलिमप्रतीतिः प्रभातिरिक्तदेशेनैव भवति | अत एव अतिदूरे नभसि नीलिमप्रतीतिरिति तमसः भाभावरूपत्वे न किञ्चिद्बाधकमिति |

{\fontsize{11.7}{0}\selectfont\s  द्रव्यगुणकर्मनिष्पत्तिवैधर्म्याद्भाभावस्तम इति | सोऽपि कथमालोकमन्तरेण प्रतियोगिस्मरणाधिकरणग्रहणविरहे विधिमुखेन चाक्षुष इति चेन्न हि यद्ग्रहे यदपेक्षं चक्षुः तदभावग्रहेऽपि तदपेक्षते | एवं हि तदितरसामग्रीसाकल्यं स्यात् | तदालोकाभावेऽप्यालोकापेक्षा स्यात् , यद्यालोके तदपेक्षा स्यात् | न त्वेतदस्ति | प्रत्युत विरोध एव | तस्मिन् सति तदभाव एव न स्यात् , किं तदपेक्षेण चक्षुषा गृह्यते | दिवा च प्रतियोगिनः प्रभामण्डलस्य ग्रहण एव प्रदेशान्तरे तद्ग्रह इति न किञ्चिदनुपपन्नम् |\footnote{कि. ९८-१०२}}

{\fontsize{11.7}{0}\selectfont\s तथा च द्रव्यादिभिन्नं तमः न द्रव्यान्तरम् , किन्यु महाप्रभात्वावच्छिन्नात्यन्ताभावः |\footnote{सू. ३४}}

\subsection{विमर्शः}

ननु 'नीलं तमः श्चलति' इत्यादिप्रतीतेः अन्धकारादौ सम्भवात् तमसि नीलरूपक्रियादिदर्शनात् तस्यापि द्रव्यत्वं सिध्यति | अत्र यद्यपि 'नीलं तमः' इति प्रतीत्या एव तमसि नीलरूपवत्त्वात् द्रव्यत्वं सिध्यति तथापि वह्न्यादितेजसि औपाधिकपीतरक्तादिरूपभ्रमस्य सर्ववादिसिद्धतया प्रकृतेऽपि रूपवत्ताप्रतीतौ भ्रमत्वकल्पनसम्भवात् | क्रियाप्रकारकप्रतीतौ तु तादृशभ्रमादर्शनात् 'तमः श्चलति' इति प्रतीतेः ग्रहणम् | न च वेगवतः पुंसः स्थिरे वृक्षादौ भ्रमरूपा क्रियावत्ताप्रतीतिरनुभूयते इति वाच्यम् | तत्र क्रियावृत्ताप्रतीतिस्तु न औपाधिकी, अपि तु पुरुषदोषवशादुत्पद्यमाना | तस्मात् क्रियावत्त्वात्तमसः न मूर्तद्रव्यातिरिक्तद्रव्यत्वसम्भवः | नीलरूपवत्त्वात् न पृथिवीतरत् , गन्धाभावाच्च न पृथिवीति तस्यातिरिक्तत्वमपि सिध्यति | 

न च तमो यदि रूपवद्द्रव्यं स्यात् तर्हि रूपवद्द्रव्यस्य स्पर्शवत्त्वनियमात् तस्मिन् स्पर्शोऽपि स्यात् | स्पर्शवत्वाभ्युपगमे तु महतः स्पर्शवद्द्रव्यस्य प्रतिघातकत्वनियमात् भित्तिवत्तमसः अपि चलनादौ प्रतिबन्धः स्यात् | किन्तु तमसि तादृशप्रतिबन्धकत्वाभावात् तन्न द्रव्यम् | महतः स्पर्शवद्द्रव्यस्य प्रतिघातकत्वनियमो व्यभिचरतः | अन्यथा आलोकदशायामपि प्रतिघातः स्यात् |

नाभावस्तमः | घटाभावस्य यथा प्रतियोगिमन्तरा न व्यवहारः तद्वत् तमसः अपि स्वप्रतियोगिमन्तरा व्यावहारानुपपत्तिः |  किञ्चास्य अभावत्वे तु तेजोभावरूपत्वमेव वक्तव्यम् | तेन सह विरोधदर्शनात् | तत्र किं तमः तेजसः अन्योन्याभावः ? उत संसर्गाभावः ? आद्ये भास्करकरनिकराक्रान्तेषु प्राङ्गणादिषु तमोव्यवहारापत्तेः, तत्र तेजोभेदस्य सत्त्वात् | द्वितीये किं यत्किञ्चित्प्रतियोगिकाभावः उत तेजस्सामान्याभावः | आद्ये यत्किञ्चित्तेजःप्रतियोगिकसंसर्गाभावस्य तेजोधिकरणेऽपि सत्त्वात् तत्र तमःप्रतीत्यापत्तिः | द्वितीये तु तमः प्रतीतिकालेऽपि यत्किञ्चित्तेजसः सत्त्वात् तमःप्रतीत्यनुपपत्तिः | नापि तेजोविशेषाभावः | अभावज्ञानं प्रति प्रतियोगिज्ञानस्य कारणत्वात् तादृशानां तमःप्रतियोगितेजोविशेषाणां तमःप्रतीतेः पूर्वमज्ञानात् | तस्मात्तमो न तेजोभावरूपः | अपि तु अन्य एव | 

तथा च मीमांसकाः - {\fontsize{11.7}{0}\selectfont\s 'गुणकर्मादिसद्भावादस्तीति प्रतिभासतः | प्रतियोग्यस्मृतेश्चैव भावरूपं ध्रुवं तमः ||'\footnote{मा.मे. १५२}} इति | किञ्च {\fontsize{11.7}{0}\selectfont\s 'तमः कृष्णं व्यक्थमस्थित'\footnote{कृ.य.ब्रा. }} इति श्रुतिप्रमाणस्य सत्त्वात् तमः अतिरिक्तं द्रव्यमिति वदन्ति |

वेदान्तिनस्तु {\fontsize{11.7}{0}\selectfont\s 'यदि तावत् सहानवस्थानलक्षणो विरोधः, ततः प्रकाशभावे तमसो भावानुपपत्तिः, तदसत् ; दृश्यते हि मन्दप्रदीपे वेश्मनि अस्पष्टं रूपदर्शनं, इतरत्र च स्पष्टम् । तेन ज्ञायते मन्दप्रदीपे वेश्मनि तमसोऽपि ईषदनुवृत्तिरिति ; तथा छायायामपि औष्ण्यं तारतम्येन उपलभ्यमानं आतपस्यापि तत्र अवस्थानं सूचयति । एतेन शीतोष्णयोरपि युगपदुपलब्धेः सहावस्थानमुक्तं वेदितव्यम् ।'\footnote{प.पा.}} इति वदन्ति |
  
केचित्तु - नीलाकारेण प्रतीयमानस्य तमसः अभावरूपत्वं नैव सम्भवति | मध्यन्दिनेऽपि तेजसः सद्भावेऽपि दूरगगनव्यापिनः नीलिम्नः तमसः प्रतिभानात् न तस्य तेजोभावरूपत्वम् | किञ्च यदि तेजोभाव एव तमः स्यात् तस्याप्रत्यक्षत्वप्रसङ्गः | कथम् ? अभावो हि इन्द्रियसंयुक्ततदधिकरणविशेष्यतया विशेषणतया वा प्रतीयते | न स्वतन्त्रः | तमः प्रत्ययस्थले अन्यस्य कस्यापि अप्रतीतेः | एवं तमः तेजोभाव एव इति प्रतीतिरपि न पश्चादुदेति | तस्मान्नाभावेदम् | न च आलोकादर्शनं यदा तदा तम इति प्रतीतिः | तस्मादालोकादर्शनाभाव एव तमः इति वाच्यम् | तमः, छाया इत्यादिना कृष्णाकारस्य प्रतिभानात् | तस्मात् प्रौढप्रकाशकतेजसः अभावे सति प्रतीयमानः  सर्वतः समारोपितः  नीलरूपविशेष एव तमःपदार्थः इति |

अन्ये तु - तमसः नीलरूपवत्त्वात् क्रियादिमत्त्वाच्च मेयान्तरमिति वदन्ति | तथा हि - भगवता सूत्रकारेण कणादेन धर्मविशेषप्रसूतादित्यादिसूत्रे पदार्थानां सर्वेषां निबन्धनं नाकारि | अन्यथा अभावानभिधानात्मकः दोषः स्यात् | अपि तु भावपदार्थानामेव | तमस्तु मेयान्तरमेव | अतः तन्निबन्धनं न कृतम् | ननु आलोकाभाव एव तमः, आलोके सति अननुभूयते, तदभावे चानुभूयते इति अन्वयव्यतिरेकस्य विद्यमानत्वादिति वाच्यम् | नञर्थोल्लेखेन न प्रतीयते तमः | यदि भासामभावः‌ स्यात् तदा 'इदानीं न प्रभा' इति नञर्थोल्लेखेनैव तस्य प्रतीतिः स्यात् | न च आरोपितं नीलरूपमेव तमः इति वाच्यम् | बाधकं विना आरोपानुपपत्तेः | न च आलोकाभावदशायां चाक्षुषत्वमेव बाधकम् , न हि भावस्तदानीं गृह्यते  इति वाच्यम् | विलक्षणमेव मेयान्तरं तमः यद्गृहे चक्षुरालोकं नापेक्षते | अपि तु आलोकाभाव एव अस्य व्यञ्जकः | न च तमसि वास्तवनिलरूपवत्त्वे आलोकाभावे तमसः चाक्षुषत्वं बाधकम् | शुक्लरूपेतररूपवद्द्रव्यस्य प्रत्यक्षं प्रति आलोकसहकृतचक्षुषः अन्वयव्यतिरेकाभ्यां कारणत्वकल्पनात् | 'नीलं तमः' इत्यादिप्रतीतिस्तु भ्रमात्मिका इति वाच्यम् | आलोकाभाववादिमतेऽपि तादृशप्रतीतेः भ्रमत्वसम्पादनार्थं नीलरूपारोपो अभ्युपेयः | स च न सम्भवति | आलोकाभावात् आरोपाधिकरणस्य च अप्रत्यक्षत्वात् | न च तमसः नीलरूपवत्त्वात् दशमद्रव्यत्वमस्तु इति वाच्यम् | नवैव द्रव्याणि इति यदुद्दिष्टं लक्षितं परीक्षितं च शास्त्रे तस्य व्याघातः स्यात् | तर्हि नवस्वेवास्य अन्तर्भावोऽस्तु इत्यपि न चिन्तनीयम् | गन्धस्पर्शशून्यत्वात् नीलरूपवत्त्वाच्च न पृथिव्यादिनवद्रव्येषु अस्यान्तर्भावः | एवं न गुणस्तमः | चतुर्विंशतिगुणानां प्रतिज्ञातत्वात् तद्व्याघातप्रसङ्गः, गुणे गुणानङ्गीकारच्च | तस्मात्तमः षड्भावपदार्थातिरिक्तः भावपदार्थः आलोकनिरपेक्षचक्षुर्ग्राह्यत्वादिति सिद्धम् | एतेन षडतिरिक्तस्य अभावपदार्थस्य सत्त्वान्न सिद्धसाधनमिति |

अत्रोच्यते - उद्भूतरूपविशिष्टस्य तेजसः अभावे सति तमःप्रत्ययात् , तदागमने च तमःप्रत्ययाभावात् लाघवाच्च तादृशस्य प्रौढप्रकाशकतेजस्सामान्याभाव एव तमः | न त्वतिरिक्तं द्रव्यम् | न च यत्किञ्चित्तेजसः तमःप्रदेशे सत्त्वात् तेजस्सामान्याभावस्य तमस्त्वानुपपत्तिरिति वाच्यम् | उद्भूतरूपविशिष्टस्य तेजसः तमःप्रत्ययकालेऽप्यभावात् | अन्यथा तेजस्तमसोः यो विरोधः तस्यानुपपत्तिः | अत एव दिवापि गगने अतिदूरदेशे नैल्यप्रतीतिरेव उदेति, न तु समीपदेशे | 
 
अधिकरणप्रत्यक्षं विना अभावो न गृह्यते | अधिकरणेन सह इन्द्रियसन्निकर्षे सति तत्र विशेषणतया अभावस्य ग्रहणं सम्भवति | तमःप्रत्यक्षस्थले नान्यस्य कस्यचिद्वस्तुनः प्रत्यक्षात् तमो नाभावः इति यदुक्तं तदप्यसत् | अधिकरणप्रत्यक्षं विनापि अभावप्रत्यक्षसम्भवत् | अधिकरणेन सह इन्द्रियसम्बन्धमात्रमपेक्षितम् | अन्यथा श्रोत्रेण शब्दाभावस्याप्रत्यक्षत्वप्रसङ्गः | शब्दाधिरणस्याकाशस्यातीन्द्रियत्वात् | अधिकरणेन सह इन्द्रियसम्बन्धस्तु तमःप्रत्यक्षकालेऽप्यस्त्येव |

यदुक्तं 'नीलं तमः' इति प्रत्ययानन्तरं तद्बाधकप्रत्ययान्तरस्य कस्याप्यनुदयात् बाधकाभावात् तस्य प्रमात्वं स्वीकरणीयम् | अन्यथा 'नीलो घटः' इति प्रत्ययस्यापि भ्रमत्वापत्तिरिति | तदप्यसारम् | गुणशून्ये तेजोऽभावे नीलरूपवत्त्वमेवात्र बाधकम् | न च तस्य कथं तेजोऽभावत्वमिति वाच्यम् | तेन सह विरोधात् | किञ्च तस्य द्रव्यत्वे तत्परमाणावः स्पर्शवन्तो न वा ? आद्ये तमस्यपि स्पर्शोपलब्धिप्रसङ्गः, स्पर्शवतः नीलरूपविशिष्टद्रव्यस्य सर्वत्र इन्द्रियसन्निकर्षविघटकत्वं लोके दृष्टम् | किन्तु तमसि स्थितस्यापि वर्णरञ्जितमञ्चके नाटकादीनां दर्शनानुभवात्तस्य इन्द्रियसन्निकर्षाविघटकत्वात् नेदं स्पर्शवद्द्रव्यारब्धमिति ज्ञायते | द्वितीये तु अस्पर्शवतः वस्तुनः कार्यद्रव्यानरम्भकत्वात् तमसः अपि उत्पत्तिरेव न स्यात् | न च मनोवदणुपरिमाणकमिदमिति वाच्यम् | प्रत्यक्षत्वात् | अत एव न विभुपरिमाणकम् | तस्मादस्य द्रव्यत्वासिद्धौ तेजसा सह विरोधात् तदभावरूपत्वमेव स्वीकरणीयम् | तस्मादस्य अभावत्वमेव रूपवत्तप्रतीतौ बाधकम् | एतेन तमः मेयान्तरमित्यपि निरस्तम् |

न च नीलरूपस्य तदाभानात् आरोपितं नीलरूपमेव तमः इति वाच्यम् | अधिकरणं दर्शनं विना आरोपितस्य वस्तुनः साक्षात्कारासम्भवात् | किञ्च तमसि आलोकासहकृतचक्षुर्ग्राह्यत्वात् नीलरूपाभावोऽपि सिध्यति | न च चाक्षुषं प्रति आलोकसहकारस्य कारणत्वं व्यभिचरितम् | तथा हि - {\fontsize{11.7}{0}\selectfont\s 'दिवान्धाः प्राणिनः केचिद्रात्रावन्धास्तथापरे | केचिद्दिवा तथा रात्रौ प्राणिनस्तुल्यदृष्टयः ||'\footnote{स.श.}} इति वचनात् अनुभवाच्च केचन प्राणिनः रात्रावपि पश्यन्ति | केचित् प्राणिनः दिवापि न पश्यन्ति इति सिध्यति | तथा च आलोकाभावेऽपि चाक्षुषत्वसम्भवात् आलोके सत्यपि चाक्षुषत्वासम्भवात् अन्वयव्यतिरेकव्यभिचारो दृष्टः इति वाच्यम् | नक्तञ्चराणां चक्षुरिन्द्रियस्यैव आलोकात्मकत्वात् | तदुक्तं - {\fontsize{11.7}{0}\selectfont\s 'नक्तञ्चरनयनरश्मिदर्शनाच्च'\footnote{न्या.सू. ३.१.४३}} इति सूत्रव्याख्यानावसरे न्यायभाष्ये {\fontsize{11.7}{0}\selectfont\s 'दृश्यते हि नक्तं नयनरश्मयो नक्तञ्चराणां वृषदंशप्रभृतीनां तेन शेषस्यानुमानमिति'\footnote{न्या. भा. २५४}} इति | तथा च यत्र येषां प्राणिनां रात्रावेव चाक्षुषप्रतीतिः तेषामिन्द्रियाणामालोकात्मकत्वात् आलोकत्वेन रूपेण तत्संयोगस्य कारणत्वम् | इन्द्रियत्वेन रूपेण च अपरा कारणता इति | किञ्च प्रत्यक्षं प्रति विलक्षणालोकसंयोगस्य हेतुत्वन्तु अवश्यमभ्युपेयम् इन्द्रियसामर्थ्यात् | अत एव कदाचिदत्युत्कटप्रभासंयुक्तपदार्थेन इन्द्रियसम्बन्धे सति कदाचित् मन्दालोकदशायामस्माकमपि चाक्षुषप्रतीतिः नोदेति | एवञ्च केषाञ्चित्प्राणिनां आलोकसामान्ये सति चाक्षुषप्रत्ययदर्शनात् तेषां दिवा रात्रौ च चाक्षुषप्रतीतिः सम्भवति | अस्मदृशानान्तु आलोकाभावात् रात्रौ न चाक्षुषप्रत्ययः | वृषदंशादीनान्तु चक्षुषः एव आलोकात्मकत्वात् मन्दालोकसंयोगस्यैव प्रत्यक्षजनकत्वात् रात्रावैव प्रत्यक्षप्रतीतिरिति न किञ्चिदनुपपन्नम् | 

एतेन मन्दप्रदीपे वस्तुनः अदर्शात् तत्रापि प्रतिबन्धकीभूतस्य तमसः सत्त्वमभ्युपगन्तव्यम् | तथा च तम आलोकयोः परस्परविरोधाभावोऽप्यनेन सिध्यति | एवञ्च विरोधाभावान्न तेजोभावस्तमः इत्यपि निरस्तम् | विलक्षणतेजोभावस्यैव तमः पदार्थत्वात् | तथैव लोके अनुभवात् | अत एव प्रौढप्रकाशकतेजोभाव इत्यत्र प्रकाशे प्रौढत्वं तत्तत्तमो विरोधिरूपं विलक्षणमेव ग्राह्यम् |

किञ्च तेजोभावत्वेऽपि तमसः आलोकासहकृतचक्षुषाग्रहणं कथमिति चेत् , यस्य साक्षात्कारे यादृशसामग्रीविशिष्टमिन्द्रियमपेक्ष्यते तदभावग्रहेऽपि तादृशसामग्रीविशिष्टमेव इन्द्रियमपेक्ष्यते इति नियमात् | तथा च आलोकग्रहे आलोकनिरपेक्षचक्षुषः एव कारणत्वात् तदभावग्रहेऽपि आलोकनिरपेक्षं चक्षुरेव कारणम् | आलोकाभावविशिष्टेन्द्रियस्य तद्ग्राहकत्वाभ्युपगमादेव अधिकरणसाक्षात्कारं विना तत्प्रत्यक्षमुपपद्यते | इन्द्रियस्य तादृशसामर्थ्यकल्पनात् |

एवञ्च एवंरीत्या तस्य भाभावरूपत्वसिद्धौ तस्य द्रव्यत्वाभावादेव तस्मिन् नीलरूपादिकं न सिध्यति | नीलरूपादीनामभावे तु नीलादिप्रतीतीनां पित्तदोषवतः 'शङ्खः पीतः' इति प्रतीतिवत् भ्रान्तत्वं सिद्धम् |





















