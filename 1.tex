\section{उपमानप्रमाणविचारः}

प्रमाणेषु पदपदार्थयोः सम्बन्धेति प्रसिद्धस्य सङ्केतस्य ग्राहकमिति प्रसिद्धमिदमुपमानप्रमाणम् | 'गोसदृशो गवयः' इत्यादिना सादृश्यज्ञानेन पदपदार्थयोः सम्बन्धग्रहोत्पद्यत इत्यतः सादृश्यज्ञानात्मकस्यास्य इतरप्रमाणापेक्षया विलक्षणता | तथा हि - गवयाभिज्ञेनारण्यकेन गवपदव्युत्पित्सुं नागरिकं प्रति 'गोसदृशो गवयः' इति उक्तम् | ततः परं नागरिकः कालान्तरे अरण्यादौ गवयं पश्यन् आरण्यकेनोक्तं वाक्यस्यार्थं स्मरति | ततः परमस्य 'अयमेव गवयपदवाच्यः' इति ज्ञानमुदेति | इयमेव उपमितिरित्युच्यते | अस्मिन्विषये न्यायवैशेषिकयोः मतिभेदो दृश्यते | वैशेषिकास्तावत् आप्तवाक्यश्रवणानन्तरमुत्पद्यमानः पदपदार्थविषयकसम्बन्धग्रहो नानुमानादतिरिक्तप्रमाणजन्य इति प्रतिपादयन्ति |

तथा हि - शब्दादनुमानाद्वा सम्बन्धग्रहः इति प्रधानतया मार्गद्वयं दृश्यते | तत्र 'गो सदृशो गवयः' इति केनचिदुक्तवाक्येन गवयपदशक्तिग्रहादुपमानमाप्तवचनमेव | शब्दादुत्पद्यमानं ज्ञानं तावदनुमित्यात्मक\footnote{}मित्यतः उपमानमनुमानमेव इति | गवयपदेऽव्युत्पन्नं नागरिकं प्रत्यारण्यकः 'गोसदृशो गवय' इत्यवोचत् | तदानीमेव लक्षणया गवयपदात् गवयपदवाच्यत्वविशिष्टस्योपस्थितिः ततः  'गोसदृशाभिन्नो गवयपदवाच्यः' इति गवयपदतदर्थयोः सङ्केतविषयकबोधश्च, गोसदृशपदस्य गवयत्वविशिष्टवाचकत्वे लक्षणाभ्युपगमेन 'गवयत्वविशिष्टवाचकाभिन्नो गवयः' इति वा शाब्दधीरुदेतीत्यतः तेनैव गवयपदतदर्थयोः सम्बन्धप्रतिपत्तिर्जायते  |

अथ वा वनगमनानन्तरं नागरिकस्य गवयपिण्डदर्शने सति गवयपदं गवयत्वविशिष्टैतत्पिण्डवाचकम् असति वृत्त्यन्तरे तदभियुक्तेनात्र प्रयुज्यमानत्वादित्यनुमानात् गवयपदशक्तिग्रहसम्भवादुपमानमनुमानमेवेति | अथ वा वनगमनानन्तरं गवयपिण्डे दृष्टे 'अयं गवयपदवाच्यः गोसादृश्यात् यन्नैवं तन्नैवं यथा महिषमातङ्गादिः' इत्यनेन व्यतिरेकानुमानेन वा पदपदार्थयोः संसर्गग्रहः |

तत्रादौ गवयशब्दस्य गवयपदवाच्यत्वं कथमिति चेदुच्यते -

\subsection{गोसदृशपदज्ञानादेव गवयत्वभानम्}

नागरिकं प्रति यदा आरण्यकः 'गोसदृशो गवयः' इति ब्रूते तदानीमेव नागरिकस्य गवयपदवाच्यत्वविषयकग्रहो उदेति | तथा हि - पृष्टुः गवयपदार्थः कः इत्यत्रैव जिज्ञासा वर्तते | अतः आरण्यकवचनमपि तदुपशमनसमर्थमेव | अन्यथा पुनः प्रश्नस्य उत्थितिः स्यात् | किन्तु तथा न दृश्यते | तस्मादारण्यवचनादेव तादृशजिज्ञासानिवर्तकः गवयपदतदर्थयोः सम्बन्धविषयकग्रहो जायते | तादृशशाब्दबोधस्तु स्वरसतया न सिध्यति, 'गोसदृशो गवयः' इत्यस्मिन् वाक्ये गवयपदवाच्यत्वविशिष्टस्यार्थस्य शक्त्या उपस्थापकं पदं नास्तीत्यतः | अतः तत्र तादृशबोधनिर्वाहाय गवयपदस्य गवयपदवाच्यत्वे लक्षणाभ्युपगम्यते | 

ननु गवयत्वेन प्रवृत्तिनिमित्तेन गवयपदवाच्यत्वं यदा जानाति तदानीमेव तस्य गवये गवयपदवाच्यत्वग्रहः सम्भवति । प्रकृते तु गवयस्यैव अदर्शनात् तद्वृत्तिजातेरपि अग्रहात् गवयपदे लक्षणास्विकारेऽपि 'गोसादृश्यविशिष्टाभिन्नः गवयपदवाच्यः' इत्येव बोधः, न तु 'गवयत्वविशिष्टः गवयपदवाच्याभिन्नः' इति | तथा च गोसादृश्यं प्रवृत्तिमित्तीकृत्यैव गवयं जानातीत्यतः न तेन वाक्येन शक्तिग्रहसम्भव इति चेन्न |  गोसदृशः इत्यनेन गवयत्वविशिष्ट एव लक्ष्यते । लक्षणाबीजन्तु तात्पर्यानुपपत्तिरेव | तथा हि  गोसादृश्यस्य अखण्डत्वेन गुरुत्वात् प्रवृत्तिनिमित्तत्वं न सम्भवति । अतः गवयत्वस्यैव जातित्वात् प्रवृत्तिनिमित्तत्वं स्वीकरणीयम् | तथा च गोसदृशो गवय इत्यस्य गवयत्वविशिष्टो धर्मी गवयपदवाच्य इत्यर्थः । गवयत्वं प्रवृत्तिनिमित्तीकृत्यैव बोधो जायते इत्यतः तत्र शक्तेरपि भाने न किमपि बाधकमस्ति | 

ननु यद्यन्वयानुपपत्तिः स्यात् तदानीमेव लक्षणा सम्भवति | प्रकृते यो गोसदृशः स गवयपदवाच्य इत्यन्वयसम्भवान्न लक्षणाप्रसक्तिरिति चेन्न । तात्पर्यानुपपत्तितः लक्षणासम्भवात् ।  किमत्र तात्पर्यमनुपपन्नम् ? सति लघुनि गवयत्वरूपप्रवृत्तिनिमित्ते गुरुणा गोसादृश्यरूपप्रवृत्तिनिमित्तेन गवयपदवाच्यत्वव्युत्पादनमनुपपन्नम् । तस्मादारण्यकवाक्यादेवासौ गवये गवयपदवाच्यत्वं ग्रह्णाति |


{\fontsize{11.7}{0}\selectfont\s  आप्तेनाप्रसिद्धस्य गवयस्य गवा गवयप्रतिपादनादुपमानमाप्तवचनमेव ।\footnote{प्र.भा.}

आप्तिः साक्षादर्थस्य प्राप्तिः , यथार्थोपलम्भः , तया वर्तत इत्याप्तः , साक्षात्कृतधर्मा , यथार्थदृष्टस्यार्थस्य चिख्यापयिषया प्रयोक्तोपदेष्टा , तेनाप्तेन वनेचरेण विदितगवयेन अज्ञातगवयस्य नागरिकस्य कीदृग्गवय इति पृच्छतो गोसारूप्येण गवयस्य प्रतिपादनादुपमानं यथा गौर्गवयस्तथेति वाक्यमाप्तवचनमेव । वक्तृप्रामाण्यादेव तथा प्रतीतेः । आप्तवचनं चानुमानम् । तस्मादुपमानमप्यनुमानाव्यतिरिक्तमित्यभिप्रायः ।\footnote{न्या.कं.}}

\subsection{आप्तवचनादेव गवयपदशक्तिग्रहः}

आरण्यकोक्तं 'गोसदृशो गवय' इति वाक्यं श्रुत्वा नागरिकः वनं गत्वा गवयपिण्डं पश्यति | तदा सः आरण्यकोक्तवाक्यार्थं स्मरन् अनुमीते 'गवयपदं गवयत्वविशिष्टपिण्डवाचकं असतिवृत्त्यन्तरे तदभियुक्तेनात्र प्रयुज्यमानत्वात् महिषादिवत् इति | 'यो यत्र असतिवृत्त्यन्तरे शब्दं प्रयुज्तते स तस्य वाचकः' इति तत्र व्याप्तिः | यथा 'महोषोऽयम्' इति यत्राभियुक्तेन प्रयोगः क्रियते तत्र महिषादिपदानां महिषत्वविशिष्टपिण्डवाचकत्वं दृष्टम् तथैव प्रकृते गवयपदस्यापि अभियुक्तेनारण्यकेन प्रयुक्तत्वात् तस्य गवयत्वविशिष्टपिण्डवाचकत्वं सिद्धम् | अत्र सैन्धवादिपदानां नानार्थकत्वात् तत्र केवललवणत्वविशिष्टवाचकत्वं वक्तुं न शक्यते शक्त्यन्तरस्य सद्भावात् | प्रकरणवशादर्थनिर्णयः कर्तव्यः | तथा च यो यत्र अभियुक्तेन प्रयुज्यते स तस्य वाचकः इति व्याप्तिः व्यभिचरितः | तस्मादसतिवृत्त्यन्तरे इति हेतुकुक्षौ विशेषणं दत्तम् | तथा च अनुमानादेव शक्तिग्रहादुपमानं नातिरिक्तं प्रमाणम् |

{\fontsize{11.7}{0}\selectfont\s उपमानं च न मानान्तरम् , अनुमानादेव तदर्थसिद्धेः । यो यत्रासति वृत्यन्तरे प्रयुज्यते स तस्य वाचको यथा गोशब्दो गोजातीयस्य , प्रयुज्यते चायमसति प्रतिभासमानजातीय इति । न चायमसिद्धः ,  मुख्यानुपपत्तिं विनोपचारस्यासम्भवात् । सादृश्यवति प्रयोगस्य कल्पनागौरवापत्तिहृतत्त्वात् । व्यक्तिषु प्रयोगस्यानन्त्यदूषित्वात् । नूनमयमेतज्जातीयाभिधानाय प्रयुज्यते इति निश्चयोपपत्तेः । न चेदेवमुपमानेऽपि वृत्त्यन्तरनिमित्तान्तरविषयविशिष्टप्रयोगसम्भावनायामपेक्षितासिद्धिप्रसङ्गात् ।\footnote{न्या.ली.५३१-५३६}}

\subsection{व्यतिरेकानुमानात् गवयपदवाच्यत्वग्रहः}

गवयपदार्थजिज्ञासुं नागरिकं प्रति आरण्यकः गवये विद्यमानं गोसादृश्यरूपं लिङ्गं उपदिशति 'गोसदृशो गवय' इति वाक्येन | ततः ज्ञातातिदेशवाक्यार्थः सः अरण्ये गवयपिण्डं दृष्ट्वा अनुमीते 'अयं गवयपदवाच्यः गोसादृश्यात् यन्नैवं तन्नैवं यथा महिषादिः' इति | अत्र गोसादृश्यस्य महिषादौ क्वचिदंशे सत्त्वेऽपि वैधर्म्यबाहुल्यादेव तत्र तद्वाचकत्वग्रहो नोदेति | अथ वा पूर्वमेव महिषादिषु महिषादिपदवाच्यत्वस्यावगतत्वात् न तत्र तद्विषयकानुमितिः | गवये तु गोसादृश्यस्य अधिकतया दर्शनात् गवयपदवाच्यत्वानुमितिरिति |

{\fontsize{11.7}{0}\selectfont\s यद्वा गवयपदार्थः कः ? इति नागरिकेण पृष्टः आरण्यकः गवये विध्यमानं गोसादृश्यं लिङ्गत्वेनातिदिशति 'गोसदृशो गवय' इति । ततः प्रतिपन्नातिदेशवाक्यार्थः श्रोता वनं गच्छति | तत्र करितुरगमहिषादिषु कथञ्चिद्गोसादृश्यं पश्यन्नपि न तेषां गवयपदवाचकत्वं जानाति । वैधर्म्यबाहुल्येन तिरस्कृतत्वात् | कदाचित् गोसादृश्यं गवयपिण्डं पश्यन्नूनमयं गवयपदवाच्यः, गोसदृशत्वात् , यन्न गवयपदवाच्यः नासौ गोसदृशो यथा महिषमातङ्गादिरित व्यतिरेकानुमानेन तत्र गवयपदवाच्यत्वं निश्चिनोति । यथा का पृथिवीति पृथिवीपदव्युत्पित्सुं प्रति 'गन्धवती पृथिवी' इति उच्यते चेदनन्तरं स पृथिवीदर्शने सति तत्रत्य गन्धं गृहीत्वा इदं पृथिवीपदवाच्यं गन्धवत्वादिति जानाति ।\footnote{क.र.}}


\subsection{उपमानस्यातिरिक्तप्रमाणत्वप्रदर्शनम्}

न्यायनये तावत् उपमानस्यातिरिक्तप्रमाणत्वमङ्गीकृतं वर्तते | अत एव {\fontsize{11.7}{0}\selectfont\s "प्रत्यक्षानुमानोपमानशब्दाः प्रमाणानि"\footnote{न्या.सू.१.१.३}} इति न्यायसूत्रं प्रणीतं भगवता गौतमेन | तथा चायमाशयः - गवयपिण्डे गवयशब्दवाचकत्वप्रतीतिः अरण्ये गवयपिण्डे गोसादृश्यप्रत्यक्षानन्तरं अतिदेशवाक्यार्थस्मरणद्वारा वा भवति | अत्र तादृशसङ्केतविषयकप्रतीतेः करणं न व्याप्तिज्ञानम् , नापि पदज्ञानम् , अपि तु सादृश्यज्ञानमित्यतः अस्य अनुमानागमाभ्यामतिरिक्तत्वम् इति |

\subsection{आप्तवाक्यमेवोपमानम्}

तत्र प्राचीननैयायिकास्तावत् सादृश्यज्ञानमुपमितिकरणमिति वदन्ति | अयमाशयः - आप्तवचनाद्यत्र सादृश्यप्रकारकशाब्दधीरुदेति ततः कालान्तरे गवये सादृश्यप्रतक्षानन्तरमस्य गवयशब्दः संज्ञा इति बोधो जायते | अत्र पदपदार्थयोः सङ्केतविषयकमितिं प्रति सादृश्यप्रत्यक्षद्वारा सादृश्यज्ञानस्य करणत्वात् तदेवोपमानमिति |

{\fontsize{11.7}{0}\selectfont\s समाख्यासम्बन्धप्रतिपत्तिः उपमानार्थ इत्याह । यथा गौरेवं गवय इत्युपमाने प्रयुक्ते गवा समानधर्माणम् अर्थमिन्द्रियार्थसन्निकर्षादुपलभमानोऽस्य गवयशब्दः संज्ञेति संज्ञासंज्ञिसम्बन्धं प्रतिपद्यते इति । यथा मुद्गस्तथा मुद्गपर्णी, यथा माषस्तथा माषपर्णीत्युपमाने प्रयुक्ते उपमानात्संज्ञासंज्ञिसम्बन्धं प्रतिपद्यमानस्तामोषधीं भैषज्याय आहरति । एवमन्योऽप्युपमानस्य लोके विषयो बुभुत्सितव्य इति ।\footnote{न्या.भा.}}

\subsection{प्रत्यक्षादतिरिक्तमुपमानम्}

संज्ञासंज्ञिनोः सम्बन्धप्रतिपत्तिस्तावत् यत्र आप्तवचनश्रवणानन्तरं जायते तच्च चक्षुर्व्यापारजन्यं न | अन्यथा सत्यपि चक्षुर्व्यापारे आप्तवचनस्मरणाभावे उपमित्यापत्तिः | तत्र उपमितेरनुदयात् केवलचक्षुर्व्यापारस्य तदजनकत्वमेव | तस्मात् न उपमानं प्रत्यक्षप्रमाणात्मकम् | तदेवं प्रतिपादयन्ति आचार्याः -

{\fontsize{11.7}{0}\selectfont\s अस्ति तर्हि सादृश्यादिज्ञानकाले विस्फारितस्य चक्षुषो व्यापारः | न | उपलब्धगोसादृश्यविशिष्टगवयपिण्डस्य वाक्यतदर्थस्मृतिमतः कालान्तरेऽप्यनुसन्धानबलात् समयपरिच्छेदोपपत्तेः |\footnote{न्या.कु. ३७८}}

\subsection{अनुमानादतिरिक्तमुपमानम्}

न तावदुपमानस्य अनुमानादनतिरिक्तत्वम् | 'गवयशब्दः गवयवाचकः असतिवृत्त्यन्तरे अभियुक्तैस्तत्र प्रयुज्यमानत्वात् , गवि गोशब्दवत्' इत्यनुमानं प्रामाणमिति चेदसिद्धेः | मुख्यवृत्तिज्ञाने हि लक्षणावृत्तेः ज्ञानं सम्भवति, लक्षणायाः शक्यसम्बन्धरूपत्वात् | वृत्त्यन्तराग्रहे च विशेषणाज्ञानात् हेतुग्रहाभाव इति विशेषणासिद्धिः | अभियुक्तप्रयोगस्तु लक्षणावृत्तिस्थलेऽपि दृष्ट इति स्वरूपासिद्धिरिति वदत्याचार्याः - 

{\fontsize{11.7}{0}\selectfont\s अस्त्वनुमानम् - तथा हि - गवयशब्दो गवयस्य वाचकः असति वृत्त्यन्तरे अभियुक्तैस्तत्र प्रयुज्यमानत्वात् , गवि गोशब्दवदिति चेन्न | असिद्धेः | \footnote{न्या.कु. ३८२}}

\subsection{शब्दादतिरिक्तमुपमानम्}

ननु यथा वेदवाक्यानां वेदाध्ययनानन्तरमङ्गादिज्ञानसहकृतानामेव कालान्तरे धर्माद्यर्थविषयकबोधजनकत्वम् , तद्वत् आरण्यकोक्तवचनश्रवणानन्तरं कालान्तरे पिण्डप्रत्यक्षेण तद्गतगवयत्वज्ञानात् तत्सहकृतमेव वाक्यस्मरणं संज्ञासंज्ञिसम्बन्धबोधं जनयतीति अस्य शब्दत्वमेवेति चेन्न | आकाङ्क्षायोग्यतादीनां सत्त्वे वाक्यश्रवणानन्तरं शाब्दबोधो भवत्येव | कारणानां सर्वेषां समवधाने सति कार्यस्यावश्यमुदयात् | वेदाध्ययनस्थले तु तात्पर्यादीनामज्ञानात् कारणाभावादेव न कार्योत्पत्तिः | प्रकृते गोगवयसादृश्ययोः सामानाधिकरण्यात् आकाङ्क्षादिकारणानां समवधानाच्च 'गोसदृशाभिन्नो गवयशब्दवाच्यः' इत्याकारकबोधो भवत्येव | न च गवयपदप्रवृत्तिनिमित्तस्याज्ञानात् न तत्र बोध इति वाच्यम् | यथा 'घटो रक्तो न वा' इति सन्देहदशायामपि 'घटो भवति' इति वाक्यात् शाब्दबोधो भवति तद्वत् प्रकृतेऽपि शाब्दबोधसम्भवात् | अन्यथा यथाश्रुतान्वये सति जिज्ञासायां पुनरन्वयः इष्यते तदा वाक्यभेदापत्तिः |

न च 'गङ्गायां घोष' इति वाक्यात् आदौ गङ्गादिपदार्थानां विभक्त्याद्यर्थेषु अन्वयो भवत्येव | ततः घोषपदार्थे अन्वयानुपपत्त्या वृत्त्यन्तरेणार्थोपस्थितिः ततः शाब्दबोधः | तथा च अनेन वाक्येनैव यथा पदार्थान्तरविषयकशाब्दबोधो यथा तथैव प्रकृतेऽपि गवयत्वादिविषयकशाब्दबोधः इति वाच्यम् | एवंरीत्या यद्युच्यते तर्हि 'पीनो देवदत्तः दिवा न भुङ्के' इति वाक्यादेव रात्रिभोजनविषयकबोधापत्तिः | तथा च 'गङ्गायां घोष' इत्यत्र अन्वययोग्यताभावात् वृत्त्यन्तरोपस्थापितेनार्थेनान्वयोभवति, न तु अपदार्थेन | प्रकृते तादृशानुपपत्तीनामभावात् कारणसमुदायस्य च सत्त्वात् गोसादृश्यप्राकारकबोध एव जायते | स च न संज्ञासंज्ञिसम्बन्धविषयक इति न तदुपमानप्रमाणम् | तदुक्तं कुसुमाञ्जल्याम् -

{\fontsize{11.7}{0}\selectfont\s 'श्रुतान्वयादनाकाङ्क्षं न वाक्यं ह्यन्यदिच्छति |\\ पदार्थान्वयवैधुर्यात्तदाक्षिप्तेन सङ्गतिः ||'\\ 'गो सदृशो गवयशब्द वाच्य' इति सामानाधिकरण्यमात्रेणान्वययोपपत्तौ विशेषसन्देहेऽपि वाक्यस्य पर्यवसितत्वेन मानान्तरोपनीतानपेक्षणात् | रक्तारक्तसन्देहेऽपि 'घटो भवति' इति वाक्यवत् , अन्यथा वाक्यभेददोषात् | न च 'गङ्गायां घोष' इतिवत् पदर्था एवान्वयायोग्याः, येन प्रमाणान्तरोपनीतेनान्वयः स्यात् | प्रतीतवाक्यार्थबलायातोऽप्यर्थो यदि वाक्यस्यैव, दिवाभोजननिषेधवाक्यस्यापि रात्रिभोजनमर्थः स्यात् |तस्माद्यथा गवयशब्दः कस्यचिद्वाचकः शिष्टप्रयोगादिति सामान्यतः निश्चितेऽपि विशेषे मानान्तरापेक्षा, तथा गोसदृशस्य गवयशब्दो वाचक इति वाक्यान्निश्चितेऽपि सामान्ये, विशेषवाचकत्वेऽस्य मानान्तरमनुसरणीयम् |\footnote{न्या.कु. ३८०, ३८१}}

\section{उपमानस्य लक्षणम्}

उपनानस्य लक्षणन्तावत् अनवगसङ्गतिसंज्ञासमभिव्याहृतवाक्यार्थस्य संज्ञिन्यनुसन्धानमिति | अनवगतसङ्गतिश्चासौ संज्ञाचेति कर्मधारयः | तस्तमभिव्याहृतवाक्यार्थः आरण्यकवाक्यश्रवणादुत्पन्नः | तस्य गवयरूपसंज्ञिसाक्षात्कारे सति अनुसन्धानम् - स्मरणमेव उपमानमित्यर्थः | तथा च अतिदेशवाक्यार्थस्मरणमुपमानमिति ज्ञायते | एतेन वैधर्म्यादिज्ञानेनापि उपमिति सम्भवात् नाननुगमः | तदुक्तं कुसुमाञ्जल्याम् - 

{\fontsize{11.7}{0}\selectfont\s लक्षणन्त्वस्य अनवगतसङ्गतिसंज्ञासमभिव्याहृतवाक्यार्थस्य संज्ञिन्यनुसन्धानमुपमानम् | वाक्यार्थश्च क्वचित् साधर्म्यं क्वचिद्वैधर्म्यमतो नाव्यापकम् , तस्मान्नियतविषयत्वादेव न तेन बाधः, न त्वनतिरेकादिति स्थितिः |\footnote{न्या.कु. २८६}}

\subsection{वैधर्म्यज्ञानमप्युपमानम्}

क्वचित् वैधर्म्यज्ञानादपि उपमितिरुदेति | तथा हि - दक्षिणापथं गन्तुकामेन उत्तरापथं गन्तुकामं प्रत्युक्तं 'धिक्करभमतिदीर्घग्रीवं प्रलम्बचपलौष्ठं कठोरतीक्ष्णकण्टकाशिनं कुत्सितावयवसन्निवेशमपसदं पशूनाम्' इति | अत्र पशूनामपसदम् इत्यस्य पशूनां मध्ये अपकृष्टमित्यर्थः | तेन पशुवैधर्म्यमेव करभे भासते | एतादृशवैधर्म्यज्ञानेनापि कश्चन  उत्तरापथं गत्वा तादृशपिण्डं दृष्ट्वा करभपदतदर्थयोः सम्बन्धं परिच्छिनत्ति | अतः वैधर्म्यज्ञानमपि उपमानमेव | तदुक्तं तत्त्वचिन्तामणौ - 

{\fontsize{11.7}{0}\selectfont\s  यदोदीच्येन क्रमेण कं निर्गत्योक्तं धिक्करभमतिदीर्घग्रीवं प्रलम्बचपलौष्ठं कठोरतीक्ष्णकण्टकाशिनं कुत्सितावयवसन्निवेशमपदं पशूनामिति तदुपश्रुत्य दाक्षिणात्य उत्तरापथं गत्वा तादृशं वस्तूपलभ्य नूनमसौ करभ इति प्रत्येति | तत्र किं मानम् ? न तावदुपमानं सादृश्याभावात् , न च प्रमाणान्तरं सम्भवति |\footnote{त.चि. ६१,६२}}

\subsection{कारणताज्ञानमप्युपमानम्}

\subsection{असाधारणज्ञानमप्युपमानम्}
