\section{संशयस्य परोक्षत्वविचारः}

अयथार्थानुभवस्त्रिविधः संशयविपर्ययतर्कभेदात् | तत्र संशयस्तावत् एकस्मिन् धर्मिणिविरुद्धनानाधर्मावगाहिज्ञानम् | यथा 'अयं स्थाणुः पुरुषो वा', 'पर्वतो वह्निमान्न वा' इत्यादिः | स चान्तः बाह्यश्चेति द्विविधः |

\subsection{संशयस्यानुमानगम्यत्वमभ्युपगच्छन्ति वैशेषिकाः}

तत्र बाह्यसंशयोऽपि प्रत्यक्षपरोक्षभेदात् द्विविधः | साधारणधर्मदर्शनात् तद्धर्मिणि विरुद्धानेकधर्मितावच्छेदकप्रकारकसंशयः परोक्षः इति वैशेषिकानामाशयः |

\subsubsection{विषाणदर्शनेन गोत्वगवयत्वप्रकारकसंशयः}

अतिदूरे विद्यमानपदार्थं दृष्ट्वा 'अयं स्थणुर्वा पुरुषो वा' इति संशयो यो जायते स प्रत्यक्षात्मकः संशयः | कदाचिदटव्यां विषाणमात्रदर्शनेन 'अयं गौर्गवयो वा' इति यः संशयः स परोक्षः अनुमित्यात्मकः इति | तत्र गोः गवयस्य च विषाणस्य आकारसाम्यात् गोत्वेन गवयत्वेन च साकं विषाणस्य व्याप्तिः गृहीता | किन्त्वरण्ये केवलविषाणदर्शनमुभयविधव्याप्तिमपि स्मारयति आकारसाम्यात् | विषाणस्य प्राणिवृत्तित्वस्य निश्चितत्वात् पक्षधर्मताज्ञानादिकमुदेति | तथा च 'अयं गोत्वव्याप्यविषाणवत्', 'अयं गवयत्वव्याप्यविषाणवत्' इति परामर्शद्वयसंवलने सति 'अयं गौः गवयोवा' इति संशयो जायते इति |

{\fontsize{11.7}{0}\selectfont\s अत्र प्रशस्तपादः – बहिःसंशयः द्विविधः । प्रत्यक्षविषये चाप्रत्यक्षविषये च । तत्राप्रत्यक्षविषये तावत् साधारणलिङ्गदर्शनादुभयविशेषानुस्मरणादधर्माच्च संशयो भवति । यथाऽटव्यां विषाणमात्रदर्शनाद् 'गौर्गवयो वे'ति । प्रत्यक्षविषयेऽपि स्थाणुपुरुषयोरूर्ध्वतामात्रसादृश्यदर्शनाद् वक्रादिविशेषानुपलब्धितः स्थाणुत्वादिसामान्यविशेषानभिव्यक्तावुभयविशेषानुस्मरणादुभयत्राकृष्यमाणस्यात्मनः प्रत्ययो दोलायते -  किं नु खल्वयं स्थाणुः स्यात् पुरुषो वेति ?\footnote{प्र.भा.}}

{\fontsize{11.7}{0}\selectfont\s वाटान्तरितस्य  पिण्डस्याप्रत्यक्षस्य सामान्येन विषाणमात्रदर्शनानुमितस्य संशयविषयत्वादप्रत्यक्षविषयोऽयं संशयः ।\footnote{न्या.कं.}}

\subsubsection{प्रत्यक्षापेक्षया अस्य वैलक्षण्यम्}

सामान्यधर्मद्वारा धर्मिणां प्रत्यक्षे जाते तत्र विशेषधर्मपुरस्सरेण तत्प्रत्यक्षं प्रति तद्गतासाधारणधर्मः सहकरोति | यथा कृष्णवर्णरूपसामान्यधर्मेण दृष्टयोः काकपिकयोः मध्ये विशिष्यपिकत्वेन प्रतीतिस्तु तद्गतमधुरध्वनिरूपासाधारणधर्मश्रवणे सत्येव भवति | प्रकृते गोगवयविषाणेषु सम्यात् तदेकत्रासाधारणधर्मो न भवति | अतस्तेन विशिष्य गोत्वेन वा गवयत्वेन वा साक्षात्कारो न भवति | किन्त्वत्र तदाश्रयतया धर्मिद्वयोपस्थितिस्तु जायते | यतः तदुभयवृत्तित्वस्य पूर्वं गृहीतत्वात् | एवमेव उभयत्र तद्वृत्तित्वं तु बाधितमेव | एवं प्रकृते धर्मिणा सह सन्निकर्षाभावात् साधारणरूपेणापि धर्मिज्ञानं न जायते | तस्मादत्र संशयः धर्मिसाक्षात्काराभावात् परोक्ष एव |

{\fontsize{11.7}{0}\selectfont\s बहिर्द्विविधः | केन रूपेण ?‌ प्रत्यक्षविषये चाप्रत्यक्षविषये चेति | तत्राप्रत्यक्षविषये तावदभिधीयते संशयः साधारणलिङ्गदर्शनादिति | यल्लिङ्गं विरुद्धविशेषैः‌ सहोपलब्धं तददर्शनाद्विशेषानुपलब्धेरुभयविशेषानुस्मरणादधर्माद्दिक्कालादिभ्यश्च संशयो भवति | यथा अटव्यां‌ विषाणमात्रदर्शनात्तद्गतविशेषानुपलब्धेः तत्स्मरणात्संशयो भवति 'गो र्गवयो वा' इति | अथ प्रत्यक्षादस्य विशेषः कथम् ? सामान्यविशेषितस्य धर्मिणः प्रत्यक्षत्वं विशेषलक्षणाभिधानात् | अथ किं गोगवयविषाणेषु सामान्यम् , तुल्यावयवरचनायोगः | यथाभूता ह्यवयवरचना गोविषाणे दृष्टा तथा गवयविषाणेऽपीति | ये तु गोविषाणविशेषा गवयविषाणविशेषाश्च पूर्वोपलब्धास्तेषामत्रानुपलम्भ इति | अतः सादृश्यमात्रविशेषितं विषाणमुपलभमानस्य विशेषानुपलब्धेः तत्स्मरणाच्च किं गौः स्यात् गवयो वेति संशयः |\footnote{व्यो . ५३६,५३७}}

\subsubsection{रत्नकोशकारमतम्}

परस्परविरुद्धप्रकारकपरामर्शसमवधाने सति संशयनुमितिरिति रत्नकोशकाराः | तथा हि - वह्नितदभावव्याप्यवत्तावगाहिपरामर्शसंवलने सति वह्न्यनुमितिः न जायते | वह्न्यभावव्याप्यवत्तापरामर्शस्य प्रतिबन्धकत्वात् | वह्न्यभाववत्तानुमितिरपि नोदेति | वह्निव्याप्यवत्तानिश्चयस्य प्रतिबन्धकत्वात् | तस्मात् परस्परविरोधिसामग्र्युपस्थितेः संशयजनकत्वोपगमात् प्रकृते संशयानुमितिः जायते इति |

{\fontsize{11.7}{0}\selectfont\s रत्नकोषकारस्तु सत्प्रतिपक्षाभ्यां प्रत्येकं स्वसाध्यानुमितिः संशयरूपा जायते, विरुद्धोभयज्ञानसामग्र्याः संशयजनकत्वात् | संशयद्वारास्य दूषकत्वम् | न च संशयरूपा नानुमितिः बाधस्येव विरोध्युपस्थितेरनुमितिसामग्रीविघटकत्वेनावधारणात् अन्यथा बाधेऽप्यनुमित्यापत्तेरिति वाच्यम् | अधिकबलतया बाधेन प्रतिबन्धात् , तुल्यबलत्वादनुमितिः स्यादेव सामग्रीसत्त्वात् | साध्याभावबोधस्य च तत्र प्रतिबन्धकत्वं न तु तद्बोधकस्य चक्षुरादेः | प्रत्येकं निर्णायकत्वेनावधारितात्कथं संशय इति चेन्न | प्रत्येकाद्धि ज्ञानमुत्पद्यमानं अर्थात् संशयो न तु प्रत्येकं संशयजनकत्वमिति मेने |\footnote{स.त.म. ६०}}

\subsubsection{शब्दज्ञानादपि संशयो भवति}

शब्दज्ञानादपि संशयो जायते इत्यपि केचित् | 'पर्वतो वह्निमान्' , 'पर्वतो वह्न्यभाववान्' इति वाक्यद्वयात् 'पर्वतो वह्निमान्नवा' इति संशयो जायते | परस्परविरुद्धसामग्रीसमवधानस्य संशयजनकत्वनियमात् इति | 

{\fontsize{11.7}{0}\selectfont\s पदजन्यधर्मिककोटिद्वयतदुभयविरोधिज्ञानसंशयात्मकयोग्यताज्ञानसहितात् शब्दादाहत्यैव संशयः | “समानानेके"त्यादिसूत्रं प्रमाणयतो महर्षेरपि सम्मतमिदम् |\footnote{प.त.नि.१४२}}

\subsection{तत्र न्यायविदामाशयः}

परस्परविरुद्धानेकप्रकारकज्ञानात्मकः संशयः सर्वदा अपरोक्ष एव | स्वप्रकारकज्ञानजनने स्वविरुद्धप्रकारकसामग्र्याः प्रतिबन्धकत्वात् न परोक्षः संशयसम्भवः | प्रत्यक्षस्थले तु विषययस्य कार्यसहभावेन कारणत्वात् परस्परविरुद्धसामग्रीसत्त्वे संशयो भवत्येव इति न्यायविदामाशयः |

\subsubsection{संशयानुमितिनिराकरणम्}

'पक्षः साध्यवान्न वा' इत्याकारकसंशयाकारानुमित्युत्पत्तौ का सामग्री इति जिज्ञासायां साध्यतदभावप्रकारकपरामर्शः सामग्री इति वक्तव्यम् | किन्तु तादृशपरामर्शद्वयसंवलने सति साध्यवत्ताबुद्धिर्नो देति, साध्याभावव्याप्यवत्तानिश्चयस्य तदभावव्याप्यवत्तानिश्चयविधया प्रतिबन्धकत्वात् | साध्याभाववत्ताबुद्धिरपि नोदेति, साध्यप्रकारकपरामर्शस्य तदभावव्याप्यवत्तानिश्चयविधया प्रतिबन्धकत्वात् | यथा सुरभ्यसुरभ्यवयवारब्धे अवयविनि गन्धो नोत्पद्यते तथैवात्र परस्परप्रतिबन्धादनुमितिरेव नोदेति | अतः न संशयानुमितिरिति |

{\fontsize{11.7}{0}\selectfont\s साध्यतदभावयोर्विरोधेन यथा एकज्ञानस्यापरधीप्रतिबन्धकत्वं तथा साध्याभावव्याप्यवत्त्वस्यापि साध्यविरोधित्वात्तद्बुद्धेरपि साध्यधीप्रतिबन्धकत्वात् , विरोधिज्ञानत्वस्य प्रतिबन्धकत्वे तन्त्रत्वात् |\footnote{स.त.म. ६१}}

\subsubsection{अव्याप्यवृत्तिसाध्यकस्थले परोक्षसंशयसम्भवः}

अव्याप्यवृत्तिसाध्यकस्थले संशयानुमितिः भवितुमर्हति | तद्वत्ता बुद्धिं प्रति तदभाववत्तानिश्चयस्य प्रतिबन्धकत्वं अनुभवसिद्धमेव | तद्वत् तत्रैव अव्याप्यवत्तित्वग्रहदशायां तद्वत्ताबुद्धिरुदेति | तेन अव्याप्यवृत्तित्वग्रहः तदभाववत्तानिश्चयस्य प्रतिबन्धकत्वस्थले उत्तेजकः इति ज्ञायाते | तद्वत् तदभावव्याप्यवत्तानिश्चयस्य यत्र प्रतिबन्धकत्वं तत्रापि बाधकाभावात् अव्याप्यवृत्तित्वग्रहस्य उत्तेजकत्वं स्वीकरणीयम् | तथा च साध्यतदभाववावगाहिपरामर्शदशायां 'साध्यमव्याप्यवृत्ति' इति यदि निश्चयः तदा तत्र प्रतिबन्धकत्वाभावात् अनुमितिरुदेति इति वक्तव्यम् | सा च संशयात्मिका इति तादृशस्थलविशेषे संशयानुमितिप्रसिद्धिरिति | यथा 'वृक्षः कपिसंयोगी एतद्वृक्षत्वात्' इत्यत्र संयोगस्य दैशिकाव्याप्यवृत्तित्वग्रहदशायां साध्यतदभावप्रकारकपरमर्शसत्त्वे शाखावच्छेदेन 'वृक्षः कपिसंयोगी न वा' इति संशयो भवति | एवं 'घटः गन्धवान्' इत्यत्र गन्धस्य कालिकाव्याप्यवृत्तित्वग्रहदशायां गन्धतदभावप्रकारकपरामर्शसत्त्वे एतत्क्षणावच्छेदेन 'घटः गन्धवान्न वा' इत्याकारकसंशयो जायते इति |

{\fontsize{11.7}{0}\selectfont\s परे तु ग्राह्याभावेऽव्याप्यवृत्तित्वग्रहदशायां ग्राह्याभावनिश्चयविषये धर्मिणि ग्राह्यज्ञानोत्पत्त्या यथा तादृशाव्याप्यवृत्तित्वग्रहस्य तत्प्रतिबन्धकतायामुत्तेजकत्वं, तथा तद्धिर्मिकाव्याप्यवृत्तित्वग्रहदशायां तद्व्याप्यवत्तानिश्चयस्यापि तद्विपरीतज्ञानाप्रतिबन्धकतया तत्रापि तद्धर्मिकाव्याप्यवृत्तित्वग्रहस्योत्तेजकत्वमुपेयम् | एवं च कपिसंयोगतदभावयोर्दैशिकाव्याप्यवृत्तित्वग्रहदशायां पक्षतावच्छेदकशाखादिरूपैकावच्छेदेन तदुभयकोट्यवगाहिनी गन्धतदभावयोः कालिकाव्याप्यवृत्तित्वग्रहदशायां पक्षतावच्छेदकतत्क्षणावच्छेदेन तदुभयकोटिमवगाहमाना च संशयानुमितिः ग्राह्याभावज्ञानप्रतिबन्धकतया न शक्यते वारयितुम् |\footnote{स.गा.९३}}


\subsubsection{विमर्शः}

एकस्मिन् धर्मिणि परस्परविरुद्धानेकोट्यवगाह्यनुभवात्मकः संशयः प्रधानतया बाह्यः मानसश्चेति द्विविधः | तत्र ज्ञानाद्यात्मगुणविषयकस्तावदन्तः संशयः | सैव मानसः इत्युच्यते | तथा हि ज्योतिर्विद्भिः ग्रहादिगत्यनुसारेण ऊहादिकं यत्र क्रियते तत्र तत्र बहिस्थपदार्थेन सह इन्द्रियव्यापाराभावेऽपि ज्ञानमुत्पद्यते | तत्र तेषाम् 'इदं सत्यं मिथा वा' इत्यादिसंशयो यो जायते सः मानसः | बाह्यः पुनर्द्विविधः प्रत्यक्षविषयकः अप्रत्यक्षविषयकश्चेति | तत्र हि अरण्यादौ अतिदूरे विद्यमानं ऊर्धताधर्मविशिष्टं वस्तु दृष्ट्वा तत्र विशेषेण करादीनां पुरुषविशेषधर्माणाम् , वक्रकोटरादीनां स्थाणुविशेषधर्माणां चादृष्ट्वा संशीते 'अयं स्थाणुः पुरुषो वा' इति | अत्र स्थाणुपुरुषसाधारणोर्धतामात्र विशिष्टेन संशयविषयीभूतेन धर्मिणा सह इन्द्रियसन्निकर्षादस्य प्रत्यक्षत्वमुपपद्यते |

द्वितीयस्तावत् अरण्ये विषाणमात्रदर्शनेन 'गौर्गवयो वेति' संशयः | तत्र हि इन्द्रियसन्निकृष्टे विषाणे गोः गवयस्य वा विशेषधर्माग्रहात् उभयविषाणेषु आकारसाम्यस्य पूर्वं ज्ञातत्वात् संशयो जायते | अत्र जायमानसंशयधर्मी पिण्डविशेषः नेन्द्रियसन्निकृष्टः | अपि तु विषाणरूपहेतुना अनुमितः इति अस्य परोक्षत्वमिति | तथा च यत्र संशयविषयीभूतधर्मी इन्द्रियसन्निकृष्टः तत्र संशयः प्रत्यक्षः | यत्र तु धर्मी प्रमाणान्तरसिद्धः तत्र संशयः परोक्षः इति वैशेषिकाः |

रन्तकोषकारास्तु - परस्परविरुद्धप्रकारकपरामर्शसमवधाने सति संशयनुमितिरिति वदन्ति | तथा हि - वह्नितदभावव्याप्यवत्तावगाहिपरामर्शसंवलने सति वह्न्यनुमितिः न जायते | वह्न्यभावव्याप्यवत्तापरामर्शस्य प्रतिबन्धकत्वात् | वह्न्यभाववत्तानुमितिरपि नोदेति | वह्निव्याप्यवत्तानिश्चयस्य प्रतिबन्धकत्वात् | तस्मात् परस्परविरोधिसामग्र्युपस्थितेः संशयजनकत्वोपगमात् प्रकृते संशयानुमितिः जायते इति | न च वह्नितदभावप्रकारकपरामर्शयोः परस्परविरोधात् सुरभ्यसुरभ्यारब्धद्रव्ये यथा गन्धाभावः तद्वत् कार्यमेव नोत्पद्यते इति वाच्यम् | क्वचित् चित्रपटादौ परस्परविरोधसामग्रीदशायामपि कार्योत्पत्तेः दर्शनात् | न च विरोधिनिश्चयसत्त्वे कार्योत्पत्त्यभ्युपगमे वह्न्यभाववत्तानिश्चयदशायामपि वह्निमत्ताबुद्धिप्रसङ्गः | कार्योत्पत्तौ विरोधिनिश्चयस्य अकिञ्चित्करत्वापत्तिः इति वाच्यम् | यदा निश्चयस्याधिकबलत्वं तदानीमेव तस्य कार्योत्पत्तिविघटकत्वम् | बाधस्थले तु वह्निमत्तबुध्युत्पत्तौ वह्न्यभाववत्तानिश्चयस्य अधिकबलत्वात् प्रतिबन्धकत्वमेव | यदा तु समानबलत्वं तदा निश्चययोः कार्योत्पत्तेरनुभवात् न तयोः प्रतिबन्धकत्वम् | तथा च तत्र यत्कार्यमुत्पद्यते तदर्थात् संशयात्मकं भवति इति |

शब्दज्ञानादपि संशयो जायते इत्यपि केचित् | वादस्थले वादिना यदा प्रतिज्ञा क्रियते 'पर्वतो वह्निमान्' इति, ततः परं प्रतिवादिना 'पर्वतो वह्न्यभाववान्' इति या प्रतिज्ञा क्रियते तदुभश्रवणात् मध्यस्थस्य यः संशयो जायते सः परोक्षः शब्दादुत्पन्नः इत्यतः शाब्दः एव | तत्रापि परस्परविरुद्धवह्नितदभावावगाहिशाब्दबुद्धिजनकवाक्ययोः समवधानात् | परस्परविरुद्धसामग्रीसमवधानस्य च संशयजनकत्वनियमात् इति | तथा चात्रापि परस्परविरुद्धसामग्रीद्वयसमवधाने सति तयोः कार्यविघटकत्वाभावात् संशयो जायते इत्याशयः | मैवम् |

संशयः सर्वोऽप्ययरोक्ष एव | न हि अरण्यादौ विषाणमात्रदर्शनाद् 'गौर्गवयो वा' इति संशयः परोक्षः | तत्र विषाणरूपहेतुना प्राणिविशेषस्यैव निश्चितत्त्वात् | निश्चितप्राणिविशेषे च विशेषधर्मजिज्ञासायां गोः गवयस्य च संस्कारबलात् एकसम्बन्धिज्ञानमिति रीत्या स्मरणं जायते | ततः जिज्ञासावशात् नष्टस्य निश्चयस्य पुनः संस्कारवशात् स्मरणे जाते मानसव्यापारेण संशयो जायते | तथा च तस्य मानसप्रत्यक्षत्वमेव | न हि ज्ञानाद्यात्मागन्तुकगुणतदभावकोटिक  एव संशयः मानसो भवति | मनसा आत्मवृत्तिज्ञानेन सह सन्निकर्षे जाते तद्गतविषयस्यापि अवगाहनात् | निर्विषयकज्ञानस्याप्रसिद्धत्वाच्च | किञ्च विषाणरूपहेतुना प्राणिविशेषस्यैव कल्पनासम्भवात् गोत्वादिधर्मेण सह तस्य व्याप्त्यभावात् | तथा च तत्र प्राणिविषयकनिश्चय एव जायते | न तु संशयानुमितिरिति |

न हि विरुद्धोभयपरामर्शसंवलनदाशायां संशयानुमितिर्जायते | तत्र कार्योत्पत्तौ परस्परप्रतिबन्धकस्य सत्त्वात् कार्यमेव नोत्पद्यते | न च चित्ररूपादौ परस्परप्रतिबन्धकस्य सत्त्वे कार्योत्पत्तिरङ्गीकृता | तत्रापि नीलपीततन्तूनां समवधाने सति पटे नीलरूपमुत्पद्यते | पीततन्तूनां प्रतिबन्धकत्वात् | न हि पीतरूपं नीलतन्तूनां प्रतिबन्धकत्वात् | तथा च अन्यदेव कार्यमुत्पद्यते | तत्रापि पटप्रत्यक्षत्वानुपपत्त्या रूपान्तरमङ्गीकृतम् | अत एव सुरभ्यसुरभ्यवयवारब्धे अवयविनि लाघवाच्चित्र गन्धः न कल्पितः | प्रकृतेऽपि बाधकाभावात् विरुद्धोभयपरामर्शसंवलनदशायां कार्योत्पत्तिः नोच्यते | तस्मात् कार्यस्यैव तत्रानुत्पादान्न संशयानुमितिसम्भवः |

यदुक्तं वादस्थले वह्नितदभावादिप्रकारकशब्दज्ञानवशात् शाब्दः संशयो जायते इति | तदप्यसत् | तत्र हि वाद्युक्तवह्निप्रकारकवचनश्रवणात् तदर्थोपस्थित्युत्तरक्षणे वह्निप्रकारकनिश्चय एव उत्पद्यते | ततः प्रतिवाद्युक्तवह्न्यभावप्रकारकवचनश्रवणात् अर्थोपस्थित्त्या वह्न्यभावप्रकारकशाब्दः निश्चयः | तदनन्तरं वह्नितदभावयोः स्मरणे तयोः विरोधस्मरणात् मानस एव संशयो जायते | अतः न शाब्दोऽसंशयः सम्भवति | न च 'पर्वतो वह्निमान्न वा' इति वाक्यात् संशयाकारशाब्दोत्पत्तिर्भवति | वह्निप्रकारकशाब्दबुद्धिं प्रति वह्न्यभावप्रकारकनिश्चयस्य वह्न्यभावप्रकारकशाब्दंप्रति वह्निप्रकारकनिश्चयस्य च प्रतिबन्धकत्वकल्पनात् | अन्यथा  ह्रदादौ प्रमाणान्तरेण वह्न्यभाववत्तानिश्चयदशायामपि 'ह्रदो वह्निमान्' इति वाक्यात् शाब्दबोधप्रसङ्गः | तस्मान्न संशयस्य न शाब्दत्वमिति |

ननु अव्याप्यवृत्तिसाध्यकस्थले 'कपिसंयोगी एतद्वृक्षत्वात्' इत्यादौ कपिसंयोगतदभावावगाहिपरामर्शयोः संवलनदशायां सति 'कपिसंयोगो अव्याप्यवृत्तिः' इति ज्ञाने संशयानुमितिरुत्पद्यते | अव्याप्यवृत्तित्वप्रकारकज्ञानस्य उत्तेजकत्वात् | यथा 'कपिसंयोगाभाववान्' इति निश्चयदशायामव्याप्यवृत्तित्वप्रकारकज्ञाने सति कपिसंयोगवत्ताबुद्धिरुदेति तत्र यथा अव्याप्यवृत्तित्वप्रकारकज्ञानस्य उत्तेजकत्वं तथैव तदभावव्याप्यवत्तानिश्चयदशायामपि तस्य उत्तेजकत्वाभ्युपगम्यते | न च अव्याप्यवृत्तित्वग्रहदशायां तस्य संशयप्रतिबन्धकत्वात् समुच्चय एव स्यादिति वाच्यम् | अव्याप्यवृत्तित्वग्रहस्य कोट्योः विरोधांशे एव प्रतिबन्धकत्वम् , न तु संशयप्रतिबन्धकत्वम् | तस्माद्विरोधानवगाहिसंशयो जायते | न च संशयस्य नियमेन विरोधविषयकत्वं स्वीकर्तव्यम् | अन्यथा संशयसमुच्चययोः भेद एव न स्यादिति वाच्यम् | अरण्यादौ निरुक्तरीत्या 'सः गौः गवयो वा' इति संशये, 'चैत्रः मैत्रो वा' इत्यादिसंशये च कोट्योः विरोधाभावात् | न च तत्र वस्तुतः विरोधाभावेऽपि विरोधो अवगाहते इति वाच्यम् | 'भूतलं घटवत् पटवदुभयवद्वा' इत्यादिसंशयस्यापि आनुभविकतया तत्र घटपटोभयवृत्तित्वस्यापि संशये भानात् | अन्यथा घटपटयोः विरोधभाने उभयवृत्तित्वप्रकारकसंशय एव न स्यात् | 

तर्हि संशयसमुच्चययोः किं वैलक्षण्यमिति चेत् , संशये विरोधभानस अनियततया संशयत्वञ्च 'सन्देह्मि' इत्यनुव्यवसायबलात् संशयवृत्तिकोटिताख्यविषयताविशेषः एव कल्प्यते | तथा च समुच्चयोत्तरं 'सन्देह्मि' इत्यनुव्यवसायानुदयात् तत्र एतादृशविषयताविशेषाभावात् तयोः वैलक्षण्यमिति | 

तथा च अव्याप्यवृत्तिसाध्यकस्थले 'कपिसंयोगी एतद्वृक्षत्वात्' इत्यादौ साध्यतदभावविषयकपरामर्शयोः संवलनदशायाम् अव्याप्यवृत्तित्वप्रकारकग्रहे च सति 'अग्रावच्छेदेन वृक्षः कपिसंयोगी न वा' इत्याकारकसंशयात्मकानुमितिः भवति | अन्यत्र सर्वत्रापि अपरोक्ष एव संशयो जयते इति सिद्धम्  |
