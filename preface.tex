\chapter{प्रतिज्ञापत्रम्}

राष्ट्रियसंस्कृतसंस्थानस्य राजीवगान्धीपरिसरे सहाचार्यपदमलङ्कुर्वाणानां प्रातःस्मरणीयानां नवीवनन्यायपाठनरतानामस्माकं विद्यागुरूणां डा. नवीनहोळ्ळमहोदयानां दिङ्निर्देशेन अनुसन्धाताहं “प्राचीनन्याय-वैशेषिक-नव्यन्यायशास्त्रेषु तत्तद्वाख्याकाराणां सैद्धान्तिकमतभेदानां सङ्कलनपूर्वकं विमर्शात्मकमध्यययनम्” इति शीर्षकालङ्कृतस्य शोधप्रबन्धस्य समर्पणावसरे समुद्घोषयामि यदेषः शोधप्रबन्धः ऐदं प्राथम्येन राघवेन्द्र पी आरोल्लीनाम्ना  मया गूरूणामाशिषा स्वप्रयत्नेन अपेक्षितसकलसामग्रीः सङ्कलय्य विश्वविद्यालयानुदानायोगस्य २००८ संवत्सरस्याधिनियमानुसारं रचितः । तथा राष्ट्रियसंस्कृतसंस्थानस्य शृङ्गगिरिस्थराजीवगान्धीपरिसरे विद्यावारिधिः() इत्युपाधये समर्प्यत इति प्रतिजाने । यदसौ विषयः न केनापि कुत्रचिद्विश्वविद्यालये शोधकार्यं कर्तुं स्वीकृतः इत्यपि प्रतिजाने ।
\bigskip

\noindent									
स्थानम् –\hfill अनुसन्धाता\\
दिनाङ्कः -\hfill {\large\bfseries राघवेन्द्र पी. आरोल्ली}

\chapter{कार्तज्ञ्यवचनम्}

	लोके हि सर्वेऽपि व्यवहाराः शब्दमूलकाः एव । पदार्थोपस्थितिस्तु शब्दज्ञानादेव इति शब्दार्थयोः कश्चन सम्बन्धः अङ्गीकर्तव्यः । अस्मिन् विषये शास्त्रकारेषु महती विप्रतिपत्तिः दरीदृश्यते । स च सम्बन्धः बोधकत्वरूप इति, प्राप्ति रूप इति, वाच्यवाचकभाव इति बहुधा उररीक्रियते । नैय्यायिकैस्तु अभिधानाभिधेयनियमनियोग एव शब्दाद् अर्थप्रत्यये सम्बन्धत्वेन स्वीक्रियते । स च सङ्केतरूपः । ईश्वरेच्छा इत्यपि अभिधीयते । आधुनिकं चेत् तत्पारिभाषिकमिति कथ्यते । किञ्च शब्देन सम्बद्धोऽर्थः व्यक्तिः उत जातिः उत जात्युपहितव्यक्तिः इत्यत्रापि महती विप्रतिपत्तिः । जातिशक्तिवादिनो मीमांसकाः, व्यक्तिशक्तिवादिनो नैयायिका इति प्रथितम् । मणिग्रन्थे अत्र प्राभाकरमतं, भाट्टमतं, कुब्जशक्तिवादिमतं, श्रीकरमतं, मण्डनमतं यथावसरं समुपादाय रोचको विचारः प्रवर्तितो दृश्यते । एवं पङ्कजपदस्य रूढिः स्वीकर्तव्या न वेति विचारः न्यायमीमांसामतावलम्बनेन बहुविधयुक्त्युपन्यासपुरस्सरं उपस्थापितः । एवं लक्षणाविषये पदसाधुविषयेऽपि । अत्र स्वपक्षविपक्षसाधकबाधकयुक्तीनाम् आकलनं परिशीलनञ्च महत्त्यै व्युत्पत्तये तत्त्वनिर्णयाय च भवति । अतः विषयेऽस्मिन् अध्ययनं संशोधनं विमर्शश्च तत्त्वबुभुत्सूनां व्युत्पित्सूनां च महते उपकाराय भवति । अत एव “तत्त्वचिन्तामणौ शक्तिवादस्य विमर्शात्मकमध्ययनम्” इति विषयः संशोधनाय स्वीकृतः । 




\chapter{उपोद्घातः}
	
	इह खलु समेऽपि प्रेक्षावन्तः सुखावाप्तिं दुःखनिवृत्तिञ्च अभिलषन्तः तत्साधनं प्रथममन्विष्यन्ति । निरतिशयसुखप्राप्तिः दुःखनिवृत्तिर्वा परमपुरुषार्थः मोक्ष इति जेगीयते । स एव विवेकिनामिच्छाविषयः । मोक्षसाधनत्वेन अनेकानि शास्त्राणि प्रावर्तन्त । तेषु गौतमीयं न्यायशास्त्रं मूर्धन्यत्वेन विभाव्यते । तदिदं निजगाद मणिकारः प्रामाण्यवादोपक्रमे – 

	“अथ जगदेव दुःखपङ्कनिमग्नमुद्दिधीर्षुः परमकारुणिको मुनिः अष्टादशविद्यास्थानेषु अभ्यर्हिततमामान्वीक्षिकीं प्रणिनाय” इति । तत्र “प्रमाणप्रमेयसंशयप्रयोजनदृष्टान्तसिद्धान्तावयवतर्कनिर्णयवादजल्पवितण्डाहेत्वा-भासच्छलजातिनिग्रहास्थानानां तत्त्वज्ञानान्निःश्रेयसाधिगमः”1 इति । सूत्रेण शास्त्रस्यास्य परमप्रयोजनं निश्श्रेयसाधिगमं निरदिक्षत् महर्षिगौतमः ।

	अनु ईक्षा – अन्वीक्षा तद्व्युत्पादकं शास्त्रम् – आन्वीक्षिकीविद्या इति तदर्थः । आन्वीक्षिकी च न्यायशास्त्रमेव, गौतममुनिप्रणीतमिदं न्यायशास्त्रम् अष्टादशविद्यास्थानेषु प्रधानतया गणितम् । 

	अष्टादशविद्यास्थानानि च – \\
		अङ्गानि वेदाश्चत्वारो मीमांसा न्यायविस्तरः\\ 
		पुराणं धर्मशास्त्रञ्च विद्या ह्येताश्चतुर्दश ।\\
		आयुर्वेदो धनुर्वेदः गान्धर्वश्चेति ते त्रयः \\
		अर्थशास्त्रञ्च विज्ञेयं विद्या अष्टादशैव तु ॥ \\

\chapter{शब्दप्रामाण्यविचारः }
	
	न्यायशास्त्रे चत्वारि प्रमाणानि निरूपितानि । तान्येव सूत्रितानि भगवता – “प्रत्यक्षानुमानोपमानशब्दाः प्रमाणानि”2 इति । प्रमाणत्वञ्च प्रमितिकरणत्वम् । तानि प्रमाणानि च चत्वारि, ततः प्रमाः अपि चतस्रः । ताश्च प्रत्यक्षानुमित्युपमितिशाब्दाः। तत्र प्रत्यक्षप्रमितिं प्रति प्रत्यक्षप्रमाणं कारणम् । अनुमितिं प्रति अनुमानं कारणम् , उपमितिं प्रति उपमानं कारणं, शाब्दं प्रति शब्दः कारणं भवति । न हि वैशेषिकैः शब्दः पृथक्प्र3माणतया अङ्गीक्रियते । शब्दमपि अनुमाने एव अन्तर्भावयन्ति ते । तथा हि प्रमाणभेदे प्रवृत्तिभेदः बीजम् । प्रवृत्तिभेदसत्वात् शब्दानुमानयोः पार्थक्यं व्यवस्थापयति आचार्यः गौतमः । तथा च मुनिरसौ शब्दप्रमाणस्यानुमाने अन्तर्भावमाशङ्क्य निराकरोति । तथा च पूर्वपक्षसूत्रम् – “शब्दोनुमानमर्थस्यानुपलब्धेरनुमेयत्वात्”4 इति । न हि शब्दः प्रमाणान्तरम् । तस्य च अनुमानविधयैव प्रत्ययजनकत्वात् । यथा हि प्रत्यक्षप्रमाणेनाज्ञायमानं सत् साध्यं ज्ञातेन लिङ्गेन अनुमीयते अनुमानस्थले, तद्वत् यः शब्दः श्रूयते, तेन ज्ञातेन शब्देन च पश्चाद् अर्थोऽवगम्यते । स च प्रत्यक्षादिना   नावगम्यते । तेन शब्दोऽपि लिङ्गविधयैव प्रतीतिजनकः । अतः शब्दोनुमानमेव  इति ।
	
	किञ्च शब्दोनुमानमित्याशङ्क्य सूत्रान्तरं प्रादर्शि भगवता – “उपलब्धेरद्विप्रवृत्तित्वात्”5 इति । तथा च यत्र अर्थबोधनप्रकारः विलक्षणः दृश्यते तत्रैव स्थले द्वयोः प्रमाणयोः भेदः अङ्गीक्रियते । यथा प्रत्यक्षानुमानयोः अनुमानोपमानयोश्च  । प्रत्यक्षे हि अज्ञायमानादिन्द्रियात् अर्थज्ञानं भवति । उपमानस्थले तु अर्थप्रतिपत्तिः सादृश्यज्ञानादिति प्रमाणप्रवृत्तिभेदात् प्रमाणभेदो उपपद्यते । न हि भवति शब्दानुमानस्थले ईदृशप्रवृत्तिभेदः । यथा ज्ञातेन लिङ्गेन अर्थप्रतीतिः तथैव ज्ञातेन शब्देन अर्थप्रतीतिर्भवति । अतः शब्दः न अनुमानात् प्रथक् प्रमाणम् इति । एवमेव “सम्बन्धाच्च”6 इति सूत्रेण उपायान्तरमपि शङ्कितम् । तथा हि परस्परसम्बन्धिनोः साध्यहेत्वोर्मध्ये हेतुदर्शनानन्तरं हेतुसाध्ययोः स्मृत्यात्मकसम्बन्धज्ञाने सति साध्यप्रमितिर्जायते । तद्वत् परस्परसम्बद्धयोश्च पदार्थयोः शब्दविषयकश्रावणप्रत्यक्षानन्तरं सम्बन्धस्मृतौ सत्यामर्थप्रतीतिः जायते शब्दस्थले इति शब्दोऽनुमानमेव इति । 
	
	अत्र सिद्धान्तसूत्रम् – “आप्तोपदेशसामर्थ्यात् शब्दादर्थसंप्रत्ययः”7 इति । शब्दार्थप्रत्ययो हि आप्तोपदेशसामर्थ्यादेव भवितुमर्हति । न च स्वर्गः, अप्सरसः , उत्तराः कुरवः, सप्तद्वीपाः, समुद्रः इत्याद्यर्थाः प्रत्यक्षेणावगम्यन्ते । न च शब्दमात्रात् तेषां सम्प्रत्ययः जायते । किं तर्हि “शब्दोऽयं आप्तैरुक्तः” इति, या शब्दगता आप्तोक्तता तस्याः ज्ञानादेव समीचीनः प्रत्ययः सम्भवति । न ह्येवं अनुमानादिस्थले भवतीति अवश्यं प्रवृत्तिभेदो अङ्गीकर्तव्यः । प्रवृत्तिभेदे च अभ्युपगते शब्दस्यानुमानात् पृथक् प्रामाण्यमपि सिध्यतीति पूर्वोक्तसूत्रद्वयशङ्का निराक्रियते । यत् “सम्बन्धाच्च” इत्याशङ्कितं तदपि निराक्रियते ।
