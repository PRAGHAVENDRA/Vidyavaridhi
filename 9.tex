\section{मोक्षकारणविचारः}

\subsection{न्यायकन्दली} प्रत्यहं दुःखैरभिहन्यमानस्य तत्त्वतो विज्ञातेषु दुःखैकनिदानेषु विषयेषु विरक्तस्यात्यन्तिकं दुःखवियोगमिच्छतः 'आत्मख्यातिरविप्लवा हानोपायः', 'तस्याश्च समाधिवोशेषो निबन्धनम्' इति श्रुतवतः 'संन्यस्य सर्वकाम्यकर्माणि समाधिमनुतिष्ठामः' तत्प्रत्यनीकभूयिष्ठं ग्रामं परित्यज्य वनमाश्रित्य यमनियमाभ्यां कृतात्मसंस्कारस्य समाध्यभ्यासान्निवर्तको धर्मो जायते | तस्मादस्य प्रकृष्टः समाधिस्ततोऽन्यः प्रकृष्टतरो धर्मः, तस्मादप्यन्यः‌ प्रकृष्टतमः समाधिरित्यनेन क्रमेणान्त्ये स तादृशः समाधिविशेषः परिणमति, यो द्वन्द्वेनाप्यभिभवितुं न शक्यते | दृष्टो हि किञ्चिदभिमतं विषयमादरेणाचिन्तयतस्तदेकाग्रीभूतचित्तस्य सन्निहितेषु प्रबलेष्वपि विषयेषु सम्बाधः, यथेषुकार इषौ लब्धलक्ष्याभ्यासो गच्छन्तमपि राजानं न बुध्यते | तथा च भगवान् पतञ्जलिः 'अभ्यासवैराग्याभ्यां तन्निरोधः' इति | एवं परिणते समाधावात्मस्वरूपसाक्षात्कारिविज्ञानमुदेति |\footnote{न्या. कं. ६७४,६७५}


\subsection{न्या. सा.} ततोऽचिरेणैव कालेन भगवन्तमनौपम्यस्वभावं शिवमवितथं प्रत्यक्षतः पश्यति । तं दृष्ट्वा निरतिशयं श्रेयः प्राप्नोति । तथा चोक्तं – “यदा चर्मवदाकाशं वेष्टयिष्यन्ति मानवाः । तदाशिवमविज्ञाय दुःखस्यान्तो भविष्यति”, “तमेव विदित्वातिमृत्युमेति” इत्यादि च । तस्माच्छिवसन्दर्शनादेव मोक्ष इति ।\footnote{न्या.सा.१४१}


\subsection{न्यायमञ्जर्यान्तु} कार्यत्वाच्च दुःखस्य कारणोच्छेदात्तदुच्छेदः | कारणं चास्य जन्म | जन्मनि सति हि दुःखं भवति | जायते इति हि जन्म देहेन्द्रियादिसम्बन्धः आत्मनः | तदपि जन्म कारणोच्छेदादेवोच्छेद्यम् | अतस्तत्कारणं प्रवृत्तिरुच्छेद्या | तस्या अपि हेतूच्छेदादुच्छेद इति तद्धेतवो दोषा उच्छेद्याः | तेषां तु निमित्तं मिथ्याज्ञानम् | तस्मिन्नुच्छिन्ने दोषा उच्छिन्ना भवन्तीति मिथ्याज्ञानमुच्छेद्यम् | तदुच्छेदे च तत्त्वज्ञानमुपायः | प्रसिद्धो ह्ययमर्थः समर्थितश्च पूर्वं विस्तरतः तत्त्वज्ञानं मिथ्याज्ञानस्य बाधकमिति | तस्मात् तत्त्वज्ञानान्मिथ्याज्ञानदोषप्रवृत्तिजन्मदुःखनिवृत्तिक्रमेणापवर्ग इति |\footnote{न्या.मं. ४३८}
 


\subsection{विमर्शः}

\subsection{तत्र वेदान्तिनः} ततो हि ईश्वराद्धेतोः, अस्य जीवस्य, बन्धमोक्षौ भवतः — ईश्वरस्वरूपापरिज्ञानात् बन्धः, तत्स्वरूपपरिज्ञानात्तु मोक्षः । तथा च श्रुतिः — ‘ज्ञात्वा देवं सर्वपाशापहानिः क्षीणैः क्लेशैर्जन्ममृत्युप्रहाणिः । तस्याभिध्यानात्तृतीयं देहभेदे विश्वैश्वर्यं केवल आप्तकामः’ (श्वे. उ. १-११) इत्येवमाद्या ॥ ५ ॥\footnote{ब्र.शां.}
