\chapter{उपसंहारः}
	
	अवयवसंयोगरूपाकृतिरपि पदशक्या~। गोपदात् जात्याकृतिविशिष्टस्यैव अनुभवात्~। 

	अत्र चिन्त्यते – व्यक्तिवत् अननुगतानां अवयवसंयोगरूपाकृतीनां शक्यत्वे शक्यानन्त्यं दुर्वारम्~। न च जातिविशेषवान् यो अवयवसंयोगः संस्थानविशेषः तद्रूपाकृतीनां शक्यत्वम्~। तथा च जातिविशेषस्यैव अनुगमकतया नोक्तदोष इति वाच्यम्~। अन्यतरकर्मजत्वादिना सङ्करेण तादृशजात्यप्रसिद्धेः~। न च उपाधिरेवानुगतः सुलभः इति वाच्यम्~। तादृशोपाधेः अनिरुक्तेः~। न च ‘जातिविशेषवदवयवसंयोगः’ इति मणौ जातिविशेषवत्त्वं अवयवविशेषणम्~। तथा च कपालसंयोगादिकमेव तथेति वाच्यम्~। गवादिपदे सास्नाद्यवयवस्याप्यननुगतत्वेन दोषतादवस्थ्यात्~। 

	अत्रोच्यते – जातिविशेषवतः गवादेः अवयवसंयोगस्य आकृतेः शक्यत्वम्~। विशेषपदोपादानेन पृथिवीपदादौ नाकृतिः शक्या इति लब्धम्~। तथा च गवावयवसंयोगत्वादिकमेव अनुगतम् इति न शक्त्यानन्त्यम्~। न चैवं तत्प्रकारकधीः स्यादिति वाच्यम्~। इष्टापत्तेः~। गवादौ आकृतिवैशिष्ट्यञ्च परम्परासम्बन्धेन~। ‘पिष्टकमय्यो गावः’ इत्यत्र तु गोपदस्य गवाकृतिसदृशाकृतौ लक्षणा~। पिष्टकसंयोगविशेषस्याशक्यत्वात्~। एवं जात्याकृतिव्यक्तीनां प्रत्येकमात्रपरत्वे लक्षणैव~। समुदायशक्तस्य पदस्य प्रत्येके लाक्षणिकत्वात्~। न च ‘गौरुत्पन्नः’ इत्यत्र व्यक्तिमात्रपरत्वेऽपि न लक्षणेति वाच्यम्~। तत्र व्यक्तिमात्रपरत्वाभावात्~। तद्व्यक्तित्वादिना तदुपस्थितौ यत्र तात्पर्यं तत्र लक्षणा इष्यत एव~। ‘व्यक्त्याकृतिजातयस्तु पदार्थः’1 इति पारमर्षं सूत्रम्~। तत्र च एकयैव शक्त्या व्यक्त्याकृतिजातीनां बोध इति सूचनाय पदार्थ इत्येकवचनम्~।


