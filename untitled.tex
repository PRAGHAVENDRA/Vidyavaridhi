\section{शब्दप्रमाणविचारः}

शब्दोऽप्यनुमानमेव इति काणादानां समय इति प्रसिद्धमेव । तथा हि - 'तद्वचनादाम्नायस्य प्रामाण्यम्'\{वै. सू. १.१.३} इति कणादाचनेन वेदादीनां शब्दस्वरूपाणामपि प्रामाण्यमस्तीति ज्ञायते । तत्र शब्दानां व्याप्तिज्ञानविधया  प्रमात्मकज्ञानजनकत्वम् । 'शब्दादीनामप्यनुमाने अन्तर्भावः'\{प्र. भा. ४९८} इति प्रशस्तपादवचनात् । तथा च लिङ्गदर्शनस्य व्याप्तिस्मरणद्वारा वह्न्यादिज्ञानजनत्वस्य परोक्षस्थले दर्शनात् पदप्रत्यक्षस्यापि पदपदार्थयोः अविनाभावसम्बन्धस्मरणद्वारा शाब्दज्ञानजनकत्वमङ्गीकर्तव्यम् । तथा चानुमानम् - एते पदार्थाः‌ मिथः संसर्गवन्तः आकाङ्क्षादिमत्पदस्मारितत्वात् गामभ्याजेति पदार्थवत् इति, एतानि पदानि स्मारितार्थसंसर्गवन्ति आका

 अनुमानस्य ईश्वराबाधकत्वसिद्ध्या एव चरितार्थत्वात् । उपमानशब्दयोः अनुमानाभिन्नत्वादिति वैशेषिकानामाशयः । अत्र शब्दस्य अनुमानाभिन्नत्वं विचार्यते ।

ननु शब्दस्य कथमनुमानत्वम् ?‌ इत्थम् - पदानां पदार्थानां वा आकाङ्क्षादिसहकृतानां वाक्यार्थबोधहेतुत्वमिति अनुभवसिद्धम् । तत्र किन्तेषां पदज्ञानत्वेन पदार्थसंसर्गविषयकज्ञानात्मकशाब्दहेतुत्वमुत लिङ्गविधया इति चेत् प्रमाणान्तराकल्पनाप्रयुक्तलाघवात् द्वितीयपक्षमेव स्वीकुर्मः । तथा हि - शब्दश्रवणादिव्यापारानन्तरं जायमानज्ञाने संसर्गस्यैव इन्द्रियासन्निकृष्टस्य विषयत्वात् तदेव साध्यम् । संसर्गस्य च पदार्थवृत्तित्वात् पदार्थ एव पक्षः इति सामान्यतयावगम्यते । अस्मिन् अनुमाने हेतुस्तु न तावत् वाक्यत्वम् , वाक्यस्य पदवृत्तित्वात् । पदार्थत्वस्य हेतुत्वे 'गौरश्वः' इत्याद्यनाकाङ्क्षितवाक्यात् उपस्थितेष्वर्थेषु पदार्थत्वस्य सत्त्वेऽपि संसर्गवत्वाभावात् व्यभिचारः । न च पदेष्वेव संसर्गवत्वं साध्यते इति वाच्यम् । पदेषु संयोगसमवायादिना पदार्थसंसर्गस्याभावात् । न हि संसर्गवत्वं नाम संसर्गज्ञाप्यत्वम् । संसर्गज्ञाप्यत्वं हि पदानां शाब्दकारणत्वे सम्भवति । तच्च प्रकृते नेष्टम् । नापि संसर्गानुमापकहेतुत्वम् । संसर्गानुमापकत्वरूपसाध्यप्रसिद्धौ वाक्यत्वेन हेतुना तत्साधनं सम्भवति, वाक्यत्वेन हेतुना च संसर्गानुमापकत्वसिद्धिरिति अन्योन्याश्रयः । तस्माद्युक्त्यभावात् न शब्दस्यानुमानत्वमिति चेन्न ।

'एते पदार्थाः मिथः संसर्गवन्तः आकाङ्क्षादिमत्पदस्मारितत्वात् गामानयेति पदार्थवत्', 'एतानि पदानि स्मारितार्थसंसर्गज्ञानपूर्वकाणि आकाङ्क्षादिमत्त्वे सति तत्स्मारकत्वात् गामभ्याजेति पदसमूहवत्' इति वा अनुमानसम्भवात् । प्रथमानुमाने 'गौरश्वः' इत्याद्यनाकाङ्क्षितवाक्योपस्थितार्थेषु साध्याभावेऽपि आकाङ्क्षादिमत्पदस्मारितत्वरूपहेत्वभावात् न व्यभिचारः । अर्थं बुद्ध्वा शब्दरचना इति न्यायानुसारं स्मारितार्थसंसर्गज्ञानपूर्वकत्वस्य पदेषु सम्भवात् न साध्याप्रसिद्धिः । न च प्रथमानुमाने पदार्थेषु संसर्गत्वेन रूपेण संसर्गसिद्धौ शाब्दबोधविषयीभूताभेदादिविशेषसंसर्गासिद्धिरिति वाच्यम् । पदार्थानां तत्तस्थलानुरोधेन भेदात् तद्बलादेव तत्र तत्र व्यवहारौपयिकसंसर्गविशेषस्यापि भानसम्भवात् । तस्मात्पदानां अनुमानत्वेनैव शाब्धधीजनकत्वमिति चेत् । मैवम् ।

अत्रोच्यते - 'अनैकान्तः परिच्छेदे सम्भवे च न निश्चयः । आकाङ्क्षा सत्तया हेतुर्योग्यासत्तिरबन्धना ।\footnote{न्या. कु. ३.१३} संसर्गसाधकानुमानेन किं नियतः संसर्गः सिध्यति उत संसर्गयोग्यता ? नाद्यः । 'नद्यास्तीरे फलानि सन्ति' इत्यनाप्तोक्तपदार्थान्तर्भावेण व्यभिचारः । तत्र संसर्गाभाववत्यपि तादृशपदार्थेषु आकाङ्क्षादिमत्पदस्मारितत्वस्य सत्त्वात् । न चाप्तोक्तत्वं हेतुविशेषणम् । तेन तादृषपदार्थानामनप्तोक्तत्वात् हेत्वभाव इति वाच्यम् । शब्दप्रमाणव्यापारानन्तरं हि तदुच्चारयितरि आप्तत्वनाप्तत्वनिश्चयः सम्भवति । दोषाणां‌ पुरुषधर्मत्वात् क्वचित् तस्य आप्तत्वं क्वचिदनाप्तत्वमिति फलबलकल्प्यमेव तत् । तस्मात् स्वरूपासिद्धिः । नापि द्वितियः । संसर्गयोग्यत्वं हि पक्षधर्मताबलादेव ज्ञातमिति सिद्धसाधनम् । यदि हेतुकुक्षौ योग्यत्वं न निवेश्यते तर्हि 'अग्निना सिञ्चति' इत्यादिपदार्थमादाय व्यभिचारः‌ । एवञ्च पदार्थपक्षकानुमानेन न शाब्दनिर्वाहः ।

पदपक्षकानुमानेऽपि हेतुशरीरस्थाकाङ्क्षायाः वक्तुमशक्यत्वात् हेत्वसिद्धिः । कथञ्चित् तत्स्वरूपनिर्वचनसम्भवेऽपि किं‌ सत्तामात्रेण तस्य विशेषणत्वम् उत ज्ञायमानं सत् ।




श्रीमद्भिः उदयनाचार्यैः सकलकार्यप्रपञ्चकारणीभूतस्य जीवात्मातिरिक्तस्य सर्वज्ञस्येश्वरस्य सद्भावे प्रमाणानि प्रमाणान्तरबाधकत्वाबाधश्च स्वकीये न्यायकुसुमाञ्जलिनामके ग्रन्थे विस्तरेण न्यरूपि । तत्र हि द्वितीयस्तबके वेदानां प्रामाण्यं महाजनपरिग्रहादिना निश्चित्य, तस्य च आप्तोक्तत्वाभावे असम्भवात् आप्तोक्तत्वं संसाध्य, कपिलादिषु वेदाकर्तृत्वञ्च निराकृत्य तत्कर्तृत्वेन ईश्वरः साधितः । एवमपि प्रमाणान्तरेण ईश्वराभावे निश्चिते बाधसम्भवात् तदपनोदनार्थं तृतीयस्तबकमारब्धम् ।

रूपादिशून्यपदार्थसत्त्वे तदभावे वा इन्द्रियाणां ग्रहणासामर्थ्यात् प्रत्यक्षं नेश्वरबाधकम् । एवं इश्वरासाधकलिङ्गाद्यभावात् अनुमानमपि न बाधकम् । एवमुपमानशब्दयोरपि नेश्वरबाधकत्वम् ।
