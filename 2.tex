\section{वायोः त्वाचत्वविषये मतवैलक्षण्यम्}

रूपरहितस्पर्शवद्द्रव्यं वायुरिति प्रसिद्धमेव | मूर्तेषु तद्गतरूपस्याप्रत्यक्षत्वेऽपि तदीयस्पर्शस्य उपलब्धिसम्भात् तस्य प्रत्यक्षत्वमस्ति वा इति सन्देहो जायते | यतः प्रायः चक्षुरिन्द्रियेण यद्द्रव्यं गृह्यते तत्रैव प्रत्यक्षत्वव्यवहारः लोके दृश्यते | वायौ तु रूपाभावात् अचाक्षुषत्वात्तस्य प्रत्यक्षगोचरत्वं नास्त्येव | किन्तु रूपवतां घटादीनामपि केवलत्वाचप्रत्यक्षत्वसम्भवात् प्रत्यक्षं प्रति रूपं प्रयोजकमिति वक्तुं नैव शक्यते | अतो वायुसद्भावे प्रत्यक्षं प्रमाणं वा इति संशयो उदेति | तत्र काणादास्तु तस्य अप्रत्यक्षत्वमेव उपपादयामसुः |

\subsection{वायुरप्रत्यक्ष इति वैशेषिकाः}

तथा हि -  यत्र द्रव्यगतसङ्ख्यापरिमाणादीनां सामान्यगुणानामिन्द्रिययोग्यत्वं सम्भवति तद्द्रवस्यापि तदिन्द्रियग्राह्यत्वमिति अभ्युपेयम् |‌ यथा घटगतसङ्ख्यादीनां चक्षुषा त्वचा च ग्रहणसम्भवात् तस्य चाक्षुषत्वं त्वाचत्वं च सिध्यति | वायुगतसङ्ख्यादीनां ग्रहणन्तु न केनापि इन्द्रियेण सम्भवति | अतः वायुरप्रत्यक्षैव | तत्सद्भावे स्पर्शशब्दादिलिङ्गकमनुमानमेव प्रमाणमित्याशयः |

\subsubsection{वायुविषयकं ज्ञानं व्याप्तिस्मरणाद्यनपेक्षमनुमानमेव}

 ननु 'वायुर्वाति', 'शीतो वायुः' इताद्याः प्रीतीतयः वायुसन्निकर्षे सति जायन्ते खलु इति चेन्न | वायुना सह त्वक्सन्निकर्षे सति वायुगतस्पर्श एव अनुभूयते | सैव प्रत्यक्षविषयः, नान्यत् | यदपि 'वायुर्वाति' इत्यद्याः प्रतीतयः तदप्यनुमानादेव | न च व्यप्तिस्मरणपरामर्शादिकं तत्र नानुभूयते खलु इति वाच्यम् | वृक्षादिषु विद्यमानक्रियायाः साक्षात्कारे सति यथा झटिति वायुविषयकप्रतीतिरुदेति तथैवात्रापि व्यप्तिस्मरणादीनामभावेऽपि साक्षाल्लिङ्गज्ञानेनैवानुमितिरुदेति अभ्यासातिशयात् | न च यो मूर्तद्रव्यवृत्तिर्गुणः येनेन्द्रियेण गृह्यते तेनेन्द्रियेण तदधिकरणमपि गृह्यते इति वाच्यम् | ग्रीष्मोष्मेषु व्यभिचातात् | एवं यत्र यत्र द्रव्यचाक्षुषत्वं तत्रैव स्पार्शनत्वस्यापि दर्शनात् द्रव्यस्पार्शनत्वं चाक्षुषत्वेन व्याप्तमेव | तस्मात् वायुरप्रत्यक्षैव |
 
{\fontsize{11.7}{0}\selectfont\s  उपलभ्यमानस्पर्शाधिष्ठानभूत आश्रयो यः स विषय इति । किमस्यास्तित्वे प्रमाणम् ? प्रत्यक्षमेव , त्वगिन्द्रियव्यापारेण वायुर्वातीत्यपरोक्षज्ञानोत्पत्तेरिति कश्चित् , तन्न युक्तम्  । स्पर्शव्यतिरिक्तस्य वस्त्वन्तरस्यासंवेदनात् । अपरोक्षज्ञाने तु स्पर्श एव प्रतिभाति नान्यत् । यदपि वायुर्वातीति ज्ञानं तदभ्यासपाटवातिशयाद् व्याप्तिस्मरणाद्यनपेक्षं स्पर्शेनानुमानम् , चक्षुषेव वृक्षादिगतिक्रियोपलम्भात् । शीतोष्णस्पर्शभेदप्रतीतौ वायुप्रतिभिज्ञानमपि तदाश्रयोपनायकद्रव्यानुमानादेव । त्वगिन्द्रियेण तु शीतोष्णस्पर्शाभ्यामन्यस्य न प्रतिभासोऽस्ति । स्पार्शनप्रत्यक्षो वायुरुपलभ्यमानस्पर्शाधिष्ठानत्वाद् घटवदित्यनुमानं शशादिषु पशुत्वेन शृङ्गानुमानवदनुपलब्धिबाधितम् । द्रव्यस्य स्पार्शनत्वं चाक्षुषत्वेन व्याप्तमवगतं घटादिषु चाक्षुषत्वस्य च वायावभावस्तेनात्र शक्यं स्पार्शनत्वनिवृत्यनुमानमेतत्  , अतस्तस्याप्रत्यक्षस्य सद्भावेऽनुमानम् ।\footnote{न्या.कं.}}

\subsubsection{वायोः प्रत्यक्षत्वे तद्गतसंख्यापरिमाणादिग्रहणप्रसङ्गः}

एवं वायोः त्वाचत्वे तद्गतसङ्ख्यादीनां प्रत्यक्षत्वप्रसङ्गः | क्वचित् सङ्ख्याग्रहणात् इष्टापत्तिरिति न वक्तव्यम् | फूत्कारादौ सङ्ख्याग्रहस्यानुमानिकत्वादिति | न च यज्जातीयद्रव्यं बहिरिन्द्रियेण गृह्यते तज्जातीयद्रव्यगतसङ्ख्यादयः तेन इन्द्रियेण गृह्यन्ते इति, यत्र यद्व्यक्तेः साक्षात्कारत्वं तत्र तद्व्यक्तिगतसङ्ख्यादीनामपि साक्षात्कारत्वमिति वा नियमः स्वीकार्यः | आद्ये कदाचित् वायावपि 'एको वायुः' इत्यादिप्रतीतिबलात् सङ्ख्याग्रहणसम्भवात् वायोरपि प्रत्यक्षत्वे न किञ्चिद्बाधकम् , द्वितीये तु पृष्टे संलग्नस्य वस्त्रादेः प्रत्यक्षत्त्वेऽपि तद्गतसङ्ख्यादीनामग्रहात् नियमोऽयं व्यभिचरित इति वाच्यम् | वायौ सङ्ख्याग्रहस्तु न प्रत्यक्षादपि तु  व्याप्तिस्मरणाद्यनमपेक्षादनुमानादेव | द्वितीये तु पृष्टे संलग्नस्य वस्त्रादेः‌ सङ्ख्याग्रहेऽपि तस्यैव चक्षुषा ग्रहणे तद्गतसङ्ख्याग्रहो भवत्येव |‌ किञ्च सर्वत्र द्रव्यप्रत्यक्षे सङ्ख्यायाः अविषयत्वात् याव्यक्तिः प्रत्यक्षा तदीयसङ्ख्यादिकं गृह्यते इति नियमः स्वीकार्यः | वायौ कदापि सङ्ख्यायाः अग्रहणात् वायुरप्रत्यक्षैव |

{\fontsize{11.7}{0}\selectfont\s किञ्च वायोः प्रत्यक्षत्वे तद्गतसंख्यापरिमाणादिग्रहणप्रसङ्गः । ननु यज्जातीयं प्रत्यक्षं तज्जातीयस्य सङ्ख्यादिकं प्रत्यक्षमिति व्याप्तिश्चेदभिमता तदा फूत्कारे सङ्ख्यादिग्रहोऽस्त्येव । अथ या व्यक्तिः प्रत्यक्षा तदीयं सङ्ख्यादिकं गृह्यत एवेति नियमस्तदा पृष्ठलग्नवस्त्रादौ व्यभिचार इति चेन्न फूत्कारादपि सङ्ख्याप्रतीतेरानुमानिकत्वात् पृष्ठलग्नवस्त्रादेरपि करपरामर्शेण सङ्ख्यापरिमाणादेर्ग्रहणात् । द्वितीयनियमेऽपि न व्यभिचारः , चक्षुषा वस्त्रसङ्ख्यादिग्रहणात् । न च ह्येतादृशो नियमो  यत् प्रत्यक्षं तत्सङ्ख्यादिविषयकमेव किं तर्हि या व्यक्तिः प्रत्यक्षा तदीयसङ्ख्यादिकं गृह्यत एवेति । वायौ च तथा नास्तीति वायुरप्रत्यक्ष एवेति ।{क.र.}}

\subsubsection{बहिर्द्रव्यप्रत्यक्षं प्रति उद्भूतरूपवत्वम् उद्भूतस्पर्शवत्वञ्च मिलितं तन्त्रम्}

बहिर्द्रव्यप्रत्यक्षं प्रति उद्भूतरूपं तन्त्रमिति कार्यकारणभावस्वीकारादेव वायोः त्वाचत्वे विवादो दृश्यते | किन्तु आचार्यास्तु द्रव्यप्रत्यक्षं प्रति उद्भूतरूपस्पर्शयोः द्वयोरेव कारणत्वमभ्युपगच्छन्ति | अत एव तेषां मते उद्भूतस्पर्शाभावान्न प्रभादिकं प्रत्यक्षत्वमिति |

{\fontsize{11.7}{0}\selectfont\s उद्भूतरूपवत्वम् उद्भूतस्पर्शवत्वञ्च मिलितं तन्त्रमिति तात्पर्याचार्याः । तन्मते चान्द्रं तेजो नयनगतपित्तद्रव्यं पद्मरागादिप्रभादिकं च किमपि न प्रत्यक्षम् । तथा चोद्भूतरूपवत्वस्योद्भूतस्पर्शवत्वसहकृतस्य वा तस्य प्रयोजकत्वे वायुरप्रत्यक्ष एवेति ।\footnote{क.र.}}


\subsubsection{प्रत्यक्षस्पर्शाश्रयत्वेन तत्प्रत्यक्षत्वसाधने उपाधिप्रदर्शनम्}

वायोः प्रत्यक्षत्वं प्रत्यक्षस्पर्शादिलिङ्गकहेतुना सिध्यति | न च तत्रोद्भूतरूपमुपाधिः |  साध्यव्यापकत्वाभावात् इति वाच्यम् | साधनावच्छिन्नसाध्यवापकत्वस्यापि उपाधिशरीरप्रविष्टत्वादिति | प्रत्यक्षस्पर्शाश्रयत्वावच्छिन्नप्रत्यक्षत्वस्य आत्मन्यभावात् अन्यत्र सर्वत्र उद्भूतरूपस्य सत्वात् |

{\fontsize{11.7}{0}\selectfont\s ननु वायुः प्रत्यक्ष एव किन्न स्यात् । त्वगिन्द्रियव्यापारानन्तरं वायुर्वातीति प्रतीतेः । तथा च प्रयोगः वायुः प्रत्यक्षः प्रत्यक्षस्पर्शाश्रयत्वात् यदेवं तदेवं यथा घटः इति चेन्न उद्भूतरूपवत्वस्योपाधित्वात् । आत्मनि साध्याव्यापकमिदमिति चेन्न साधनावच्छिन्नसाध्यव्यापकत्वात् । तदुक्तम् –\\
वाद्युक्तनियमच्युतोऽपि कथकैरुपाधिरुद्भाव्यः ।\\[-1mm]
पर्यवसितं नियमयन् दूषकताबीजसाम्राज्यात् ॥\\
चाक्षुषप्रत्यक्षतायामेव तत्तन्त्रं न तु स्पार्शनप्रत्यक्षतायामपि इति चेन्न सामान्ये बाधकाभावात् ।{क.र.}}


\subsection{तत्र नवीनाः}

\subsubsection{वायोरप्रत्यक्षतामेव ब्रूते मणिकारः}

नवीननैयायिकेषु मणिकारस्तु वायोरप्रत्यक्षतामेव मनुते | अत एव सङ्ख्यादीनामग्रहरूपम् एव तत्र प्रधानतया युक्तिं प्रदर्शयति |

{\fontsize{11.7}{0}\selectfont\s उच्यते | द्रव्यस्य स्पार्शनत्वे उद्भूतस्पर्शमात्रं न तन्त्रम् | निदाघोष्मणि वायूपनीतशीतोष्णद्रव्ये च प्रत्यक्षत्वेन तद्गतसङ्ख्यापरिमाणसंयोगविभागकर्मणां प्रत्यक्षत्वप्रसङ्गात् | योग्यव्यक्तिवृत्तित्वेन तेषां योग्यतया द्रव्यग्राहकसामग्रीग्राह्यत्वावधारणात् | न चोष्मादिजातीये दोषाभावेऽपि घटादाविव करपरामर्शे कदाचित् केनापि सङ्ख्या गृह्यते | तथोद्भूतरूपवत्त्वमात्रस्य तथात्वे चान्द्राद्युद्योतस्य नयनगतपित्तद्रव्यस्य च प्रत्यक्षत्वे तद्गतसङ्ख्याग्रहोऽपि स्यात् | न च घटादाविव निपुणं निभालयन्तोऽपि तद्गतसङ्ख्याद्वित्वादि हस्तवितस्त्यादिपरिमाणं कर्म वा वीक्षामहे इत्येकैकव्यभिचाराद्विनिगमकाभावात् उभयमपि बहिरिन्द्रियद्रव्यप्रत्यक्षत्वे प्रयोजकमिति वायुरप्रत्यक्षः | न चैवमपसिद्धान्तः | पीतः शङ्खः इत्यादौ नयनपित्तपीतिमैव गृह्यते न तु पित्तद्रव्यम् | विभक्ता(विषक्ता)वयवत्वात् प्रभायामिव तेजस इति टीकाकृतामभिधानादिति नवीनाः \footnote{त.म.}}

\subsubsection{द्रव्यस्पार्शनत्वे स्पर्शवत्वमेव प्रयोजकम्}

द्रव्यस्पार्शनप्रत्यक्षे स्पर्शवत्वमेव प्रयोजकम् | अत एव 'शीतो वायुरि'त्यादिप्रत्ययोऽपि स्पार्शनः साधु सङ्गच्छते | त्रुटेरस्पार्शनत्वे तु प्रकृष्टतमं परिमाणमपि तथा गौरवान्मानाभावात् |	त्वग्व्यापारानन्तरं वायुर्वातीति सार्वलौकिकप्रत्यक्षस्य अन्यथानुपपत्त्या च रूपं तत्र न निवेशनीयम् | फूत्कारादौ च स्फुटतरप्रत्यक्षाः सङ्ख्यादयः चाक्षुषद्रव्यप्रत्यक्षे रूपं तथा | गौरवान्मानाभावाच्च | निस्पर्शायामपि प्रभायां चलनादिप्रत्ययाच्च  स्पर्शोऽपि न तथा | द्रव्यस्य समवेतेन्द्रियप्रत्यक्षे तु अनात्मसमवेतशब्दरसगन्धजातीतरयोग्यधर्मसमवायित्वं तथा | सुखादिसमवायिकारणतावच्छेदकत्वेन सिद्धमात्मत्वं जातिर्नेश्वर इति तदीय ज्ञानादिपिशाचादिसंयोगवारणाय योग्येति | विषयिधर्मासामानाधिकरणेत्यभिधाने तु शब्दो नोपादेयः | अमदादिनयनसंसृष्टपित्तद्रव्यस्य परिमाणशून्यत्वं, परिमाणवत्वमते तु तादृशप्रत्यक्षे परिमाणवत्वमेव तथा | यद्वा नयनसंसृष्टपित्तद्रव्यं निरूपमेव | अन्यथा पुरुषान्तरेण तत्पीतिमोपलम्भापत्तिः | स्मर्यमाणस्तु पीतिमा दोशवषाच्छङ्खादावारोप्यत इति | द्रव्यप्रत्यक्षे च शब्दरसन्धजातीतरयोग्यधर्मसमवायित्वं तथा | रसनगतञ्च पित्तद्रव्यं न रूपवत् न वा स्पर्शवत् रसना च न द्रव्यग्राहिकेति न तत् प्रत्यक्षम् |\footnote{प.नि. ७७}

\subsection{विमर्शः}

अरूपिद्रव्यस्य वायोः त्वाचत्वमत्र विमृश्यते | ग्रीष्मकाले 'वायुर्वाति' इत्यादिप्रतीतयः जायन्ते | त्वगिन्द्रियव्यापारानन्तरमेव एतासां जाननात् वायुः त्वाचप्रत्यक्षविषयः इति ज्ञायते | किन्तु पृथिव्यादीनां साक्षात्कारः यत्र यत्र भवति तत्र सर्वत्रापि उद्भूतरूपं वर्तत एव इत्यतः बहिरिन्द्रियद्रव्यप्रत्यक्षे उद्भूतरूपस्य प्रयोजकत्वं क्लृप्तमेव | वायौ तु तदभावात् तस्य त्वाचत्वं नेष्यते | 

ननु तादृशप्रतीतीनां का गतिरिति चेत् , तत्र अभ्यासातिशयेन विना व्याप्तिस्मरणं पक्षधर्मताज्ञानानपेक्षं च परामर्शात्मकं लिङ्गज्ञानमुदेति | तेन चानुमितिरिति | तथा च इन्द्रियव्यापारेण स्पर्शमात्रं गृह्यते, न तु वायुः‌ इति |

न च येन द्रव्यग्राहकेन्द्रियेण मूर्तद्रव्यवृत्तिः‌ यो गुणः गृह्यते तेनैव इन्द्रियेण तदधिकरणमपि गृह्यते इति नियमो दृष्टः | यथा रूपादीनां घटादिवृत्तीनां रूपादीनां यथा चक्षुषा ग्रहः तथैव घटादीनामपि चक्षुषा ग्रहोऽनुभवसिद्ध एव | आकाशादीनामग्रहणात् मूर्तेति | घ्राणादिना गन्धाद्यधिकरणानामग्रहणात् द्रव्यग्राहकेति | तथा च  वायुवृत्तिस्पर्शस्य त्वचा ग्रहे तु तस्यापि ग्रहणं स्यादेवेति वाच्यम् | ग्रीष्मोष्मादौ उष्णस्पर्शस्य त्वचा ग्रहेऽपि तदधिकरणस्य तेजसः त्वचा अग्रहात् नियमोऽयं व्यभिचरितः | 

किञ्च वायोः तद्गतसङ्ख्यापरिमाणादीनां ग्रहणप्रसङ्गः | तथा हि यत्र द्रव्यस्य चाक्षुषत्वं स्पार्शनत्वं वा भवति तत्रावश्यं तद्गतसङ्ख्यादीनां ग्रहणं सम्भवति | वायुवृत्तिसङ्ख्यादीनामग्रहणात् न वायुः त्वाचप्रत्यक्षविषयः |

न च यज्जातीयद्रव्यं बहिरिन्द्रियेण गृह्यते तज्जातीयद्रव्यगतसङ्ख्यादयः तेन इन्द्रियेण गृह्यन्ते इति, यत्र यद्व्यक्तेः साक्षात्कारत्वं तत्र तद्व्यक्तिगतसङ्ख्यादीनामपि साक्षात्कारत्वमिति वा नियमः स्वीकार्यः | आद्ये कदाचित् वायावपि 'एको वायुः' इत्यादिप्रतीतिबलात् सङ्ख्याग्रहणसम्भवात् वायोरपि प्रत्यक्षत्वे न किञ्चिद्बाधकम् , द्वितीये तु पृष्टे संलग्नस्य वस्त्रादेः प्रत्यक्षत्त्वेऽपि तद्गतसङ्ख्यादीनामग्रहात् नियमोऽयं व्यभिचरित इति वाच्यम् | वायौ सङ्ख्याग्रहस्तु न प्रत्यक्षादपि तु  व्याप्तिस्मरणाद्यनमपेक्षादनुमानादेव | द्वितीये तु पृष्टे संलग्नस्य वस्त्रादेः‌ सङ्ख्याग्रहेऽपि तस्यैव चक्षुषा ग्रहणे तद्गतसङ्ख्याग्रहो भवत्येव |‌ तस्मात् नियमद्वयस्वीकारेऽपि न दोषः | 

केचित्तु बहिरिन्द्रियप्रत्यक्षं प्रति उद्भूतरूपवत्त्वमुद्भूतस्पर्शवत्त्वमुभयं प्रयोजकमिति वदन्ति | तेषां मतानुसारेण उद्भूतस्पर्शवत्त्वाभावात् प्रभादिकं किमपि न प्रत्यक्षम् , उद्भूतरूपाभावाच्च ऊष्मादिकं न प्रत्यक्षमिति | तन्न | 'प्रभां साक्षात्करोमि' इति अनुव्यवसायबलात् सिद्धाया उद्भूतस्पर्शशून्ययाः प्रभाया अप्रत्यक्षत्वकथनं न युक्तम् |

ननु यद्यपि 'वायुं साक्षात्करोमि' इति अनुव्यावसायवशादेव वायोः त्वाचत्वं सिध्यति | तथापि तदृशानुव्यवसायस्य प्रमात्वसन्देहे 'वायुः प्रत्यक्षः प्रत्यक्षस्पर्शाश्रयत्वात् घटवत्' इत्यनुमानं प्रमाणमिति वदामः | घटादौ उद्भूतस्पर्शस्य सत्त्वात् तेषां प्रत्यक्षत्वम् यथा तथैव वायावपि उद्भूतस्पर्शस्य सत्त्वात् तस्यापि प्रत्यक्षत्वं प्रमाणतः सिद्धमेव इति चेन्न | तस्मिन्ननुमाने उद्भूतरूपस्य उपाधित्वात् | तथा हि - बहिरिन्द्रियप्रत्यक्षविषयेषु वायुभिन्नेषु सर्वेष्वपि द्रव्येषु उद्भूतरूपस्य विद्यमानत्वात् साध्यव्यापकत्वं, वायौ उद्भूतस्पर्शस्य विद्यमानत्वेऽपि उद्भूतरूपाभावात् साधनाव्यापकत्वम्  | न च आत्मनि प्रत्यक्षत्वसत्त्वेऽपि उद्भूतरूपाभावान्न साध्यव्यापकत्वमिति वाच्यम् | साधनावच्छिन्नसाध्यव्यापकत्वात् | उद्भूस्पर्शाश्रयत्वस्यात्मन्यभावान्न आत्मान्तर्भावेण साध्यव्यापकत्वभङ्गः | मैवम् |

त्वगिन्द्रियव्यापारानन्तरं 'शीतो वायुः' इत्यादिप्रतीतेरुदयात् वायुः प्रत्यक्ष एव | न च सदागतिमत्वाद्वायोः इन्द्रियव्यापारस्य च सर्वदा विद्यमानत्वात् सर्वदा वायुविषयकप्रतीतिप्रसङ्गः इति वाच्यम् | वायुप्रत्यक्षं प्रति उत्कृष्टक्रियाविशिष्टवायोः कारणत्वं कल्प्यते | अत एव ग्रीष्मकालादौ पङ्कादिना प्राप्तोत्कृष्टक्रियस्य वायोः ग्रहणं, यानादौ च तादृशक्रियाविशिष्टवायुसन्निकर्षात् स्पष्टतया वायोः त्वाचत्वानुभवः | तस्मात् त्वाचप्रत्यक्षानन्तरमनुभूयमानानुव्यवसायस्य प्रबलप्रमाणस्य अपलापासम्भवात् वायुः प्रत्यक्ष एव |

न च तत्प्रत्यक्षत्वे तद्गतसङ्ख्यापरिमाणादीनां प्रत्यक्षत्वप्रसङ्ग इति वाच्यम् | सर्वत्र प्रत्यक्षविषयीभूतेषु द्रव्येषु सङ्ख्यादिकं गृह्यते इति न नियमः | अत एव चक्षुःसन्निकृष्टेऽपि दूरस्थपुरुषे स्थाणौ वा सङ्ख्यादिकं न गृह्यते | न च तत्र दोषवशात् सङ्ख्यादीनामग्रहणम् | अत एव चक्षुःसमीपस्थद्रव्ये सङ्ख्यादिकं गृह्यते इति वाच्यम् | प्रकृतेऽपि अवयवसंयोगवैरल्यस्य सङ्ख्यादिग्रहणे दोषत्वं स्वीकुर्मः‌ | अत एव यथा घटादौ सङ्ख्यादिग्रहः तथैव जलादौ प्रभादौ च न भवति | तस्मान्नायं प्रत्यक्षाभावापादम् | अस्तु वा फूत्कारादौ कदाचित्सङ्ख्याग्रहणं सम्भवति इति वदामः इति |

बहिरिन्द्रियप्रत्यक्षं प्रति उद्भूतरूपं प्रयोजकम् | वायौ तदभावात् वायुः‌ न बहिरिन्द्रियप्रत्यक्षविषयः‌ इति केचित् | तन्न | त्वाच प्रत्यक्षं प्रति उद्भूतरूपमकिञ्चित्करमेव | अन्यथा त्वचापि रूपग्रहोत्पत्तेः | येन इन्द्रियेण द्रव्यप्रत्यक्षं तं प्रति प्रयोजकीभूतगुणस्यापि तेनैव इन्द्रियेण ग्रहणात् | यथा चक्षुषा घटप्रत्यक्षजनने प्रयोजकीभूतस्य रूपस्य चक्षुषैव ग्रहणम् , त्वचा तु तत्प्रत्यक्षे रूपस्याप्रयोजकत्वात्तदग्रहणमिति | अपि तु त्वचा तद्ग्रहणे स्पर्शस्य प्रयोजकत्वात् तद्ग्रहणमिति | तस्मात् त्वाचप्रत्यक्षं प्रति स्पर्श एव प्रयोजकः | तत्सत्वात् वायुः प्रत्यक्षः‌ |

एतेन बहिरिन्द्रियप्रत्यक्षं प्रति उद्भूतरूपमुद्भूतस्पर्शश्च प्रयोजकमिति पक्षोऽपि निरस्तः | सर्वेषामपि लोकानामनुभवसिद्धस्य प्रभादिविषयकप्रतीतेः स्पर्शविरहादनुपपत्त्या उभयोः तन्त्रत्वं नैव युज्यत इति |

शास्त्रान्तरेऽपि वायोः प्रत्यक्षत्वमेव प्रतिपादितम् | तथा हि - {\fontsize{11.7}{0}\selectfont\s शीतादिषु स्पर्शविशेषोपलभ्यमानेषु शीतो वायुरुष्णो वायुरनुष्णाशीतो वायुरिति वायुद्रव्यस्यैकस्य प्रत्यभिज्ञायमानत्वात् कृष्णो घटः, पीतो घटः, श्वेतो घट इतिवत् सकलस्पर्शानुगतमेकमेव वायुद्रव्यं प्रत्यभिज्ञायतां भवतां स्पर्शमात्रमेव वयं प्रत्यभिजानीमो नान्यत् किञ्चिदिति वचनमनुभवविरुद्धमेव | प्रयोगश्च भवति - वायुः प्रत्यक्षः महत्त्ववत्त्वेऽनिन्द्रियत्वे च सति स्पर्शवत्त्वाद् भूतत्वाद्वा घटवदिति \footnote{मा.मे.}}

तस्मात् त्वगिन्द्रियमात्र ग्राह्यो वायुः | तत्प्रत्यक्षे च उद्भूतस्पर्शवत्त्वं प्रयोजकमिति सिद्धम् |
