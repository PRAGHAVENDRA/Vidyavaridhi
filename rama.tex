\usepackage{fontspec,graphicx}
\usepackage{graphicx}
\graphicspath{ {images/} }
\usepackage{indentfirst}
\parindent 1.5cm
\usepackage{setspace}

\usepackage{polyglossia}
\usepackage[utf8]{inputenc}
\usepackage{imakeidx}
\usepackage{cleveref}
\makeindex[columns=2, title=श्लोकानुक्रमणिका, intoc]


\usepackage{hyperref}
\hypersetup{
    colorlinks=true,
    linkcolor=blue,      
    urlcolor=cyan,
    pdftitle={mainfile},
}
\usepackage{tabu}

\usepackage{titlesec}
\titleformat{\section}
  {\normalfont\fontsize{14}{18}\bfseries}{\thesection}{1em}{}
\titleformat{\subsection}
  {\normalfont\fontsize{13}{16}\bfseries}{\thesubsection}{1em}{}
 \titleformat{\subsubsection}
  {\normalfont\fontsize{12}{14}\bfseries}{\thesubsubsubsection}{1em}{}
\usepackage{metalogo}
\usepackage{perpage}
\MakePerPage{footnote}
 
%\usepackage{thispagestyle}

%\usepackage{sectsty}
%\sectionfont{\fontfamily{s}\fontseries{Lohit Devanagari}\fontsize{25pt}{35pt}\selectfont}
%\subsectionfont{\fontfamily{phv}\fontseries{b}\fontsize{11pt}{20pt}\selectfont}
%\subsubsectionfont{\fontfamily{phv}\fontseries{b}\fontsize{11pt}{20pt}\selectfont}



%\fontsize{13}{15}\selectfont
%\setmainfont{Sanskrit2003}
\setmainfont[WordSpace=1.4, Scale=1.8]{Lohit Devanagari}
\setdefaultlanguage{sanskrit}
\newfontfamily\AA[Script=Devanagari]{Lohit Devanagari}

\newfontfamily\s[WordSpace=2.1, Scale=2.0, Script=Devanagari]{Sanskrit2003}



\usepackage{pdfpages}
\usepackage{unicode-math}
\usepackage{pdflscape}

\newcommand{\devanagarinumeral}[1]{%
  \devanagaridigits{\number\csname c@#1\endcsname}}
\renewcommand{\thesection}{\devanagarinumeral{section}}
\renewcommand{\thesubsection}{\devanagarinumeral{subsection}}
\renewcommand{\thepage}{\devanagarinumeral{page}}
\renewcommand{\theenumi}{\devanagarinumeral{enumi}}
\renewcommand{\thefootnote}{\devanagarinumeral{footnote}}

\usepackage{blindtext}
\usepackage{multicol}
\setlength{\columnsep}{10cm}
\usepackage{fancyhdr}
\usepackage{xltxtra}
\usepackage{lipsum}
\usepackage{titlesec}
\titlespacing*{\section}{0pt}{0.5\baselineskip}{0.5\baselineskip}

\renewcommand*\contentsname {विषयानुक्रमणिका}

\usepackage[papersize={210 mm,297mm},textwidth=100mm,
textheight=200mm,headheight=14.5mm,headsep=12.7mm,topmargin=1.8cm,botmargin=2.1cm,
leftmargin=3.5cm,rightmargin=2.5cm,footskip=1.0cm]{zwpagelayout}


\linespread{2.8}



\fancypagestyle{}{%
\chead[]{}
\lhead[]{}
\rhead[]{}
\cfoot[]{}
}

 
\pagestyle{fancy}
\fancyhf{}
\fancyhead[LE,RO]{\leftmark}
\fancyhead[RE,LO]{\rightmark}
\fancyfoot[CE,CO]{\bfseries \small \thepage}

\renewcommand{\chaptermark}[1]{\markboth{#1}{}}
\renewcommand{\sectionmark}[2]{\markboth{#1}{}}
\renewcommand\subsectionmark[1]{}
\renewcommand\subsubsectionmark[1]{}


\pretolerance=9990
