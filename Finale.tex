\titleformat {\chapter}[display]{\normalfont\Large} % format
{अथ प्रथमोऽध्यायः\\[1mm]} % label
{-3.8ex}{ \rule{\textwidth}{1pt}\vspace{-5ex}
\centering
} % before-code
[
\vspace{-6.7ex}%
\rule{\textwidth}{1pt}
]
\titlespacing*{\chapter} {10pt}{-60pt}{50pt}


\chapter{प्रमाणप्रपञ्चे सैद्धान्तिकमतभेदाः}

अथ संसारान्धकारपारावारपतितानां लोकानामुद्दिधीर्षुणा महर्षिणाक्षपादेन प्रमाणप्रमेयेत्यादिषोडशपदार्थतत्त्वानिरूपणपरं न्यायशास्त्रमसूत्रि~। ततः परं  वात्स्यायनादियुक्तिकोविदाः तेषां सूत्राणां भाष्यादिक्रमेण व्यख्योपव्याख्याः रचयामासुः~। एवमान्विक्षिकीति प्रथिता न्यायविद्या विस्तृततया प्रमाणादितत्त्वनिरूपणपरा अभूत्~।

शिवसाक्षात्कारेण हि परमपुरुषार्थोऽपवर्गः सिध्यति~। शिवसाक्षात्कारश्च द्रव्यादिपदार्थतत्त्वज्ञानाधीन इति मन्वानः कणादमुनिः द्रव्यादिप्रमेयान् प्रधानतया 'अथातो धर्मं व्याख्यास्यामः'\footnote{वै. सू. १.१.१} इत्यादिना असूत्रयत्~। ततः प्रशस्तपादादीनां व्याख्योपव्याख्यानैः वैशेषिकदर्शनं प्रधानतया द्रव्यादिप्रमेयतत्त्वानवबोधयत्~।

अनयोः दर्शनयोः सैद्धान्तिकतत्त्वान् स्वीकृत्य तत्त्वचिन्तामणिनामकं ग्रन्थं विरच्य शास्त्रप्रपञ्चान्तरं निर्मितं श्रीमता गङ्गेशोपाध्यायेन~। न्यव्यन्यायेति प्रसिद्धं शास्त्रमिदं प्रधानतया प्रमाणतत्त्वम्~, तत्रापि अनुमानशब्दप्रमाणे विस्तृततया प्रत्यपादयत्~। 

यद्यप्येतेषां सर्वेषामपि विदुषां परमपुरुषार्थप्रयोजकयुक्तियुक्तपदार्थतत्त्वनिरूपणमेव मुख्यमुद्देश्यम्~, तथापि तत्र तत्र पदार्थतत्त्वनिरूपणावसरे मतिभेदात् मतभेदाः दरीदृश्यन्ते~। तेषां निरूपणं तत्र विमर्शश्च यथामति अस्मिन् प्रबन्धे निरूप्यते~। तत्रादौ 'प्रत्यक्षानुमानोपमानशब्दाः प्रमाणानि'\footnote{न्या. सू. १.१.३} इति गौतमसूत्रानुसारं प्रमाणतत्त्वे मतभेदाः निरूप्यन्ते~।

चक्षुरादिभिरिन्द्रियैः जायमानः साक्षात्करोमीत्यनुव्यवसायविषयीभूतोऽनुभवः प्रत्यक्षमिति प्रसिद्धम्~। तत्करणानि चक्षुरादिषडिन्द्रियाणि~। तानि च विषयसन्निकृष्टानि मनसासम्बद्धानि विलक्षणात्ममनस्संयोगद्वारा प्रत्यक्षमुत्पादयन्ति इति सर्वविदितमेव~। तस्मादस्मिन् विषये नास्ति विदुषां विप्रतिपत्तिः~।

इन्द्रियासन्निकृष्टोऽप्यर्थः वह्न्यादिः युक्तिवशाज्ज्ञायन्ते~। धूमादिना सह सहचरितः वह्न्यादिः इति यो जानाति सह पर्वतादौ धूममात्रदर्शनेन असन्निकृष्टमपि वह्निं जानाति व्यप्तिपक्षधर्मतादिज्ञानेन~। अत इदं द्वितीयमनुमानं प्रमाणम्~। तत्करणं व्याप्तिविशिष्टपक्षधर्मताज्ञानम्~, परामर्श इत्युच्यते~। यथा 'वह्निव्याप्यधूमवान् पर्वत' इति ज्ञानम्~। अस्मिन्नपि विषये नास्ति युक्तिकोविदेषु विप्रतिपत्तिः~।

	\section{उपमानप्रमाणविचारः}

	प्रमाणेषु पदपदार्थयोः सम्बन्धेति प्रसिद्धस्य सङ्केतस्य ग्राहकमिति प्रसिद्धमिदमुपमानप्रमाणम्~। 'गोसदृशो गवयः' इत्यादिना सादृश्यज्ञानेन पदपदार्थयोः सम्बन्धग्रहोत्पद्यत इत्यतः सादृश्यज्ञानात्मकस्यास्य इतरप्रमाणापेक्षया विलक्षणता~। तथा हि~- गवयाभिज्ञेनारण्यकेन गवपदव्युत्पित्सुं नागरिकं प्रति 'गोसदृशो गवयः' इत्युक्तम्~। ततः परं नागरिकः कालान्तरे अरण्यादौ गवयं पश्यन् आरण्यकेनोक्तं वाक्यस्यार्थं स्मरति~। ततः परमस्य 'अयमेव गवयपदवाच्यः' इति ज्ञानमुदेति~। इयमेव उपमितिरित्युच्यते~। अत्र विवदन्ते वैशेषिकाः~। ते तु  युक्तिबलादेव पदपदार्थविषयकसम्बन्धग्रहः सम्भवे तदर्थं नातिरिक्तप्रमाणापेक्षा इति मन्विरे~।

	तथा हि~- शब्दादनुमानाद्वा सम्बन्धग्रहः इति प्रधानतया मार्गद्वयं दृश्यते~। तत्र 'गो सदृशो गवयः' इति केनचिदुक्तवाक्येन गवयपदशक्तिग्रहादुपमानमाप्तवचनमेव~। शब्दादुत्पद्यमानं ज्ञानं तावदनुमित्यात्मकमित्यतः उपमानमनुमानमेव इति~। गवयपदेऽव्युत्पन्नं नागरिकं प्रत्यारण्यकः 'गोसदृशो गवय' इत्यवोचत्~। तदानीमेव लक्षणया गवयपदात् गवयपदवाच्यत्वविशिष्टस्योपस्थितिः, ततः  'गोसदृशाभिन्नो गवयपदवाच्यः' इति गवयपदतदर्थयोः सङ्केतविषयकबोधः, गोसदृशपदस्य गवयत्वविशिष्टवाचकत्वे लक्षणाभ्युपगमेन 'गवयत्वविशिष्टवाचकाभिन्नो गवयः' इति वा शाब्दधीरुदेतीत्यतः तेनैव गवयपदतदर्थयोः सम्बन्धप्रतिपत्तिर्जायते~।

	अथ वा वनगमनानन्तरं नागरिकस्य गवयपिण्डदर्शने सति गवयपदं गवयत्वविशिष्टैतत्पिण्डवाचकम् असति वृत्त्यन्तरे तदभियुक्तेनात्र प्रयुज्यमानत्वादित्यनुमानात् गवयपदशक्तिग्रहसम्भवादुपमानमनुमानमेवेति~। अथ वा वनगमनानन्तरं गवयपिण्डे दृष्टे 'अयं गवयपदवाच्यः गोसादृश्यात् यन्नैवं तन्नैवं यथा महिषमातङ्गादिः' इत्यनेन व्यतिरेकानुमानेन वा पदपदार्थयोः संसर्गग्रहः इति~।

	तत्रादौ गवयशब्दस्य गवयपदवाच्यत्वं कथमिति चेदुच्यते~-

		\subsection{गोसदृशपदज्ञानादेव गवयत्वभानम्}

		नागरिकं प्रति यदा आरण्यकः 'गोसदृशो गवयः' इति ब्रूते तदानीमेव नागरिकस्य गवयपदवाच्यत्वविषयकग्रहो उदेति~। तथा हि~- प्रष्टुः गवयपदार्थः कः इत्यत्रैव जिज्ञासा वर्तते~। अतः आरण्यकवचनमपि तदुपशमनसमर्थमेव भवेत्~। अन्यथा पुनः प्रश्नस्य उत्थितिः स्यात्~। किन्तु तथा न दृश्यते~। तस्मादारण्यवचनादेव तादृशजिज्ञासानिवर्तकः गवयपदतदर्थयोः सम्बन्धविषयकग्रहो जायते~। तादृशशाब्दबोधस्तु स्वरसतया न सिध्यति, 'गोसदृशो गवयः' इत्यस्मिन् वाक्ये गवयपदवाच्यत्वविशिष्टस्यार्थस्य शक्त्या उपस्थापकपदाभावात्~। अतः तत्र तादृशबोधनिर्वाहाय गवयपदस्य गवयपदवाच्यत्वे लक्षणाभ्युपगम्यते~। 

		ननु गवयत्वेन प्रवृत्तिनिमित्तेन गवयपदवाच्यत्वं यदा जानाति तदानीमेव तस्य गवये गवयपदवाच्यत्वग्रहः सम्भवति~। प्रकृते तु गवयस्यैव अदर्शनात् तद्वृत्तिजातेरपि अग्रहात् गवयपदे लक्षणास्विकारेऽपि 'गोसादृश्यविशिष्टाभिन्नः गवयपदवाच्यः' इत्येव बोधः, न तु 'गवयत्वविशिष्टाभिन्नो गवयपदवाच्यः' इति~। तथा च गोसादृश्यं प्रवृत्तिमित्तीकृत्यैव गवयं जानातीत्यतः न तेन वाक्येन शक्तिग्रहसम्भव इति चेन्न~।  गोसदृशः इत्यनेन गवयत्वविशिष्ट एव लक्ष्यते~। लक्षणाबीजन्तु तात्पर्यानुपपत्तिरेव~। तथा हि~- गोसादृश्यस्य अखण्डत्वेन गुरुत्वात् प्रवृत्तिनिमित्तत्वं न सम्भवति~। अतः गवयत्वस्यैव जातित्वात् प्रवृत्तिनिमित्तत्वं स्वीकरणीयम्~। तथा च गोसदृशो गवय इत्यस्य गवयत्वविशिष्टो धर्मी गवयपदवाच्य इत्यर्थः~। गवयत्वं प्रवृत्तिनिमित्तीकृत्यैव बोधो जायते इत्यतः तत्र शक्तेरपि भाने न किमपि बाधकमस्ति~। 

		ननु यद्यन्वयानुपपत्तिः स्यात् तदानीमेव लक्षणा सम्भवति~। प्रकृते यो गोसदृशः स गवयपदवाच्य इत्यन्वयसम्भवान्न लक्षणाप्रसक्तिरिति चेन्न~। तात्पर्यानुपपत्तितः लक्षणासम्भवात्~।  किमत्र तात्पर्यमनुपपन्नम्~? सति लघुनि गवयत्वरूपप्रवृत्तिनिमित्ते गुरुणा गोसादृश्यरूपप्रवृत्तिनिमित्तेन गवयपदवाच्यत्वव्युत्पादनमनुपपन्नम्~। तस्मादारण्यकवाक्यादेवासौ गवये गवयपदवाच्यत्वं ग्रह्णाति~। तदुक्तं वैशेषिकग्रन्थेषु~-

		{\fontsize{11.7}{0}\selectfont\s  आप्तेनाप्रसिद्धस्य गवयस्य गवा गवयप्रतिपादनादुपमानमाप्तवचनमेव~।\footnote{प्र.भा. ५१९}}

		{\fontsize{11.7}{0}\selectfont\s  आप्तिः साक्षादर्थस्य प्राप्तिः~, यथार्थोपलम्भः~, तया वर्तत इत्याप्तः~, साक्षात्कृतधर्मा~, यथार्थदृष्टस्यार्थस्य चिख्यापयिषया प्रयोक्तोपदेष्टा~, तेनाप्तेन वनेचरेण विदितगवयेन अज्ञातगवयस्य नागरिकस्य कीदृग्गवय इति पृच्छतो गोसारूप्येण गवयस्य प्रतिपादनादुपमानं यथा गौर्गवयस्तथेति वाक्यमाप्तवचनमेव~। वक्तृप्रामाण्यादेव तथा प्रतीतेः~। आप्तवचनं चानुमानम्~। तस्मादुपमानमप्यनुमानाव्यतिरिक्तमित्यभिप्रायः~।\footnote{न्या.कं. ५१९}}

		\subsection{उपमानस्याप्तवचनत्वे युक्तिः}

		आरण्यकोक्तं 'गोसदृशो गवय' इति वाक्यं श्रुत्वा नागरिकः वनं गत्वा गवयपिण्डं पश्यति~। तदा सः आरण्यकोक्तवाक्यार्थं स्मरन् अनुमीते 'गवयपदं गवयत्वविशिष्टपिण्डवाचकं असतिवृत्त्यन्तरे तदभियुक्तेनात्र प्रयुज्यमानत्वात् महिषादिवत्' इति~। 'यो यत्र असतिवृत्त्यन्तरे शब्दं प्रयुज्तते स तस्य वाचकः' इति तत्र व्याप्तिः~। यथा 'महोषोऽयम्' इति यत्राभियुक्तेन प्रयोगः क्रियते तत्र महिषादिपदानां महिषत्वविशिष्टपिण्डवाचकत्वं दृष्टम्~, तथैव प्रकृते गवयपदस्यापि अभियुक्तेनारण्यकेन प्रयुक्तत्वात् तस्य गवयत्वविशिष्टपिण्डवाचकत्वं सिद्धम्~। अत्र 'गङ्गायां घोष' इत्यादिवाक्येषु अभियुक्तैः प्रयुक्तेषु गङ्गादिपदानां प्रवाहादिवाचकत्वाभावात् 'यो यत्र अभियुक्तेन प्रयुज्यते स तस्य वाचकः' इति व्याप्तिः व्यभिचरितः~। तस्मादसतिवृत्त्यन्तरे इति हेतुकुक्षौ विशेषणं दत्तम्~। तथा च अनुमानादेव शक्तिग्रहादुपमानं नातिरिक्तं प्रमाणम्~। तदुक्तं लीलावत्याम्~-

		{\fontsize{11.7}{0}\selectfont\s उपमानं च न मानान्तरम्~, अनुमानादेव तदर्थसिद्धेः~। यो यत्रासति वृत्यन्तरे प्रयुज्यते स तस्य वाचको यथा गोशब्दो गोजातीयस्य~, प्रयुज्यते चायमसति प्रतिभासमानजातीय इति~। न चायमसिद्धः~,  मुख्यानुपपत्तिं विनोपचारस्यासम्भवात्~। सादृश्यवति प्रयोगस्य कल्पनागौरवापत्तिहृतत्त्वात्~। व्यक्तिषु प्रयोगस्यानन्त्यदूषित्वात्~। नूनमयमेतज्जातीयाभिधानाय प्रयुज्यते इति निश्चयोपपत्तेः~। न चेदेवमुपमानेऽपि वृत्त्यन्तरनिमित्तान्तरविषयविशिष्टप्रयोगसम्भावनायामपेक्षितासिद्धिप्रसङ्गात्~।\footnote{न्या.ली.५३१-५३६}}

		\subsection{अत्र गौतमसूत्रविरोधपरिहारः}

		न चोपमानस्याप्तवचनत्वे 'प्रत्यक्षानुमानोपमानशब्दाः प्रमाणानि'\footnote{न्या. सू. १.१.३} इति गौतमसूत्रविरोधः~। यथा प्रमाणप्रमेयेत्यादिप्रथमसूत्रे प्रमाणनिग्रहस्थानविशेषयोः दृष्टान्तहेत्वाभासयोः प्रयोजनान्तरवशात्पृथगभिधानं कृतं तथैव प्रकृतेऽपि~। किं तत् प्रयोजनम्~? शब्दप्रमाणसमर्थनमेव~।

		अत्र केचित्~- पदेन वाक्येन वा संज्ञासंज्ञिनोः सम्बन्धप्रतिपत्तिः नैव भवितुमर्हन्ति~। तथा हि प्रत्यक्षानुमानाभ्यां या याः प्रमितयः जायन्ते तत्र ये विषयाः ते एव शब्दजन्यप्रमितावपि विषयाः भवन्ति~। अन्यत्र तु संज्ञिनः अज्ञानात् पदेन सम्बन्धग्रहो नैव भवति~। अन्यथा पदेनार्थाभिधाने सम्बन्धग्रहणं तद्गृहणे पदेनार्थाभिधानमिति इतरेतराश्रयप्रसङ्गः~। तस्मात् पदेन सम्बन्धग्रहो नैव सम्भवति~। वाक्यार्थस्य तु प्रसिद्धपदार्थान्वयरूपत्वात् तेनापि संज्ञासंज्ञिनोः न सम्बन्धग्रह इति वदन्ति~। तदयुक्तम्~। एतन्निराकरणार्थमेव उपमानस्य पृथगभिधानं कृतम्~। यथा अप्रसिद्धस्य शक्रस्य प्रसिद्धं सहस्राक्षत्वादिधर्ममुपादाय 'सहस्राक्षः शक्रः' इति वाक्येन शक्रपदतदर्थयोः सम्बन्धप्रतिपत्तिः क्रियते~। तत्र यथा तस्याप्तवचनत्वं तद्वत् प्रकृतेऽपि अप्रसिद्धस्य गवयस्य प्रसिद्धं गोसादृश्यमुपादाय उपमानाख्येन वाक्येन गवयपदतदर्थयोः सम्बन्धप्रतिपत्तिः क्रियते~। तस्मान्न सूत्रविरोधः~। तदुक्तं न्यायसारे~-

		{\fontsize{11.7}{0}\selectfont\s सूत्रविरोध इति चेत्~। न~। प्रमाणनिग्रहस्थानाभ्यां दृष्टान्तहेत्वाभासादीनामिव प्रयोजनवशेन पृथगभिधानात्~। तर्हि प्रयोजनं वाच्यम्~। उच्यते~। शब्दप्रामाण्यसमर्थनं प्रयोजनम्~। कथम्~। केचिदाहुः – प्रत्यक्षानुमानविषयत्वे शब्दस्यानुवादकत्वम्~। तदविषयत्वे सम्बन्धाग्रहणादवाचकत्वम्~। पदार्थस्याप्रसिद्धत्वान्न पदेन सम्बन्धग्रहणमितरेतराश्रयत्वात्~। वाक्यर्थस्तु प्रसिद्धानां पदानामन्वयमात्रमिति~। तन्निराकरणार्थमुपमानं निदर्शनार्थत्वेन पृथगुक्तम्~। यथा कार्यार्थिनोऽप्रसिद्धगवयस्य प्रसिद्धं गोसादृश्यमुपादायोपमानाख्येन वाक्येन संज्ञासंज्ञिसम्बन्धप्रतिपत्तिः क्रियते, तथा किञ्चिन्निमित्तमुपादाय शक्रादिपदपदार्थयोरपि~। तस्मादन्यार्थत्वान्न सूत्रविरोधः~।\footnote{न्या.सा. १०६-१११}}

		\subsection{व्यतिरेकानुमानात् गवयपदवाच्यत्वग्रहः}

		अथ वा गवयपदार्थजिज्ञासुं नागरिकं प्रति आरण्यकः गवये विद्यमानं गोसादृश्यरूपं लिङ्गं उपदिशति 'गोसदृशो गवय' इति वाक्येन~। ततः ज्ञातातिदेशवाक्यार्थः सः अरण्ये गवयपिण्डं दृष्ट्वा अनुमीते 'अयं गवयपदवाच्यः गोसादृश्यात् यन्नैवं तन्नैवं यथा महिषादिः' इति~। अत्र गोसादृश्यस्य महिषादौ क्वचिदंशे सत्त्वेऽपि वैधर्म्यबाहुल्यादेव तत्र तद्वाचकत्वग्रहो नोदेति~। अथ वा पूर्वमेव महिषादिषु महिषादिपदवाच्यत्वस्यावगतत्वात् न तत्र तद्विषयकानुमितिः~। गवये तु गोसादृश्यस्य अधिकतया दर्शनात् गवयपदवाच्यत्वानुमितिरिति~। तदुक्तं कणादरहस्ये~-

		{\fontsize{11.7}{0}\selectfont\s यद्वा गवयपदार्थः कः~? इति नागरिकेण पृष्टः आरण्यकः गवये विध्यमानं गोसादृश्यं लिङ्गत्वेनातिदिशति 'गोसदृशो गवय' इति~। ततः प्रतिपन्नातिदेशवाक्यार्थः श्रोता वनं गच्छति~। तत्र करितुरगमहिषादिषु कथञ्चिद्गोसादृश्यं पश्यन्नपि न तेषां गवयपदवाचकत्वं जानाति~। वैधर्म्यबाहुल्येन तिरस्कृतत्वात्~। कदाचित् गोसादृश्यं गवयपिण्डं पश्यन्नूनमयं गवयपदवाच्यः, गोसदृशत्वात्~, यन्न गवयपदवाच्यः नासौ गोसदृशो यथा महिषमातङ्गादिरित व्यतिरेकानुमानेन तत्र गवयपदवाच्यत्वं निश्चिनोति~। यथा का पृथिवीति पृथिवीपदव्युत्पित्सुं प्रति 'गन्धवती पृथिवी' इति उच्यते चेदनन्तरं स पृथिवीदर्शने सति तत्रत्य गन्धं गृहीत्वा इदं पृथिवीपदवाच्यं गन्धवत्वादिति जानाति~।\footnote{क.र.}}


		\subsection{उपमानस्यातिरिक्तप्रमाणत्वप्रदर्शनम्}

		न्यायनये तावत् उपमानस्यातिरिक्तप्रमाणत्वमङ्गीकृतं वर्तते~। अत एव "प्रत्यक्षानुमानोपमानशब्दाः प्रमाणानि"\footnote{न्या.सू.१.१.३} इति न्यायसूत्रं प्रणीतं भगवता गौतमेन~। तथा चायमाशयः~- गवयपिण्डे गवयशब्दवाचकत्वप्रतीतिः अरण्ये गवयपिण्डे गोसादृश्यप्रत्यक्षानन्तरं अतिदेशवाक्यार्थस्मरणद्वारा भवति~। अत्र तादृशसङ्केतविषयकप्रतीतेः करणं न व्याप्तिज्ञानम्~, नापि पदज्ञानम्~, अपि तु सादृश्यज्ञानमित्यतः अस्य अनुमानागमाभ्यामतिरिक्तत्वम् इति~।

		\subsection{आप्तवाक्यार्थमेवोपमानम्}

		तत्र प्राचीननैयायिकास्तावत् सादृश्यज्ञानमुपमितिकरणमिति वदन्ति~। अयमाशयः~- आप्तवचनाद्यत्र सादृश्यप्रकारकशाब्दधीरुदेति ततः कालान्तरे गवये सादृश्यप्रतक्षानन्तरमस्य गवयशब्दः संज्ञा इति बोधो जायते~। अत्र पदपदार्थयोः सङ्केतविषयकमितिं प्रति सादृश्यप्रत्यक्षद्वारा सादृश्यज्ञानस्य करणत्वात् तदेवोपमानमिति~। तदुक्तं  भाष्यकारैः~-

		{\fontsize{11.7}{0}\selectfont\s समाख्यासम्बन्धप्रतिपत्तिः उपमानार्थ इत्याह~। यथा गौरेवं गवय इत्युपमाने प्रयुक्ते गवा समानधर्माणम् अर्थमिन्द्रियार्थसन्निकर्षादुपलभमानोऽस्य गवयशब्दः संज्ञेति संज्ञासंज्ञिसम्बन्धं प्रतिपद्यते इति~। यथा मुद्गस्तथा मुद्गपर्णी, यथा माषस्तथा माषपर्णीत्युपमाने प्रयुक्ते उपमानात्संज्ञासंज्ञिसम्बन्धं प्रतिपद्यमानस्तामोषधीं भैषज्याय आहरति~। एवमन्योऽप्युपमानस्य लोके विषयो बुभुत्सितव्य इति~।\footnote{न्या.भा. २७}}

		\subsection{प्रत्यक्षादतिरिक्तमुपमानम्}

		संज्ञासंज्ञिनोः सम्बन्धप्रतिपत्तिस्तावत् यत्र आप्तवचनश्रवणानन्तरं जायते तच्च चक्षुर्व्यापारजन्यं न~। अन्यथा सत्यपि चक्षुर्व्यापारे आप्तवचनस्मरणाभावे उपमित्यापत्तिः~। तत्र उपमितेरनुदयात् केवलचक्षुर्व्यापारस्य तदजनकत्वमेव~। तस्मात् न उपमानं प्रत्यक्षप्रमाणात्मकम्~। तदेवं प्रतिपादयन्ति आचार्याः~-

		{\fontsize{11.7}{0}\selectfont\s अस्ति तर्हि सादृश्यादिज्ञानकाले विस्फारितस्य चक्षुषो व्यापारः~। न~। उपलब्धगोसादृश्यविशिष्टगवयपिण्डस्य वाक्यतदर्थस्मृतिमतः कालान्तरेऽप्यनुसन्धानबलात् समयपरिच्छेदोपपत्तेः~।\footnote{न्या.कु. ३७८}}

		\subsection{अनुमानादतिरिक्तमुपमानम्}

		न तावदुपमानस्य अनुमानादनतिरिक्तत्वम्~। 'गवयशब्दः गवयवाचकः असतिवृत्त्यन्तरे अभियुक्तैस्तत्र प्रयुज्यमानत्वात्~, गवि गोशब्दवत्' इत्यनुमानं प्रामाणमिति चेदसिद्धेः~। मुख्यवृत्तिज्ञाने सति हि लक्षणावृत्तेः ज्ञानं सम्भवति, लक्षणायाः शक्यसम्बन्धरूपत्वात्~। वृत्त्यन्तराग्रहे च विशेषणाज्ञानात् हेतुग्रहाभाव इति विशेषणासिद्धिः~। अभियुक्तप्रयोगस्तु लक्षणावृत्तिस्थलेऽपि दृष्ट इति स्वरूपासिद्धिरिति वदत्याचार्याः~- 

		{\fontsize{11.7}{0}\selectfont\s अस्त्वनुमानम्~- तथा हि~- गवयशब्दो गवयस्य वाचकः असति वृत्त्यन्तरे अभियुक्तैस्तत्र प्रयुज्यमानत्वात्~, गवि गोशब्दवदिति चेन्न~। असिद्धेः~। \footnote{न्या.कु. ३८२}}

		\subsection{शब्दादतिरिक्तमुपमानम्}

		ननु यथा वेदवाक्यानां वेदाध्ययनानन्तरमङ्गादिज्ञानसहकृतानामेव कालान्तरे धर्माद्यर्थविषयकबोधजनकत्वम्~, तद्वत् आरण्यकोक्तवचनश्रवणानन्तरं कालान्तरे पिण्डप्रत्यक्षेण तद्गतगवयत्वज्ञानात् तत्सहकृतमेव वाक्यस्मरणं संज्ञासंज्ञिसम्बन्धबोधं जनयतीति अस्य शब्दत्वमेवेति चेन्न~। आकाङ्क्षायोग्यतादीनां सत्त्वे वाक्यश्रवणानन्तरं शाब्दबोधो भवत्येव~। कारणानां सर्वेषां समवधाने सति कार्यस्यावश्यमुदयात्~। वेदाध्ययनस्थले तु तात्पर्यादीनामज्ञानात् कारणाभावादेव न कार्योत्पत्तिः~। प्रकृते गोगवयसादृश्ययोः सामानाधिकरण्यात् आकाङ्क्षादिकारणानां समवधानाच्च 'गोसदृशाभिन्नो गवयशब्दवाच्यः' इत्याकारकबोधो भवत्येव~। न च गवयपदप्रवृत्तिनिमित्तस्याज्ञानात् न तत्र बोध इति वाच्यम्~। यथा 'घटो रक्तो न वा' इति सन्देहदशायामपि 'घटो भवति' इति वाक्यात् शाब्दबोधो भवति तद्वत् प्रकृतेऽपि शाब्दबोधसम्भवात्~। अन्यथा यथाश्रुतान्वये सति जिज्ञासायां पुनरन्वयः इष्यते तदा वाक्यभेदापत्तिः~।

		न च 'गङ्गायां घोष' इति वाक्यात् आदौ गङ्गादिपदार्थानां विभक्त्याद्यर्थेषु अन्वयो भवत्येव~। ततः घोषपदार्थे अन्वयानुपपत्त्या वृत्त्यन्तरेणार्थोपस्थितिः ततः शाब्दबोधः~। तथा च अनेन वाक्येनैव यथा पदार्थान्तरविषयकशाब्दबोधो यथा तथैव प्रकृतेऽपि गवयत्वादिविषयकशाब्दबोधः इति वाच्यम्~। एवंरीत्या यद्युच्यते तर्हि 'पीनो देवदत्तः दिवा न भुङ्के' इति वाक्यादेव रात्रिभोजनविषयकबोधापत्तिः~। तथा च 'गङ्गायां घोष' इत्यत्र अन्वययोग्यताभावात् वृत्त्यन्तरोपस्थापितेनार्थेनान्वयोभवति, न तु अपदार्थेन~। प्रकृते तादृशानुपपत्तीनामभावात् कारणसमुदायस्य च सत्त्वात् गोसादृश्यप्राकारकबोध एव जायते~। स च न संज्ञासंज्ञिसम्बन्धविषयक इति न तदुपमानप्रमाणम्~। तदुक्तं कुसुमाञ्जल्याम्~-

		{\fontsize{11.7}{0}\selectfont\s \begin{center}'श्रुतान्वयादनाकाङ्क्षं न वाक्यं ह्यन्यदिच्छति~।\\ पदार्थान्वयवैधुर्यात्तदाक्षिप्तेन सङ्गतिः~॥'\\\end{center} 'गो सदृशो गवयशब्द वाच्य' इति सामानाधिकरण्यमात्रेणान्वययोपपत्तौ विशेषसन्देहेऽपि वाक्यस्य पर्यवसितत्वेन मानान्तरोपनीतानपेक्षणात्~। रक्तारक्तसन्देहेऽपि 'घटो भवति' इति वाक्यवत्~, अन्यथा वाक्यभेददोषात्~। न च 'गङ्गायां घोष' इतिवत् पदर्था एवान्वयायोग्याः, येन प्रमाणान्तरोपनीतेनान्वयः स्यात्~। प्रतीतवाक्यार्थबलायातोऽप्यर्थो यदि वाक्यस्यैव, दिवाभोजननिषेधवाक्यस्यापि रात्रिभोजनमर्थः स्यात्~।तस्माद्यथा गवयशब्दः कस्यचिद्वाचकः शिष्टप्रयोगादिति सामान्यतः निश्चितेऽपि विशेषे मानान्तरापेक्षा, तथा गोसदृशस्य गवयशब्दो वाचक इति वाक्यान्निश्चितेऽपि सामान्ये, विशेषवाचकत्वेऽस्य मानान्तरमनुसरणीयम्~।\footnote{न्या.कु. ३८०, ३८१}}

		\subsection{वैधर्म्यज्ञानमप्युपमानम्}

		क्वचित् वैधर्म्यज्ञानादपि उपमितिरुदेति~। तथा हि~- दक्षिणापथं गन्तुकामेन उत्तरापथं गन्तुकामं प्रत्युक्तं 'धिक्करभमतिदीर्घग्रीवं प्रलम्बचपलौष्ठं कठोरतीक्ष्णकण्टकाशिनं कुत्सितावयवसन्निवेशमपसदं पशूनाम्' इति~। अत्र पशूनामपसदम् इत्यस्य पशूनां मध्ये अपकृष्टमित्यर्थः~। तेन पशुवैधर्म्यमेव करभे भासते~। एतादृशवैधर्म्यज्ञानेनापि कश्चन  उत्तरापथं गत्वा तादृशपिण्डं दृष्ट्वा करभपदतदर्थयोः सम्बन्धं परिच्छिनत्ति~। अतः वैधर्म्यज्ञानमपि उपमानमेव~। तदुक्तं तत्त्वचिन्तामणौ~- 

		{\fontsize{11.7}{0}\selectfont\s  यदोदीच्येन क्रमेण कं निर्गत्योक्तं धिक्करभमतिदीर्घग्रीवं प्रलम्बचपलौष्ठं कठोरतीक्ष्णकण्टकाशिनं कुत्सितावयवसन्निवेशमपदं पशूनामिति तदुपश्रुत्य दाक्षिणात्य उत्तरापथं गत्वा तादृशं वस्तूपलभ्य नूनमसौ करभ इति प्रत्येति~। तत्र किं मानम्~? न तावदुपमानं सादृश्याभावात्~, न च प्रमाणान्तरं सम्भवति~।\footnote{त.चि. ६१,६२}}

		\subsection{उपमानस्य लक्षणम्}

		उपनानस्य लक्षणन्तावत् अनवगसङ्गतिसंज्ञासमभिव्याहृतवाक्यार्थस्य संज्ञिन्यनुसन्धानमिति~। अनवगतसङ्गतिश्चासौ संज्ञा चेति कर्मधारयः~। तस्यामभिव्याहृतवाक्यार्थः आरण्यकवाक्यश्रवणादुत्पन्नः~। तस्य गवयरूपसंज्ञिसाक्षात्कारे सति अनुसन्धानम्~- अतिदेशवाक्यार्थस्मरणमेव उपमानमित्यर्थः~। एतेन वैधर्म्यादिज्ञानेनापि उपमिति सम्भवात् नाननुगमशङ्कावतारः सम्भवति~। तदुक्तं कुसुमाञ्जल्याम्~- 

		{\fontsize{11.7}{0}\selectfont\s लक्षणन्त्वस्य अनवगतसङ्गतिसंज्ञासमभिव्याहृतवाक्यार्थस्य संज्ञिन्यनुसन्धानमुपमानम्~। वाक्यार्थश्च क्वचित् साधर्म्यं क्वचिद्वैधर्म्यमतो नाव्यापकम्~, तस्मान्नियतविषयत्वादेव न तेन बाधः, न त्वनतिरेकादिति स्थितिः~।\footnote{न्या.कु. २८६}}

	\section{विमर्शः}

		\subsection{उपमानस्याप्तवचनत्वविमर्शः}

		गवयपदमात्रमजानन् कश्चित् गवयाभिज्ञं पृच्छति~- कीदृग्गवय~? इति~। तदा गवयाभिज्ञो वदति 'गोसदृशो गवय' इति~। ततः कालान्तरे गवयदर्शने सति तस्य गवयपदतदर्थयोः सम्बन्धग्रहो जायते~। अत्र एतादृशसम्बन्धग्राहकं प्रमाणं क्लृप्तमनुमानं शब्दं वा, उतातरिक्तं किञ्चित् कल्पनीयं वा इति जिज्ञासायां काणादास्तावत् प्रथमपक्षमेव समर्थयन्ति~। तथा हि~- 'गोसदृशो गवय' इति वाक्यादेव गवयपदतदर्थयोः सम्बन्धप्रतीतिजननात् शब्दमेवेदं प्रमाणमिति~। 

		न च किञ्चिदनुगतधर्मविशिष्टे एव पिण्डे गवयपदवाच्यत्वग्रहः उपपादनीयः~। अन्यथा शक्त्यानन्त्यप्रसङ्गः~। प्रकृते गवयत्वस्य जातित्वात् तद्विशिष्टे एव गवयपदवाच्यत्वग्रहः उपपादनीयः~। किन्तु गवयपदप्रवृत्तिनिमित्तस्य तस्य पूर्वमज्ञानात् न वाक्यात् तादृशबोधः‌ इति वाच्यम्~। 'गोसदृशो गवयः' इति वाक्यात् जायमानबोधे गवयपदप्रवृत्तिनिमित्ततया कश्चन धर्मः भासत एव~। स धर्मः न गोसादृश्यम्~। सति लघुधर्मे गुरुधर्मस्य प्रवृत्तिनिमित्तत्वासम्भवात्~। कोऽसौ लघुधर्म इति चेत् गवयत्वमेव~। कथं तस्य ज्ञानमिति चेत् पदादेव~। किन्तद्बोधकं पदमिति चेत् गोसदृशपदमेव~। लक्षणया गोसदृशपदात् गवयत्वविशिष्टस्यैव उपस्थितिः~। लक्षणा बीजन्तु तात्पर्यानुपपत्तिः~। तथा च गोसदृशपदस्य गवयत्वविशिष्टे लक्षणाभ्युपगमात् वाक्यादेव तादृशप्रमिति सम्भवात् उपमानमाप्तवचनमेवेति~।

		अत्रोच्यते~- न तद्वाक्यात् गवयपदतदर्थयोः सम्बन्धप्रतीतिः भवितुमर्हति~। गवयत्वस्य तदा अज्ञानात्~। न च लक्षणया गवयत्वस्य उपस्थितिः सम्भवति~। लक्षणाव्यापारस्तु तदा एव सम्भवति यदा तेनोपस्थाप्योऽर्थः ज्ञातं स्यात्~। यथा कश्चित् तीरमेव न जानाति चेत् तस्य 'गङ्गायां घोष' इति वाक्यात् बोध एव न जायते~। प्रकृतेऽपि गवयत्वस्य पूर्वमज्ञानात् आप्तेनापि गवयत्वाबोधनात् न तत्र गवयत्वे लक्षणा युज्यते~।

		किञ्च आप्तवाक्यश्रवणानन्तरं कालान्तरे यदा स गवयपिण्डं पश्यति तदा पूर्वोक्तवाक्यं स्मरतीत्यनुभवः~। तदनुपपन्नम्~। पुनस्तद्विषयकवाक्यार्थस्मरणस्य निष्फलत्वात्~। अस्तु वा उद्बोधकवशात् स्मरणं तथापि तदुत्तरं 'गवयः गवयपदवाच्यः' इति संज्ञासंज्ञिसम्बन्धप्रतिपत्तिर्या उदेति तदनुपपत्तिः~। तस्य पूर्वमेव गृहीतत्वात्~। तथापि पुनरुत्पत्तिः प्रतिबन्धकाभावादिति चेत्~, तस्य अनुभवात्मकत्वात् तस्य करणं वक्तव्यम्~। तच्च शब्दादतिरिक्तमेवेति~।

		\subsubsection{उपमानस्यानुमानत्वविमर्शः}

		ननु 'अयं गवयपदवाच्यः गोसादृश्यात् यन्नैवं तन्नैवं यथा महिषमातङ्गादिः' इत्यनुमानेन गवये गवयपदवाच्यत्वप्रतीतिरुदेति~। आरण्यकोक्तवचनात् गवयपिण्डे गोसादृश्यज्ञानान्न स्वरूपासिद्धिः~। गवयपदवाच्यत्वसन्देहान्न व्यभिचारः~। तस्मादुपमानमनुमानमेवेति चेन्न~। महिषादिषु कथञ्चिद्गोसादृश्यस्य सत्त्वात् तत्रापि गवयपदवाच्यत्वप्रतीतिप्रसङ्गः~। किमर्थं तत्र तथा न प्रतीतिरिति चेत् महिषादेः महिषशब्दवाच्यत्वस्य निश्चितत्वात्~। अन्यथा महिषादौ गोसादृश्यग्रहणात् तत्रैव गव्यपदवाच्यत्वप्रतीतिप्रसङ्गः~। तस्माद्गवयपदवाच्यत्वाभाववति महिषादौ गोसादृश्यस्य सत्त्वाद्व्यभिचारः~। न च यथा पृथिवीपदार्थजिज्ञासुं प्रति 'गन्धवती पृथिवी' इति प्रयुङ्क्ते तत्र गन्धवत्वरूपहेतुना पृथिव्यां पृथिवीपदवाच्यत्वप्रतीतिः तद्वदत्रापि इति वाच्यम्~। येन पृथिवीदर्शनेन पृथिवीत्वपदप्रवृत्तिनिमित्तधर्मं ज्ञातं किन्तु तस्य पृथिवीपदवाच्यत्वं न गृहीतं तत्रैव तथा सम्भवः~। येन तु पृथिवीत्वज्ञानमेव नास्ति तस्य तु पृथिव्यां गन्धग्रहणे सति अवान्तरवाक्यार्थस्मरणादेव पृथिव्यां पृथिवीपदवाच्यत्वग्रह इति~। तस्मात् पदपदार्थसम्बन्धग्रहे व्याप्त्यभावान्नानुमानत्वमुपमानस्य~।

		\subsubsection{प्रत्यक्षाद्यतिरिक्तमेवोपमानम्}

		व्याप्तिज्ञानपरामर्शादीनां संज्ञासंज्ञिसम्बन्धप्रतिपत्त्यजनकत्वात् तदकरणत्वाच्च उपमानमनुमानाद्यतिरिक्तं प्रमाणम्~। अत एव गौतमसूत्रं "प्रत्यक्षानुमानोपमानशब्दाः प्रमाणानि"\footnote{न्या.सू.१.१.३} इति~। अरण्ये साक्षात्कृतगवयपिण्डस्य गवयपदार्थजिज्ञासुः अवान्तरवाक्यार्थस्मरणं विना इन्द्रियव्यापारेणैव गवयपदतदर्थयोः सम्बन्धप्रतिपत्त्यापत्तिः~। तस्मादुपमानं प्रत्यक्षादतिरिक्तमेव~।

		'गवयशब्दः गवयवाचकः असतिवृत्त्यन्तरे वृद्धैस्तत्र प्रयुज्यमानत्वात् गविगोशब्दवत्' इत्यनुमाने तु शक्तिज्ञानं विना लक्षणावृत्तेरनवगमात् वृत्त्यन्तराभावस्य पक्षेऽनवगमात्स्वरूपासिद्धिः~। व्यतिरेकिलिङ्गकानुमाने च व्यभिचारो प्रदर्शितः~।‌ तस्मान्नास्यानुमानत्वम्~।

		आरण्यकोक्तवाक्यज्ञानादेव संज्ञासंज्ञिप्रतिपात्तिरित्यप्ययुक्तमेव~। तदा गवयत्वस्य गवयपदप्रवृत्तिनिमित्तस्य अज्ञानात्~। अज्ञातपदप्रवृत्तिनिमित्तकशब्देन मृदङ्गादिध्वनिवत् शाब्दबोधासम्भवः~। किञ्च आरण्यकवाक्यश्रवणसमनन्तरं संज्ञासंज्ञिसम्बन्धप्रतिपत्त्यभावात् न तस्य उपमितिकरणत्वम्~। न च वेदाध्ययनकाले यथा वेदवाक्यानां शाब्दबोधाजनकत्वम्~, किन्तु अङ्गाध्ययनानन्तरं तेषां तज्जनकत्वं तद्वत् प्रकृतेऽपि आरण्यकवाक्यस्यैव कालान्तरे शाब्दबोधजनकत्वमिति वाच्यम्~। वेदाध्ययनकाले तात्पर्यज्ञानादिशाब्दसामग्र्यन्तरासत्त्वान्न शाब्दधीरुदेति~। कालान्तरे तेषां समवधाने सति शाब्दधीरुत्पद्यते~। तत्रापि वेदवाक्यानां पुनस्स्मरणादेव शाब्दधीरुत्पद्यते~। न तत्र अध्ययनकाले उत्पन्नवेदवाक्यानां ज्ञानं शाब्दधीकरणम्~। प्रकृते तु अरण्ये गवयदर्शनानन्तरं आरण्यकवचनजन्यबोध एव स्मर्यते, न तु तेनोक्तं वाक्यमित्यतः न तस्य वेदवाक्यवच्छाब्दधीजनकत्वम्~। तस्मात् गोसादृश्यप्रत्यक्षमेव अवान्तरवाक्यार्थस्मरणद्वारा उपमितिं जनयति इति~।

		\subsubsection{वैधर्म्यादिज्ञानानामुपमानत्वविमर्शः}

		करभपदार्थज्ञानाभाववान् कश्चित् उत्तरापथगन्तुकामः तदभिज्ञं पृच्छति करभोनाम कः~? इति~। तदानीं सः वदति 'धिक्करभमतिदीर्घग्रीवं प्रलम्बचपलौष्ठं कठोरतीक्ष्णकण्टकाशिनं कुत्सितावयवसन्निवेशमपसदं पशूनाम्' इति~। अत्र अतिदीर्घग्रीवमित्यादिपश्वसाधारणधर्मान्युक्त्वा अन्ते अपसदं पशूनामिति पशुवैधर्म्यमुक्तं तेन~। एतच्छ्रुत्वा कालान्तरे करभपिण्डे पशुवैधर्म्यदर्शने सति तस्य अवान्तरवाक्यार्थस्मरणद्वारा करभपदतदर्थयोः सम्बन्धज्ञानमुदेति~। अत्रापि इन्द्रियादीनां प्रमाणत्वाभावात् वैधर्म्यप्रत्यक्षस्य अवान्तरवाक्यार्थस्मरणद्वारा संज्ञासंज्ञिसम्बन्धप्रतिपत्तिजनकत्वात् उपमानप्रमाणत्वमेव अङ्गीक्रियते~।

		क्वचिदसाधारणधर्मज्ञानादपि उपमतिसम्भवात् तदप्युपमानमेव~। तथा हि~- जलपदार्थजिज्ञासुः 'शीतस्पर्शवत्य आपः' इति वाक्यं श्रुत्वा कालान्तरे जले शीतस्पर्शमनुभूय वाक्यार्थं स्मृत्वा जलत्वावच्छेदेन जलपदवाच्यत्वं परिच्छिनत्ति~। तत्र शीतस्पर्शरूपासाधारणधर्मप्रत्यक्षमेवोपमानम्~।

		\subsubsection{प्रमाणाधिक्यवारणम्}

		न च साधृश्यादिनानाधर्मविषयकप्रत्यक्षाणां उपमितिकरणत्वे प्रमाणाधिक्यप्रसङ्गः~। तथा च सूत्रविरोधः~। न हि अन्वयव्यतिरेकव्याप्त्यादिघटितपरामर्शभेदेऽपि अनुमितिकरणत्वेन तत्र प्रमाणैक्यम्~, तद्वदत्रापि उपमितिकरणत्वेन सादृश्यादिप्रत्यक्षाणां भेदेऽपि उपयमानैक्यं सम्भवति~। अनुमानस्थले अनुमित्यनन्तरमनुमिनोमि इत्यनुव्यवसायबलात् तद्विषयीभूतानुमितेः करणत्वेनानुगमः सम्भवति~। अनुव्यवसाये पक्षसाध्यवानित्येव ज्ञानस्य विषयत्वात्~। प्रकृते तु उपमिनोमि इत्यनुव्यवसाये तु उपमितेः विषयत्वेऽपि तस्याः सादृश्यार्थकत्वमेव लोके प्रसिद्धम्~। न तु संज्ञासंज्ञिसम्बन्धज्ञानार्थकत्वम्~। 
		
		अत एव {\fontsize{11.7}{0}\selectfont\s \begin{center}'दृश्यमानार्थसादृश्यात् स्मर्यमाणार्थगोचरम्~।\\ असन्निकृष्टसादृश्यज्ञानं ह्युपमितिर्मता~॥\\ यथा गवये गोसादृश्यदर्शनानन्तरं स्मर्यमाणे गवि गवयसादृश्यज्ञानम्~।\end{center}}\footnote{मा.मे. १०६} इति वदन्ति~। 
		
		तत्रैव वेदान्तिनः {\fontsize{11.7}{0}\selectfont\s 'तत्र सादृश्यप्रमाकरणमुपमानम्~। तथा हि प्राङ्गणेषु दृष्टगोपिण्डस्य पुरुषस्य वनं गतस्य गवेन्द्रियसन्निकर्षे सति भवति 'पिण्डो गोसदृश' इति प्रत्यक्षम्~। तदनन्तरं च भवति निश्चयो 'अनेन सदृशी मदीया' गौरिति~। तत्रान्वयव्यतिरेकाभ्यां गवयनिष्ठगोसादृश्यज्ञानं करणं गोनिष्ठगवयसादृश्यज्ञानं फलम्~। न चेदं प्रत्यक्षेण सम्भवति गोपिण्डस्य तदेन्द्रियासन्निकर्षात्~, नाप्यनुमानेन गवयनिष्ठसादृश्यस्यातल्लिङ्गत्वात्~। नापि मदीया गौरेतद्गवयसदृशी एतन्निष्ठसादृश्यप्रतियोगित्वात्~, यो यद्गतसादृश्यप्रतियोगी स तत्सदृशः, यथा मैत्रनिष्ठसादृश्यप्रतियोगी चैत्रः मैत्रसदृशः इत्यनुमानात्तत्संभव इति वाच्यम्~। एवंविधानुमानानवतारेप्यनेन सदृशी मदीया गौरिति प्रतीतेरनुभवसिद्धत्वात्~। उपमिनोमीत्यनुव्ययसायाच्च~। तस्मादुपमानम्मानान्तरम्~।}\footnote{वे.प.२२६ - २३७} इति कथयन्ति~।
		
		यदिदं पूर्वोत्तरमीमांसकानामुपमानवर्णनं यदस्ति तत्र सादृश्यविषयकज्ञानस्यैव प्रमाणव्यापारफलत्वात् तदुत्तरमुपमिनोमि इत्यनुव्यवसायो युक्त इति वाच्यम्~। संज्ञासंज्ञिसम्बन्धज्ञाने उपमितिपदस्य पारिभाषिकत्वस्वीकारात्~। अरण्ये गवयपिण्डे गोसादृश्यप्रत्यक्षादिव्यापारानन्तरं यत् संज्ञासंज्ञिसम्बन्धज्ञानमुदेति तदुत्तरं साक्षात्करोम्यनुमिनोमीत्याद्यनुव्यवसायो नोदेति~। अपि तु एतदपेक्षया विलक्षण एवानुव्यवसाय इति स्वीकर्तव्यमेव, तस्य प्रत्यक्षाद्यकरणकत्वात्~। तद्विषयत्वमेवोपमितिरित्युच्यते~। तत्करणत्वेन सादृश्यज्ञानादीनामनुगमसम्भवात् न प्रमाणाधिक्यापत्तिरिति~।

		अथ वा अनवगतसङ्गतिसंज्ञासमभिव्याहृतवाक्यार्थस्य संज्ञिन्यनुसन्धानमुपमानम्~। 'गोसदृशो गवयपदवाच्य' इत्यादिवाक्यानामनवगतसङ्केतसंज्ञासमभिव्याहृतत्वमस्ति~। अनवगतसङ्केतः गवयपदसङ्केतः, संज्ञा च गवयपदम्~, तयोः समभिव्याहारः गवयपदवाच्य इत्यंशे वर्तते~।‌ तस्य कालान्तरे गवयपिण्डे गोसादृश्यदर्शनानन्तरमनुसन्धानं अवान्तरवाक्यार्थस्मरणमेव~। तदेवोपमानमिति~। एवमेव वैधर्म्यादिज्ञानानामपि सम्भवात् तेषामपि उपमानत्वं सम्भवति~। अस्मिन् पक्षे अवान्तरवाक्यार्थस्मरणमेवोपमानम्~।

		\subsubsection{सादृश्यज्ञानस्य उपमितित्ववारणम्}

		किञ्च सादृश्यविषयकज्ञानस्य उपमितिपदार्थत्वे काव्यादिप्रमाणसत्त्वेऽपि न तत्करणस्य प्रत्यक्षागमाभ्यामतिरिक्तत्वं सम्भवति~। यदुक्तं गवये गोसादृश्यदर्शनानन्तरं स्मर्यमाणे गवि गवयसादृश्यज्ञानमेवोपमानमिति~। तत्र प्रत्यक्षे धर्मिणा सह सन्निकर्षस्य अपेक्षितत्वात् प्रकृते गोरभावात् न तेन सन्निकर्षः, तस्मादिदं प्रक्षात्मकं न भवति~। नापि शाब्दं आप्तेन गवि गवयसादृश्यस्याप्रतिपादनादिति, तदयुक्तम्~। गोप्रतियोगिकसादृश्यस्य गवये ज्ञाते गवयप्रतियोगिकसादृश्यस्यापि अर्थाद् गवि  ज्ञातमेव भवति~। तदर्थं न प्रमाणान्तरापेक्षा~। किञ्च सादृश्यमतिरिक्तपदार्थं न वा~। तस्यातिरिक्तपदार्थत्वे द्विनिष्ठत्वात्तस्य गवये गोसादृश्यदर्शने सति प्रत्यभिज्ञानवत् स्मृते गवि गवयसादृश्यस्य प्रत्यक्षतः प्रतीतिर्जायते~। तस्यानतिरिक्तपदार्थत्वे तु तद्भिन्नत्वे सति तद्गतभूयोधर्मवत्वस्य सादृश्यत्वे गोभेदस्य गवये प्रत्यक्षतः ज्ञाते गवयभेदोऽपि गवि ज्ञातमेव~। यो यत्प्रतियोगिकभेदवान् सः तदनुयोगिकभेदवांश्चेति व्याप्तेः सत्त्वात्~। गोगतधर्माणां गवये सत्त्वे ते धर्माः गव्यपि पूर्वमेव ज्ञातेति न तत्र प्रमाणान्तरापेक्षा~। तस्माद्गवि गवयसादृश्यज्ञानार्थं न प्रमाणान्तरापेक्षा~।

		किञ्च सादृश्यज्ञानस्य उपमितित्वे इन्द्रियव्यापरेण शब्देन वा तत्सम्भवान्न प्रमाणान्तरापेक्षा इति न्यायवार्तिककारा अप्याहुः~। तथा हि~-  {\fontsize{11.7}{0}\selectfont\s 'प्रत्यक्षागमाभ्यां नोपमानं भिद्यते~। कथमिति~? यदा तावुभौ गोगवयौ प्रत्यक्षेण पश्यति, तदायमनेन सरूप इति प्रत्यक्षतः प्रतिपद्यते~। यदापि शृणोति यथा गौरेवं गवय इति तदास्य शृण्वतः एव बुद्धिरुपजायते~। केचिद् गोधर्मा गवयेन्वयिन उपलभ्यन्ते, केचिद् व्यतिरेकेण इति~। अन्यथा हि यथा तथेत्येतन्न स्यात्~। भूयस्तु सारूप्यं गवा गवयस्य इत्येवं प्रतिपद्यते~। तस्मात् नोपमानं प्रत्यक्षागमाभ्यां भिद्यते~।}'\footnote{न्या. वा. }

		\subsubsection{उपमितिकरणस्वरूपविमर्शः}

		केचित्तु 'गोसदृशो गवयपदवाच्य' इति आरण्यकवाक्याज्जायमानबोध एव उपमानम्~। तस्य सादृश्यविषयकप्रत्यक्षसहकृतावान्तरवाक्यार्थस्मरणद्वारा उपमिति जनकत्वम्~। तत एव उपमानव्यापारारम्भात् तस्यैव उपमितिकरणत्वम्~। न च तस्य व्यवधानात् उपमितिकारणत्वस्यैव असम्भवात् उपमानत्वासम्भव इति वाच्यम्~। वाक्यार्थबोधजन्यस्ंस्कारवत्तासम्बन्धेन उपमितिकारणत्वसम्भवात् उपमानत्वं सम्भवतीति वदन्ति~। तदप्ययुक्तमेव~। वाक्यार्थबोधस्य उपमितिकारणत्वासम्भवात्~। तादृशसंस्कारवत्तासम्बन्धेन च तस्य अवान्तरवाक्यार्थस्मरणं प्रत्येव हेतुत्वं सम्भवति~। न च विनश्यदवस्थापन्नस्य तस्य तेन सम्बन्धेन उपमितिहेतुत्वं सम्भवतीति वाच्यम्~। अनुभवनाशकाले एव संस्कारोत्पत्त्या विनश्यदवस्थापन्नस्यैव वाक्यार्थबोधस्य संस्कारद्वारा वाक्यार्थस्मरणकारणत्वात्~। किन्तु गवये गोसादृश्यदर्शनमेव उपमानम्~। तस्य च अवान्तरवाक्यार्थस्मरणद्वारा उपमितिहेतुत्वम्~। अवान्तरवाक्यार्थबोधं प्रति उद्बोधकविधया तस्य हेतुत्वाच्च~।‌ 

		इदन्तु बोध्यम्~- व्यापारवदसाधारणकारणस्य करणत्वमते गोसादृश्यदर्शनस्य उपमानत्वं सम्भवति~। किन्तु फलायोगव्यवच्छिन्नकारणस्य करणत्वमते तु गोसादृश्यप्रत्यक्षस्य तु उपमितिकाले नाशान्न तस्य करणत्वं सम्भवति~। अपि तु अवान्तरवाक्यार्थस्मरणस्यैव इति~। तस्य उपमित्युत्पत्त्युत्तरक्षणे नाशादिति~।



	\section{आकृतेः पदशक्यत्वविचारः}

	शाब्दप्रमितेः‌ करणं तावत् शब्दज्ञानमेव~। शब्दाः स्वार्थेन सम्बद्धा एव श्रुतिस्मृतिपुराणादिषु निबद्धाः, अस्माभिश्च प्रयुज्यन्ते~। एतादृशस्य शब्दस्य अर्थेन सह सम्बन्धः‌ शक्तिरित्युच्यते~। शब्दानां कुत्र शक्तिः~? जातौ, व्यक्तौ उत आकृताविति विकल्पे प्राप्ते जात्याकृतिव्यक्तिषु शक्तिः इति केचन वदन्ति~। जातिविशिष्टव्यक्तौ शक्तिः, आकृतिस्तावन्न पदार्थः इत्यन्ये~। जातावेव शक्तिः‌ इति शास्त्रान्तरीयाः‌~। तदत्र निरूप्यते~।

		\subsection{आकृतिः न पदशक्या}

		आकृतिः न पदशक्या~। न च तदा सूत्रविरोधः सम्भवति~। गवादिपदानां गोत्वविशिष्टव्यक्तिबोधकत्वस्वीकारपक्षेऽपि गोत्वस्य व्यक्त्या सह सम्बन्धः समवायः गोपदादेव उपस्थितः इति वक्तव्यम्~। तच्च सूत्रकृतानुक्तमिति न्यूनता स्यात्~। तस्मात् सौत्रमाकृतिपदं न अवयवसंस्थाविशेषपरम्~, अपि तु जातिव्यक्त्योः संसर्गविशेषपरम्~। तस्मात् सूत्रेणापि आकृतेः पदशक्यत्वं‌ न सिध्यति इति~। तदुक्तं शब्दशक्तिप्रकाशिकायाम्~- 

		{\fontsize{11.7}{0}\selectfont\s  सौत्रमाकृतिपदं न संस्थानपरम्~, परन्तु करणव्युत्पत्त्या आकारनिरूपकार्थकं जातिव्यक्त्योः संसर्गपरमेव, अन्यथा समवायादेरपि सम्बन्धविधया गवादिपदशक्यत्वेन तदनुक्त्या मुनेर्न्यूनत्वापत्तेः~। कादाचित्कस्तु जातिसंस्थानाभ्यां गवादेरवगमो गवादिशब्दस्य जात्याकृतिविशिष्टे शक्तिभ्रमेण लक्षणया वा सम्पाद्य इति~।\footnote{श.प्र. ५९}}

		\subsection{आकृतेः पदशक्यत्वम्} 

		जात्याकृतिव्यक्तिषु पदानामेका शक्तिः‌~। व्यक्त्याकृतिं विना‌ जात्युपस्थित्यसम्भवात् पदार्थोपस्थितौ जातिभानस्य नियतत्वाच्च आकृतिरपि पदार्थ‌ इति~। आकृतिश्च सत्त्वावयवानां तदवयवानाञ्च नियतसंयोगः~। तज्ज्ञाते तद्गतजातिरपि ज्ञायते~। तस्मादाकृतिरपि पदार्थः~। 
		
		ननु 'पिष्टकमय्यो गावः' इत्यादौ यत्र वाक्येनैव अवयवसंयोगविशेषः आकृतिर्ज्ञायते, सा च गोशब्दोपस्थाप्याकृतिः न, अपि तु पिष्टकसंयोगविशेषः, तथाचान्वयानुपपत्त्या शाब्दबोधानुपपत्तिरिति चेत् तादृशस्थलेषु गोपदस्य लक्षणया गवाकृतिसदृशाकृत्युपस्थापकत्वं कल्प्यते~। यत्र योग्यतादिशाब्दसामग्र्यन्तरबाधः तत्र लक्षणाभ्युपगमात्~। किञ्च 'व्यक्ताकृतिजातयस्तु पदार्थः'\footnote{न्या. सू. २.२.६७} इति न्यायसूत्रे पदार्थ इत्येकवचनं श्रूयते~। तेन व्यक्ताकृतिजातिषु एका शक्तिरित्यवगम्यते~। अन्यथा 'प्रत्यक्षानुमानोपमानशब्दाः प्रमाणानि'\footnote{न्या. सू. १.१.३} इतिवद्बहुवचनमेव प्रयुञ्जीत~। तस्मात् व्यक्त्याकृतिजातिषु पदानामेका शक्तिः इति ज्ञापनार्थमेव तत्र एकवचनप्रयोगः इति~। तदुक्तं तत्त्वचिन्तामणौ~-

		{\fontsize{11.7}{0}\selectfont\s जातिविशेषवदवयवसंयोगरूपाकृतिरपि पदशक्या गोपदात् जात्याकृतिविशिष्टस्यैवानुभवात्~। 'पिष्टकमय्यो गाव' इत्यादौ गवाकृतिसदृशाकृतौ लक्षणा पिष्टकसंयोगविशेषस्याशक्यत्वात्~। जात्याकृतिव्यक्तीनां प्रत्येकमात्रपरत्वे लक्षणैव~। प्रत्येकस्य जात्याकृतिविशिष्टादन्यत्वात्~। यथा गुरूणां कार्यशक्ताया लिङो लोके कार्यत्वपरत्वे~। अत एव ’व्यक्त्याकृतिजातयस्तु पदार्थ’ इति पारमर्षसूत्रम्~। एकयैव शक्त्या एकवित्तिवेद्यत्वसूचनाय पदार्थ इत्येकवचनम्~।\footnote{त.म. }}

		\subsection{व्यक्त्याकृतिविशिष्टजातिः पदशक्या}

		पदस्य जातावेव शक्तिरिति केचित्~। तन्न~। गवादिपदानां प्रयोगे सति स्वरसतया व्यक्तिविषयकोपस्थितिसम्भवात्~। यथा 'गङ्गायां घोष' इति वाक्यश्रवणे सति गङ्गापदात् तत्सम्बन्धिनः तीरस्य न स्वरसतया उपस्थितिः, तदर्थं व्यापारान्तरापेक्षा दृष्टा, तद्वन्न 'गामानय' इति वाक्यश्रवणात् गवादिपदानां गोत्वविशिष्टव्यक्त्युपस्थापकत्वे व्यापारान्तरापेक्षा~। तस्माज्जातिविशिष्टव्यक्तिरपि पदशक्या एव~।

		तत्र गौतमीयसूत्रमपि प्रमाणम्~। सूत्रे तावत् व्यक्त्याकृतिजातीनां पदार्थत्वमङ्गीकृतम्~। व्यक्त्याकृतिजातयस्तु इत्यत्र तु शब्दः विशेषणार्थः~। तथा हि~- किं पदानां जात्याकृतिविशिष्टव्यक्तिवाचकत्वमुत व्यक्त्याकृतिविशिष्टजातिवाचकत्वमाहोस्वित् जातिव्यक्तिविशिष्टाकृतिवाचकत्वम् इति विशेष्यविशेषणभावे विनिगमनाविरहात् सर्वमपि स्थलानुरोधेन स्वीकार्यमित्यर्थः~। तदुक्तं न्यायमञ्जर्याम्~-

		{\fontsize{11.7}{0}\selectfont\s \begin{center}यथा विध्यन्तपर्यन्तो वाक्यव्यापार इष्यते~।\\ तथैव व्यक्तिपर्यन्तः पदव्यापार इष्यताम्~॥\end{center}}

		{\fontsize{11.7}{0}\selectfont\s \begin{center}अनवरतव्यापारे शब्दे तदवगमात्~।\\ येनान्विताभिधानञ्च पदानामभ्युपेयते~।\\ सुतरां तेन वक्तव्या व्यक्त्यन्ता पदतो मतिः~॥\\ न हि व्यक्त्यनपेक्षाणां जातीनामितरेतरम्~।\\ अन्वयोऽनन्वितानाञ्च नाभिधानमिति स्थितिः~॥\\ गङ्गायां घोष इत्यादौ यथा सामीप्यलक्षणा~।\\ नैवं गौः शुक्ल इत्यादौ गम्यते व्यक्तिलक्षणा~॥\\ प्रयोगप्रतिपत्तिभ्यां प्रयोगोऽध्यवसीयते~।\\ तस्माद्गवादिशब्दानां तद्वानर्थ इति स्थितम्~॥\end{center}}

		{\fontsize{11.7}{0}\selectfont\s तदिदमुक्तं सूत्रकृता 'व्यक्ताकृतिजातयस्तु पदार्थः' इति, तुशब्दो विशेषणार्थः~। किं विशिष्यते~? गुणप्रधानभावस्य नियमेन शब्दार्थत्वम्~, स्थितेऽपि तद्वतो वाच्यत्वे क्वचित्प्रयोगे जातेः प्राधान्यं व्यक्तेरङ्गभावः यथा गौर्न पदा स्पृष्टव्येति सर्वगवीषु प्रतिषेधो गम्यते, क्वचिद्व्यक्तेः प्राधान्यं जातेरङ्गभावः यथा 'गां मुञ्च', 'गां बधान' इति नियतां काञ्चिद्व्यक्तिमुद्दिश्य प्रयुज्यते, क्वचिदाकृतेः प्राधान्यं व्यक्तेरङ्गभावो जातिर्नास्त्येव यथा पिष्टकमय्यो गावः क्रियन्तामिति सन्निवेषचिकीर्षया प्रयोग इति सर्वगतत्वेऽपि जातेः न मृद्गवादौ वृत्तिरित्युक्तम्~, तदेवं गवाश्वादिशब्दानां तावत्तद्वानर्थ इति सिद्धम्~।\footnote{न्या. मं. २९७}}


		\subsection{जात्याकृतिविशिष्टव्यक्तौ एकैवशक्तिः}

		गवादिपदश्रवणे सति यथा गोत्वविशिष्टगोव्यक्तेर्भानं तद्वत् अवयवसंस्थानविशेषस्यापि भानमित्यनुभवः~। तथा च अवयवसंस्थानविशेषरूपाकृतेः वाच्येऽर्थे विशेषणतया अनुभवान्यथानुपपत्त्या तत्रापि गोपदशक्तिरङ्गीकर्तव्या~। आकृतेः गोपदवाच्यत्वेऽपि न प्रवृत्तिनिमित्तता~। गोपदवाच्येऽर्थे साक्षाद्विशेषणस्यैव प्रवृत्तिनिमित्तत्वाभ्युपगमात्~। आकृतेस्तु समवायद्वयघटितसामानाधिकरण्यसम्बन्धेनैव गवादिव्यक्तौ विशेषणत्वात्~। तदपेक्षया जातेरेव साक्षात्सम्बन्धेन विशेषणत्वात् गोपदप्रवृत्तिनिमित्तत्वमिति~।

		ननु व्यक्त्याकृतिजातिषु किमेका शक्तिः, उत नाना~? आद्ये कदाचिज्जातिमात्रस्योपस्थितिः कदाचिद्व्यक्तिमात्रस्येति अव्यवस्था स्यात्~। द्वितीये तु किं जातिविशिष्टव्यक्तौ उत व्यक्तिविशिष्टजाताविति विनिगमनाविरहः~। तथा च अनेकशक्तिकल्पनारूपं गौरवमिति चेन्न~। गवादिपदानां जात्याकृत्योः एकतरविरहे अपरभानासम्भवात् लाघवाच्च उभयविशिष्टे गवादावेकैव शक्तिः कल्प्यते~। विशेषणभेदेन न शक्तिभेदः, पुष्पवन्तधेन्वादिपदवत् नानार्थेषु एकशक्त्यभ्युपगमात्~। यत्र तु केवलजातिविशिष्टे गवादिपदतात्पर्यं 'पिष्टकमय्यो गाव' इत्यादौ तत्र तु केवलजातिविशिष्टे गवि लक्षणैव~। एतेन 'व्यक्त्याकृतिजातयस्तु पदार्थः' इति सूत्रे पदार्थपदोत्तरैकवचनमप्युपपद्यते~। तदुक्तं शक्तिवादे~- 

		{\fontsize{11.7}{0}\selectfont\s  गवादिशब्दात् संस्थानरूपाकृतेरपि बोधस्यानुभविकतया गोत्वादिजातिवत् सापि तद्वाच्ये विशेषणम्~, तस्याश्च वाच्यविशेषणत्वेऽपि न प्रवृत्तिनिमित्तता~। साक्षात्सम्बन्धेन वाच्यवृत्तित्वाभावात्~, अवयवसंयोगरूपायास्तस्याः सामानाधिकरण्येनैव गवादौ सत्त्वात्~, अत्र शक्त्या जात्याकृत्योरेकतरविनिर्मोकेणापरभानविरहात् लाघवाच्चोभयविशिष्टे गवादिपदस्यैकैवशक्तिः स्वीक्रियते, शक्यविशेषणभेदेऽपि पुष्पवन्तादिपदवत् धेन्वादिपदवच्च शक्त्यैक्यस्य दुरुपवादत्वात्~। एकविशिष्टापरावच्छिन्ने शक्तिरिति तु न सत्~। विशेष्यविशेषणभावे विनिगमनाविरहात् गवाद्यंशे साक्षादुभयप्रकारकबोधस्यानुभवसिद्धस्य दुरुपपादत्वाच्च~। तत्र केवलाकृतिविशिष्टे गवादिपदतात्पर्यं यथा 'पिष्टकमय्यो गाव' इत्यादौ तत्र शुद्धगोत्वाद्यवच्छिन्नपरे धेन्वादिपदवत् लक्षणैव, जात्याकृतिविशिष्टायां व्यक्तौ शक्तेरैक्यं 'जात्याकृतिव्यक्तयस्तु पदार्थ' इति न्यायसूत्रे बहुवचनमुपेक्ष्य पदार्थ इत्येकवचनान्तं निर्दिष्टवतो महर्षेरप्यनुमतम्~।\footnote{श. वा. १७१,१७२}}


	\section{विमर्शः}
	
		\subsection{आकृतेः पदशक्यत्वे दोषाः}
		
		आकृतौ शक्तिरिति मतं न युज्यते~। अवयवसंस्थानरूपस्याः अकृतेः प्रतिव्यक्तिभिन्नत्वात्  तस्याः आनन्त्यव्यभिचाराभ्यां संबन्धज्ञानासम्भवात्~। तथा च  नियतशाबलेयसन्निवेशः गोशब्दवाच्य इति वक्तुं नैव शक्यते~। बाहुलेयसंनिवेशे  तादृशसंनिवेशाभावेऽपि  गोशब्दप्रयोगस्य  दर्शनात्  तत्तु व्यभिचरितम्~।  न हि गोशब्दस्य  त्रैलोक्यान्तर्गतसकलगोपिणडगतसन्निवेशवाचकत्वं  भवितुमर्हति~। आनन्त्यात्  सकलसंनिवेशेषु  संबन्धग्रहासम्भवात्~।  तथा च  नाकृतिः शब्दार्थः~, प्रेषणादिक्रियान्वयस्य  आकृतौ अनुपपत्तेः~। न हि यः कश्चित् ”गामानय” इत्यादिवाक्यं श्रुत्वा  मृत्स्नामयीं गामानयति~। तत्र हि गवाकृतिरस्त्येव~। अपि तु आकृतिविशिष्टामेव व्यक्तिमानयति~। ननु तर्हि  जातिरेव पदार्थः इति मतेऽपि  गोत्वादिजातेः व्यापकत्वात् गामानय इत्युक्ते किमर्थं मृद्गवकं  न आनीयते~? इति चेत्  तर्हि हस्ती कुतो नानीयते~। तत्र गोत्वादिजातेः  सर्वत्रव्यापकत्वेऽपि  व्यञ्जकव्यक्तिनियमात्~, या व्यक्तिः गोत्वजातेः व्यञ्जिका सा एव आनीयते इति चेत् उच्यते~,  हन्त तर्हि सास्नादिमत्प्राणी  गोत्वजातेरभिव्यञ्जक  इति सास्नादिमत्प्राण्येव आनीयते  न  मृद्गव  इति नातिप्रसङ्गः~। आकृतेश्च संनिवेशरूपत्वेन  गोशब्दवाच्यत्वे  तु  मृद्गवेऽपि तत्सद्भावात्  'गौरानीयताम्' इत्युक्ते  मृद्गवानयनप्रसङ्गः  वारयितुं न शक्यते~।

		किञ्च यदि गोशब्दः आकृतिवाचकः स्यात्  शुक्लादिगुणवाचिभिः  पदान्तरैः सह गोशब्दस्य   ”शुक्ला गौश्चरति” इत्यादिषु यत् सामानाधिकरण्यं वर्तते तत् न सङ्गच्छेत~।  आकृतौ गुणाद्याश्रयत्वाभावात्  व्यक्तावेव तत्सम्भवात्~। अतः व्यक्तावेव शब्दार्थत्वकल्पनम्  वरम्~।  क्रियान्वयस्तु व्यक्तौ सम्भवति~। तथा च जातौ सम्बन्धग्रहः सुलभः~। आकृतेश्च पदार्थत्वे तु क्रियान्वयसम्बन्धग्रहयोः तत्र असम्भवात्  नाकृतिः पदार्थः इति~। मैवम्~।

		\subsection{आकृतेः पदशक्यत्वसमर्थनम्}

		ननु गवादिपदानां श्रवणे सति गोत्वविशिष्टगोव्यक्त्यादेरेव स्मरणात् अवयवसंस्थाविशेषस्य न गवादिपदशक्यत्वम्~। न हि व्यक्त्याकृतीत्यादिसूत्रविरोधः~। गोत्वस्य व्यक्त्या सह सम्बन्धोऽपि शक्त्या एव स्मर्यतेत्यतः तदनभिधानप्रयुक्ता न्यूनता स्यात्~। तस्मात् आकृतिपदस्य जातिसम्बन्धार्थकत्वमेव~। किञ्च सर्वत्र अवयविरूपव्यक्तेरुपस्थित्यनन्तरं अवयवसंयोगविशेषरूपाकृत्युपस्थितेः प्रयोजनाभावाच्च न तस्य पदशक्यत्वमिति चेन्न~।

		गोपदात् जात्याकृतिविशिष्टस्यैव अनुभवात् आकृतावपि शक्तिः स्वीक्रियते~। आकृतिश्च अवयवसंयोगः~।  तथा च गोपदस्य अवयवसंयोग~- गोत्वजाति~- एतदुभयविशिष्टव्यक्तौ शक्तिरिति फलितम्~।
				
		अत्र चिन्त्यते – व्यक्तिवत् अननुगतानामवयवसंयोगरूपाकृतीनां शक्यत्वे शक्यानन्त्यं दुर्वारम्~। न च जातिविशेषवान् यो अवयवसंयोगः संस्थानविशेषः तद्रूपाकृतीनां शक्यत्वम्~। तथा च जातिविशेषस्यैव अनुगमकतया नोक्तदोष इति वाच्यम्~। अन्यतरकर्मजत्वादिना सङ्करेण तादृशजात्यप्रसिद्धेः~। न च उपाधिरेवानुगतः सुलभः इति वाच्यम्~। तादृशोपाधेः अनिरुक्तेः~। न च ‘जातिविशेषवदवयवसंयोगः’ इति मणौ जातिविशेषवत्त्वं अवयवविशेषणम्~। तथा च कपालसंयोगादिकमेव तथेति वाच्यम्~। गवादिपदे सास्नाद्यवयवस्याप्यननुगतत्वेन दोषतादवस्थ्यात्~। 

		अत्रोच्यते – जातिविशेषवतः गवादेः अवयवसंयोगस्य आकृतेः शक्यत्वम्~। विशेषपदोपादानेन पृथिवीपदादौ नाकृतिः शक्या इति लब्धम्~। तथा च गवावयवसंयोगत्वादिकमेव अनुगतम् इति न शक्त्यानन्त्यम्~। न चैवं सति तत्प्रकारकधीः स्यादिति वाच्यम्~। इष्टापत्तेः~। गवादौ आकृतिवैशिष्ट्यञ्च परम्परासम्बन्धेन~। ‘पिष्टकमय्यो गावः’ इत्यत्र तु गोपदस्य गवाकृतिसदृशाकृतौ लक्षणा~। पिष्टकसंयोगविशेषस्याशक्यत्वात्~। समुदायशक्तस्य पदस्य प्रत्येके लाक्षणिकत्वात्~। न च ‘गौरुत्पन्नः’ इत्यत्र व्यक्तिमात्रपरत्वेऽपि न लक्षणेति वाच्यम्~। तत्र व्यक्तिमात्रपरत्वाभावात्~। तद्व्यक्तित्वादिना तदुपस्थितौ यत्र तात्पर्यं तत्र लक्षणा इष्यत एव~।  एवं जात्याकृतिव्यक्तीनां प्रत्येकमात्रस्य बोधनेऽपि लक्षणा~।  यथा ”गौर्नित्या” इत्यत्र जातिमात्रपरत्वं गोपदस्य~।  यथा च गुरुमते कार्यशक्तायाः लिङः लोके  कार्यत्वमात्रपरत्वे~।  समुदायशक्तस्य पदस्य प्रत्येकस्मिन् लाक्षणिकत्वमिति नियमात्~।

		जात्याकृतिव्यक्तिषु मिलितासु एकैव शक्तिः सूत्रकृता ”व्यक्त्याकृतिजातयस्तु पदार्थः” इत्येकवचनप्रयोगेण सूच्यते~।  यत्पदात् नियमतो यत् प्रतीयते तस्य तत्पदशक्यत्वम् इति नियमात् गोपदात् जात्याकृतिव्यक्तीनां तिसृणामपि प्रतीतेः तत्त्रितयस्य गोपदशक्यत्वम् आवश्यकम्~। 

		गवादिव्यक्त्या सह गोत्वादिजातीनां यः सम्बन्धः तस्यापि पदशक्यत्वात् सूत्रकृतस्तदकथनप्रयुक्ता न्यूनता या उक्ता सापि नैव सङ्गच्छते~। संसर्गातिरिक्तविषयस्यैव सूत्रेणोक्तत्वात्~। विशेषणविशेष्ययोः ज्ञाने सति तत्सम्बन्धस्य सुलभतया ज्ञातुं शक्यत्वात्~।

		\subsection{आकृतेः पदशक्यत्वे महाभाष्यकारसम्मतिः}

		”सरूपाणामेकशेष एकविभक्तौ ” इति सूत्रस्य विचारणावसरे भगवता पतञ्जलिना पदार्थविचारावसरे द्रव्यं वा आकृतिर्वा पदार्थः इति  विचारितः~। तत्रादौ वाजप्यायनमतानुसारेण  आकृतिः पदार्थः इति उपक्षिप्तम्~।~- ”आकृत्यभिधानाद्वा एकं विभक्तौ वाजप्यायनः” तत्र हेतवः- प्रख्या विशेषात् आकृतेः पदार्थत्वम्~। प्रख्या नाम बुद्धिः, तस्याः बुद्धेः अविशेषात् समानरूपत्वात्  तद्विषयस्यापि ऐक्यं  प्रतीयते~। तत्र एकत्वं प्रतीयमानं कथमिति चेत्-  गुणप्रमाणादिभिन्नेष्वपि  गोपिण्डेषु  गौर्गौरिति एकाकारप्रतीतिः जायते~। अतः अवश्यमेकेनालम्बनेन सामान्येन  भाव्यमिति जातिसद्भावः एकत्वं चावसीयते~। यद्यपि एका आकृतिः प्रख्याविशेषात् ज्ञायते इति~। 
			
		कुतस्त्वेतत् साभिधीयत इति, तत्राह – अव्यपवर्गगतेश्च~। अव्यपवर्गो नाम अभेदः~। न हि गौरित्युक्ते व्यपवर्गो गम्यते~, शुक्ला, नीला, कपिलेति~। अव्यपवर्गगतेश्च मन्यामहे आकृतिरभिधीयत इति~। ज्ञायते चैकोपदिष्टम्~। गौरस्य कदाचिदुपदिष्टो भवति~। स तमन्यस्मिन् देशे अन्यस्मिन्  काले अन्यस्यां च वयोवस्थायां  दृष्ट्वा जानाति  अयं गौरिति~। तथा च देशकालावस्थापिण्डान्तरेष्वबाधितप्रत्यभिज्ञाप्रत्ययोदयान्यथानुपपत्त्या  सामान्यसद्भावो अनुमीयते~। धर्मशास्त्रं च तथा एवं च कृत्वा (आकृतेः पदार्थत्वमित्यभिप्रेत्य) धर्मशास्त्रं प्रवृत्तम्- ”ब्राह्मणो न हन्तव्यः”~, ”सुरा न पेया” इति~।  ब्राह्मणमात्रं च  न हन्यते~। सुरामात्रं च न पीयते~। यदि द्रव्यं पदार्थः  स्यात्~, एकं ब्राह्मणमहत्वा  एकां च सुरामपीत्वा अन्यत्र  कामचारः स्यात्~। नन्वेकमनेकस्थं कथं भवति~।  न हि देवदत्तो  युगपत् मथुरायां  स्रुघ्ने च भवतीत्यत आह~- अस्ति चैकमनेकाधिकरणस्थं  युगपत् तद्यथा  एक  आदित्योऽनेकाधिकरणस्थो  युगपदुपलभ्यते~। इतीन्द्रवद्विषयः तद्यथा एक इन्द्रोऽनेकस्मिन क्रतुशते आहूतो युगपत्  सर्वत्र  भवति~। एवमाकृतिर्युगपत्  सर्वत्र भविष्यति~। अवश्यं चैतदेवं विज्ञेयमाकृतिरभिधीयत इति – द्रव्याभिधाने ह्याकृत्यसम्प्रत्ययः  तत्रासर्वद्रव्यगतिः,  प्राप्नोति~। असर्वगतौ को दोषः~? ”गौरनुबन्ध्यः, अजोऽग्नीषोमीयः” इति एकः शास्त्रार्थं कुर्वीत~, अपरोऽशास्त्रोक्तम्~। अशास्त्रोक्ते च क्रियमाणे  विगुणं कर्म भवति~। विगुणे च कर्मणि फलानवाप्तिः~। अयमाशयः- प्रत्यर्थं शब्दनिवेशः इति न्यायात् एकः शब्दः एकं गोरूपं द्रव्यं बोधयेत्~। तस्य केनचिदालम्भे कृते~, परैस्तेन वा गवान्तरालम्भे अशास्त्रार्थः स्यात्~। एवं न ब्राह्मणमित्यादौ अपि एकस्य ब्राह्मणस्यावधो विषयः, तत्रान्यविषये सर्वेषां कामचारेण वधप्रवृत्तावपि न प्रायश्चित्तं स्यात् इति~। सर्वपुरुषान् प्रति एकव्यक्तेरेव बोधनमित्यभिधानः~। चोदनायां चैकस्योपाधिवृत्तेः अष्टाकपालत्वादिरुपाधिः तस्य वृत्तिः प्रवृत्तिः~।(आग्नेयमष्टाकपालं निर्वपति~) 
			
		एकं निरुप्य द्वीतीयस्तृतीयश्च निरूप्यते~। यदि च द्रव्यं पदार्थः स्यात् एकं निरूप्य द्वीतीयस्य  निर्वपणं न प्रकल्पेत~। एतदुक्तं भवति – यदि आग्नेयादिशब्देन सामान्यमभिधीयते  तदा तेनाशेषद्रव्याणामाक्षेपात् प्रतिद्रव्यमष्टाकपालत्वमुपाधिः युज्यते~। द्रव्ये तु पदार्थे  एकेनैव यजमानेन सकृदष्टाकपालः पुरोडाशो निरूप्येत  नान्येन~, नापि  तेनान्यदित्यर्थः इति~। तदुक्तं महाभाष्ये~-

		किं पुनाराकृतिः पदार्थः आहोस्विद्द्रव्यम्~। उभयमित्याह~। कथं ज्ञायते~। उभयथा ह्याचार्येण सूत्राणि पठितानि~। आकृतिं पदार्थं मत्वा 'जात्याख्यायामेकस्मिन् बहुवचनमन्यतरस्याम्'\footnote{व्या. सू. १.२.५८} इत्युच्यते~। द्रव्यं पदार्थं मत्वा 'सरूपाणाम्'\footnote{१.२.६४} इत्येकशेष आरभ्यते~।\footnote{व्या. म. } इत्यादिना~।
		
		
\begin{center}\begin{small}॥ इति प्राचीनन्याय-वैशेषिक-नव्यन्यायशास्त्रेषु तत्तद्व्याख्याकाराणां सैद्धान्तिकमतभेदानां विमर्शात्मकमध्ययनमिति प्रबन्धे  प्रमाणप्रपञ्चे सैद्धान्तिकमतभेदाः इत्याख्यः प्रथमोऽध्यायः~॥\end{small}\end{center}


\end{document}


\titleformat {\chapter}[display]{\normalfont\Large} % format
{अथ द्वितीयोऽध्यायः\\[1mm]} % label
{-3.8ex}{ \rule{\textwidth}{1pt}\vspace{-5ex}
\centering
} % before-code
[
\vspace{-6.7ex}%
\rule{\textwidth}{1pt}
]
\titlespacing*{\chapter} {10pt}{-60pt}{50pt}



\chapter{द्रव्यप्रपञ्चे सैद्धान्तिकमतभेदाः}

	प्रमाणेषु मतभेदनिरूपणानन्तरं काणादाभिमतद्रव्यादिप्रमेयेषु मतभेदाः निरूप्यन्ते~। पृथिव्यादिभेदेन द्रव्यं नवधा विभक्तम्~। तत्र पृथिव्यप्तेजो भिन्नस्य क्रियावद्भूतद्रव्यस्य अस्तित्वे प्रमाणं प्रदर्शयन्ति काणादाः स्पर्शानुमेयो वायुः इत्यादिना~। तथा च रूपरहितस्य वायोः इन्द्रियागोचरत्वात् न तत्र प्रत्यक्षं प्रमाणमिति ज्ञायते~। वृक्षादीनां कम्पनहेतुत्वेन पृथिवीवृत्तिस्पर्शविलक्षणस्पर्शाश्रयत्वेन च किञ्चित्क्रियाविशिष्टं द्रव्यं कल्प्यते~। सैव वायुरित्युच्यते इति~। अत्र बहिरिन्द्रियद्रव्यप्रत्यक्षं प्रति रूपस्य हेतुत्वस्वीकारात् वायौ च रूपाभावात् तस्य न प्रत्यक्षत्वम् सम्भवति इति प्राचीनानामाशयः~। द्रव्यस्य त्वाचप्रत्यक्षं प्रति रूपस्याप्रयोजकत्वात् स्पर्शस्य च हेतुत्वात् वायौ च स्पर्शसद्भावात् वायुः त्वाचप्रत्यक्ष इति दीधितिकारप्रभृतीनां नव्यानामाशयः~। 

	\section{वायोः त्वाचत्वविषये मतवैलक्षण्यम्}

	वायुगतसङ्ख्याद्यग्रहात् न तस्य त्वाचत्वमिति काणादाः~। तथा हि~-  यत्र द्रव्यगतसङ्ख्यापरिमाणादीनां सामान्यगुणानामिन्द्रिययोग्यत्वं सम्भवति तद्द्रवस्यापि तदिन्द्रियग्राह्यत्वमिति अभ्युपेयम्~।‌ यथा घटगतसङ्ख्यादीनां चक्षुषा त्वचा च ग्रहणसम्भवात् तस्य चाक्षुषत्वं त्वाचत्वं च सिध्यति~। वायुगतसङ्ख्यादीनां ग्रहणन्तु न केनापि इन्द्रियेण सम्भवति~। अतः वायुरप्रत्यक्ष एव~। तत्सद्भावे स्पर्शशब्दादिलिङ्गकमनुमानमेव प्रमाणमित्याशयः~।

		\subsection{वायुविषयकं ज्ञानमनुमानात्मकम्}

		 ननु 'वायुर्वाति', 'शीतो वायुः' इताद्याः प्रीतीतयः वायुसन्निकर्षे सति जायन्ते खलु इति चेन्न~। वायुना सह त्वक्सन्निकर्षे सति वायुगतस्पर्श एव अनुभूयते~। सैव प्रत्यक्षविषयः, नान्यः~। यदपि 'वायुर्वाति' इत्यद्याः प्रतीतयः तदप्यनुमानादेव~। न च व्यप्तिस्मरणपरामर्शादिकं तत्र नानुभूयते खलु इति वाच्यम्~। वृक्षादिषु विद्यमानक्रियायाः साक्षात्कारे सति यथा झटिति वायुविषयकप्रतीतिरुदेति तथैवात्रापि व्यप्तिस्मरणादीनामभावेऽपि साक्षाल्लिङ्गज्ञानेनैवानुमितिरुदेति अभ्यासातिशयात्~। 
		 
		 न च यो मूर्तद्रव्यवृत्तिर्गुणः येनेन्द्रियेण गृह्यते तेनेन्द्रियेण तदधिकरणमपि गृह्यते इति वाच्यम्~। ग्रीष्मोष्मेषु व्यभिचातात्~। एवं यत्र यत्र द्रव्यचाक्षुषत्वं तत्रैव स्पार्शनत्वस्यापि दर्शनात् द्रव्यस्पार्शनत्वं चाक्षुषत्वेन व्याप्तमेव~। तस्मात् वायुरप्रत्यक्षैव~।
		 
		{\fontsize{11.7}{0}\selectfont\s  उपलभ्यमानस्पर्शाधिष्ठानभूत आश्रयो यः स विषय इति~। किमस्यास्तित्वे प्रमाणम्~? प्रत्यक्षमेव~, त्वगिन्द्रियव्यापारेण वायुर्वातीत्यपरोक्षज्ञानोत्पत्तेरिति कश्चित्~, तन्न युक्तम्~। स्पर्शव्यतिरिक्तस्य वस्त्वन्तरस्यासंवेदनात्~। अपरोक्षज्ञाने तु स्पर्श एव प्रतिभाति नान्यत्~। यदपि वायुर्वातीति ज्ञानं तदभ्यासपाटवातिशयाद् व्याप्तिस्मरणाद्यनपेक्षं स्पर्शेनानुमानम्~, चक्षुषेव वृक्षादिगतिक्रियोपलम्भात्~। शीतोष्णस्पर्शभेदप्रतीतौ वायुप्रत्यभिज्ञानमपि तदाश्रयोपनायकद्रव्यानुमानादेव~। त्वगिन्द्रियेण तु शीतोष्णस्पर्शाभ्यामन्यस्य न प्रतिभासोऽस्ति~। स्पार्शनप्रत्यक्षो वायुरुपलभ्यमानस्पर्शाधिष्ठानत्वाद् घटवदित्यनुमानं शशादिषु पशुत्वेन शृङ्गानुमानवदनुपलब्धिबाधितम्~। द्रव्यस्य स्पार्शनत्वं चाक्षुषत्वेन व्याप्तमवगतं घटादिषु चाक्षुषत्वस्य च वायावभावस्तेनात्र शक्यं स्पार्शनत्वनिवृत्यनुमानमेतत्~, अतस्तस्याप्रत्यक्षस्य सद्भावेऽनुमानम्~।\footnote{न्या.कं. १२७, १२८}}

		\subsection{वायोः त्वाचत्वे संख्यापरिमाणादिग्रहणप्रसङ्गः}

		एवं वायोः त्वाचत्वे तद्गतसङ्ख्यादीनां प्रत्यक्षत्वप्रसङ्गः~। क्वचित् सङ्ख्याग्रहणात् इष्टापत्तिरिति न वक्तव्यम्~। फूत्कारादौ सङ्ख्याग्रहस्यानुमानिकत्वादिति~। न च यज्जातीयद्रव्यं बहिरिन्द्रियेण गृह्यते तज्जातीयद्रव्यगतसङ्ख्यादयः तेन इन्द्रियेण गृह्यन्ते इति, यत्र यद्व्यक्तेः साक्षात्कारत्वं तत्र तद्व्यक्तिगतसङ्ख्यादीनामपि साक्षात्कारत्वमिति वा नियमः स्वीकार्यः~। आद्ये कदाचित् वायावपि 'एको वायुः' इत्यादिप्रतीतिबलात् सङ्ख्याग्रहणसम्भवात् वायोरपि प्रत्यक्षत्वे न किञ्चिद्बाधकम्~, द्वितीये तु पृष्टे संलग्नस्य वस्त्रादेः प्रत्यक्षत्त्वेऽपि तद्गतसङ्ख्यादीनामग्रहात् नियमोऽयं व्यभिचरित इति वाच्यम्~। वायौ सङ्ख्याग्रहस्तु न प्रत्यक्षादपि तु  व्याप्तिस्मरणाद्यनमपेक्षादनुमानादेव~। द्वितीये तु पृष्टे संलग्नस्य वस्त्रादेः‌ सङ्ख्याग्रहेऽपि तस्यैव चक्षुषा ग्रहणे तद्गतसङ्ख्याग्रहो भवत्येव~।‌ किञ्च सर्वत्र द्रव्यप्रत्यक्षे सङ्ख्यायाः अविषयत्वात् याव्यक्तिः प्रत्यक्षा तदीयसङ्ख्यादिकं गृह्यते इति नियमः स्वीकार्यः~। वायौ कदापि सङ्ख्यायाः अग्रहणात् वायुरप्रत्यक्षैव~।

		{\fontsize{11.7}{0}\selectfont\s किञ्च वायोः प्रत्यक्षत्वे तद्गतसंख्यापरिमाणादिग्रहणप्रसङ्गः~। ननु यज्जातीयं प्रत्यक्षं तज्जातीयस्य सङ्ख्यादिकं प्रत्यक्षमिति व्याप्तिश्चेदभिमता तदा फूत्कारे सङ्ख्यादिग्रहोऽस्त्येव~। अथ या व्यक्तिः प्रत्यक्षा तदीयं सङ्ख्यादिकं गृह्यत एवेति नियमस्तदा पृष्ठलग्नवस्त्रादौ व्यभिचार इति चेन्न फूत्कारादपि सङ्ख्याप्रतीतेरानुमानिकत्वात् पृष्ठलग्नवस्त्रादेरपि करपरामर्शेण सङ्ख्यापरिमाणादेर्ग्रहणात्~। द्वितीयनियमेऽपि न व्यभिचारः~, चक्षुषा वस्त्रसङ्ख्यादिग्रहणात्~। न च ह्येतादृशो नियमो  यत् प्रत्यक्षं तत्सङ्ख्यादिविषयकमेव किं तर्हि या व्यक्तिः प्रत्यक्षा तदीयसङ्ख्यादिकं गृह्यत एवेति~। वायौ च तथा नास्तीति वायुरप्रत्यक्ष एवेति~।\footnote{क.र.}}

		\subsection{बहिर्द्रव्यप्रत्यक्षं प्रति उद्भूतरूपवत्वम् उद्भूतस्पर्शवत्वञ्च मिलितं तन्त्रम्}

		बहिर्द्रव्यप्रत्यक्षं प्रति उद्भूतरूपं तन्त्रमिति कार्यकारणभावस्वीकारादेव वायोः त्वाचत्वे विवादो दृश्यते~। किन्तु आचार्यास्तु द्रव्यप्रत्यक्षं प्रति उद्भूतरूपस्पर्शयोः द्वयोरेव कारणत्वमभ्युपगच्छन्ति~। अत एव तेषां मते उद्भूतस्पर्शाभावान्न प्रभादिकं प्रत्यक्षत्वमिति~।

		{\fontsize{11.7}{0}\selectfont\s तथा हि~- गुणिधर्मो वासौ स्यात्~, गुणधर्मो वा स्यात्~? तत्र गुणिधर्मत्वे तब्दले रूपस्पर्शयोरविशेषेण ग्रहणम्~, न तु क्वचिद्ग्रहणमग्रहणं च क्वचिदिति~। तदेकार्थसमवेतयोर्ग्रहणमेवेति चेन्न~, निशि निदाघसमये तेजःस्पर्शग्रहेऽपि रूपाग्रहात्~, चन्द्रातपरूपग्रहेऽपि स्पर्शाग्रहात्~, गुणधर्मस्त्वनित्यो न भवत्येव, नित्यस्तु सामान्यरूपो न संस्कारशब्दवाच्यो नाप्युपलब्धिफलः, तस्य स्वोपलब्धेरन्यत्रानिमित्तत्वादित्यत आह~- धर्माधर्मनिमित्त इति~।\footnote{न्या.वा.ता.प.}}


		\subsection{वायोः प्रत्यक्षत्वसाधकानुमाने उपाधिप्रदर्शनम्}

		वायोः प्रत्यक्षत्वं प्रत्यक्षस्पर्शादिलिङ्गकहेतुना सिध्यति~। न च तत्रोद्भूतरूपमुपाधिः~।  साध्यव्यापकत्वाभावात् इति वाच्यम्~। साधनावच्छिन्नसाध्यवापकत्वस्यापि उपाधिशरीरप्रविष्टत्वादिति~। प्रत्यक्षस्पर्शाश्रयत्वावच्छिन्नप्रत्यक्षत्वस्य आत्मन्यभावात् अन्यत्र सर्वत्र उद्भूतरूपस्य सत्त्वात्~।

		{\fontsize{11.7}{0}\selectfont\s ननु वायुः प्रत्यक्ष एव किन्न स्यात्~। त्वगिन्द्रियव्यापारानन्तरं वायुर्वातीति प्रतीतेः~। तथा च प्रयोगः वायुः प्रत्यक्षः प्रत्यक्षस्पर्शाश्रयत्वात् यदेवं तदेवं यथा घटः इति चेन्न उद्भूतरूपवत्वस्योपाधित्वात्~। आत्मनि साध्याव्यापकमिदमिति चेन्न साधनावच्छिन्नसाध्यव्यापकत्वात्~। तदुक्तम् –\\ \begin{center}वाद्युक्तनियमच्युतोऽपि कथकैरुपाधिरुद्भाव्यः~।\\[-1mm] पर्यवसितं नियमयन् दूषकताबीजसाम्राज्यात्~॥\\ चाक्षुषप्रत्यक्षतायामेव तत्तन्त्रं न तु स्पार्शनप्रत्यक्षतायामपि इति चेन्न सामान्ये बाधकाभावात्~।\footnote{क.र.}\end{center}}

		\subsection{वायोरप्रत्यक्षतामेव ब्रूते मणिकारः}

		नवीननैयायिकेषु मणिकारस्तु वायोरप्रत्यक्षतामेव मनुते~। अत एव सङ्ख्यादीनामग्रहरूपम् एव तत्र प्रधानतया युक्तिं प्रदर्शयति~।

		{\fontsize{11.7}{0}\selectfont\s उच्यते~। द्रव्यस्य स्पार्शनत्वे उद्भूतस्पर्शमात्रं न तन्त्रम्~। निदाघोष्मणि वायूपनीतशीतोष्णद्रव्ये च प्रत्यक्षत्वेन तद्गतसङ्ख्यापरिमाणसंयोगविभागकर्मणां प्रत्यक्षत्वप्रसङ्गात्~। योग्यव्यक्तिवृत्तित्वेन तेषां योग्यतया द्रव्यग्राहकसामग्रीग्राह्यत्वावधारणात्~। न चोष्मादिजातीये दोषाभावेऽपि घटादाविव करपरामर्शे कदाचित् केनापि सङ्ख्या गृह्यते~। तथोद्भूतरूपवत्त्वमात्रस्य तथात्वे चान्द्राद्युद्योतस्य नयनगतपित्तद्रव्यस्य च प्रत्यक्षत्वे तद्गतसङ्ख्याग्रहोऽपि स्यात्~। न च घटादाविव निपुणं निभालयन्तोऽपि तद्गतसङ्ख्याद्वित्वादि हस्तवितस्त्यादिपरिमाणं कर्म वा वीक्षामहे इत्येकैकव्यभिचाराद्विनिगमकाभावात् उभयमपि बहिरिन्द्रियद्रव्यप्रत्यक्षत्वे प्रयोजकमिति वायुरप्रत्यक्षः~। न चैवमपसिद्धान्तः~। पीतः शङ्खः इत्यादौ नयनपित्तपीतिमैव गृह्यते न तु पित्तद्रव्यम्~। विभक्ता(विषक्ता)वयवत्वात् प्रभायामिव तेजस इति टीकाकृतामभिधानादिति नवीनाः \footnote{त.म.}}

		\subsection{द्रव्यस्पार्शनत्वे स्पर्शवत्वमेव प्रयोजकम्}

		त्वगिन्द्रियव्यापारेण घटादीनां प्रत्यक्षं यज्जायते तत्र प्रयोजकं रूपं स्पर्शं वा~। द्रव्यसाक्षात्कारं प्रति यो गुणः प्रयोजकः अवश्यं स गुणः तेनेन्द्रियेण गृह्यते~। यथा आत्मासाक्षात्कारं प्रति ज्ञानादीनां प्रयोजकत्वम्~, तेषाञ्च मानसत्वमिति~। घटादीनां त्वाचप्रत्यक्षत्वे यदि रूपं प्रयोजकं स्यात्तर्हि तस्य त्वचा ग्रहणं स्यात्~। किन्तु नैव गृह्यते~। स्पर्शस्तु गृह्यते~। तस्मात् स्पर्श एव त्वाचप्रत्यक्षप्रयोजको गुणः~। तस्य च वायौ सत्त्वात् वायुः त्वाचप्रत्यक्षविषय इति परेषां नैयायिकानामाशयः~। तदुक्तं पदार्थतत्त्वनिरूपणे~-

		{\fontsize{11.7}{0}\selectfont\s द्रव्यस्पार्शनप्रत्यक्षे स्पर्शवत्वमेव प्रयोजकम्~। अत एव 'शीतो वायुरि'त्यादिप्रत्ययोऽपि स्पार्शनः साधु सङ्गच्छते~। त्रुटेरस्पार्शनत्वे तु प्रकृष्टतमं परिमाणमपि तथा गौरवान्मानाभावात्~।	त्वग्व्यापारानन्तरं वायुर्वातीति सार्वलौकिकप्रत्यक्षस्य अन्यथानुपपत्त्या च रूपं तत्र न निवेशनीयम्~। फूत्कारादौ च स्फुटतरप्रत्यक्षाः सङ्ख्यादयः चाक्षुषद्रव्यप्रत्यक्षे रूपं तथा~। गौरवान्मानाभावाच्च~। निस्पर्शायामपि प्रभायां चलनादिप्रत्ययाच्च  स्पर्शोऽपि न तथा~। द्रव्यस्य समवेतेन्द्रियप्रत्यक्षे तु अनात्मसमवेतशब्दरसगन्धजातीतरयोग्यधर्मसमवायित्वं तथा~। सुखादिसमवायिकारणतावच्छेदकत्वेन सिद्धमात्मत्वं जातिर्नेश्वर इति तदीय ज्ञानादिपिशाचादिसंयोगवारणाय योग्येति~। विषयिधर्मासामानाधिकरणेत्यभिधाने तु शब्दो नोपादेयः~। अमदादिनयनसंसृष्टपित्तद्रव्यस्य परिमाणशून्यत्वं, परिमाणवत्वमते तु तादृशप्रत्यक्षे परिमाणवत्वमेव तथा~। यद्वा नयनसंसृष्टपित्तद्रव्यं निरूपमेव~। अन्यथा पुरुषान्तरेण तत्पीतिमोपलम्भापत्तिः~। स्मर्यमाणस्तु पीतिमा दोशवषाच्छङ्खादावारोप्यत इति~। द्रव्यप्रत्यक्षे च शब्दरसन्धजातीतरयोग्यधर्मसमवायित्वं तथा~। रसनगतञ्च पित्तद्रव्यं न रूपवत् न वा स्पर्शवत् रसना च न द्रव्यग्राहिकेति न तत् प्रत्यक्षम्~।\footnote{प.नि. ७७}}

	\section{विमर्शः}

		\subsection{रूपस्य बहिर्द्रव्यप्रत्यक्षप्रयोजकत्वविमर्शः}

		अरूपिद्रव्यस्य वायोः त्वाचत्वमत्र विमृश्यते~। ग्रीष्मकाले 'वायुर्वाति' इत्यादिप्रतीतयः जायन्ते~। त्वगिन्द्रियव्यापारानन्तरमेव एतासां जाननात् वायुः त्वाचप्रत्यक्षविषयः इति ज्ञायते~। किन्तु पृथिव्यादीनां साक्षात्कारः यत्र यत्र भवति तत्र सर्वत्रापि उद्भूतरूपं वर्तत एव इत्यतः बहिरिन्द्रियद्रव्यप्रत्यक्षे उद्भूतरूपस्य प्रयोजकत्वं क्लृप्तमेव~। वायौ तु तदभावात् तस्य त्वाचत्वं नेष्यते~। 

		ननु तादृशप्रतीतीनां का गतिरिति चेत्~, तत्र अभ्यासातिशयेन विना व्याप्तिस्मरणं पक्षधर्मताज्ञानानपेक्षं च परामर्शात्मकं लिङ्गज्ञानमुदेति~। तेन चानुमितिरिति~। तथा च इन्द्रियव्यापारेण स्पर्शमात्रं गृह्यते, न तु वायुः‌ इति~।

		\subsection{वायोः त्वाचत्वे बाधकानि}

		न च येन द्रव्यग्राहकेन्द्रियेण मूर्तद्रव्यवृत्तिः‌ यो गुणः गृह्यते तेनैव इन्द्रियेण तदधिकरणमपि गृह्यते इति नियमो दृष्टः~। यथा रूपादीनां घटादिवृत्तीनां यथा चक्षुषा ग्रहः तथैव घटादीनामपि चक्षुषा ग्रहोऽनुभवसिद्ध एव~। आकाशादीनामग्रहणात् मूर्तेति~। घ्राणादिना गन्धाद्यधिकरणानामग्रहणात् द्रव्यग्राहकेति~। तथा च  वायुवृत्तिस्पर्शस्य त्वचा ग्रहे तु तस्यापि ग्रहणं स्यादेवेति वाच्यम्~। ग्रीष्मोष्मादौ उष्णस्पर्शस्य त्वचा ग्रहेऽपि तदधिकरणस्य तेजसः त्वचा अग्रहात् नियमोऽयं व्यभिचरितः~। 

		किञ्च वायोः तद्गतसङ्ख्यापरिमाणादीनां ग्रहणप्रसङ्गः~। तथा हि यत्र द्रव्यस्य चाक्षुषत्वं स्पार्शनत्वं वा भवति तत्रावश्यं तद्गतसङ्ख्यादीनां ग्रहणं सम्भवति~। वायुवृत्तिसङ्ख्यादीनामग्रहणात् न वायुः त्वाचप्रत्यक्षविषयः~।

		न च यज्जातीयद्रव्यं बहिरिन्द्रियेण गृह्यते तज्जातीयद्रव्यगतसङ्ख्यादयः तेन इन्द्रियेण गृह्यन्ते इति, यत्र यद्व्यक्तेः साक्षात्कारत्वं तत्र तद्व्यक्तिगतसङ्ख्यादीनामपि साक्षात्कारत्वमिति वा नियमः स्वीकार्यः~। आद्ये कदाचित् वायावपि 'एको वायुः' इत्यादिप्रतीतिबलात् सङ्ख्याग्रहणसम्भवात् वायोरपि प्रत्यक्षत्वे न किञ्चिद्बाधकम्~, द्वितीये तु पृष्टे संलग्नस्य वस्त्रादेः प्रत्यक्षत्त्वेऽपि तद्गतसङ्ख्यादीनामग्रहात् नियमोऽयं व्यभिचरित इति वाच्यम्~। वायौ सङ्ख्याग्रहस्तु न प्रत्यक्षादपि तु  व्याप्तिस्मरणाद्यनमपेक्षादनुमानादेव~। द्वितीये तु पृष्टे संलग्नस्य वस्त्रादेः‌ सङ्ख्याग्रहेऽपि तस्यैव चक्षुषा ग्रहणे तद्गतसङ्ख्याग्रहो भवत्येव~।‌ तस्मात् नियमद्वयस्वीकारेऽपि न दोषः~। 

		\subsection{रूपस्पर्शयोः बहिर्द्रव्यप्रत्यक्षप्रयोजकत्वम्}

		केचित्तु बहिरिन्द्रियप्रत्यक्षं प्रति उद्भूतरूपवत्त्वमुद्भूतस्पर्शवत्त्वमुभयं प्रयोजकमिति वदन्ति~। तेषां मतानुसारेण उद्भूतस्पर्शवत्त्वाभावात् प्रभादिकं किमपि न प्रत्यक्षम्~, उद्भूतरूपाभावाच्च ऊष्मादिकं न प्रत्यक्षमिति~। तन्न~। 'प्रभां साक्षात्करोमि' इति अनुव्यवसायबलात् सिद्धाया उद्भूतस्पर्शशून्ययाः प्रभाया अप्रत्यक्षत्वकथनं न युक्तम्~।

		\subsection{वायुप्रत्यक्षत्वानुमानविमर्शः}

		वायोः प्रत्यक्षत्वे 'वायुः प्रत्यक्षः प्रत्यक्षस्पर्शाश्रयत्वात् घटवत्' इत्यनुमानं प्रमाणमिति वदामः~। घटादौ उद्भूतस्पर्शस्य सत्त्वात् तेषां प्रत्यक्षत्वं यथा, तथैव वायावपि उद्भूतस्पर्शस्य सत्त्वात् तस्यापि प्रत्यक्षत्वं प्रमाणतः सिद्धमेव इति चेन्न~। तस्मिन्ननुमाने उद्भूतरूपस्य उपाधित्वात्~। तथा हि~- बहिरिन्द्रियप्रत्यक्षविषयेषु वायुभिन्नेषु सर्वेष्वपि द्रव्येषु उद्भूतरूपस्य विद्यमानत्वात् साध्यव्यापकत्वम्~, वायौ उद्भूतस्पर्शस्य विद्यमानत्वेऽपि उद्भूतरूपाभावात् साधनाव्यापकत्वम्~। न च आत्मनि प्रत्यक्षत्वसत्त्वेऽपि उद्भूतरूपाभावान्न साध्यव्यापकत्वमिति वाच्यम्~। साधनावच्छिन्नसाध्यव्यापकत्वात्~। उद्भूस्पर्शाश्रयत्वस्यात्मन्यभावान्न आत्मान्तर्भावेण साध्यव्यापकत्वभङ्गः~।

		\subsection{वायोः प्रत्यक्षत्वे सामग्र्युपपादनम्}

		अत्रोच्यते~- त्वगिन्द्रियव्यापारानन्तरं 'शीतो वायुः' इत्यादिप्रतीतेरुदयात् वायुः प्रत्यक्ष एव~। न च सदागतिमत्वाद्वायोः इन्द्रियव्यापारस्य च सर्वदा विद्यमानत्वात् सर्वदा वायुविषयकप्रतीतिप्रसङ्गः इति वाच्यम्~। वायुप्रत्यक्षं प्रति उत्कृष्टक्रियाविशिष्टवायोः कारणत्वं कल्प्यते~। अत एव ग्रीष्मकालादौ पङ्कादिना प्राप्तोत्कृष्टक्रियस्य वायोः ग्रहणम्~, यानादौ च तादृशक्रियाविशिष्टवायुसन्निकर्षात् स्पष्टतया वायोः त्वाचत्वानुभवः~। तस्मात् त्वाचप्रत्यक्षानन्तरमनुभूयमानानुव्यवसायस्य अपलापासम्भवात् वायुः प्रत्यक्ष एव~।

		न च तत्प्रत्यक्षत्वे तद्गतसङ्ख्यापरिमाणादीनां प्रत्यक्षत्वप्रसङ्ग इति वाच्यम्~। सर्वत्र प्रत्यक्षविषयीभूतेषु द्रव्येषु सङ्ख्यादिकं गृह्यते इति न नियमः~। अत एव चक्षुःसन्निकृष्टेऽपि दूरस्थपुरुषे स्थाणौ वा सङ्ख्यादिकं न गृह्यते~। न च तत्र दोषवशात् सङ्ख्यादीनामग्रहणम्~। अत एव चक्षुःसमीपस्थद्रव्ये सङ्ख्यादिकं गृह्यते इति वाच्यम्~। प्रकृतेऽपि अवयवसंयोगवैरल्यस्य सङ्ख्यादिग्रहणे दोषत्वं स्वीकुर्मः‌~। अत एव यथा घटादौ सङ्ख्यादिग्रहः तथैव जलादौ प्रभादौ च न भवति~। तस्मान्नायं वायोः प्रत्यक्षत्वाभावापादकम्~। अस्तु वा फूत्कारादौ कदाचित्सङ्ख्याग्रहणं सम्भवति इति वदामः इति~।

		\subsection{त्वाचप्रत्यक्षे स्पर्श एव प्रयोजकः}

		बहिरिन्द्रियप्रत्यक्षं प्रति उद्भूतरूपं प्रयोजकम्~। वायौ तदभावात् वायुः‌ न बहिरिन्द्रियप्रत्यक्षविषयः‌ इति केचित्~। तन्न~। त्वाच प्रत्यक्षं प्रति उद्भूतरूपमकिञ्चित्करमेव~। अन्यथा त्वचापि रूपग्रहोत्पत्तेः~। येन इन्द्रियेण द्रव्यप्रत्यक्षम्~, तं प्रति प्रयोजकीभूतगुणस्यापि तेनैव इन्द्रियेण ग्रहणात्~। यथा चक्षुषा घटप्रत्यक्षजनने प्रयोजकीभूतस्य रूपस्य चक्षुषैव ग्रहणम्~, त्वचा तु तत्प्रत्यक्षे रूपस्याप्रयोजकत्वात्तदग्रहणमिति~। अपि तु त्वचा तद्ग्रहणे स्पर्शस्य प्रयोजकत्वात् तद्ग्रहणमिति~। तस्मात् त्वाचप्रत्यक्षं प्रति स्पर्श एव प्रयोजकः~। तत्सत्त्वात् वायुः प्रत्यक्षः‌~।

		एतेन बहिरिन्द्रियप्रत्यक्षं प्रति उद्भूतरूपमुद्भूतस्पर्शश्च प्रयोजकमिति पक्षोऽपि निरस्तः~। सर्वेषामपि लोकानामनुभवसिद्धस्य प्रभादिविषयकप्रतीतेः स्पर्शविरहादनुपपत्त्या उभयोः तन्त्रत्वं नैव युज्यत इति~।

		\subsection{शास्त्रान्तरेऽपि वायुप्रत्यक्षत्वोपपादनम्}

		शास्त्रान्तरेऽपि वायोः प्रत्यक्षत्वमेव प्रतिपादितम्~। तथा हि~- {\fontsize{11.7}{0}\selectfont\s शीतादिषु स्पर्शविशेषोपलभ्यमानेषु शीतो वायुरुष्णो वायुरनुष्णाशीतो वायुरिति वायुद्रव्यस्यैकस्य प्रत्यभिज्ञायमानत्वात् कृष्णो घटः, पीतो घटः, श्वेतो घट इतिवत् सकलस्पर्शानुगतमेकमेव वायुद्रव्यं प्रत्यभिज्ञायतां भवतां स्पर्शमात्रमेव वयं प्रत्यभिजानीमो नान्यत् किञ्चिदिति वचनमनुभवविरुद्धमेव~। प्रयोगश्च भवति~- वायुः प्रत्यक्षः महत्त्ववत्त्वेऽनिन्द्रियत्वे च सति स्पर्शवत्त्वाद् भूतत्वाद्वा घटवदिति \footnote{मा.मे. १५०,१५१}} तस्मात् त्वगिन्द्रियमात्रग्राह्यो वायुः~। तत्प्रत्यक्षे च उद्भूतस्पर्शवत्त्वं प्रयोजकमिति सिद्धम्~।



	\section{परमाणुः}

	अथैतेषां परिदृश्यमाननां पृथिव्यप्तेजोवायूनां नित्यत्वमनित्यत्वमिति द्वेधाविभागो वर्तते~। जगदारम्भकत्वेन द्व्यणुकाद्यवयभूताः नित्याः परमाणवः तत्र तत्र वैशेषिकग्रन्थेषु न्यायग्रन्थेषु च निरूपिताः~। सर्वेषां परिदृश्यमानानामवयविनां किञ्चिदवयवारम्भकत्वात् त्रसरेण्वादीनामपि तदबाधितमेव~। तस्यापि त्रसरेण्ववयवस्य द्व्यणुकस्य महदारमभकत्वं सावयवत्वे एवोपपद्यते इत्यतः द्व्यणुकावयवत्वेन परमाणुः सिध्यति इति~। यद्यपि प्रमाणपदार्थस्य विशदतया प्रतिपादने तत्पराः गङ्गेशोपाध्यायप्रभृतयः नव्यनैयायिकाः अस्मिन् विषये तथा विचारं न कृतवन्तः, तथापि श्रीरघुनाथशिरोपणिप्रभृतिभिः परमाणुः निराकृतो वर्तते~। एवं परमाणुवृत्तिसंयोगस्य अव्याप्यवृत्तित्वविषये च मतभेदो दृश्यते~। अतस्तदत्र निरूप्यते~।

	तत्र वैशेषिकास्तावत् परमाणूनामनुमानगम्यत्वमनेकत्वं निरवयवत्वञ्चाभ्युपगच्छन्ति~।

		\subsection{परणुसत्त्वे प्रमाणम्}

		किं तावत्परमाणुसत्त्वे प्रमाणमिति चेत्~, अनुमानम्~। तथा हि~- अणुपरिमाणतारतम्यं क्वचिद्विश्रान्तं परिमाणतारतम्यत्वात् महत्परिमाणतारतम्यवत्' इति अनुमानस्वरूपम्~। घटपटादिपरिमाणे तन्तुकपालादिपरिमाणापेक्षया औत्कट्यं दृश्यते~। एवं परिमाणतारतम्यं यत्र यत्र द्रव्येषु दृष्टं तत्र सर्वत्रापि महावयविनि घटादौ तस्य विश्रान्तिरपि दृश्यते~। यथा कपालाद्यवयवधारा उत्कृष्टपरिमाणवति घटे विश्रान्ता, तन्त्वाद्यवयधारा पटे विश्रान्ता~। एवमेव अणुपरिमाणदारापि क्वचिद्विश्रान्ता, यत्र सा विश्रान्ता स परमाणुः~। न च अणुपरिमाणस्य तद्वतश्च सद्भावे किं मानमिति चेत्~, येषां घटादीनां साक्षात्कारो भवति ते सर्वेऽपि सावयवाः, अतिसूक्ष्मोऽपि त्रसरेणुः यो दृश्यते सोऽपि सावयव एव परिदृश्यमानत्वात्~। तच्च न महत्परिमाणकम्~, तस्य प्रत्यक्षत्वाभावात्~। नापि परममहत्परिमाणकम्~, तदा तस्य विभुत्वेन त्रसरेण्वोत्पत्त्यनुपपत्तेः~। तस्मात्तदतिरिक्तपरिमाणकमेव~। तच्च परिमाणमणुपरिमाणमिति~। न च त्रसरेण्ववयवाः एव परमाणवः~। तस्यापि अवयवस्य सत्त्वात्~। अन्यथा महद्रव्यारम्भकत्वमनुपपन्नम्~। तस्मात्तस्यावयवाः परमाणवः इति~। तदुक्तं न्यायकन्दल्याम्~- 

		{\fontsize{11.7}{0}\selectfont\s परमाणुस्वभावायाः पृथिव्याः सत्त्वे किं प्रमाणम्~? अनुमानम्~। अणुपरिमाणतारतम्यं क्वचिद्विश्रान्तं परिमाणतारतम्यत्वात् महत्परिमाणतारतम्यवत्~, यत्रेदं विश्रान्तं यतः परमाणुर्नास्ति सः परमाणुः~। अत एव नित्यो द्रव्यत्वे सत्यनवयवत्वात् आकाशवत् अथायं सावयवो न तर्हि परमाणुः, कार्यपरिमाणापेक्षया तदवयवपरिमाणस्य लोकेऽल्पीयस्त्वप्रतीतेः~। यश्च तस्यावयः सः परमाणुर्भविष्यति~।\footnote{न्या.कं. ९४}}

		\subsection{परमाणूनां सावयवत्वनिरासः}

		अथ द्व्यणुकावयवत्वेन सिद्धमपि तन्न परमाणुः~, तस्यापि सावयवत्वादिति चेत्~, तदा अवयवधारायाः विश्रान्त्यभावात् अनवस्था स्यात्~। न चेष्टापत्तिः~। अवयवसङ्ख्याप्रयुक्तानि खलु  अवयविनि परिमाणतारतम्यानि~। न्यूनसङ्ख्याकतन्तुभिर्जन्यः पटः अल्पः, अधिकसङ्ख्याकतन्तुभिर्जन्यः बृहदिति लोके व्यवहारः~। तर्हि अवयवधारायाः अनन्तत्वात् अनन्तावयवकत्वं सर्वेषां कार्यद्रव्याणां सिध्यति~। तथा च परिमाणतारतम्यानुपपत्तिः~। अस्ति च लोके परिमाणतारतम्यम्~, अतो अवयवधारापि क्वचिद्विश्रान्ता~। यत्र सा विश्रान्ता, स एव परमाणुः~। निरवयवत्वात्सः नित्यः~। तदुक्तं न्यायकन्दल्याम्~- 

		{\fontsize{11.7}{0}\selectfont\s अथ सोऽपि न भवति अवयवान्तरसद्भावात्~? एवं तर्ह्यनवस्था, ततश्चावयविनामल्पतरतमादिभावो न स्यात्~, सर्वेषामनन्तकारणजन्यत्वाविशेषेण परिमाणप्रकर्षाप्रकर्षहेत्वोः कारणसङ्ख्याभूयस्त्वयोरसम्भवात्~। अस्ति तावदयं परिमाणभेदः, तस्मादणुपरिमाणं क्वचिन्निरतिशयमिति सिद्धो नित्यः परमाणुः~।\footnote{न्या.कं. ९४}}

	न्यायनयेऽपि आरम्भवादे जगदुत्पादकः कः इति जिज्ञासायां सूत्रकारप्रभृतिभिः परमाणुपदार्थः जगदारम्भकत्वेन स्वीकृतः~। अतीन्द्रियस्य तस्य सत्त्वे युक्तिरपि प्रदर्शिता~। किन्त्वेतैः त्र्यणुकावयवत्वेनैव परमाणुः स्वीकृता इति भाष्यादिग्रन्थदर्शनेन ज्ञायते~। तथा हि~-

		\subsection{प्राचीनन्याये परमाणुः}

		अस्याः जगतः सृष्टिसंहारविचारे सर्वनाशात्मकः प्रलयो नैव भवितुमर्हति इति 'न प्रलयः, अणुसद्भावात्'\footnote{न्या. सू. ४.२.१६} इति सूत्रेण महर्षिः गौतमः प्रतिपादयाञ्चकार~। तत्रैव भाष्यकारैः अवयवावयविभावापन्नानि जन्यद्रव्याणि स्वावयवनाशे सति नश्यन्ति~। एवमेव तत्रविद्यमानाः गुणादयोऽपि नश्यन्ति~। तस्माज्जगतः नाशो भवति इति शङ्कायां निवयवस्य परमाणुपदार्थस्य नाशासम्भवात् सर्वनाशो न भवति इति वदन्ति~। तर्हि कः परमाणुः~? कस्यावयवोऽयमिति शङ्कायां 'परं वा त्रुटेः'\footnote{न्या. सू. ४.२.१७} अत्र त्रुटिशब्दस्य 'स्त्रियां मात्रा त्रुटिः पुंसि लवलेशकणाणवः'\footnote{अमर. } इति कोशवाक्येन 'त्रुटिस्त्रसरेणुरित्यनर्थान्तरम्'\footnote{न्या. वा. ता. टी.} तात्पर्यटीकाकाराणां वचनेन च त्रसरेणुरित्यर्थो लभ्यते~। तदपेक्षया सूक्ष्मः परमाणुः इति सूत्रवशात् भाष्यावलोकनेन च अवगम्यते~। किन्त्वत्र त्रुटेरवयवाकथनात् न द्व्यणुकपदार्थः स्वीकृतः, तत्स्थाने एव निरवयवः परमाणुः स्वीकृत इति ज्ञायते~। तथा च भाष्यम्~-

		{\fontsize{11.7}{0}\selectfont\s अवयवविभागमाश्रित्य वृत्तिप्रतिषेधादभावः प्रसज्यमानो निरवयवात् परमाणोर्निवर्तते, न सर्वप्रलयायकल्पते~। निरवयवत्वं खलु परमाणोः विभागैरल्पतरप्रसङ्गस्य अभावात् यतो नाल्पीयस्स्तत्रावस्थानात्~। लोष्टस्य खलु प्रविभज्यमानावयवस्याल्पतरमल्पतममुत्तरमुत्तरं भवति~। स चामल्यतरप्रसङ्गः यस्मान्नाल्पतरमस्ति, यः परमोऽल्पस्तत्र निवर्तते~। यत श्च नाल्पीयोऽस्ति तं परमाणुं प्रचक्ष्महे इति~।\footnote{न्या. भा. ४.२.१६} अवयवविभागस्यानवस्थानाद् द्रव्याणामसङ्ख्येयत्वात् त्रुटित्वनिवृत्तिरिति~।\footnote{न्या. भा. ४.२.१७}}

		अथात्र विकल्पप्रदर्शनपुरस्सरं परमाणुं व्यवस्थापयति न्यायमञ्जरीकारः~। तथा हि~- घटादयः महावयविनः निरवयवाः अनन्तावयवकाः परमाण्वन्तावयवकाः वा~? घटपटादीनां कपालतन्त्वाद्यवयवकत्वं प्रत्यक्षप्रमाणेनोपलब्धमित्यतः तेषामनित्यत्वाच्च प्रथमकल्पोऽवसितः~। द्वीतीयकल्पोऽपि न युक्तः~। लोके अवयवसङ्ख्यातारतम्यात् परिमाणतारतम्यो दृष्टः~। अनन्तावयवकत्वे परिमाणतारतम्यानुपपत्त्या मेरुसर्षपयोः समानपरिमाणकत्वप्रसङ्गः~। किन्तु तेषामसमानपरिमाणकत्वं प्रत्यक्षम्~। तस्मात्परमाण्वन्तावयवका एव~। तदुक्तं न्यायमञ्जर्याम्~- 

		{\fontsize{11.7}{0}\selectfont\s तथा हि पार्थिवमाप्यं तैजसं वायवीयमिति चतुर्विधं कार्यं स्वावयवाश्रितमुपलभ्यते, तत्र यथा घटः सावयवः कपालेश्वाश्रित  एवं कपालान्यपि सावयवत्वत्तदवयवेषु तदवयवा अपि तदवयवान्तरेष्वित्येवं तावद्यावत्परमाणवो निरवयवा इति, यत्र यावतः कार्यजातस्य स्वावयवाश्रितस्य प्रत्यक्षेण ग्रहणं तत्र तदेव प्रमाणम्~, तदपि हि कार्यं स्वावयवत्वात्~, परिदृश्यमानकार्यवत्~। निरवयवत्वे तु तस्य परमाणुत्वमेव~। परमाणुषु च सावयवत्वस्य च हेतोरसिद्धत्वान्नावयवान्तरकल्पना~। तेषां हि सावयवत्वे तदवयवाः परमाणवो भवेयुः~। न ते उत्पत्तिक्रमवत् विनाशक्रमेणापि परमाणवोऽनुमीयन्ते~। लोष्टस्य प्रविभज्यमानस्य भागाः, तद्भागानां च भागान्तराणीत्येवं तावत् यावदशक्यभङ्गत्वमदर्शनविषयत्वं च भवति~। तद्यतः परमवयवविभागो न सम्भवति, ते परमाणव उच्यन्ते~। तेष्वपि हि विभज्यमानेषु तदवयवाः परमाणवो भवेयुर्न ते~। तदेतदेवं उत्पत्तिक्रमवत् विनाशक्रमस्येदृशो दर्शनात् सन्ति परमाणवः~।}

		{\fontsize{11.7}{0}\selectfont\s अत्र हि त्रयी गतिः~। अस्य घटादेः कार्यस्य निरवयवत्वमेव वा, अवयवानन्त्यं वा, परमाण्वन्तता वा~? तत्र निरवयवत्वमनुपपन्नम्~, अवयवानां पटे तन्तूनां घटे च कपालानां प्रत्यक्षमुपलम्भात्~। अनन्तावयवयोगित्वमपि न युक्तम्~, मेरुसर्षपयोरनन्तावयवयोगित्वाविशेषेण तुल्यपरिमाणत्वप्रसङ्गात्~। तस्मात्परमाण्वन्ततयैव युक्तिमती~।\footnote{न्या.मं. ४२०}}

		\subsection{द्व्यणुकसाधनम्}

		तात्पर्यटीकाकारास्तु द्व्यणुकमपि साधयन्ति~। अस्तु तावन्महदारम्भकत्वेन अणुः~। तथापि पुनस्तदवयवकल्पनमयुक्तमिति चेत्~, त्रुट्यवयवभूतानामणूनाम् अन्त्यावयवत्वे तेषामेव परमाणुस्वरूपत्वमिति वक्तव्यम्~। परमाणु परिमाणस्य च न महदारम्भकत्वम्~। परिमाणस्य स्वसजातीयस्वोत्कृष्टपरिमाणारम्भकत्वनियमात्~। तस्मात् त्रसरेणुगतमहत्वं प्रति बहुत्वं कारणमिति वक्तव्यम्~। न च परमाणुत्रसंयोगात् महत्वोत्पत्तिः, परमाणुत्वे सति बहुत्वसङ्ख्यायुक्तत्वात्~। अन्यथा घटादीनां नाशेऽपि कपालचूर्णादिक्रमो न स्यात्~। तस्मात् नाशक्रमानुरोधेन महत्वोत्पत्तिकारणत्वेन च त्रुट्यवयवः द्व्यणुकमिति कल्प्यते~। तस्यापि कार्यत्वं प्रकल्प्य तदवयवत्वेन परमाणुमभ्युपगच्छामः~। तदुक्तं तात्पर्यटीकायाम्~- 

		{\fontsize{11.7}{0}\selectfont\s परमाणूनां बहूनामनारम्भकत्वात्~। तथा हि त्रयः परमाणवो न कार्यद्रव्यमारभन्ते परमाणुत्वे सति बहुत्वसंख्यायुक्तत्वाद् घटोपगृहीतपरमाणुप्रचयवत्~। आरम्भकत्वे तेषां घटोपगृहीतानां कपालशर्कराचूर्णक्रमो घटनाशे नोपलभ्येत द्व्यणुके च विजातीयानारम्भकत्वे सिद्धे तेनैव दृष्टान्तेनान्यत्रापि विजातीयेनारम्भो निषेध्यः~।\footnote{न्या.वा.ता.टी. ३.१.३०}}

		\subsection{परमाणावव्याप्यवृत्तिसंयोगसाधनम्}

		 ननु परमाणूनां निरवयवत्वं न युक्तम्~। निरवयवस्य हि द्रव्यस्य दिगवच्छेदेन भेदो न दृष्टः, सर्वगतत्वात् गगनावत्~। परमाणूनान्तु सर्वगतत्वाभावात् दिगवच्छेदेन तस्य सावयवत्वमेव~। तथा च षड्दिगुपादिभिः विशिष्टस्य परमाणोः पुनः षडवयवाः सिध्यन्ति इति चेन्न~। दिशः सर्वगतत्वादेकत्वात्तस्य षट्सङ्ख्याकत्वमनुपपन्नम्~। षडुपाधीनां विरहादेव परमाणूनामपि न षडवयवकत्वम्~। अस्तु वा 'अयमस्माद्विप्रकृष्टः' इत्यादिव्यवहारनिर्वाहार्थं सुमेरुपर्वताद्युपाधिभेदेन दिशः षट्त्वम्~। तेन परमाणूनां न सावयवत्वं सिध्यति~। तद्यथा एकस्यापि निरवयवस्य नानासंयोगाः भवितुमर्हन्ति~। यथा देशभेदेन घटपटादिना सह गगनस्य संयोगः~। तद्वत् परमाणूनामपि सर्वत्र परितः दिशस्संयोगो वर्तते~। दिशः उपाधिभेदेन षटसङ्ख्याकत्वात् षटसंयोगाः भवितुमर्हन्ति~। एतेन परमाणुवृत्तिसंयोगः अव्याप्यवृत्तिरिति सिद्धम्~। एवञ्च निरवयवस्य परमाणोः अव्याप्यवृत्तिसंयोगासम्भवात् द्व्यणुकाद्यनुपपत्तिशङ्कापि निरस्ता~। तदुक्तं तात्पर्यटीकायाम्~-

		{\fontsize{11.7}{0}\selectfont\s यत्पुनरुक्तं दिग्देशभेदो यस्यास्ति तस्यैकत्वं न युक्तमिति~। परमाणोः किल भवदभिमतस्यैकस्य दिग्भागाः षट्, न चैकस्य दिग्भागे भेदोऽस्तीति षडेव परमाणवः~। एतद्दूषयति क एवमाह दिग्देशभेदो यस्यास्तीति~। स्वरूपेणैका दिक्सर्वगता च नास्या भेदोऽस्तीत्यर्थः~। यद्येकैव दिक्क्व तर्हि परमाणावस्मादयं परमाणुः पूर्वोऽयं पश्चिम इत्यादयो बुद्धिव्यपदेशभेदा इत्यत आह~। दिग्देशभेदाश्च दिशः संयोगा एकत्वेऽपि दिश आदित्योदयदेशप्रत्यासन्नदेशसंयुक्तो यः सैतरस्माद्विप्रकृष्टदेशसंयोगात्परमाणोः पूर्वःप एवमादित्यास्तमयदेशप्रत्यासन्नदेशसंयुक्तो यः स इततस्माद्विप्रकृष्टदेशसंयोगात्परमाणोः पश्चिमः तौ च पूर्वपश्चिमौ परमाणू अपेक्ष्य यः सूर्योदयास्तमयदेशविप्रकृष्टदेशसंयोगः स मध्ववतीम्~। एवमेतयोर्यौ तिर्यग्देशसम्बन्धिनौ मध्यस्य आर्जवेन व्यवस्थितौ पार्श्ववर्तिनौ तौ दक्षिणोत्तरौ परमाणू एवं मध्यन्दिनवर्तिसूर्यसन्निकर्षविप्रकर्षौ पूर्वसंख्यावच्छिन्नत्वं चाल्पत्वं परसंख्यावच्छिन्नत्वं च भूयस्त्वम्~। तस्मादेकस्यापि परमाणोः परमाण्वन्तरसंयोगा अव्याप्यवृत्तय एव भागाः~।\footnote{न्या.वा.ता.टी. ४.२.२५}}

		\subsection{परमाणुद्वयसंयोगो व्याप्यवृत्तिः}

		निरवयवानां परमाणूनामव्याप्यवृत्तिसंयोगानुपपत्तिशङ्कायां तस्य व्याप्यवृत्तिसंयोगः कल्प्यते~। न च संयोगः सर्वोऽप्यव्याप्यवृत्तिरिति नियमविरोधः~। तस्य दैशिकाव्याप्यवृत्तित्वाभावेऽपि कालिकाव्याप्यवृत्तित्वमुपपद्यते, अनित्यत्वात्~। न च संयोगत्वस्वाश्रयसमवायिदेशवृत्तित्वयोः व्याप्तिर्विद्यते~। येन परमाणुसंयोगः स्वाश्रयसमवायिदेशवृत्तिः संयोगत्वादित्यनुमानेन परमाणूनां सावयवत्वसिद्धिः~। तत्र हि सावयववृत्तिसंयोगत्वमुपाधिः~। सावयववृत्तिसंयोगत्वं च स्वाश्रयसमवायिदेशवृत्तिसंयोगे सर्वत्र वर्ततेत्यतः तस्य साध्यव्यापकत्वम्~। विवादाध्यासिते परमाणुसंयोगे अन्यत्र गगनसंयोगे वा संयोगत्वस्य सत्त्वेऽपि सावयववृत्तिसंयोगत्वस्य विरहात् साधनाव्यापकत्वम्~। न च संयोगानां सर्वेषां स्वाश्रयसमवायिदेशमवच्छेदकम्~। तथा सति सर्वेषां सावयवत्वात् परिमाणतारतम्याभावात् जगदारम्भकत्वेन परमाणुकल्पनाया एवानुपपत्तेः~। तस्मात्परमाणावपि संयोगो जायत एव~। स च व्याप्यवृत्तिरिति आरम्भवादप्रणेतॄणां बदरीनाथशुक्लमहोदयानामाशयः~- 

		{\fontsize{11.7}{0}\selectfont\s न खलु परमाण्वोः संयोगासम्भवः सम्भावनीयः, तयोर्व्याप्यवृत्तिसंयोगाङ्गीकारात्~, संयोगत्वावच्छेदेनैव अव्याप्यवृत्तित्वनियमस्य परमाणुसंयोगे कालिकाव्याप्यवृत्तित्वमादायापि निर्वाहसम्भवात्~। यो यः संयोगः स सर्वः स्वाश्रयसमवायिदेशावच्छिन्नो भवति इति नियमस्तु नास्त्येव, सावयववृत्तिसंयोगत्वस्य उपाधित्वात्~, यत्र यत्र संयोगे स्वाश्रयसमवायिदेशावच्छिन्नत्वं प्रमाणसिद्धं तत्र सर्वत्र सावयववृत्तिसंयोगत्वस्य सत्त्वेन साध्यव्यापकत्वात्~, विवादाध्यासिते परमाणुसंयोगे संयोतत्वरूपसाधनाव्यापकत्वाच्च~। संयोगत्वावच्छेदेन स्वाश्रयसमवायिदेशावच्छिन्नत्वनियमस्वीकारे च परमाणुकल्पनाया एव वैयर्थ्यप्रसङ्गात्~। यैः परमोदाप्रतिभैः परमाणवः स्वीक्रियन्ते तैर्व्याप्यवृत्तयः तेषां संयोगाः स्वीक्रियन्त एवेति तदीयमभिप्रायमनवगच्छतो भवतो न खेदमाददाति~।\footnote{आरम्भवादः १०}}


		\subsection{परमाणुसत्त्वे प्रमाणाभावः}

		दीधितिकारेति प्रसिद्धाः श्रीरघुनाथशिरोमणयः परमाणुं निराकुर्वन्ति~। त्रुटेः चाक्षुषत्वादिना, तदवयवस्य च महदारम्भकत्वादिना च सावयवत्वकल्पनमप्रयोजकशङ्काकलङ्कितम्~। अन्यथा अनन्तावयवधाराया अपि कल्पनमापद्येत~। न चानवस्थाभयान्न कल्प्यते इति वाच्यम्~। तर्हि तदर्थं त्रुटावेवावयवधारायाः विश्रान्तिः कल्प्यताम्~। तस्यैव निरवयवत्वं नित्यत्वञ्च स्वीक्रियताम्~। तस्मात् प्रमाणाभावान्नास्त्येव परमाणुः~। तदुक्तं पदार्थतत्त्वनिरूपणे~- 

		{\fontsize{11.7}{0}\selectfont\s परमाणुद्व्यणुकयोश्च मानाभावः त्रुटावेव विश्रमात्~। त्रुटिः समवेता चाक्षुषद्रव्यत्वात् घटवत्~, ते च समवायिनः समवेताः चाक्षुषद्रव्यसमवायित्वादिति चाप्रयोजकम्~। अन्यथा तादृशसमवायिसमवायित्वादिभिरनवस्थिततत्समवायिपरम्परासिद्धिप्रसङ्गात्~।\footnote{प.त.नि २२}}

		\subsection{परमाणुव्यवहारानुपपत्तिशङ्कावारणम्}

		न च अणुरिति प्रामाणिकानां व्यवहारानुपपत्तिरिति वाच्यम्~। कपालाद्यपेक्षया घटादौ परिमाणाधिक्यात् यथा 'महत्तम' इति व्यवहारः तद्वत् परिमाणाल्पत्वे अणुरिति व्यवहारोऽपि उपपद्यते~।

		{\fontsize{11.7}{0}\selectfont\s अणुव्यवहारश्चापकृष्टपरिमाणनिबन्धनो महत्यपि महत्तमादणुव्यवहारात्~।\footnote{प.त.नि २२}}


	\section{विमर्शः}

		\subsection{परमाणुसत्त्वे प्रमाणविमर्शः}

		परमाणुसत्त्वे प्रमाणन्तु न प्रत्यक्षम्~, अपि तु अनुमानादिकमेव~। किं तावदनुमानमिति चेत् 'अणुपरिमाणतारतम्यं‌ क्वचिद्विश्रान्तं परिमाणतारतम्यत्वात् महत्परिमाणतारतम्यवत्' इति केचन वदन्ति~। 'त्रसरेणुः कार्यः स्वावयवत्वात् परिदृश्यमानकार्यवत्' इत्यपरे~। 'त्रुटिः सावयवः चाक्षुषत्वात् घटवत्', 'तदवयवः सावयवः महदारम्भकत्वात् कपालवत्' इत्यनुमानाभ्यां क्रमेण द्व्यणुकपरमाण्वोः सिद्धिरित्यन्ये~। अत्र प्रथमानुमाने पक्षकुक्षौ प्रविष्टस्याणुपरिमाणस्य सद्भावे प्रमाणान्तरं वक्तव्यम्~। तथा हि अणुरिति व्यवहार एव अणुपरिमाणसत्त्वे प्रमाणम्~। तदाश्रितस्यैकत्वे च त्रुट्यनुपपत्तिः~। अनेकैरवयवैरेव कार्यद्रव्योत्पत्तिः~। तस्मादणुपरिमाणाश्रयः अनेकः त्र्युट्यारम्भकत्वात् पटवत् इत्यनुमानेन तस्यानेकत्वसिद्धिः~। महत्परिमाणतारतम्यस्य यथात्युत्कृष्टमहति घटादौ विश्रमात्तेषाम् अणुतारतम्यस्यापि अत्युत्कृष्टाणावेव विश्रमः कल्प्यते~। सैव परमाणुरित्युच्यते~।

		द्वीतीयानुमानन्तु 'न प्रलयः परमाणुसद्भावात्'\footnote{न्या.सू. ४.२.१६} इति सूत्रभाष्यावलोकनेन ज्ञायते~। अत्र हि कार्याणां सर्वेषां सावयवत्वनियमात् त्रुटेरपि कार्यत्वात् तस्यापि सावयवत्वसिद्धिः~। किन्त्वत्र यतः अवयवविभागो न सम्भवति सः परमाणुरित्युक्तत्वात्~, द्व्यणुकसत्त्वे प्रमाणाप्रदर्शनाच्च त्रुट्यवयवा एव परमाणव इति सिद्धान्तितम्~।

		तृतीयानुमाने तु चाक्षुषद्रव्यत्वेन हेतुना त्रुट्यवयवं द्व्यणुकं संसाध्य, तस्य च महदारम्भकत्वेन सावयवत्वं साध्यते इति विशेषः~। अत्र निरवयवस्य महदारम्भकत्वं न सम्भवतीत्यंशोऽपि भासते~।

		एवं विनाशक्रमेणापि परमाणवः अनुमीयन्ते~। तथा हि~- कपालादिस्वावयवसंयोगनाशात् घटादिकार्याणां नाशो लोके दृष्टः, द्रव्यनाशं प्रत्यसमवायिकारणनाशस्य कारणत्वात्~। एवञ्च 'त्रुटिनाशोऽपि स्वावयवसंयोगनाशजः कार्यद्रव्यत्वात् घटवत्' इत्यनुमानेन त्रुटिनाशकारणीभूतनाशप्रतियोगिसंयोगाश्रयतया किञ्चिद्द्रव्यं सिध्यति~। तदेव द्व्यणुकमिति वदामः~। महदारम्भकत्वाच्च तस्य कार्यत्वे सिद्धे तन्नाशकारणीभूतनशप्रतियोगिसंयोगाश्रयतया परमाणुः सिध्यति~। न च तस्यापि महदारम्भकद्रव्यारम्भकत्वात् कार्यत्वम्~। अप्रामाणिकानन्तावयवधाराकल्पनारूपानवस्थाप्रसङ्गात्~। तस्मात् सः अकार्यत्वान्निरवयवत्वमेव~। सैव परमाणुरिति~।

		एवं 'त्रसरेणुः सावयवः दृश्यत्वात् घटवत्' इत्यनेनानुमानेन त्रसरेणोरवयवत्वेन द्व्यणुकं सिध्यति~। 'तदपि सावयवः महदारम्भकत्वात् कपालवत्' इत्यनुमानेन द्व्यणुकावयवत्वेन परमाणुः सिध्यति इति~। 

		\subsection{परमाणूनां नित्यत्वविमर्शः}

		कार्यातिरिक्तानां गगनादिपदार्थानां नित्यत्वमिति क्लृप्तम्~। न च तेषां नित्यत्वे विभुत्वमेव प्रयोजकमिति वाच्यम्~। अविभुनः मनसोऽपि नित्यत्वात्~। तस्मान्नित्यत्वं निरवयवप्रयुक्तमेव~। गगनादीनां मनसश्च निरवयवत्वात् नित्यत्वम्~। परमाणूनामपि निरवयवत्वान्नित्यत्वं सिद्धमेव~। निरवयवत्वञ्चास्याप्रामाणिकानन्तावयवधारायाः कल्पना स्यादिति भिया न स्वीक्रियते~।  द्व्यणुकस्यावयवत्वेन परमाणूनां साधनात्तेषामनेकत्वमिति~। 

		\subsection{परमाणुवृत्तिसंयोगस्य व्याप्यवृत्तित्वविमर्शः}

		एतादृशपरमाणुषु अदृष्टवदात्मसंयोगात् क्रिया उत्पद्यते~। ततः विभागादिक्रमेण परमाणुद्वयसंयोगो जायते~। ततः कार्यद्रव्यं द्व्यणुकमुत्पद्यते~। एवं रीत्या महावयविपर्यन्तमुत्पत्तिः इति सृष्टिक्रमः~। ननु परमाणौ संयोगो अनुपपन्नः~। तथा हि~- {\fontsize{11.7}{0}\selectfont\s 'तदेवं नियतस्य कस्यचित्कर्मनिमित्तस्याभावान्नाणुष्वाद्यं कर्म स्यात्~; कर्माभावात्तन्निबन्धनः संयोगो न स्यात्~; संयोगाभावाच्च तन्निबन्धनं द्व्यणुकादि कार्यजातं न स्यात्~। संयोगश्चाणोरण्वन्तरेण सर्वात्मना वा स्यात् एकदेशेन वा~? सर्वात्मना चेत्, उपचयानुपपत्तेरणुमात्रत्वप्रसङ्गः, दृष्टविपर्ययप्रसङ्गश्च, प्रदेशवतो द्रव्यस्य प्रदेशवता द्रव्यान्तरेण संयोगस्य दृष्टत्वात्~; एकदेशेन चेत्, सावयवत्वप्रसङ्गः~; परमाणूनां कल्पिताः प्रदेशाः स्युरिति चेत्, कल्पितानामवस्तुत्वादवस्त्वेव संयोग इति वस्तुनः कार्यस्यासमवायिकारणं न स्यात्~; असति चासमवायिकारणे द्व्यणुकादिकार्यद्रव्यं नोत्पद्येत'\footnote{ब्र.शां.भा. २.२.१२}} इति~। तन्न~। अदृष्टस्य जिवात्मवृत्तिविशेषगुणस्य सत्त्वात् तद्विशिष्टात्मसंयोग परमाणौ द्रव्यारम्भकसंयोगजनकक्रियानकत्वात्~।‌ न च अदृष्टमचेतनगुणः~। तथा सति उत्पन्नस्य शिशोः भुभुक्षादिनिवृत्तौ प्रवृत्तिरेव न स्यात्~। शरीररूपजडपदार्थस्य इदमिदानीमुत्पन्नत्वात्~, घटादीनाञ्च सामानाधिकरण्याभावात्~। तस्मात् परमाणौ क्रियाजनकसामग्रीसत्त्वात् क्रिया उपपद्यते~। यत्पुनरुक्तं परमाणुषु संयोगोऽनुपपन्न इति~। तदप्यसारम्~। यद्यपि तत्र निरवयवत्वात् किञ्चिद्देशावच्छेदेन प्रसिद्धः अव्याप्यवृत्तिसंयोगोऽनुपपन्नः~। तथापि व्याप्यवृत्तिसंयोगस्सम्भवति~। संयोगस्याव्याप्यवृत्तित्वनियमस्तु कालिकाव्याप्यवृत्तित्वमादायापि उपपद्यते~। अथवा परमाणूनां नानात्वात् परमाणौ यथा भिन्नदेशवृत्तित्वं प्रलयकाले स्वीक्रियते तत्र यथा दिशः अवच्छेदकत्वं क्लृप्तमवश्यं वक्तव्यञ्च तथैव प्रकृतेऽपि उपाधिभेदात् दिशामपि भेदं प्रकल्प्य तत्तद्दिक्संयोगस्याव्याप्यवृत्तित्वं स्वीकृत्य यत्किञ्चिद्दिगवच्छेदेन परमाण्वन्तरसंयोगो कल्प्यते~। स च संयोगो दैशिकाव्याप्यवृत्तिरिति~।

		अत्रेदं चिन्त्यते~-  त्रुटिकारणत्वेन परमाणुः न सिध्यति, तद्गतमहत्वं प्रति बहुत्वस्य कारणत्वात् इत्युक्तम्~। न च परमाणुत्रयसंयोगः कल्प्यतामिति चेन्न~। महत्वाश्रयद्रव्यस्य अव्याप्यवृत्तिसंयोगजन्यत्वनियमात्~। घटपटादिषु तथैव दृष्टत्वात्~। परमाणूनान्तु निरवयवत्वात् तत्र व्याप्यवृत्तिसंयोग एव जायते~। तस्मादवान्तरद्रव्यं किञ्चित्कल्पनीयमिति युक्तम्~। न च व्यप्यवृत्तिसंयोगो असिद्ध इति वाच्यम्~। तूलकादौ तूलकान्तरस्य व्याप्यवृत्तिसंयोगस्य कथञ्चिदनुभवात्~, अत एव यदा तत्र व्याप्यवृत्तिसंयोगः तदा तत्र परिमाणे उत्कृष्टता नानुभूयते~। अस्मिन् पक्षे परमाणुद्व्यणुकयोः परिमाणे तथा तारतम्यं न भवितुमर्हति व्याप्यवृत्तिसंयोगजन्यद्रव्ये परिमाणसाम्यात्~। यदि तत्र तारतम्यमपि एष्टव्यमित्याग्रहः तदा अव्याप्यवृत्तिसंयोग एव आदरणीयः इत्यलं विस्तरेण~।

		\subsection{परमाणुसत्त्वे प्रमाणाभावो विमृश्यते}

		ननु परमाणुरिति पदार्थो नास्त्येव, तत्सत्त्वे प्रमाणाभावात्~। न च 'त्रुटिः सावयवः चाक्षुषत्वात् घटवत्', 'द्व्यणुकं सावयवं महदारम्भकत्वात् कपालवत्' इत्यनुमानेन परमाणुः सिध्यति~। अप्रयोजकत्वात्~। एवं तर्हि परमाणूनामप्यवयवः कल्प्यताम्~। न च तत्रानास्थाभिया न कल्प्यते इति वाच्यम्~। तर्हि त्रुटीनामेव निरवयवत्वं कल्प्यताम्~। तस्मात्त्रुटिरेव निरवयवो नित्यः इति केचित्~। तदेव शास्त्रान्तरेऽपि वर्णितं यथा~- {\fontsize{11.7}{0}\selectfont\s जालरन्ध्रविसरद्रवितेजोजालभासुरपदार्थविशेषान्~।\\ अल्पकानिह पुनः परमाणून् कल्पयन्ति हि कुमारिलशिष्याः~॥}\footnote{मा.मे. १५७} इति~।

		\subsection{त्रुट्यवयव एव परमाणुः}

		अत्रेदं चिन्त्यते~- त्रुटीनां निरवयवत्वस्वीकारे तद्वृत्तिद्रव्यान्तरारम्भकसंयोगः अव्याप्यवृत्तिर्न वा~। नाद्यः निरवयवमूर्तस्य अव्याप्यवृत्तित्वसंयोगासम्भवात्~। न द्वीतीयः तज्जन्यद्रव्यस्य तदपेक्षया उत्कृष्टपरिमाणत्वानुपपत्तिः~। ननु  तत्राप्यव्याप्यवृत्तिसंयोगः प्रत्यक्षसिद्धः इति चेत्~, तस्य सावयवत्वमपि युक्त्या सिद्धमेव~। यदुक्तमप्रयोजकमिति तदपि न~। यदि त्रुटिः सावयवं न स्यात् तर्हि तस्य चाक्षुषत्वमपि न स्यात्~, चाक्षुषद्रव्यस्य अवयवजन्यत्वनियमादिति तर्क एव अप्रयोजकशङ्कानिवर्तकः~। न च तथापि परमाणुस्तावत्सिध्यति इति वाच्यम्~। द्वितीयानुमानेऽपि यदि द्व्यणुकं सावयवं न स्यात् तर्हि तस्य महदारम्भकत्वमपि न स्यादिति तर्केण द्वितीयानुमानेऽपि अप्रयोजकशङ्कावारणात्~। तस्मात् प्रमाणतः प्रतिपन्न एव परमाणुरिति~।

		एवमपि अव्याप्यवृत्तिसंयोगस्यैव त्रुटिजनकत्वमित्यत्र मानाभावात्~, त्रुटिगतपरिमाणं प्रति बहुत्वस्यैव कारणत्वस्वीकाराच्च त्रुट्यवयव एव परमाणुः~। न च त्रुटिरेव परमाणुरिति उच्यतामिति चेत्~, त्रुटेः परमाणुत्वे तत्प्रत्यक्षत्वमेव बाधकम्~। यो यः साक्षात्क्रियते तत्सर्वं न श्यति इति युक्तेः अपलापासम्भवात् न अवयवधारायाः त्रुटौ विश्रान्तिः युज्यते~। न च सावयवस्यैव महदारम्भकत्वात् तस्यापि अवयवः सिध्यति इति वाच्यम्~। सावयवद्रव्यमेव महदारम्भकमिति न कल्प्यते, अप्रयोजकत्वात्~। तस्मात्त्र्युट्यवयव एव परमाणुरिति सिद्धम्~।

	\section{तमसः अभावरूपत्वे मतवैलक्षण्यम्}
	
	इदानीं तमः पदार्थो विचार्यते~।'नीलं तमश्चलति' इत्यादिप्रतीतिवशात् तमसि नीलरूपं चलनक्रिया च भासते~। क्रियाविशिष्टस्य रूपविशिष्टस्य च द्रव्यत्वात् तमः द्रव्यमिति सिध्यति~। किन्तु क्लृप्तेषु नवसु द्रव्येषु तन्नान्तर्भवति~। तथा हि~- गन्धाभावात् न पृथिवी, नीलरूपवत्त्वाच्च न जलादिकामिति~। तस्मादतिरिक्तं द्रव्यं तमः‌ इति प्राप्ते नातिरिक्तं द्रव्यं तदिति वैशेषिकानां नैयायिकानां च सिद्धान्तः~। तत्र वैशेषिकेषु एकदेशिनः आरोपितं नीलरूपमेव तमः इति कथयन्ति~। अन्ये तावत् भासामभावस्तमः इति वदन्ति~। तदत्र निरूप्यते~।

		\subsection{आरोपितनीलरूपं तमः}

		तमसि स्पर्शानुपपत्त्या तस्य स्पर्शवद्द्रव्यानारम्भकस्य द्रव्यत्वानुपपत्तौ तस्य तेजोभावरूपत्वशङ्कायां तन्निराकुर्वन्ति श्रीधराचार्याः~। तथा हि~- नीलाकारेण प्रतीयमानस्य तमसः अभावरूपत्वं नैव सम्भवति~। मध्यन्दिने तेजसः सद्भावेऽपि दूरगगनव्यापिनः नीलिम्नः तमसः प्रतिभानात् न तस्य तेजोभावरूपत्वम्~। किञ्च यदि तेजोभाव एव तमः स्यात् तस्याप्रत्यक्षत्वप्रसङ्गः~। कथम्~? अभावो हि इन्द्रियसंयुक्ततदधिकरणविशेष्यतया विशेषणतया वा प्रतीयते~। न स्वतन्त्रः~। तमः प्रत्ययस्थले अन्यस्य कस्यापि अप्रतीतेः~। एवं तमः तेजोभाव एव इति प्रतीतिरपि न पश्चादुदेति~। तस्मान्नाभावेदम्~। न च आलोकादर्शनं यदा तदा तम इति प्रतीतिः~। तस्मादालोकदर्शनाभाव एव तमः इति वाच्यम्~। तमः, छाया इत्यादिना कृष्णाकारस्य प्रतिभानात्~। तस्मात् प्रौढप्रकाशकतेजसः अभावे सति प्रतीयमानः  सर्वतः समारोपितः  नीलरूपविशेषः एव तमःपदार्थः इति~।

		{\fontsize{11.7}{0}\selectfont\s किन्त्वारम्भानुपपत्तेः नीलिममात्रप्रतीतेश्च द्रव्यमिदं न भवतीति ब्रूमः~। तर्हि भासामभाव एवायं प्रतीयते~? न, तस्य नीलाकारेण प्रतिभासायोगात्~, मध्यन्दिनेऽपि दूरगगनाभोगव्यापिनो नीलिम्नश्च प्रतीतेः~। किञ्च गृह्यमाणे प्रतियोगिनि संयुक्तविशेषणतया तदन्यप्रतिषेधमुखेनाभावो गृह्यते, न स्वतन्त्रः~। तमसि च गृह्यमाणे नान्यस्य ग्रहणमस्ति~। न च प्रतिषेधमुखः प्रत्ययः~। तस्मान्नाभावोऽयम्~। न चालोकादर्शनमात्रमेवैतत्~, बहिर्मुखतया तम इति, छायेति च कृष्णाकारप्रतिभासनात्~। तस्माद्रूपविशेषोऽयमत्यन्तं तेजोभावे सति सर्वतः समारोपितस्तम इति प्रतीयते~।\footnote{न्या.कं. ३३,३४}}


		\subsection{मेयान्तरं तमः}

		तमसः नीलरूपवत्त्वात् क्रियादिमत्त्वाच्च मेयान्तरत्वव्यवस्थापयन्ति वल्लभाचार्याः~। तथा हि~- भगवता सूत्रकारेण कणादेन धर्मविशेषप्रसूतादित्यादिसूत्रे पदार्थानां सर्वेषां निबन्धनं नाकारि~। अन्यथा अभावानभिधानात्मकः दोषः स्यात्~। अपि तु भावपदार्थानामेव~। तमस्तु मेयान्तरमेव~। अतः तन्निबन्धनं न कृतम्~। ननु आलोकाभाव एव तमः, आलोके सति नानुभूतये, तदभावे चानुभूयते इति अन्वयव्यतिरेकस्य विद्यमानत्वादिति वाच्यम्~। नञर्थोल्लेखेन न प्रतीयते तमः~। यदि भासामभावः‌ स्यात् तदा 'इदानीं न प्रभा' इति नञर्थोल्लेखेनैव तस्य प्रतीतिः स्यात्~। न च आरोपितं नीलरूपमेव तमः इति वाच्यम्~। बाधकं विना आरोपानुपपत्तेः~। न च आलोकाभावदशायां चाक्षुषत्वमेव बाधकम्~, न हि भावस्तदानीं गृह्यते  इति वाच्यम्~। विलक्षणमेव मेयान्तरं तमः यद्ग्रहे चक्षुरालोकं नापेक्षते~। अपि तु आलोकाभाव एव अस्य व्यञ्जकः~। न च तमसि वास्तवनीलरूपवत्त्वे आलोकाभावे तमसः चाक्षुषत्वं बाधकम्~। शुक्लरूपेतररूपवद्द्रव्यस्य प्रत्यक्षं प्रति आलोकसहकृतचक्षुषः अन्वयव्यतिरेकाभ्यां कारणत्वकल्पनात्~। 'नीलं तमः' इत्यादिप्रतीतिस्तु भ्रमात्मिका इति वाच्यम्~। आलोकाभाववादिमतेऽपि तादृशप्रतीतेः भ्रमत्वसम्पादनार्थं नीलरूपारोपो अभ्युपेयः~। स च न सम्भवति~। आलोकाभावात् आरोपाधिकरणस्य च अप्रत्यक्षत्वात्~। न च तमसः नीलरूपवत्त्वात् दशमद्रव्यत्वमस्तु इति वाच्यम्~। नवैव द्रव्याणि इति यदुद्दिष्टं लक्षितं परीक्षितं च शास्त्रे तस्य व्याघातः स्यात्~। तर्हि नवस्वेवास्य अन्तर्भावोऽस्तु इत्यपि न चिन्तनीयम्~। गन्धस्पर्शशून्यत्वात् नीलरूपवत्त्वाच्च न पृथिव्यादिनवद्रव्येषु अस्यान्तर्भावः~। एवं न गुणस्तमः~। चतुर्विंशतिगुणानां प्रतिज्ञातत्वात् तद्व्याघातप्रसङ्गः, गुणे गुणानङ्गीकारच्च~। तस्मात्तमः षड्भावपदार्थातिरिक्तः भावपदार्थः आलोकनिरपेक्षचक्षुर्ग्राह्यत्वादिति सिद्धम्~। एतेन षडतिरिक्तस्य अभावपदार्थस्य सत्त्वान्न सिद्धसाधनमिति~।

		{\fontsize{11.7}{0}\selectfont\s तमस्तु मेयान्तरं निषेधत्वेनानवभासमानत्वात्~। बाधकाभावेन चारोपानुपपत्तेः आलोकाभावे चाक्षुषत्वं नास्तीति बाधकमिति चेन्न, तस्यालोकाभावव्यञ्जनीयत्वात्~। अन्यथारोपानुपपत्तेः~। भावत्वे यदि द्रव्यान्तरं नवैवेति व्याघातः अद्रव्यान्तरत्वं सर्ववादिनिषिद्धम्~। अथ गुणान्तरं चतुर्विंशतित्वव्याघात इति मेयान्तरमेव तमः~। अत्रैव सङ्ग्रहः श्लोकः~-\\ नाभावोभाववैधर्म्यान्नारोपो बाधहानितः~।\\ द्रव्यादिषट्कवैधर्म्याज्ज्ञेयं मेयान्तरं तमः~॥\footnote{न्या.ली.१८-२०}}


		\subsection{भाभावस्तमः}

		भासामभाव एव तमः इति किरणावल्यामाचार्याः प्रतिपादयामासुः~। तथा हि~- तमसः द्रव्यगुणकर्मसु अनन्तर्भावं विशदतया प्रतिपाद्य भाभावस्तमः इति न्यरूपि~। न च तमसः भाभावरूपत्वे तस्य चाक्षुषत्वोपपादनार्थं प्रतियोगिस्मरणमधिकरणस्य प्रत्यक्षत्वं चापेक्षितम्~। अन्यथा अभावस्यापि अप्रत्यक्षत्वादिति वाच्यम्~।‌ यस्य साक्षात्कारे यादृशसामग्रीविशिष्टमिन्द्रियमपेक्ष्यते तदभावग्रहेऽपि तादृशसामग्रीविशिष्टमेव इन्द्रियमपेक्ष्यते इति नियमस्य सत्त्वात्~। तथा च आलोकग्रहे आलोकनिरपेक्षचक्षुषः एव कारणत्वात् तदभावग्रहेऽपि आलोकनिरपेक्षं चक्षुरेव कारणम्~। आलोकाभावविशिष्टेन्द्रियस्य तद्ग्राहकत्वाभ्युपगमादेव अधिकरणसाक्षात्कारं विना तत्प्रत्यक्षत्वमुपपद्यते~। इन्द्रियस्य तादृशसामर्थ्यकल्पनात्~। दिवा च गगनादौ नीलिमप्रतीतिः प्रभातिरिक्तदेशेनैव भवति~। अत एव अतिदूरे नभसि नीलिमप्रतीतिरिति तमसः भाभावरूपत्वे न किञ्चिद्बाधकमिति~।

		{\fontsize{11.7}{0}\selectfont\s द्रव्यगुणकर्मनिष्पत्तिवैधर्म्याद्भाभावस्तम इति~। सोऽपि कथमालोकमन्तरेण प्रतियोगिस्मरणाधिकरणग्रहणविरहे विधिमुखेन चाक्षुष इति चेन्न हि यद्ग्रहे यदपेक्षं चक्षुः तदभावग्रहेऽपि तदपेक्षते~। एवं हि तदितरसामग्रीसाकल्यं स्यात्~। तदालोकाभावेऽप्यालोकापेक्षा स्यात्~, यद्यालोके तदपेक्षा स्यात्~। न त्वेतदस्ति~। प्रत्युत विरोध एव~। तस्मिन् सति तदभाव एव न स्यात्~, किं तदपेक्षेण चक्षुषा गृह्यते~। दिवा च प्रतियोगिनः प्रभामण्डलस्य ग्रहण एव प्रदेशान्तरे तद्ग्रह इति न किञ्चिदनुपपन्नम्~।\footnote{कि. ९८-१०२}}

	\section{विमर्शः}

		\subsection{तमसः भावरूपत्वविमर्शः}

		ननु 'नीलं तमः श्चलति' इत्यादिप्रतीतेः अन्धकारादौ सम्भवात् तमसि नीलरूपक्रियादिदर्शनात् तस्यापि द्रव्यत्वं सिध्यति~। अत्र यद्यपि 'नीलं तमः' इति प्रतीत्या एव तमसि नीलरूपवत्त्वात् द्रव्यत्वं सिध्यति तथापि वह्न्यादितेजसि औपाधिकपीतरक्तादिरूपभ्रमस्य सर्ववादिसिद्धतया प्रकृतेऽपि रूपवत्ताप्रतीतौ भ्रमत्वकल्पनसम्भवात्~, क्रियाप्रकारकप्रतीतौ तु तादृशभ्रमादर्शनात् 'तमः श्चलति' इति प्रतीतेः ग्रहणम्~। न च वेगवतः पुंसः स्थिरे वृक्षादौ भ्रमरूपा क्रियावत्ताप्रतीतिरनुभूयते इति वाच्यम्~। तत्र क्रियावत्ताप्रतीतिस्तु न औपाधिकी, अपि तु पुरुषदोषवशादुत्पद्यमाना~। तस्मात् क्रियावत्त्वात्तमसः न मूर्तद्रव्यातिरिक्तद्रव्यत्वसम्भवः~। नीलरूपवत्त्वात् न पृथिवीतरत्~, गन्धाभावाच्च न पृथिवीति तस्यातिरिक्तत्वमपि सिध्यति~। 

		न च तमो यदि रूपवद्द्रव्यं स्यात् तर्हि रूपवद्द्रव्यस्य स्पर्शवत्त्वनियमात् तस्मिन् स्पर्शोऽपि स्यात्~। स्पर्शवत्वाभ्युपगमे तु महतः स्पर्शवद्द्रव्यस्य प्रतिघातकत्वनियमात् भित्तिवत्तमसः अपि चलनादौ प्रतिबन्धः स्यात्~। किन्तु तमसि तादृशप्रतिबन्धकत्वाभावात् तन्न द्रव्यम्~। महतः स्पर्शवद्द्रव्यस्य प्रतिघातकत्वनियमो व्यभिचरतः~। अन्यथा आलोकदशायामपि प्रतिघातः स्यात्~।

		नाभावस्तमः~। घटाभावस्य यथा प्रतियोगिमन्तरा न व्यवहारः तद्वत् तमसः अपि स्वप्रतियोगिमन्तरा व्यावहारानुपपत्तिः~।  किञ्चास्य अभावत्वे तु तेजोभावरूपत्वमेव वक्तव्यम्~। तेन सह विरोधदर्शनात्~। तत्र किं तमः तेजसः अन्योन्याभावः~? उत संसर्गाभावः~? आद्ये भास्करकरनिकराक्रान्तेषु प्राङ्गणादिषु तमोव्यवहारापत्तेः, तत्र तेजोभेदस्य सत्त्वात्~। द्वितीये किं यत्किञ्चित्प्रतियोगिकाभावः उत तेजस्सामान्याभावः~। आद्ये यत्किञ्चित्तेजःप्रतियोगिकसंसर्गाभावस्य तेजोधिकरणेऽपि सत्त्वात् तत्र तमःप्रतीत्यापत्तिः~। द्वितीये तु तमः प्रतीतिकालेऽपि यत्किञ्चित्तेजसः सत्त्वात् तमःप्रतीत्यनुपपत्तिः~। नापि तेजोविशेषाभावः~। अभावज्ञानं प्रति प्रतियोगिज्ञानस्य कारणत्वात् तादृशानां तमःप्रतियोगितेजोविशेषाणां तमःप्रतीतेः पूर्वमज्ञानात्~। तस्मात्तमो न तेजोभावरूपः~। अपि तु अन्य एव~। 

		तथा च मीमांसकाः~- {\fontsize{11.7}{0}\selectfont\s 'गुणकर्मादिसद्भावादस्तीति प्रतिभासतः~। प्रतियोग्यस्मृतेश्चैव भावरूपं ध्रुवं तमः~॥'\footnote{मा.मे. १५२}} इति~। किञ्च {\fontsize{11.7}{0}\selectfont\s 'तमः कृष्णं व्यक्थमस्थित'\footnote{कृ.य.ब्रा. }} इति श्रुतिप्रमाणस्य सत्त्वात् तमः अतिरिक्तं द्रव्यमिति वदन्ति~।

		वेदान्तिनस्तु {\fontsize{11.7}{0}\selectfont\s 'यदि तावत् सहानवस्थानलक्षणो विरोधः, ततः प्रकाशभावे तमसो भावानुपपत्तिः, तदसत्~; दृश्यते हि मन्दप्रदीपे वेश्मनि अस्पष्टं रूपदर्शनं, इतरत्र च स्पष्टम्~। तेन ज्ञायते मन्दप्रदीपे वेश्मनि तमसोऽपि ईषदनुवृत्तिरिति~; तथा छायायामपि औष्ण्यं तारतम्येन उपलभ्यमानं आतपस्यापि तत्र अवस्थानं सूचयति~। एतेन शीतोष्णयोरपि युगपदुपलब्धेः सहावस्थानमुक्तं वेदितव्यम्~।'\footnote{पं.पा.}} इति वदन्ति~।
		  
		केचित्तु~- नीलाकारेण प्रतीयमानस्य तमसः अभावरूपत्वं नैव सम्भवति~। मध्यन्दिने तेजःसद्भावेऽपि दूरगगनव्यापिनः नीलिम्नः तमसः प्रतिभानात् न तस्य तेजोभावरूपत्वम्~। किञ्च यदि तेजोभाव एव तमः स्यात् तस्याप्रत्यक्षत्वप्रसङ्गः~। कथम्~? अभावो हि इन्द्रियसंयुक्ततदधिकरणविशेष्यतया विशेषणतया वा प्रतीयते~। न स्वतन्त्रः~। तमः प्रत्ययस्थले अन्यस्य कस्यापि अप्रतीतेः~। एवं तमः तेजोभाव एव इति प्रतीतिरपि न पश्चादुदेति~। तस्मान्नाभावेदम्~। न च आलोकादर्शनं यदा तदा तम इति प्रतीतिः~। तस्मादालोकादर्शनाभाव एव तमः इति वाच्यम्~। तमः, छाया इत्यादिना कृष्णाकारस्य प्रतिभानात्~। तस्मात् प्रौढप्रकाशकतेजसः अभावे सति प्रतीयमानः  सर्वतः समारोपितः  नीलरूपविशेष एव तमःपदार्थः इति~।

		अन्ये तु~- तमसः नीलरूपवत्त्वात् क्रियादिमत्त्वाच्च मेयान्तरमिति वदन्ति~। तथा हि~- भगवता सूत्रकारेण कणादेन धर्मविशेषप्रसूतादित्यादिसूत्रे पदार्थानां सर्वेषां निबन्धनं नाकारि~। अन्यथा अभावानभिधानात्मकः दोषः स्यात्~। अपि तु भावपदार्थानामेव~। तमस्तु मेयान्तरमेव~। अतः तन्निबन्धनं न कृतम्~। ननु आलोकाभाव एव तमः, आलोके सति अननुभूयते, तदभावे चानुभूयते इति अन्वयव्यतिरेकस्य विद्यमानत्वादिति वाच्यम्~। नञर्थोल्लेखेन न प्रतीयते तमः~। यदि भासामभावः‌ स्यात् तदा 'इदानीं न प्रभा' इति नञर्थोल्लेखेनैव तस्य प्रतीतिः स्यात्~। न च आरोपितं नीलरूपमेव तमः इति वाच्यम्~। बाधकं विना आरोपानुपपत्तेः~। न च आलोकाभावदशायां चाक्षुषत्वमेव बाधकम्~, न हि भावस्तदानीं गृह्यते  इति वाच्यम्~। विलक्षणमेव मेयान्तरं तमः यद्गृहे चक्षुरालोकं नापेक्षते~। अपि तु आलोकाभाव एव अस्य व्यञ्जकः~। न च तमसि वास्तवनिलरूपवत्त्वे आलोकाभावे तमसः चाक्षुषत्वं बाधकम्~। शुक्लरूपेतररूपवद्द्रव्यस्य प्रत्यक्षं प्रति आलोकसहकृतचक्षुषः अन्वयव्यतिरेकाभ्यां कारणत्वकल्पनात्~। 'नीलं तमः' इत्यादिप्रतीतिस्तु भ्रमात्मिका इति वाच्यम्~। आलोकाभाववादिमतेऽपि तादृशप्रतीतेः भ्रमत्वसम्पादनार्थं नीलरूपारोपो अभ्युपेयः~। स च न सम्भवति~। आलोकाभावात् आरोपाधिकरणस्य च अप्रत्यक्षत्वात्~। न च तमसः नीलरूपवत्त्वात् दशमद्रव्यत्वमस्तु इति वाच्यम्~। नवैव द्रव्याणि इति यदुद्दिष्टं लक्षितं परीक्षितं च शास्त्रे तस्य व्याघातः स्यात्~। तर्हि नवस्वेवास्य अन्तर्भावोऽस्तु इत्यपि न चिन्तनीयम्~। गन्धस्पर्शशून्यत्वात् नीलरूपवत्त्वाच्च न पृथिव्यादिनवद्रव्येषु अस्यान्तर्भावः~। एवं न गुणस्तमः~। चतुर्विंशतिगुणानां प्रतिज्ञातत्वात् तद्व्याघातप्रसङ्गः, गुणे गुणानङ्गीकारच्च~। तस्मात्तमः षड्भावपदार्थातिरिक्तः भावपदार्थः आलोकनिरपेक्षचक्षुर्ग्राह्यत्वादिति सिद्धम्~। एतेन षडतिरिक्तस्य अभावपदार्थस्य सत्त्वान्न सिद्धसाधनमिति~।

		\subsection{तेजोभावस्य तमस्त्वविमर्शः}

		अत्रोच्यते~- उद्भूतरूपविशिष्टस्य तेजसः अभावे सति तमःप्रत्ययात्~, तदागमने च तमःप्रत्ययाभावात् लाघवाच्च तादृशस्य प्रौढप्रकाशकतेजस्सामान्याभावस्य एव तमःप्रत्ययवाच्यत्वम्~। न त्वतिरिक्तं द्रव्यं तमः~। न च यत्किञ्चित्तेजसः तमःप्रदेशे सत्त्वात् तेजस्सामान्याभावस्य तमस्त्वानुपपत्तिरिति वाच्यम्~। उद्भूतरूपविशिष्टस्य तेजसः तमःप्रत्ययकालेऽप्यभावात्~। अन्यथा तेजस्तमसोः यो विरोधः तस्यानुपपत्तिः~। अत एव दिवापि गगने अतिदूरदेशे नैल्यप्रतीतिरेव उदेति, न तु समीपदेशे~। 
		 
		अधिकरणप्रत्यक्षं विना अभावो न गृह्यते~। अधिकरणेन सह इन्द्रियसन्निकर्षे सति तत्र विशेषणतया अभावस्य ग्रहणं सम्भवति~। तमःप्रत्यक्षस्थले नान्यस्य कस्यचिद्वस्तुनः प्रत्यक्षात् तमो नाभावः इति यदुक्तं तदप्यसत्~। अधिकरणप्रत्यक्षं विनापि अभावप्रत्यक्षसम्भवात्~। अधिकरणेन सह इन्द्रियसम्बन्धमात्रमपेक्षितम्~। अन्यथा श्रोत्रेण शब्दाभावस्याप्रत्यक्षत्वप्रसङ्गः~। शब्दाधिरणस्याकाशस्यातीन्द्रियत्वात्~। अधिकरणेन सह इन्द्रियसम्बन्धस्तु तमःप्रत्यक्षकालेऽप्यस्त्येव~।

		यदुक्तं 'नीलं तमः' इति प्रत्ययानन्तरं तद्बाधकप्रत्ययान्तरस्य कस्याप्यनुदयात् बाधकाभावात् तस्य प्रमात्वं स्वीकरणीयम्~। अन्यथा 'नीलो घटः' इति प्रत्ययस्यापि भ्रमत्वापत्तिरिति~। तदप्यसारम्~। गुणशून्ये तेजोऽभावे नीलरूपवत्त्वमेवात्र बाधकम्~। न च तस्य कथं तेजोऽभावत्वमिति वाच्यम्~। तेन सह विरोधात्~। किञ्च तस्य द्रव्यत्वे तत्परमाणावः स्पर्शवन्तो न वा~? आद्ये तमस्यपि स्पर्शोपलब्धिप्रसङ्गः, स्पर्शवतः नीलरूपविशिष्टद्रव्यस्य सर्वत्र इन्द्रियसन्निकर्षविघटकत्वं लोके दृष्टम्~। किन्तु तमसि स्थितस्यापि वर्णरञ्जितमञ्चके नाटकादीनां दर्शनानुभवात्तस्य इन्द्रियसन्निकर्षाविघटकत्वात् नेदं स्पर्शवद्द्रव्यारब्धमिति ज्ञायते~। द्वितीये तु अस्पर्शवतः वस्तुनः कार्यद्रव्यानरम्भकत्वात् तमसः अपि उत्पत्तिरेव न स्यात्~। न च मनोवदणुपरिमाणकमिदमिति वाच्यम्~। प्रत्यक्षत्वात्~। अत एव न विभुपरिमाणकम्~। तस्मादस्य द्रव्यत्वासिद्धौ तेजसा सह विरोधात् तदभावरूपत्वमेव स्वीकरणीयम्~। तस्मादस्य अभावत्वमेव रूपवत्ताप्रतीतौ बाधकम्~। एतेन तमः मेयान्तरमित्यपि निरस्तम्~।

		न च नीलरूपस्य तदा भानात् आरोपितं नीलरूपमेव तमः इति वाच्यम्~। अधिकरणदर्शनं विना आरोपितस्य वस्तुनः साक्षात्कारासम्भवात्~। किञ्च तमसि आलोकासहकृतचक्षुर्ग्राह्यत्वात् नीलरूपाभावोऽपि सिध्यति~। न च चाक्षुषं प्रति आलोकसहकारस्य कारणत्वं व्यभिचरितम्~। तथा हि~- {\fontsize{11.7}{0}\selectfont\s 'दिवान्धाः प्राणिनः केचिद्रात्रावन्धास्तथापरे~। केचिद्दिवा तथा रात्रौ प्राणिनस्तुल्यदृष्टयः~॥'\footnote{सप्तशती}} इति वचनात् अनुभवाच्च केचन प्राणिनः रात्रावपि पश्यन्ति~। केचित् प्राणिनः दिवापि न पश्यन्ति इति सिध्यति~। तथा च आलोकाभावेऽपि चाक्षुषत्वसम्भवात् आलोके सत्यपि चाक्षुषत्वासम्भवात् अन्वयव्यतिरेकव्यभिचारो दृष्टः इति वाच्यम्~। नक्तञ्चराणां चक्षुरिन्द्रियस्यैव आलोकात्मकत्वात्~। तदुक्तं~- 'नक्तञ्चरनयनरश्मिदर्शनाच्च'\footnote{न्या.सू. ३.१.४३} इति सूत्रव्याख्यानावसरे न्यायभाष्ये {\fontsize{11.7}{0}\selectfont\s 'दृश्यते हि नक्तं नयनरश्मयो नक्तञ्चराणां वृषदंशप्रभृतीनाम्~, तेन शेषस्यानुमानमिति'\footnote{न्या. भा. २५४}} इति~। तथा च यत्र येषां प्राणिनां रात्रावेव चाक्षुषप्रतीतिः तेषामिन्द्रियाणामालोकात्मकत्वात् आलोकत्वेन रूपेण तत्संयोगस्य कारणत्वम्~। इन्द्रियत्वेन रूपेण च अपरा कारणता इति~। किञ्च प्रत्यक्षं प्रति विलक्षणालोकसंयोगस्य हेतुत्वन्तु अवश्यमभ्युपेयम् इन्द्रियसामर्थ्यात्~। अत एव कदाचिदत्युत्कटप्रभासंयुक्तपदार्थेन इन्द्रियसम्बन्धे सति कदाचित् मन्दालोकदशायाञ्च अस्माकमपि चाक्षुषप्रतीतिः नोदेति~। एवञ्च केषाञ्चित्प्राणिनां आलोकसामान्ये सति इन्द्रियसामर्त्यवशात् चाक्षुषप्रत्ययदर्शनात् तेषां दिवा रात्रौ च चाक्षुषप्रतीतिः सम्भवति~। अस्मदृशानान्तु आलोकाभावात् रात्रौ न चाक्षुषप्रत्ययः~। वृषदंशादीनान्तु चक्षुषः एव आलोकात्मकत्वात् मन्दालोकसंयोगस्यैव प्रत्यक्षजनकत्वात् रात्रावैव प्रत्यक्षप्रतीतिरिति न किञ्चिदनुपपन्नम्~। 

		एतेन मन्दप्रदीपे वस्तुनः अदर्शनात् तत्रापि प्रतिबन्धकीभूतस्य तमसः सत्त्वमभ्युपगन्तव्यम्~। तथा च तमआलोकयोः परस्परविरोधाभावोऽप्यनेन सिध्यति~। एवञ्च विरोधाभावान्न तेजोभावस्तमः इत्यपि निरस्तम्~। विलक्षणतेजोभावस्यैव तमः पदार्थत्वात्~। तथैव लोके अनुभवात्~। अत एव प्रौढप्रकाशकतेजोभाव इत्यत्र प्रकाशे प्रौढत्वं तत्तत्तमो विरोधिरूपं विलक्षणमेव ग्राह्यम्~।

		किञ्च तेजोभावत्वेऽपि तमसः आलोकासहकृतचक्षुषा ग्रहणं कथमिति चेत्~, यस्य साक्षात्कारे यादृशसामग्रीविशिष्टमिन्द्रियमपेक्ष्यते तदभावग्रहेऽपि तादृशसामग्रीविशिष्टमेव इन्द्रियमपेक्ष्यते इति नियमात्~। तथा च आलोकग्रहे आलोकनिरपेक्षचक्षुषः एव कारणत्वात् तदभावग्रहेऽपि आलोकनिरपेक्षं चक्षुरेव कारणम्~। आलोकाभावविशिष्टेन्द्रियस्य तद्ग्राहकत्वाभ्युपगमादेव अधिकरणसाक्षात्कारं विना तत्प्रत्यक्षमुपपद्यते~। इन्द्रियस्य तादृशसामर्थ्यकल्पनात्~।

		एवंरीत्या तस्य भाभावरूपत्वसिद्धौ तस्य द्रव्यत्वाभावादेव तस्मिन् नीलरूपादिकं न सिध्यति~। नीलरूपादीनामभावे तु नीलादिप्रतीतीनां पित्तदोषवतः 'शङ्खः पीतः' इति प्रतीतिवत् भ्रान्तत्वं सिद्धम्~।


\begin{center}\begin{small}॥ इति प्राचीनन्याय-वैशेषिक-नव्यन्यायशास्त्रेषु तत्तद्व्याख्याकाराणां सैद्धान्तिकमतभेदानां विमर्शात्मकमध्ययनमिति प्रबन्धे  द्रव्यप्रपञ्चे सैद्धान्तिकमतभेदाः इत्याख्यः द्वितीयोऽध्यायः~॥\end{small}\end{center}






\titleformat {\chapter}[display]{\normalfont\Large} % format
{अथ तृतीयोऽध्यायः\\[1mm]} % label
{-3.8ex}{ \rule{\textwidth}{1pt}\vspace{-5ex}
\centering
} % before-code
[
\vspace{-6.7ex}%
\rule{\textwidth}{1pt}
]
\titlespacing*{\chapter} {10pt}{-60pt}{50pt}


\chapter{गुणप्रपञ्चे सैद्धान्तिकमतभेदाः}

	रूपरसादिभेदेन चतुर्विंशतिसङ्ख्याकाः गुणाः प्रमाणसिद्धाः इत्यत्र नास्ति विप्रतिपत्तिः~। तत्र रूपरसगन्धस्पर्शाणां पृथिव्यां पाकजत्वमिति दार्शनिकसमयः~। अत्रापि पाकवशात् परमाणुष्वेव रूपादय उत्पद्यन्ते, न तु द्व्यणुकादाविति पीलुपालवादिनो वैशेषिकाः आमनन्ति~। द्व्यणुकाद्यवयविष्वपि पाकाद्रूपादीनामुत्पत्तिरिति पिठरपाकवादिनो नैयायिकाः~। तदत्र निरूप्यते~-

	\section{पाकाधिकरणविषये मतवैलक्षण्यम्}

	नवशरावादौ अग्निसंयोगात् पूर्वरूपादीनां नाशः रूपान्तरस्य उत्पत्तिः लोके दृष्टः~। तच्च परिवर्तनं पीलावेव भवति इति पीलुपाकवादिनो वैशेषिकाः~। 

	तथा हि~- तेजस्संयोगात् अवयविनाशात् स्वतन्त्रेषु परमाणुषु पूर्वरूपादीनां नाशः, पुनस्तेजस्संयोगात् रूपातन्तरोत्पत्तिरिति वैशेषिकानामाशयः~। तत्र रूपातन्तरोत्पत्तौ पञ्चषडष्टनवदशादिक्षणाः अपेक्षिताः इति नानाप्रक्रियाः वैशेषिकग्रन्थेषु उपलभ्यन्ते~। तेष्वष्टसु क्षणेषु रूपान्तरोत्पत्तिः अवयविनि जायते इति प्रसिद्धं वर्तते~। अग्निसंयोगवशात् परमाणौ द्व्यणुकनाशानुकूलक्रिया, ततः पूर्वदेशविभागः तदुत्तरं संयोगनाशः, तस्मात् असमवायिकारणनाशात् द्व्यणुकनाशो जायते~। इदानीं स्वतन्त्रेषु परमाणुषु प्रथमक्षणे पाकवशात् श्यामरूपस्य नाशो जायते~। ततः द्वितीयक्षणे पाकवशात् रक्तोत्पत्तिर्भवति~। ततः तृतीयक्षणे अदृष्टवदात्मसंयोगात् परमाणुषु द्व्यणुकारम्भकक्रिया उत्पद्यते~। ततः‌ चतुर्थे पूर्वदेशविभागो जायते~। ततः पञ्चमे पूर्वसंयोगनाशो भवति~। ततः षष्ठे परमाणुद्वयसंयोगो उत्पद्यते~। ततः सप्तमे असमवायिकारणवशात् द्व्यणुकमुत्पद्यते~। अष्टमे कारणगुणाः कार्यगुणानारभन्ते इति न्यायानुसारं द्व्यणुके परमाणुगतरक्तरूपवशात् रक्तोत्पत्तिर्भवति इति वैशेषिकसिद्धान्तः~।

		\subsection{पार्थिवपरमाणुषु पाकजानामुत्पत्तिप्रदर्शनम्}

		तत्र हि घटादिगतपरमाणौ द्व्यणुकारम्भकसंयोगनाशकविभागजनिका क्रिया उत्पद्यते~। इमां क्रियां प्रति अग्न्यभिघातः\footnote{अभिघातः शब्दजनकसंयोगः} नोदनं\footnote{नोदनं शब्दाजनकसंयोगः} वा कारणम्~। न च केवलावयविषु अग्निसंयोगस्यादौ दर्शनात् तत्रैव तादृशाग्निसंयोगात् रूपनाशः रूपान्तरोत्पत्तिश्च जायतामिति वाच्यम्~। अतितीक्ष्णस्याग्नेः क्षणाभ्यन्तरे सर्वावयवेन सह संयोगस्य अभ्युपगमात् केवलावयविवृत्तिरूपपरावर्तकत्वस्याग्नेः कल्पनासम्भवात्~। अवयविनि अपि रूपपरावर्तनं भवतु इत्यपि नाशङ्कनीयम्~। अवयवसंयोगादग्नेः अवयवेषु परमाणुषु क्रियोत्पत्त्या द्व्यणुकादिनाशक्रमेण अवयविनः एव नाशात् आश्रयाभावादेव तत्र न रूपपरावृत्तिः जायते~। तस्मात् अवयविनाशः अवश्यमभ्युपगन्तव्यः~। ततः स्वतन्त्रेषु परमाणुषु अन्यस्मादग्निसंयोगात् पूर्वरूपस्य श्यामादेः नाशो जायते~। ततः अन्यस्मादग्निसंयोगात् रूपान्तरं रक्तादिकमुत्पद्यते~। अत्र श्यामादिनाशकः एव अग्निसंयोगः न रूपाद्युत्पादकः‌~। तथा सति उत्पादकस्यैव नाशकत्वप्रसङ्गः~। ततः द्व्यणुकारम्भकसंयोगजनिका क्रिया परमाणुषु उत्पद्यन्ते~। तत्र अदृष्टवदात्मसंयोग एव कारणम्~। केवलात्मसंयोगस्य तद्धेतुत्वे यदा कदापि परमाणुषु क्रियोत्पत्तिप्रसङ्गः~। आत्मनः विभुत्वात् तेन सह मूर्तानां परमाणूनां संयोगः सर्वदा वर्तत एव~। अतः अदृष्टविशिष्टस्यात्मनः इत्युक्तम्~। केवलादृष्टस्य तु क्रियासाधारणकारणत्वाभावात् आत्मसंयोगः अवश्यमभ्युपगन्तव्यः~। ततः विभागादिक्रमेण द्व्यणुकोत्पत्तौ तत्र परमाणुगतरक्तरूपादिना रूपाद्युत्पत्तिरिति~। अवयविगतरूपं प्रति अवयवगतरूपस्य स्वसमवायिसमवेतत्त्वसम्बन्धेन कारणत्वाभ्युपगमात् पटे स्वसमवायिसमवेतत्त्वसम्बन्धेन तन्तुरूपस्य श्वेतादेः विद्यमानत्वात् पटेऽपि समवायेन श्वेतोत्पत्तिदर्शनाच्च परमाणुगतरूपस्यैव द्व्यणुकगतरूपं प्रति असमवायिकारणत्वं स्वीक्रियते~। न तु पाकस्य~। अत एव 'कारणगुणाः कार्यगुणानारभन्ते' इति न्यायोऽपि सङ्गच्छते~। तथा च प्रशस्तपादभाष्यम्~-

		{\fontsize{11.7}{0}\selectfont\s पार्थिवपरमाणुरूपादीनां पाकजोत्पत्तिविधानम्~। घटादेः आमद्रव्यस्याग्निना सम्बद्धस्याग्न्यभिघातान्नोदनाद्वा तदारम्भकेष्वणुषु कर्माण्युत्पद्यन्ते~। तेभ्यो विभागा~, विभागेभ्यः संयोगविनाशाः~, संयोगविनाशेभ्यश्च कार्यद्रव्यं विनश्यति~। तस्मिन् विनष्टे स्वतन्त्रेषु परमाणुष्वाग्निसंयोगादौष्ण्यापेक्षाच्छ्यामादीनां विनाशः~, पुनरन्यस्मादग्निसंयोगादौष्ण्यापेक्षात् पाकजा जायन्ते~। तदनन्तरं भोगिनामदृष्टापेक्षादात्माणुसंयोगादुत्पन्नपाकजेष्वणुषु कर्मोत्पत्तौ तेषां परस्परसंयोगाद् द्व्यणुकादिक्रमेण कार्यद्रव्यमुत्पद्यते~। तत्र च कारणगुणप्रक्रमेण रूपाद्युत्पत्तिः~।  न च कार्यद्रव्य एव रूपाद्युत्पत्तिर्विनाशो वा सम्भवति~, सर्वावयवेष्वन्तर्बहिश्च वर्तमानस्याग्निना व्याप्त्यभावात्~। अणुप्रवेशादपि च व्याप्तिर्न सम्भवति~, कार्यद्रव्यविनाशादिति~।\footnote{प्र.भा. २६५ - २७०}}

		\subsection{अवयविषु पाकोत्पत्तिनिराकरणपूर्वकं स्वमतस्थापनम्}

		नन्वेवमपि अवयविनि अग्नोसंयोगस्य सम्भवात् तत्रापि पाकादेव रूपपरावर्तनं भवतु~। यतः दृश्यन्ते च अग्नौ प्रक्षिप्तस्यामघटस्य छेदनाभावः, एवं पाकोत्तरमपि 'स एवायं घटः' इति प्रत्यभिज्ञानदर्शनात् अवयविन्यपि पाकाद्रूपपरावृत्तिर्भवति इति चेन्न~। अवयवित्वं तावत् घटस्यान्तःप्रदेशेऽपि विद्यते, तत्प्रदेशस्यापि अवयवसमवेतत्वात्~। तत्र बहिर्विद्यमानस्याग्नेः संयोगासम्भवात् अवयविनि सर्वत्र अग्निसंयोगो न व्याप्तः~। तस्मात् कारणाभावादेव अवयविषु पाकानुत्पत्तिः सिद्ध एव~। 

		ननु अवयव्यवष्टब्धेष्वेवावयवेषु पाकाद्रूपोत्पत्तेरवयविनाशो न कल्पनीयः~। अवयवसंयोगस्य एकदेशे विद्यमानत्वेऽपि देशान्तरावच्छेदेन अग्निसंयोगजननसम्भवात् अवयव्यवष्टब्धेष्वेव अवयवेषु अग्निसंयोगात् रूपादिपरावृत्तिसम्भवात् अवयविनाशो न कल्पनीय इति वाच्यम्~। यद्यपि स्थूलद्रव्यावयवेषु एतादृशस्यान्तरस्य अप्रतिभानात् परमाण्वादिष्वेव तत्कल्पनीयम्~। तत्र च एतादृशातिरिक्तांशकल्पनापेक्षया अवयविनाशकल्पने एव लाघवम्~। एवं क्रियाविशिष्टाग्निसंयोगात् अवयवेषु क्रियानुत्पत्तिरपि अनुभवविरुद्धं कल्पनीयम्~। तदपेक्षया अवयविनाशकल्पनमेव वरमिति स्वतन्त्रेष्वेव परमाणुषु पाकाद्रूपपरावृत्तिरिति~। तदेवाभिहितं कन्दल्याम्~-

		{\fontsize{11.7}{0}\selectfont\s अथ कथं कार्यद्रव्ये एव रूपादीनामग्निसंयोगादुत्पादविनाशौ कल्प्येते~? प्रतीयन्ते हि पाकार्थमुपक्षिप्ता घटादयः सर्वावस्थासु प्रत्यक्षाश्छिद्रविनिवेशितदृशाः~, प्रत्यभिज्ञायन्ते च पाकोत्तरकालमपि त एवामी घटादय इति चेत्~। तत्रोच्यते अन्तर्बहिश्च सर्वेष्वयवेषु वर्तमानस्य समवेतस्यावयविनो बाह्ये वर्तमानेन  वह्निना व्याप्तेर्वापकस्य संयोगस्याभावात् कार्यरूपादीनामुत्पत्तिविनाशयोरक्लृप्तेरन्तर्वर्तिनामपाकप्रसङ्गादिति भावः~। सच्छिद्राण्येवावयविद्रव्याणि~। न तावत्परमाणवः सान्तराः~, निर्भागत्वात्~। द्व्यणुकस्य सान्तरत्वे चानुत्पत्तिरेव~, तस्य परमाण्वोरसंयोगात्~। संयुक्तौ चेदिमौ निरन्तरावेव~। सभाग्ययोर्हि वस्तुनोः केनचिदंशेन संयोगात् केनचिदसंयोगात् सान्तरः संयोगः~। निर्भागयोस्तु नायं विधिरवकल्पते~। स्थूलद्रव्येषु प्रतीयमानेष्वन्तरं न प्रतिभात्येव~, त्रणुकेष्वेवान्तरम्~। तच्चानुपलब्धियोग्यत्वान्न प्रतीयत इति गुर्वीयं कल्पना~। तस्मान्निरन्तरा एव घटादयः~। तेषामन्तस्तावदग्निप्रवेशो नास्ति यावत्पार्थिवावयवानां व्याप्ति भेदो न स्यात्~। स्पर्शवति द्रव्ये तथाभूतस्य द्रव्यान्तरस्य प्रतिघाताद् व्यतिभिद्यमानेषु चावयवेषु क्रियाविभागादिन्यायेन द्रव्यारम्भकसंयोगविनाशादवश्यं द्रव्यविनाश इति कुतस्तस्याणुप्रवेशादभिव्यक्तिः~।\footnote{न्या.कं. २६९ - २७१}}

	पाकजाः रूपादयः पृथिव्यवयविषु अपि जायन्ते~। अवयवनाशो नानुभवसिद्धः~। पाकजरूपाद्युत्पत्त्यनन्तरमपि पूर्वघटसादृश्यस्य तत्र भानात्~। तस्मात् अवयविनाशादिकं न कल्पनीयम्~। अनन्तावयवनाशतदुत्पत्त्यादीनां कल्पनापेक्षया अवयविष्वपि पाकजरूपाद्युत्पत्तिकल्पनमेव लघुभूतमिति न्यायविदामाशयः~।

		\subsection{अवयविनि अवयवे च पाकजरूपाद्युत्पत्तिः}

		समानविषयेषु कल्पनापेक्षया प्रत्यक्षस्य प्रबलप्रमाणत्वात् अग्निनिक्षिप्तामघटादेः रूपादिपरावर्तनकाले नाशादर्शनात्~, रूपपरावर्तनानन्तरमपि 'स एवायं घटः' इति प्रत्यभिज्ञादर्शनात् अवयव्यवष्टब्धेष्वपि अवयवेषु पाकाद्रूपादिपरावर्तनम् अभ्युपगन्तव्यम्~। न च तदा सर्वत्रावयविषु तदवष्टब्धावयवेषु च अग्निसंयोगाभावदर्शनात् सर्वत्र रूपादिपरावर्तनं न सम्भवतीति वाच्यम्~। शरावान्तःप्रवशिष्टजलादिगतशीतस्पर्शस्य शरावबहिर्भागेऽपि त्वाचप्रत्यक्षत्वात् अवयव्यवष्टब्धावयववैरपि द्रव्यान्तरसंयोगो भवतीति सिद्धम्~। अन्यथा अत्रापि बहिः शीतस्पर्शोपलब्ध्युपपादनाय शरावनाशः कल्पनीयः स्यात्~। तस्मात् अवयव्यवष्टब्धेष्ववयवेषु अवयवीषु च पाकाद्रूपादिपरावर्तनमङ्गीकरणीयमेव~। क्वचित् अग्निसंयोगात् अवयविनाशोऽपि दृष्टः~। यथा अग्निसंयोगात् पटादीनां नाशः~।  तत्र तु अवयवान्तरोत्पत्तिरपि अनुभवविरुद्ध एव~। तत्रापि पटादीनां नाशः‌ अग्निसंयोगात् भवति चेदपि घटादीनां नाशस्तु नाग्निसंयोगात् अपि तु काष्टाभिघातादेव~। तदुक्तं न्यायमञ्जर्याम्~-

		{\fontsize{11.7}{0}\selectfont\s अपरे पुनः प्रत्यक्षबलवत्तया घटादेरविनाशमेव पच्यमानस्य मन्यन्ते~। सुषिरद्रव्यारम्भाच्च अन्तर्बहिश्च पाकोऽप्युपपत्स्यते~। दृश्यते च पक्वेऽपि कलशे निषिक्तानामपां बहिः शीतस्पर्शग्रहणम्~। अतश्च पाककाले ज्वलदनलशिखाकलापानुप्रवेशकृतविनाशवत् तदपि शिशिरतरनीरकणनिकरानुप्रवेशकृतविनाशप्रसङ्गः~। न चेदृशी प्रमाणदृष्टिः~। अतः प्रकृतिशुषिरतयैव कार्यद्रव्यस्य घटादेरारम्भात् अन्तरान्तरा तेजः कणानुप्रवेशकृतपाकोपपत्तेरलं विनाशकल्पनया~। पिठरपाकपक्ष एव पेशलः~॥\\ यादृतेव हि निक्षिप्तः घटः पाकाय कन्दुके~।\\ पाकेऽपि तादृगेवासौ उद्धतो दृश्यते ततः~॥\\ क्वचित्तु सन्निवेशान्तरदर्शनं काष्ठाद्यभिघातकृतमुपपत्स्यते~। पावकसम्पर्ककारित्वे तु सर्वत्र तथाभावः स्यात्~। तस्मादविनष्टा एव घटादयः पच्यन्ते~।\footnote{न्या.म. २८७,२८८}}

	\section{विमर्शः}

		\subsection{पाकजरूपादयः परमाणुष्वेवेत्यत्र विमर्शः}

		पृथिव्यां रूपादीनां पाकजगुणानामुत्पत्तिक्रमः विमृश्यते~। तत्र हि वेगवदग्न्यभिघातात् परमाणौ द्व्यणुकारम्भकसंयोगप्रतिद्वन्द्विक्रिया जायते~। ततः विभागसंयोगनाशक्रमेण द्व्यणुकादीनां माहावयविपर्यन्तानां नाशे जाते स्वतन्त्रेषु परमाणुषु अग्निसंयोगात् श्यामादिरूपानां नाशः पाुनरन्यस्मादग्निसंयोगात् रक्तादीनामुत्पत्तिश्च जायते~। ततः भोगिनामदृष्टसहकृतात्मसंयोगवशात् परमाणुषु द्व्यणुकारम्भकक्रिया उत्पद्यते~। ततः विभागादिक्रमेण द्व्यणुकमुत्पद्यते~। तत्र च कारणगुणप्रक्रमेण रक्तरूपमुत्पद्यते~। एवं रीत्या महावयविन्यपि रक्तोत्पत्तिर्भवति इति वैशेषिकानां सिद्धान्तः~।

		न च पृथिव्यां पाकजरूपादिकं परमाणुष्वेव उत्पद्यन्ते, न तु घटादिषु इत्यत्र किं प्रमाणम्~? किञ्च घटादीनां नाशोत्पत्तौ बाधकमस्ति~। तथा हि~- अवयव्यवष्टब्धेषु परमाणुषु क्रियोत्पत्तौ न किञ्चित्प्रमाणमस्ति~। एवं पाकोत्तरमपि 'स एवायं घटः' इति प्रत्यभिज्ञोपलब्धेः~। किञ्च यदि उत्पन्नपाकजाः परमाणवः द्व्यणुकादिक्रमेण घाटादिकमारभेरन् तदा तत्र तत्सङ्ख्याकतत्परिमाणकघटादीनामेवोत्पत्तिरिति कथं स्यात्~? तथा नियमाभावात्~। किन्तु तस्मिन्नेव देशे तावत्परिमाणकतत्सङ्ख्याक एव घटः पाकानन्तरमपि दृष्टः~। तस्मात् पाकाद्रूपान्तरोत्पत्तौ न द्रव्यनाशो कल्पनीयः इति वाच्यम्~। यथा घटोपरि दण्डादिना ताडने सति तन्नश्यति, तत्र घटावष्टब्धेष्वेव कपालेषु क्रियोत्पत्या विभागादिक्रमेण घटनाशो भवति तथैव द्व्यणुकावष्टब्धेष्वपि परमाणुषु क्रियाभवितुमर्हति~। तदेवात्र कल्प्यते~। न च प्रतिभिज्ञापूर्वोत्तरपदार्थयोरभेदं सर्वत्र साधयति~। अन्यथा तदेवौषधमित्यादिप्रत्यभिज्ञानमेव न स्यात्~। तस्मात् साजात्यस्थलेऽपि प्रत्यभिज्ञायाः दर्शनात् प्रकृतेऽपि प्रत्यभिज्ञायाः साजात्यबोधकत्वमेव~। एवं वेगवतः तेजसः अभिघातात् अवयव्यवष्टब्धेष्वपि परमाणुषु क्रियोत्पत्तेः स्वीकारात् ततः अवश्यं द्व्यणुकादिक्रमेणावयविनाशो भवत्येवेति सिद्धम्~। पाकानन्तरं तद्देशे तत्सङ्ख्याकस्तत्परिमाणकश्च घटो अनुभूयते~। तस्मात् प्रत्यभिज्ञानाच्च तत्र तादृशस्यैव घटस्योत्पत्तिः कल्प्यते~।

		न च पाकनिक्षिप्तेषु तृणादिना आवृतेष्वपि घटादिषु छिद्रप्रदेशेन नीरीक्षमाणस्य कुलालस्य अक्षजप्रतीतिः 'घटः पच्यत' इति~। न तु 'परमाणुः पच्यत' इति~। एवं तस्य घाटनाशविषयकप्रतीतेरप्यनुदयात् अवयविन्यपि पाको अनुभवबलादभ्युपेयः इति वाच्यम्~। घटस्तावत् साक्षात् परमाणुसमूहादारब्धः येन परमाणुगतक्रियया विभागादिना तस्य अविलम्बेन नाशो भवति~। अपि तु द्व्यणुकादिक्रमेणैव आरब्धः~। तस्मात् द्व्यणुकनाशात् तत्स्यंयोगनाशे त्र्यणुकनाश इत्येवं क्रमेण असङ्ख्येयद्रव्यादिनाशपरम्परया घटनाशो विलम्बेनैव भवति~। एवमेकत्र अवयवाः विनश्यन्ति, अन्यत्र तस्मिन्नेव देशे पक्तावयवात् संयोगादिना अवयवान्तरमुत्पद्यते इत्यप्यनुभवः~। अत एव एकस्मिन्नेव घटे देशान्तरावच्छेदेन पक्वता अन्यदेशावच्छेदेन अपक्वता च कदाचिद्दृश्यते~। तस्मात् अवयवनाशात् घटभेदेऽपि घटान्तरस्य तत्र सम्भवात् तेन सह सन्निकर्षाच्च 'घटः पच्यत' इत्यादिव्यवहाराः उपपद्यन्ते~। एवमनया रीत्या पाकोत्पत्तेः पूर्वं घटारम्भकावयवाः यावन्तः तावन्त एव पाकानन्तरमपि घटमुत्पादयन्ति इत्यतः तस्य समानपरिमाणकत्वं समानसङ्ख्याकत्वञ्च उपपद्यते~।

		किञ्च गुणवति तत्सजातीयगुणान्तरानुत्पत्तेः रूपान्तरोत्पत्तौ पूर्वरूपनाशो अवश्यं वक्तव्यः~। तच्च द्रव्यनाशादेव भवति~। आश्रयनाशादेव रूपादीनां नाशः, कारणगुणादेव रूपादीनाञ्च उत्पादः लोके दृष्टः~। यथा दण्डाद्यभिघाते सति घटस्य नाशात् तद्गतरूपादीनां नाशो भवति~। एवं घटोत्पत्त्यनन्तरमेव घटे रूपमुत्पद्यते~। तस्मात् घटादिरूपाणां नाशं प्रति घटादीनां नाशः, पटादीनां रूपादिकं प्रति तन्त्वादिगतरूपादीनां च कारणत्वमभ्युपगम्यते~। एवं घटाद्यवयविषु अग्निसंयोगेन रूपादिपरावर्तनमप्यनुपपन्नमेव~। तद्यथा घटरूपावयवी स्वावयवेषु सर्वत्र अन्तर्बहिश्च वर्तते~। पाकानन्तरं रूपोपलब्धिरपि घटे सर्वत्र भवति~। एतच्च तदा सम्भवति यदा तादृशावयविनि सर्वत्र अग्निसंयोगः स्यात्~। किन्तु घटस्य बहिः अग्निसंयोगदर्शनात् कथमन्तः पाकजरूपाद्युत्पत्तिरिति~। न च तेजसः अतिसूक्ष्मत्वात् तेषामवयव्यवष्टब्धेष्वपि अवयवेषु प्रवेशः सम्भवति~। अतः सर्वत्र पाकाद्रूपपरावर्तनमिति वाच्यम्~। तेजसः अणुदेशप्रवेशस्तु कार्यद्रव्यनाशं विना अनुपपन्नम्~। तथा हि~- वेगवद्द्रव्यसंयोगस्य अन्यत्र क्रियाहेतुत्वदर्शनात् अत्रापि तादृशतेजसः परमाणुना सह संयोगे सति क्रिया उत्पद्यते~। तथा च नाशकस्य सत्त्वात् विभागादिक्रमेण अवयविनाशो भवत्येव~। तथा च स्वतन्त्रेषु परमाणुष्वेव पाकाद्रूपोत्पत्तिरिति~। मैवम्~।

		\subsection{अवयविष्वपि पाकजरूपोत्पत्तिविमर्शः}

		गौरवान्मानाभावाच्च घटाद्यवयवीनां नाशो नैव कल्प्यते~। यदुक्तं वेगवद्द्रव्यसम्बन्धस्य क्रियाजनकत्वमिति, तन्न सार्वत्रिकम्~। अत एव वेगवतः कन्दुकादेः भित्यासह संयोगे सति तत्र तदवयवे वा क्रिया उत्पद्यते~। तथा च यत्र तादृशसंयोगे क्रिया उत्पद्यते तत्रैव  वेगवद्द्रव्यसम्बन्धस्य क्रियाजनकत्वं स्वीकर्तव्यम्~। प्रकृते 'घटः पच्यते', 'स एवायं घटः' इत्यादिप्रतीतीनामुपपादनाय वेगवतः तेजसः परमाणुना सह संयोगे सत्यपि क्रिया नोत्पद्यते इत्युच्यते~। 

		न च प्रकारान्तरेण तादृशप्रतीतीनामुपपत्तिः सम्भवति~। यथा परमाणुसंयोगनाशोत्तरक्षणे एव घटनाशस्य अभावात्~, क्वचिदवयवान्तरनाशः, क्वचिदवयवान्तरोत्पात्तिरिति स्वीकारादिति वाच्यम्~। तथापि नाशानन्तरमुत्पन्नस्य घटस्य तत्परिमाणकत्वसत्त्वेऽपि तदाकारकत्वमनुपपन्नम्~। समानपरिमाणकत्वं न समानाकारकत्वे तन्त्रम्~। अन्यथा समानसङ्ख्याकतन्तुभिरारब्धानां पटानां सामानाकारकत्वं स्यात्~। किन्तु तथा लोके न दृश्यते~। क्वचित् अवयवसंयोगविशेषात् आकारविशेषोऽपि दृश्यते~। समानाकारकत्वानुपपत्तौ च प्रत्यभिज्ञानुपपत्तिः तदवस्थैव~। नापि समानाकारकत्वं कल्पयितुं शक्यम्~। सर्वत्र पाकजस्थले तथाकल्पनापेक्षया अवयविनि पाकजरूपकल्पनमेव लघुभूतम्~।

		यदुक्तं गुणनाशं प्रति आश्रयीभूतद्रव्यस्य नाशः‌ कारणमिति, तदपि व्यभिचरितमेव~। द्रव्यनाशाभावेऽपि संयोगविभागयोः नाशाभ्युपगमात्~। न च रूपादीनामेव तथा इति वाच्यम्~। परमाणौ व्यभिचारात्~।‌ परमाणूनां नित्यत्वात् तेषां नाशासम्भवेऽपि तत्र रूपादिनाशस्तु भवतापि स्वीकृत इति व्यभिचारः~। 

		न च अवयविनि पाकाद्रूपोत्पत्तौ अवयवगतरूपं प्रतिबन्धकम्~। तथा हि~- तन्तौ नीलरूपसत्त्वे पटेऽपि नीलरूपमेव उत्पद्यते~। न तु रूपान्तरमिति लोके दृष्टम्~। तथा च रूपान्तरोत्पत्तौ अवयवगतरूपस्य प्रतिबन्धकत्वं कल्प्यते~। अत एव नानारूपविशिष्टतन्तुभिरारब्धे पटे चित्ररूपं स्वीक्रियते इति वाच्यम्~। अवयव्यवष्टब्धावयवेष्वपि रूपान्तरोत्पत्तिः कल्प्यते~। 

		न चैवमपि सर्वत्रावयविनि अग्निसंयोगस्यासम्भवात् सर्वत्र घटे पाकाद्रूपादिपरावर्तनं न स्यादिति वाच्यम्~। तेजसः अतिसूक्ष्मत्वात् तेषां तादृशसामर्थ्यं कल्प्यते~।‌ दृष्टञ्च लोके जलपूरितघटस्य बहिरपि शीतस्पर्शानुभवः~। स च जलस्य घटावयवान्तःप्रवेशं विना नैव सम्भवति~। यदि तत्रापि वेगवज्जलस्य सम्भवात् परमाणुषु क्रिया स्याति तर्हि घटनाशः स्यात्~। नष्टे च घटे जलाधारत्वासम्भवात् जलस्य स्यन्दनं स्यात्~। तस्मात् तत्र घटनाशासम्भवात् प्रकृतेऽपि अवयव्यवष्टब्धेषु घटपरमाणुषु तेजस्संयोगात् तत्रापि रूपपरावृत्तिः भवत्येव~।‌ एवं द्व्यणुकादिक्रमेण सर्वत्र रूपान्तरोत्पत्तिरिति~।

		न च अवयविनि पाकाजगुणोत्पत्तिपक्षेऽपि  अवयव्यवष्टब्धेषु परमाण्वादिषु अग्निसंयोगः कल्पनीयः~। एवं द्व्यणुकादिषु आश्रयनाशं विना रूपादीनां नाशः, कारणगुणं विना गुणोत्पत्तिः कल्पनीयः इति महद्गौरवमिति वाच्यम्~। पीलुपाकवादिनापि परमाणावग्निसंयोगात् द्रव्यारम्भकसंयोगनाशकक्रिया, ततः विभागः, ततः संयोगनाशः, ततः अवयविनाशः इत्येवं महावयविनाशपर्यन्तं कल्पनीयम्~। पुनरुत्पत्तिः, उत्पन्नस्य पूर्वद्रव्यसमानाकारकत्वं समानसङ्ख्याकत्वं, समानपरिमाणकत्वं च कल्पनीयमिति महद्गौरवम्~। तस्मादवयविनि अवयव्यवष्टेब्धेषु अवयवेषु च पाकाद्रूपपरिवर्तनं भवति इति सिद्धम्~।

		इदन्तु बोध्यम्~- अग्निसंयोगस्य क्वचित् क्रियाजनकत्वमपि स्वीकर्तव्यम्~। यत्र उष्णस्पर्शाधिक्यं तत्र वेगवतः तादृशतेजसः संयोगे सति अवयवादिषु क्रिया जायते~। तत्र च द्रव्यं न श्यति~। किन्तु तत्र पुनः तद्द्रव्यस्योत्पत्तिः न भवति~। अत एव पटादिना सह तेजस्संयोगे सति ते भस्म भवन्ति~। क्वचित्तेजस्संयोगे रूपादिगुणानां परिवर्तनमात्रं भवति~। तथा च यत्र द्रव्यस्य नाशमात्रं भवति तत्र तेजस्संयोगस्य क्रियाजनकत्वम्~, यत्र तु रूपादिपरिवर्तनं तत्र तेजस्सम्बन्धस्य  न क्रियाजनकत्वमिति स्वीकर्तव्यम्~। एतेन कार्यभेदेन कारणभेदमप्युपपद्यते इति~।

	\section{संशयस्य परोक्षत्वविचारः}

	इन्द्रियादिभिः करणैः जायमानमनुभवात्मकं ज्ञानं द्विविधम्~। यथार्थमयथार्थञ्चेति~। तत्रापि अयथार्थानुभवस्त्रिविधः संशयविपर्ययतर्कभेदात्~। तत्र संशयस्तावत् एकस्मिन् धर्मिणिविरुद्धनानाधर्मावगाहिज्ञानम्~। यथा 'अयं स्थाणुः पुरुषो वा', 'पर्वतो वह्निमान्न वा' इत्यादिः~। स चान्तः बाह्यश्चेति द्विविधः~। तत्र बाह्यसंशयोऽपि प्रत्यक्षपरोक्षभेदात् द्विविधः~। साधारणधर्मदर्शनात् तद्धर्मिणि विरुद्धानेकधर्मितावच्छेदकप्रकारकसंशयः परोक्षः इति वैशेषिकानामाशयः~। संशयः सर्वोऽपि प्रत्यक्ष एवेति न्यायविदामाशयः~। नव्येषु दीधीतिकारप्रभृतयः संशयस्य शाब्दत्वमपि वदन्ति~। तदत्र विचार्यते~।

		\subsection{विषाणदर्शनेन गोत्वगवयत्वप्रकारकसंशयः}

		अतिदूरे विद्यमानपदार्थं दृष्ट्वा 'अयं स्थणुर्वा पुरुषो वा' इति संशयो यो जायते स प्रत्यक्षात्मकः संशयः~। कदाचिदटव्यां विषाणमात्रदर्शनेन 'अयं गौर्गवयो वा' इति यः संशयः स परोक्षः अनुमित्यात्मकः इति~। तत्र गोः गवयस्य च विषाणस्य आकारसाम्यात् गोत्वेन गवयत्वेन च साकं विषाणस्य व्याप्तिः गृहीता~। किन्त्वरण्ये केवलविषाणदर्शनमुभयविधव्याप्तिमपि स्मारयति आकारसाम्यात्~। विषाणस्य गोगवयोभयवृत्तित्वस्य निश्चितत्वात् पक्षधर्मताज्ञानादिकमुदेति~। तथा च 'अयं गोत्वव्याप्यविषाणवत्', 'अयं गवयत्वव्याप्यविषाणवत्' इति परामर्शद्वयसंवलने सति 'अयं गौः गवयो वा' इति संशयो जायते इति~। तदुक्तं वैशेषिकग्रन्थेषु~-

		{\fontsize{11.7}{0}\selectfont\s अत्र प्रशस्तपादः – बहिःसंशयः द्विविधः~। प्रत्यक्षविषये चाप्रत्यक्षविषये च~। तत्राप्रत्यक्षविषये तावत् साधारणलिङ्गदर्शनादुभयविशेषानुस्मरणादधर्माच्च संशयो भवति~। यथाऽटव्यां विषाणमात्रदर्शनाद् 'गौर्गवयो वे'ति~। प्रत्यक्षविषयेऽपि स्थाणुपुरुषयोरूर्ध्वतामात्रसादृश्यदर्शनाद् वक्रादिविशेषानुपलब्धितः स्थाणुत्वादिसामान्यविशेषानभिव्यक्तावुभयविशेषानुस्मरणादुभयत्र आकृष्यमाणस्यात्मनः प्रत्ययो दोलायते~-  किं नु खल्वयं स्थाणुः स्यात् पुरुषो वेति~?\footnote{प्र.भा. ४९५}}

		{\fontsize{11.7}{0}\selectfont\s वाटान्तरितस्य  पिण्डस्याप्रत्यक्षस्य सामान्येन विषाणमात्रदर्शनानुमितस्य संशयविषयत्वादप्रत्यक्षविषयोऽयं संशयः~।\footnote{न्या.कं. ४९५}}

		\subsection{प्रत्यक्षापेक्षया अस्य वैलक्षण्यम्}

		सामान्यधर्मद्वारा धर्मिणां प्रत्यक्षे जाते तत्र विशेषधर्मपुरस्सरेण तत्प्रत्यक्षं प्रति तद्गतासाधारणधर्मः सहकरोति~। यथा कृष्णवर्णरूपसामान्यधर्मेण दृष्टयोः काकपिकयोः मध्ये विशिष्यपिकत्वेन प्रतीतिस्तु तद्गतमधुरध्वनिरूपासाधारणधर्मश्रवणे सत्येव भवति~। प्रकृते गोगवयविषाणेषु सम्यात् तदेकत्रासाधारणधर्मज्ञानो न भवति~। अतस्तेन विशिष्य गोत्वेन वा गवयत्वेन वा साक्षात्कारो न भवति~। किन्त्वत्र तदाश्रयतया धर्मिद्वयोपस्थितिस्तु जायते~। यतः तदुभयवृत्तित्वस्य पूर्वं गृहीतत्वात्~। एवमेव उभयत्र तद्वृत्तित्वं तु बाधितमेव~। एवं प्रकृते धर्मिणा सह सन्निकर्षाभावात् साधारणरूपेणापि धर्मिज्ञानं न जायते~। तस्मादत्र संशयः धर्मिसाक्षात्काराभावात् परोक्ष एव~। तदुक्तं व्योमशिवाचार्यैः~-

		{\fontsize{11.7}{0}\selectfont\s बहिर्द्विविधः~। केन रूपेण~?‌ प्रत्यक्षविषये चाप्रत्यक्षविषये चेति~। तत्राप्रत्यक्षविषये तावदभिधीयते संशयः साधारणलिङ्गदर्शनादिति~। यल्लिङ्गं विरुद्धविशेषैः‌ सहोपलब्धं तददर्शनाद्विशेषानुपलब्धेरुभयविशेषानुस्मरणादधर्माद्दिक्कालादिभ्यश्च संशयो भवति~। यथा अटव्यां‌ विषाणमात्रदर्शनात्तद्गतविशेषानुपलब्धेः तत्स्मरणात्संशयो भवति 'गो र्गवयो वा' इति~। अथ प्रत्यक्षादस्य विशेषः कथम्~? सामान्यविशेषितस्य धर्मिणः प्रत्यक्षत्वं विशेषलक्षणाभिधानात्~। अथ किं गोगवयविषाणेषु सामान्यम्~, तुल्यावयवरचनायोगः~। यथाभूता ह्यवयवरचना गोविषाणे दृष्टा तथा गवयविषाणेऽपीति~। ये तु गोविषाणविशेषा गवयविषाणविशेषाश्च पूर्वोपलब्धास्तेषामत्रानुपलम्भ इति~। अतः सादृश्यमात्रविशेषितं विषाणमुपलभमानस्य विशेषानुपलब्धेः तत्स्मरणाच्च किं गौः स्यात् गवयो वेति संशयः~।\footnote{व्यो . ५३६,५३७}}

		\subsection{रत्नकोशकारमतम्}

		परस्परविरुद्धप्रकारकपरामर्शसमवधाने सति संशयानुमितिरिति रत्नकोशकाराः~। तथा हि~- वह्नितदभावव्याप्यवत्तावगाहिपरामर्शसंवलने सति वह्न्यनुमितिर्न जायते~। वह्न्यभावव्याप्यवत्तापरामर्शस्य प्रतिबन्धकत्वात्~। वह्न्यभाववत्तानुमितिरपि नोदेति~। वह्निव्याप्यवत्तानिश्चयस्य प्रतिबन्धकत्वात्~। तस्मात् परस्परविरोधिसामग्र्युपस्थितेः संशयजनकत्वोपगमात् प्रकृते संशयानुमितिः जायते इति~। तदेवं वर्णयति सत्प्रतिपक्षप्रकरणे मणिकारः~-

		{\fontsize{11.7}{0}\selectfont\s रत्नकोषकारस्तु सत्प्रतिपक्षाभ्यां प्रत्येकं स्वसाध्यानुमितिः संशयरूपा जायते, विरुद्धोभयज्ञानसामग्र्याः संशयजनकत्वात्~। संशयद्वारास्य दूषकत्वम्~। न च संशयरूपा नानुमितिः बाधस्येव विरोध्युपस्थितेरनुमितिसामग्रीविघटकत्वेनावधारणात् अन्यथा बाधेऽप्यनुमित्यापत्तेरिति वाच्यम्~। अधिकबलतया बाधेन प्रतिबन्धात्~, तुल्यबलत्वादनुमितिः स्यादेव सामग्रीसत्त्वात्~। साध्याभावबोधस्य च तत्र प्रतिबन्धकत्वं न तु तद्बोधकस्य चक्षुरादेः~। प्रत्येकं निर्णायकत्वेनावधारितात्कथं संशय इति चेन्न~। प्रत्येकाद्धि ज्ञानमुत्पद्यमानं अर्थात् संशयो न तु प्रत्येकं संशयजनकत्वमिति मेने~।\footnote{स.त.म. ६०}}

		\subsection{शब्दज्ञानादपि संशयो भवति}

		शब्दज्ञानादपि संशयो जायते इत्यपि केचित्~। 'पर्वतो वह्निमान्'~, 'पर्वतो वह्न्यभाववान्' इति वाक्यद्वयात् 'पर्वतो वह्निमान्न वा' इति संशयो जायते~। परस्परविरुद्धसामग्रीसमवधानस्य संशयजनकत्वनियमात् इति~। तथा हि पदार्थतत्त्वनिरूपणम्~-

		{\fontsize{11.7}{0}\selectfont\s पदजन्यधर्मिककोटिद्वयतदुभयविरोधिज्ञानसंशयात्मकयोग्यताज्ञानसहितात् शब्दादाहत्यैव संशयः~। “समानानेके"त्यादिसूत्रं प्रमाणयतो महर्षेरपि सम्मतमिदम्~।\footnote{प.त.नि.१४२}}


	परस्परविरुद्धानेकधर्मप्रकारकज्ञानात्मकः संशयः सर्वदा अपरोक्ष एव~। स्वप्रकारकज्ञानजनने स्वविरुद्धप्रकारकसामग्र्याः प्रतिबन्धकत्वात् न परोक्षसंशयसम्भवः~। प्रत्यक्षस्थले तु विषययस्य कार्यसहभावेन कारणत्वात् परस्परविरुद्धसामग्रीसत्त्वे संशयो भवत्येव इति न्यायविदामाशयः~।

		\subsection{संशयानुमितिनिराकरणम्}

		'पक्षः साध्यवान्न वा' इत्याकारकसंशयाकारानुमित्युत्पत्तौ का सामग्री इति जिज्ञासायां साध्यतदभावप्रकारकपरामर्शः सामग्री इति वक्तव्यम्~। किन्तु तादृशपरामर्शद्वयसंवलने सति साध्यवत्ताबुद्धिर्नो देति, साध्याभावव्याप्यवत्तानिश्चयस्य तदभावव्याप्यवत्तानिश्चयविधया प्रतिबन्धकत्वात्~। साध्याभाववत्ताबुद्धिरपि नोदेति, साध्यप्रकारकपरामर्शस्य तदभावव्याप्यवत्तानिश्चयविधया प्रतिबन्धकत्वात्~। यथा सुरभ्यसुरभ्यवयवारब्धे अवयविनि गन्धो नोत्पद्यते तथैवात्र परस्परप्रतिबन्धादनुमितिरेव नोदेति~। अतः न संशयानुमितिरिति~। अत्र मणिकारवचनम्~-

		{\fontsize{11.7}{0}\selectfont\s साध्यतदभावयोर्विरोधेन यथा एकज्ञानस्यापरधीप्रतिबन्धकत्वं तथा साध्याभावव्याप्यवत्त्वस्यापि साध्यविरोधित्वात्तद्बुद्धेरपि साध्यधीप्रतिबन्धकत्वात्~, विरोधिज्ञानत्वस्य प्रतिबन्धकत्वे तन्त्रत्वात्~।\footnote{स.त.म. ६१}}

		\subsection{अव्याप्यवृत्तिसाध्यकस्थले परोक्षसंशयसम्भवः}

		अव्याप्यवृत्तिसाध्यकस्थले संशयानुमितिः भवितुमर्हति~। तद्वत्ता बुद्धिं प्रति तदभाववत्तानिश्चयस्य प्रतिबन्धकत्वम् अनुभवसिद्धमेव~। तद्वत् तत्रैव अव्याप्यवत्तित्वग्रहदशायां तद्वत्ताबुद्धिरुदेति~। तेन अव्याप्यवृत्तित्वग्रहः तदभाववत्तानिश्चयस्य प्रतिबन्धकत्वस्थले उत्तेजकः इति ज्ञायाते~। तद्वत् तदभावव्याप्यवत्तानिश्चयस्य यत्र प्रतिबन्धकत्वं तत्रापि बाधकाभावात् अव्याप्यवृत्तित्वग्रहस्य उत्तेजकत्वं स्वीकरणीयम्~। तथा च साध्यतदभावावगाहिपरामर्शदशायां 'साध्यमव्याप्यवृत्ति' इति यदि निश्चयः तदा तत्र प्रतिबन्धकनिश्चयसत्त्वेऽपि उत्तेजकस्यापि सत्त्वात् अनुमितिरुदेति इति वक्तव्यम्~। सा च संशयात्मिका इति तादृशस्थलविशेषे संशयानुमितिप्रसिद्धिरिति~। यथा 'वृक्षः कपिसंयोगी एतद्वृक्षत्वात्' इत्यत्र संयोगस्य दैशिकाव्याप्यवृत्तित्वग्रहदशायां साध्यतदभावप्रकारकपरमर्शसत्त्वे शाखावच्छेदेन 'वृक्षः कपिसंयोगी न वा' इति संशयो भवति~। एवं 'घटः गन्धवान्' इत्यत्र गन्धस्य कालिकाव्याप्यवृत्तित्वग्रहदशायां गन्धतदभावप्रकारकपरामर्शसत्त्वे एतत्क्षणावच्छेदेन 'घटः गन्धवान्न वा' इत्याकारकसंशयो जायते इति~। तदुक्तं गदाधरभट्टाचार्यैः~-

		{\fontsize{11.7}{0}\selectfont\s परे तु ग्राह्याभावेऽव्याप्यवृत्तित्वग्रहदशायां ग्राह्याभावनिश्चयविषये धर्मिणि ग्राह्यज्ञानोत्पत्त्या यथा तादृशाव्याप्यवृत्तित्वग्रहस्य तत्प्रतिबन्धकतायामुत्तेजकत्वं, तथा तद्धिर्मिकाव्याप्यवृत्तित्वग्रहदशायां तद्व्याप्यवत्तानिश्चयस्यापि तद्विपरीतज्ञानाप्रतिबन्धकतया तत्रापि तद्धर्मिकाव्याप्यवृत्तित्वग्रहस्योत्तेजकत्वमुपेयम्~। एवं च कपिसंयोगतदभावयोर्दैशिकाव्याप्यवृत्तित्वग्रहदशायां पक्षतावच्छेदकशाखादिरूपैकावच्छेदेन तदुभयकोट्यवगाहिनी गन्धतदभावयोः कालिकाव्याप्यवृत्तित्वग्रहदशायां पक्षतावच्छेदकतत्क्षणावच्छेदेन तदुभयकोटिमवगाहमाना च संशयानुमितिः ग्राह्याभावज्ञानप्रतिबन्धकतया न शक्यते वारयितुम्~।\footnote{स.गा.९३}}


	\section{विमर्शः}

		\subsection{परोक्षसंशयविमर्शः}

		एकस्मिन् धर्मिणि परस्परविरुद्धानेकोट्यवगाह्यनुभवात्मकः संशयः प्रधानतया बाह्यः मानसश्चेति द्विविधः~। तत्र ज्ञानाद्यात्मगुणविषयकस्तावदन्तः संशयः~। सैव मानसः इत्युच्यते~। तथा हि ज्योतिर्विद्भिः ग्रहादिगत्यनुसारेण ऊहादिकं यत्र क्रियते तत्र बहिस्थपदार्थेन सह इन्द्रियव्यापाराभावेऽपि ज्ञानमुत्पद्यते~। तत्र तेषाम् 'इदं सत्यं मिथ्या वा' इत्यादिसंशयो यो जायते सः मानसः~। बाह्यः पुनर्द्विविधः प्रत्यक्षविषयकः अप्रत्यक्षविषयकश्चेति~। तत्र हि अरण्यादौ अतिदूरे विद्यमानं ऊर्धताधर्मविशिष्टं वस्तु दृष्ट्वा तत्र विशेषेण करादीनां पुरुषविशेषधर्माणाम्~, वक्रकोटरादीनां स्थाणुविशेषधर्माणां चादृष्ट्वा संशीते 'अयं स्थाणुः पुरुषो वा' इति~। अत्र स्थाणुपुरुषसाधारणधर्ममात्र विशिष्टेन संशयविषयीभूतेन धर्मिणा सह इन्द्रियसन्निकर्षादस्य प्रत्यक्षत्वमुपपद्यते~।

		द्वितीयस्तावत् अरण्ये विषाणमात्रदर्शनेन 'गौर्गवयो वेति' संशयः~। तत्र हि इन्द्रियसन्निकृष्टे विषाणे गोः गवयस्य वा विशेषधर्माग्रहात् उभयविषाणेषु आकारसाम्यस्य पूर्वं ज्ञातत्वात् संशयो जायते~। अत्र जायमानसंशयधर्मी पिण्डविशेषः नेन्द्रियसन्निकृष्टः~। अपि तु विषाणरूपहेतुना अनुमितः इति अस्य परोक्षत्वमिति~। तथा च यत्र संशयविषयीभूतधर्मी इन्द्रियसन्निकृष्टः तत्र संशयः प्रत्यक्षः~। यत्र तु धर्मी प्रमाणान्तरसिद्धः तत्र संशयः परोक्षः इति वैशेषिकाः~।

		रत्नकोषकारास्तु~- परस्परविरुद्धप्रकारकपरामर्शसमवधाने सति संशयनुमितिरिति वदन्ति~। तथा हि~- वह्नितदभावव्याप्यवत्तावगाहिपरामर्शसंवलने सति वह्न्यनुमितिः न जायते~। वह्न्यभावव्याप्यवत्तापरामर्शस्य प्रतिबन्धकत्वात्~। वह्न्यभाववत्तानुमितिरपि नोदेति~। वह्निव्याप्यवत्तानिश्चयस्य प्रतिबन्धकत्वात्~। तस्मात् परस्परविरोधिसामग्र्युपस्थितेः संशयजनकत्वोपगमात् प्रकृते संशयानुमितिः जायते इति~। न च वह्नितदभावप्रकारकपरामर्शयोः परस्परविरोधात् सुरभ्यसुरभ्यारब्धद्रव्ये यथा गन्धाभावः तद्वत् कार्यमेव नोत्पद्यते इति वाच्यम्~। क्वचित् चित्रपटादौ परस्परविरोधसामग्रीदशायामपि कार्योत्पत्तेः दर्शनात्~। न च विरोधिनिश्चयसत्त्वे कार्योत्पत्त्यभ्युपगमे वह्न्यभाववत्तानिश्चयदशायामपि वह्निमत्ताबुद्धिप्रसङ्गः~। कार्योत्पत्तौ विरोधिनिश्चयस्य अकिञ्चित्करत्वापत्तिः इति वाच्यम्~। यदा निश्चयस्याधिकबलत्वं तदानीमेव तस्य कार्योत्पत्तिविघटकत्वम्~। बाधस्थले तु वह्निमत्ताबुध्युत्पत्तौ वह्न्यभाववत्तानिश्चयस्य अधिकबलत्वात् प्रतिबन्धकत्वमेव~। यदा तु समानबलत्वं तदा निश्चययोः कार्योत्पत्तेरनुभवात् न तयोः प्रतिबन्धकत्वम्~। तथा च तत्र यत्कार्यमुत्पद्यते तदर्थात् संशयात्मकं भवति इति वदन्ति~।

		शब्दज्ञानादपि संशयो जायते इत्यपि केचित्~। वादस्थले वादिना यदा प्रतिज्ञा क्रियते 'पर्वतो वह्निमान्' इति, ततः परं प्रतिवादिना 'पर्वतो वह्न्यभाववान्' इति या प्रतिज्ञा क्रियते तदुभश्रवणात् मध्यस्थस्य यः संशयो जायते सः परोक्षः शब्दादुत्पन्नः इत्यतः शाब्दः एव~। तत्रापि परस्परविरुद्धवह्नितदभावावगाहिशाब्दबुद्धिजनकवाक्ययोः समवधानात्~। परस्परविरुद्धसामग्रीसमवधानस्य च संशयजनकत्वनियमात् इति~। तथा चात्रापि परस्परविरुद्धसामग्रीद्वयसमवधाने सति तयोः कार्यविघटकत्वाभावात् संशयो जायते इत्याशयः~। मैवम्~।

		\subsection{संशयस्यापरोक्षत्वविमर्शः}

		संशयः सर्वोऽप्ययरोक्ष एव~। न हि अरण्यादौ विषाणमात्रदर्शनाद् 'गौर्गवयो वा' इति संशयः परोक्षः~। तत्र विषाणरूपहेतुना प्राणिविशेषस्यैव निश्चितत्त्वात्~। निश्चितप्राणिविशेषे च विशेषधर्मजिज्ञासायां गोः गवयस्य च संस्कारबलात् एकसम्बन्धिज्ञानमिति रीत्या स्मरणं जायते~। ततः जिज्ञासावशात् नष्टस्य निश्चयस्य पुनः संस्कारवशात् स्मरणे जाते मानसव्यापारेण संशयो जायते~। तथा च तस्य मानसप्रत्यक्षत्वमेव~। न हि ज्ञानाद्यात्मागन्तुकगुणतदभावकोटिक  एव संशयः मानसो भवति~। मनसा आत्मवृत्तिज्ञानेन सह सन्निकर्षे जाते तद्गतविषयस्यापि अवगाहनात्~। निर्विषयकज्ञानस्याप्रसिद्धत्वाच्च~। किञ्च विषाणरूपहेतुना प्राणिविशेषस्यैव कल्पनासम्भवात् गोत्वादिधर्मेण सह तस्य व्याप्त्यभावात्~। तथा च तत्र प्राणिविषयकनिश्चय एव जायते~। न तु संशयानुमितिरिति~।

		न हि विरुद्धोभयपरामर्शसंवलनदाशायां संशयानुमितिर्जायते~। तत्र कार्योत्पत्तौ परस्परप्रतिबन्धकस्य सत्त्वात् कार्यमेव नोत्पद्यते~। न च चित्ररूपादौ परस्परप्रतिबन्धकस्य सत्त्वे कार्योत्पत्तिरङ्गीकृता~। तत्रापि नीलपीततन्तूनां समवधाने सति पटे नीलरूपमुत्पद्यते~। पीततन्तूनां प्रतिबन्धकत्वात्~। न हि पीतरूपं नीलतन्तूनां प्रतिबन्धकत्वात्~। तथा च अन्यदेव कार्यमुत्पद्यते~। तत्रापि पटप्रत्यक्षत्वानुपपत्त्या रूपान्तरमङ्गीकृतम्~। अत एव सुरभ्यसुरभ्यवयवारब्धे अवयविनि लाघवाच्चित्र गन्धः न कल्पितः~। प्रकृतेऽपि बाधकाभावात् विरुद्धोभयपरामर्शसंवलनदशायां कार्योत्पत्तिः नोच्यते~। तस्मात् कार्यस्यैव तत्रानुत्पादान्न संशयानुमितिसम्भवः~।

		यदुक्तं वादस्थले वह्नितदभावादिप्रकारकशब्दज्ञानवशात् शाब्दः संशयो जायते इति~। तदप्यसत्~। तत्र हि वाद्युक्तवह्निप्रकारकवचनश्रवणात् तदर्थोपस्थित्युत्तरक्षणे वह्निप्रकारकनिश्चय एव उत्पद्यते~। ततः प्रतिवाद्युक्तवह्न्यभावप्रकारकवचनश्रवणात् अर्थोपस्थित्त्या वह्न्यभावप्रकारकशाब्दः निश्चयः~। तदनन्तरं वह्नितदभावयोः स्मरणे तयोः विरोधस्मरणात् मानस एव संशयो जायते~। अतः न शाब्दोऽसंशयः सम्भवति~। न च 'पर्वतो वह्निमान्न वा' इति वाक्यात् संशयाकारशाब्दोत्पत्तिर्भवति~। वह्निप्रकारकशाब्दबुद्धिं प्रति वह्न्यभावप्रकारकनिश्चयस्य वह्न्यभावप्रकारकशाब्दं प्रति वह्निप्रकारकनिश्चयस्य च प्रतिबन्धकत्वकल्पनात्~। अन्यथा  ह्रदादौ प्रमाणान्तरेण वह्न्यभाववत्तानिश्चयदशायामपि 'ह्रदो वह्निमान्' इति वाक्यात् शाब्दबोधप्रसङ्गः~। तस्मान्न संशयस्य शाब्दत्वमिति~।

		ननु अव्याप्यवृत्तिसाध्यकस्थले 'कपिसंयोगी एतद्वृक्षत्वात्' इत्यादौ कपिसंयोगतदभावावगाहिपरामर्शयोः संवलनदशायां सति 'कपिसंयोगो अव्याप्यवृत्तिः' इति ज्ञाने संशयानुमितिरुत्पद्यते~। अव्याप्यवृत्तित्वप्रकारकज्ञानस्य उत्तेजकत्वात्~। यथा 'कपिसंयोगाभाववान्' इति निश्चयदशायामव्याप्यवृत्तित्वप्रकारकज्ञाने सति कपिसंयोगवत्ताबुद्धिरुदेति तत्र यथा अव्याप्यवृत्तित्वप्रकारकज्ञानस्य उत्तेजकत्वं तथैव तदभावव्याप्यवत्तानिश्चयदशायामपि तस्य उत्तेजकत्वमभ्युपगम्यते~। न च अव्याप्यवृत्तित्वग्रहदशायां तस्य संशयप्रतिबन्धकत्वात् समुच्चय एव स्यादिति वाच्यम्~। अव्याप्यवृत्तित्वग्रहस्य कोट्योः विरोधांशे एव प्रतिबन्धकत्वम्~, न तु संशयप्रतिबन्धकत्वम्~। तस्माद्विरोधानवगाहिसंशयो जायते~। न च संशयस्य नियमेन विरोधविषयकत्वं स्वीकर्तव्यम्~। अन्यथा संशयसमुच्चययोः भेद एव न स्यादिति वाच्यम्~। अरण्यादौ निरुक्तरीत्या 'सः गौः गवयो वा' इति संशये, 'चैत्रः मैत्रो वा' इत्यादिसंशये च कोट्योः विरोधाभावात्~। न च तत्र वस्तुतः विरोधाभावेऽपि विरोधो अवगाहते इति वाच्यम्~। 'भूतलं घटवत् पटवदुभयवद्वा' इत्यादिसंशयस्यापि आनुभविकतया तत्र घटपटोभयवृत्तित्वस्यापि संशये भानात्~। अन्यथा घटपटयोः विरोधभाने उभयवृत्तित्वप्रकारकसंशय एव न स्यात्~। 

		तर्हि संशयसमुच्चययोः किं वैलक्षण्यमिति चेत्~, संशये विरोधभानस्य अनियततया संशयत्वं 'सन्देह्मि' इत्यनुव्यवसायबलात् संशयवृत्तिकोटिताख्यविषयताविशेषः एव कल्प्यते~। तथा च समुच्चयोत्तरं 'सन्देह्मि' इत्यनुव्यवसायानुदयात् तत्र एतादृशविषयताविशेषाभावात् तयोः वैलक्षण्यमिति~। 

		तथा च अव्याप्यवृत्तिसाध्यकस्थले 'कपिसंयोगी एतद्वृक्षत्वात्' इत्यादौ साध्यतदभावविषयकपरामर्शयोः संवलनदशायाम् अव्याप्यवृत्तित्वप्रकारकग्रहे च सति 'अग्रावच्छेदेन वृक्षः कपिसंयोगी न वा' इत्याकारकसंशयात्मकानुमितिः भवति~। अन्यत्र सर्वत्रापि अपरोक्ष एव संशयो जयते इति सिद्धम्~।


\begin{center}\begin{small}॥ इति प्राचीनन्याय-वैशेषिक-नव्यन्यायशास्त्रेषु तत्तद्व्याख्याकाराणां सैद्धान्तिकमतभेदानां विमर्शात्मकमध्ययनमिति प्रबन्धे  गुणप्रपञ्चे सैद्धान्तिकमतभेदाः इत्याख्यः तृतीयोऽध्यायः~॥\end{small}\end{center}




\titleformat {\chapter}[display]{\normalfont\Large} % format
{अथ चतुर्थोऽध्यायः\\[1mm]} % label
{-3.8ex}{ \rule{\textwidth}{1pt}\vspace{-5ex}
\centering
} % before-code
[
\vspace{-6.7ex}%
\rule{\textwidth}{1pt}
]
\titlespacing*{\chapter} {10pt}{-60pt}{50pt}

\chapter{समवायप्रपञ्चे सैद्धान्तिकमतभेदाः}

जात्यादिषु सैद्धान्तिकविरोधाभावात् समायोऽत्र निरूप्यते~।'घटपटौ संयुक्तौ' इति विशिष्टप्रतीतौ संसर्गस्य संयोगस्यापि विषयत्वमनुभवसिद्धमेव~। तथा च सर्वत्रापि विशिष्टप्रतीतौ संसर्गः भासत एव~। 'फलं पतति' इत्यपि विशिष्टप्रतीतिरेव~। तत्र पतनस्य फलेन सह कः संसर्गः इति जिज्ञासायां न संयोगः‌, द्रव्ययोरेव संयोगदर्शनात्~, न निरूपकत्वादिकं वृत्त्यनियामकसंसर्गत्वात्~, तस्मादतिरिक्तः‌ एव संसर्गः‌ कल्प्यते समवाय इति~। तथा चानुमानं 'फलं पतति इति विशिष्टप्रतीतिः‌ विशेषणविशेष्यसंभन्धविषया विशिष्टप्रतीतित्वात् भूतलं घटवदिति विशिष्टप्रतीतिवत्' इति~। अत्रानेनानुमानेन लाघवादेकः‌ नित्यश्च संसर्गः कल्प्यते~। स च समवाय इति~। अत्र अस्य नित्यत्वे एकत्वे ऐन्द्रियकत्वे च ग्रन्थकर्तृषु विप्रतिपत्तिः दृश्यते~। तदत्र निरूप्यते~।

	\section{समवायसद्भावे प्रमाणपदर्शनम्} 

	'किं पुनः समवायसिद्धौ मानम्~?' इत्यादिना श्री वल्लभाचार्याः समवायसद्भावे प्रत्यक्षं प्रमाणं निराकृत्य  अनुमानप्रमाणं प्रदर्शयामसुः~। तथा हि~- समवायसद्भावे तस्यातीन्द्रियत्वात् प्रत्यक्षं प्रमाणं भवितुं नार्हति~। न च 'इह तन्तुषु पटः' इत्यादिप्रत्यक्षप्रतीतयः तत्र प्रमाणमिति वाच्यम्~। द्रव्यगुणकर्मसामान्यविशेषेषु विद्यमानस्य समवायस्य एकत्वात् एकसम्बन्धविशिष्टानां तेषां कस्य केन समवायः इति व्यवस्थानुपपत्तेः~। जलादिषु रूपादिसमवायस्यापि ग्रहणप्रसङ्गात्~। तर्हि तत्सत्त्वे किं मानमिति चेदुच्यते~- 'जात्यादि\footnote{अत्रादिशब्देन द्रव्यगुणकर्मविशेषाणां ग्रहणम्~।}विषयकः  विशिष्टव्यवहारः विशेषणविशेष्यसम्बन्धविषयः भावमात्रविषयाबाधितविशिष्टव्यवहारत्वात् 'सघटं भूतलमि'ति विशिष्टव्यवहारवत्' इत्यनुमानं प्रमाणम्~। अत्र विशिष्टव्यवहाराणां सर्वेषां विशेषणविशेष्यसम्बन्धविषयकत्वनियमात् जात्यादिविषयकविशिष्टव्यवहाराणामपि विशिष्टव्यवहारत्वात् सम्बन्धविषयकत्वं सिध्यति~। न च तत्र संयोगः सम्बन्धः, द्रव्ययोरेव संयोगसम्भवात्~। नापि स्वरूपसम्बन्धः, अभावस्यैव तन्नियमात्~। एतत्सूचनायैव हेतुकुक्षौ भावमात्रविषयत्वं योजितम् इति~।

	{\fontsize{11.7}{0}\selectfont\s अत्रोच्यते~- जात्यादिगोचरो विशिष्टव्यवहारः सम्बन्धनियतः भावमात्रविषयाबाधितविशिष्टव्यवहारत्वात् सघटं भूतलमिति विशिष्टव्यवहारवत्~।\footnote{न्या.ली. ७०९}}



	\section{समवायस्यैकत्वं नित्यत्वमतीन्द्रियत्वञ्च} 

	एकः नित्यश्च समवायः इति वैशेषिकभाष्यकाराणां प्रशस्तपादाचार्याणामाशयः~। तथा हि~- 'इह तन्तुषु पटः', 'इह घटे रूपम्' इत्यादिज्ञानेषु समवायनिमित्तस्य इहेत्यंशस्य दर्शनात्~, तस्य नानात्वे प्रत्यक्षादेः बाधकस्य सत्त्वात् लाघवाच्च एकः समवायः इति सिध्यति~। न च द्रव्यगुणकर्मसु विद्यमानानां द्रव्यत्वगुणत्वकर्मत्वादीनां सम्बन्धैकत्वात् तेषां सङ्करप्रसङ्गः~। द्रव्यत्वेनैव द्रव्यस्य भानात् गुणत्वादिना द्रव्यत्वाभानात् अन्वयव्यतिरेकाभ्यां समवायैक्येऽपि द्रव्यत्वादीनां द्रव्येष्वेव सम्बन्धः इति प्रतिनियमो सिध्यति~। यथा कुण्डदध्नोः संयोगस्य एकत्वेऽपि तयोः अधाराधेयभावः, तद्वत्~। 

	न च संयोगस्य यथा सम्बन्ध्यनित्यत्वात् अनित्यत्वं तथैव समवायस्यापि अनित्यत्वं भवतु इति वाच्यम्~। भावस्य तस्योत्पत्तौ उत्पादकाभावात् क्वचित् सम्बन्धि\footnote{परमाणुतद्गतपरिमाणादयः}नित्यत्वात् उत्पादकादिकल्पनाभावप्रयुक्तलाघवाच्च समवायः नित्यसम्बन्ध एव~। 

	ननु एतादृशः समवायः केन सम्बन्धेन तिष्ठति~? न संयोगेन, द्रव्ययोरेव संयोगनियमात्~। न समवायेन, तस्यैकत्वकथनात्~। नान्यः कोऽपि सम्बन्धो विद्यते इति चेन्न~। स्वरूपसम्बन्धविशेषख्यं तादात्म्यमेव समवायस्य सम्बन्ध इति~। 

	{\fontsize{11.7}{0}\selectfont\s ननु यद्येकः समवायः~? द्रव्यगुणकर्मणां द्रव्यत्वगुणत्वकर्मत्वादिविशेषणैः सह सम्बन्धैकत्वात् पदार्थसङ्करप्रसङ्ग इति~। न, आधाराधेयनियमात्~। यद्यप्येकः समवायः सर्वत्र स्वतन्त्रः, तथाप्याधाराधेयनियमोऽस्ति~। कथं द्रव्येष्वेव द्रव्यत्वम्~, गुणेष्वेव गुणत्वम्~, कर्मस्वेव कर्मत्वमिति~। एवमादि कस्मात्~? अन्वयव्यतिरेकदर्शनात्~। इहेति समवायनिमित्तस्य ज्ञानस्यान्वयदर्शनात् सर्वत्रैकः समवाय इति गम्यते~। द्रव्यत्वादिनिमित्तानां व्यतिरेकदर्शनात् प्रतिनियमो ज्ञायते~। यथा कुण्डदध्नोः संयोगैकत्वे भवत्याश्रयाश्रयिभावनियमः~। तथा द्रव्यत्वादीनामपि समवायैकत्वेऽपि व्यङ्ग्यव्यञ्जकशक्तिभेदादाधाराधेयनियम इति~। सम्बन्ध्यनित्यत्वेऽपि न संयोगवदनित्यत्वं भाववदकारणत्वात्~। यथा प्रमाणत उपलभ्यत इति~। कया पुनर्वृत्त्या द्रव्यादिषु समवायो वर्तते~? न संयोगः सम्भवति, तस्य गुणत्वेन द्रव्याश्रितत्वात्~। नापि समवायः, तस्यैकत्वात्~। न चान्या वृत्तिरस्ति~? न, तादात्म्यात्~। यथा द्रव्यगुणकर्मणां सदात्मकस्य भावस्य नान्यः सत्तायोगोस्ति~। एवमविभागिनो वृत्त्यात्मकस्य समवायस्य नान्या वृत्तिरस्ति, तस्मात् स्वात्मवृत्तिः~। अत एवातीन्द्रियाः सत्तादीनामिव प्रत्यक्षेषु वृत्त्यभावात्~, स्वात्मगतसंवेदनाभावाच्च~। तस्मादिह बुध्यनुमेयः समवाय इति~।\footnote{प्र.भा. ७७८-७८५}}

	\section{स्वरूपापेक्षया समवायः भिन्न एव} 

	श्रीमद्भिः गङ्गेशोपाध्यायैः तत्त्वचिन्तामणौ समवायसम्बन्धविचारः विशदतया निरूपितः~। तथा हि~- समवायसद्भावे  गुणक्रियाजातिविशिष्टबुद्धयो विशेषणसम्बन्धविषयाः विशिष्टबुद्धित्वात् दण्डीति बुद्धिवत्~, गुणक्रियाजातिविशिष्टबुद्धयो विशेषणसम्बन्धनिमित्तिकाः सत्यत्वे सति विशिष्टबुद्धित्वात्~, दण्डीति बुद्धिवत् इति वा अनुमानं  प्रमाणम्~। तत्रादौ प्रथमानुमाने 'नीलो घटः' इत्यादिविशिष्टबुद्धौ विशेषणसम्बन्धविषयकत्वं सिध्यति, तस्याः विशिष्टबुद्धित्वात्~। न च 'अघटं भूतलम्' इत्यादौ व्यभिचारः इति वाच्यम्~। तत्रापि विशिष्टबुद्धौ स्वरूपसम्बन्धस्य विषयत्वात्~। न च स्वरूपसम्बन्धमादाय अनाकांक्षितार्थाभिधानादर्थान्तरम्~। स्वरूपसम्बन्धेनैव प्रकृतपक्षेऽपि विशिष्टव्यवहारसम्भवादिति वाच्यम्~। गुणक्रियाजातिविशिष्टबुद्धीनां विशिष्टबुद्धित्वात् सम्बन्धविषयकत्वन्तु कल्पनीयमेव~। कल्प्यमानश्च सः लाघवज्ञानसहकृतपक्षधर्मताज्ञानेन एक एव सिद्ध्यति~। स च स्वरूपादतिरिक्तः~। अन्यथा अनन्तानां स्वरूपाणां सम्बन्धकल्पने आनन्त्यरूपगौरवप्रसङ्गः~।

	द्वितीयानुमानेऽपि विशेषणसम्बन्धः विशिष्टव्यवहारनिमित्तत्वेन सिध्यन् लाघवादेक एव सिध्यति~। तेन अनुगतानां तादृशविशिष्टबुद्धीनां अनुगतसम्बन्धजन्यत्वमपि उपपद्यते~। अस्मिन्ननुमाने हेतुकुक्षौ सत्यत्वदलं निवेशनीयम्~। अन्यथा भ्रमात्मकविशिष्टज्ञाने विशेषणसम्बन्धनिमित्तकत्वाभावात् व्यभिचारः स्यात्~। 

	न चेदमुभयमप्यप्रयोजकम्~। विशिष्टसाक्षात्कारस्य सम्बन्धविषयकत्वतज्जन्यत्वनियमात्~। अन्यथा गवाश्वादावपि विशिष्टबुद्धिप्रसङ्गः इति~।

	{\fontsize{11.7}{0}\selectfont\s  गुणक्रियाजातिविशिष्टबुद्धयो विशेषणसम्बन्धविषयाः विशिष्टबुद्धित्वात्~, दण्डीति बुद्धिवत्~। न च व्यभिचारः~। अभावादिविशिष्टबुद्धेरपि स्वरूपसम्बन्धविषयत्वात्~। न चैवमत्रापि तेनैवार्थान्तरम्~, यतो गुणक्रियाजातिविशिष्टबुद्धीनां पक्षधर्मताबलेन विषयः सम्बन्धः सिध्यन् लाघवादेक एव सिध्यति~। स एव समवायः, न तु स्वरूपसम्बन्धः, तत्स्वरूपाणामनन्तत्वात् सम्बन्धत्वेनाक्लृप्तत्वाच्च~।}

	{\fontsize{11.7}{0}\selectfont\s   अथ वा विशेषणसम्बन्धनिमित्तिका इति साध्यम्~। हेतौ तु सत्यत्वं विशेषणम्~। विशेषणसम्बन्धश्च कारणत्वेनैक एव सिध्यति~। लाघवात्~, अनुगतकार्यस्य अनुगतकारणनियम्यत्वाच्च~। न तु स्वरूपसम्बन्धः, तेषामननुगतत्वादनन्तत्वाच्च~। न चोभयमप्यप्रयोजकम्~। विशिष्टसाक्षात्कारस्य सम्बन्धाविषयत्वे तदजन्यत्वे वा गवाश्वादावपि विशिष्टबुद्धिप्रसङ्गात्~।\footnote{त.म. ३२}}


	\section{समवायोऽपि नैकः} 

	दीधितिकारेति प्रसिद्धाः श्रीरघुनाथशिरोमणयः स्वकीये पदार्थतत्त्वनिरूपणाख्ये ग्रन्थे समवायस्य नानात्वं प्रतिपादयामासुः~। तथा हि~- समवायः यदि एकः स्यात् तदा पृथिव्यां विद्यमानस्य गन्धस्य जलादावुपलब्धिप्रसङ्गः~। तत्रापि अनुयोगितया स्नेहसमवायस्य प्रसिद्धत्वात्~। एवं वाय्वादौ रूपाद्युपलब्धिप्रसङ्गः~। तस्मात् नानैव समवायः~। समवायत्वं तु न जातिः असम्बन्धात्~। अपि तु सकलसमवायानुगतः अखण्डोपाधिरेव इति~।

	{\fontsize{11.7}{0}\selectfont\s समवायोऽपि च नैको जलादेर्गन्धादिमत्त्वप्रसङ्गात्~। परन्तु नानैव, समवायत्वं तु पुनरनुगतमखण्डोपाधिरिति~।\footnote{प.नि. १४९}}


	\section{समवायः प्रत्यक्षः} 

	न्यायसारे तु समवायस्य प्रत्यक्षत्वमुपपादितम्~। तथा हि~- समवायस्य प्रत्यक्षत्वे कः सन्निकर्षः~? इति जिज्ञासायां वदति अन्तिमः विशेषणविशेष्यभावः सन्निकर्षः इति~। यथा 'इह भूतले घटो नास्ति' इति प्रतीतिः तथैव 'इह घटे रूपसमवायः' इति प्रतीतिरपि क्वचित् सम्भवति~। तदुपपादनाय कश्चन सन्निकर्षः अवश्यं ग्राह्यः~। संयोगस्य समवायस्य च प्रतियोगितया अनुयोगितया वा समवाये असत्त्वात् क्लप्तः विशेषणविशेष्यभाव एव सन्निकर्ष उच्यते~। तथा च घटे रूपसमवायः दैशिकविशेषणतासम्बन्धेन वर्तते इति एतेषामाशयः~।

	{\fontsize{11.7}{0}\selectfont\s एतत्पञ्चविधसम्बन्धसम्बन्धिविशेषणविशेष्यभावाद् दृश्याभावसमवाययोर्ग्रहणम्~। तद्यथा घटशून्यं भूतलम्~, इह भूतले घटो नास्ति~। एवं सर्वत्रोदाहरणीयम्~। समवायस्य तु क्वचिदेव ग्रहणम्~। यथा घटे रूपसमवायः, रूपसमवायवान् घटः इति~।\footnote{न्या.सा.१४,१५}}


	\section{विमर्शः}

	अयुतसिद्धानां सम्बन्धेतिप्रसिद्धस्य समवायस्य सत्त्वं तस्य नित्यत्वमेकत्वमैन्द्रियकत्वञ्चात्र विमृश्यते~।

		\subsection{समवायसत्त्वे प्रमाणविमर्शः}

		ननु 'नीलो घटः', 'शीतं जलम्'~, 'फलं पतति' इत्यादिप्रतीतयः लोके प्रसिद्धाः~। एतेषां संसर्गविषयकत्वन्तु सन्दिग्धमेव, विशेषणविशेष्ययोः मिथः भानासम्भवात्~, संसर्गस्य तत्र अप्रत्यक्षत्वाच्च~। न च 'दण्डी  पुरुषः' इत्यादिविशिष्टप्रतीतीनां संयोगादिसंसर्गविषयकत्वात् विशिष्टप्रतीतीनां सर्वेषां संसर्गविषयकत्वमभ्युपगन्तव्यम्~। तथा च 'नीलो घटः' इत्यादिप्रतीतीनामपि विशिष्टप्रतीतित्वात् संसर्गविषयकत्वं सिध्यति~। तथा चानुमानं 'नीलो घटः इति विशिष्टप्रतीतिः विशेषणविशेष्यसम्बन्धविषया विशिष्टप्रतीतित्वात् 'दण्डी पुरुष' इति विशिष्टप्रतीतिवत्' इति~। अनेन यः संसर्गः सिध्यति सः संयोगातिरिक्तः, द्रव्ययोरेव संयोगदर्शनात्~। न च स्वरूपसम्बन्धः, गौरवात्~। नापि विशेषणविशेष्यभावादिः, तस्योभयनिरूपितत्वाभावात् वृत्त्यनियामकत्वात्~। तस्मादतिरिक्तः कश्चन सम्बन्धः सिध्यति~। सैव समवायः इत्युच्यते~। कल्प्यमानश्चः सः लाघवादेकः नित्यश्च कल्प्यते इति चेन्न~। 

		'अघटं भूतलम्' इत्यादौ व्यभिचारात्~, स्वरूपेण अर्थान्तराच्च~। तत्र हि विशिष्टप्रतीतौ विशेषणविशेष्यापेक्षया अतिरिक्तः कोऽपि सम्बन्धः‌ न भासते~। भूतलस्वरूपस्यैव तत्र विशिष्टप्रतीतिनिर्वाहकसम्बन्धत्वात्~। तथा च विशेषणविशेष्यसम्बन्धविषयकत्वाभाववति 'अघटं भूतलम्' इत्यादौ विशिष्टप्रतीतित्वसत्त्वाद्व्यभिचारः~। न च तत्रापि भूतलस्वरूपमेव सम्बन्धः विशिष्टप्रतीतिनिर्वाहकः अस्त्येव~। तथा च केवल विशेषणविशेष्यसम्बन्धविषयकत्वमेव साध्यते~। न तु तत्र विशेषणविशेष्यातिरिक्तसम्बन्धविषयकत्वम्~। तथा च न व्यभिचार इति वाच्यम्~। तत्रापि तत्तत्स्वरूपेणैव विशिष्टप्रतीतिनिर्वाहसम्भवे अतिरिक्तसम्बन्धकल्पनमर्थान्तरम्~। न च हेतु कुक्षौ बाधकाभावे सत्यतिरिक्तसम्बन्धविषयकत्वरूपं विशेषणं दीयते~। 'अघटं भूतलम्' इत्यादिप्रतीतौ अधिकरणापेक्षयातिरिक्तस्यासम्बन्धत्वात् हेत्वभावः~। तथा च न व्यभिचार इति वाच्यम्~। भ्रमात्मकप्रतीतिमादाय व्यभिचारवारणाय बाधकाभावरूपदलस्यैव आवश्यकत्वात् अतिरिक्तसम्बन्धविषयकत्वदलं व्यर्थमेव~। 

		किञ्च बाधकाभावदळनिवेशे तु अप्रयोजकत्वम्~, साधकाभावदळेनापि विशेषणविशेष्यसम्बन्धविषयकत्वाभावसाधनसम्भवात् सत्प्रतिपक्षितमेवेदमनुमानम्~। न च साध्यकुक्षौ सम्बन्धिभिन्नत्वरूपं विशेषणं दीयते~। तथा च स्वरुपादतिरिक्तः सम्बन्धः सिध्यति इति वाच्यम्~। 'अघटं भूतलम्' इत्यादौ व्यभिचारात्~। तत्र विशिष्टप्रतीतित्वसत्त्वेऽपि सम्बन्धिभिन्नविशेषणविशेष्यसम्बन्धविषयकत्वाभावात् व्यभिचारः~।

		किञ्च अनेनानुमानेन यः सम्बन्धः सिध्यति स किं विशेषणतया भासते~? उत विशेष्यतया भासते~? स्वरूपेणभासते वा~? नाद्यः, विशिष्टप्रतीतेः पूर्वं तस्याज्ञानात्~, विशिष्टबुद्धौ विशेषणज्ञानस्य हेतुत्वात्~। न द्वितीयः, 'अनयोः समवायः' इत्यादिना समवायस्य विशेष्यतया अननुभवात्~। न तृतीयः 'समवायं जानामि' इति प्रतीतेरभावात्~। तस्मात् अतिरिक्तः सम्बन्धः गुणक्रियादिविशिष्टबुद्धौ नैव भासते~। 

		न च अवयविगुणक्रियाजातितद्वतां 'इह तन्तुषु पटः', 'इह पटे शौक्ल्यम्' इत्यादौ इहप्रत्ययः विशेषणविशेष्यसम्बन्धनिमित्तकः यथार्थेहप्रत्ययत्वात् 'इह कुण्डे बदरम्' इति इहप्रत्ययवदित्यनुमानं समवायसाधकम्~। तत्रेह प्रत्ययस्य यथा कुण्डबधरयोराधाराधेयभावनिमित्तकसंयोगाख्यसम्बन्धविषयकत्वं तथैव 'इह तन्तुषु पटः' इत्यादिप्रतीतौ इहप्रत्ययस्य विशेषणविशेष्यसम्बन्धनिमित्तकत्वमस्ति~। स च संयोगाद्यरितिक्तः समवाय एव इति वाच्यम्~। 'इह भूतले घटाभावः' इत्यादौ व्यभिचारात्~। तत्रापि स्वरूपसम्बन्धनिमित्तकस्य इहप्रत्ययस्य सत्त्वात् विशेषणविशेष्यापेक्षयातिरिक्तसम्बन्धस्याभावात्~। 

		न च 'शब्दजातिरूपादिः इन्द्रियसम्बद्धः प्रत्यक्षत्वात् घटवत्'  इत्यनुमानेन इन्द्रियसम्बन्धघटकतया समवायः सिध्यति~। तथा हि~- 'अयं घटः' इत्याद्याकारकद्रव्यप्रत्यक्षं यज्जायते तत्तु संयोगरूपेन्द्रियसन्निकर्षेणैव जायते~। इन्द्रियघटयोः द्रव्यत्वात् संयोगः तत्र उपपद्यते~। तद्वत् 'इदं नीलरूपम्' इत्यादयः शब्दजातिगुणादिविषयकाः अपि साक्षात्काराः अनुभूयन्ते~। तेऽपि प्रत्यक्षत्वात् इन्द्रियसम्बन्द्धाः एव~। रूपादिना सह इन्द्रियस्य संयोगासम्भवात् अन्यः एव सन्निकर्षः तत्र वक्तव्यः~। द्रव्येन सह सन्निकर्षानन्तरं तादृशप्रतीतेरुदयात् द्रव्यस्य गुणस्य च सम्बन्धः कश्चन सन्निकर्षघटकः इत्यभुपगन्तव्यम्~। स च संयुक्तसमवायः~। तथा च तद्घटकतया समवायः सिध्यति इति वाच्यम्~। अनया रीत्या समवायाभ्युपगमेऽपि 'भूतलं घटाभाववत्' इति प्रत्यक्षप्रतीतिनिर्वाहाय स्वरूपसम्बन्धस्य सन्निकर्षत्वेनापि अवश्यमभ्युपगम्यमानत्वात् तेनैव गुणादिव्यवहाराणां सिद्धौ किमतिरिक्तसम्बन्धकल्पनेन~? तथा हि रूपादीनां इन्द्रियसम्बद्धद्रव्यविशेषणतया, तत्रत्यरूपत्वादिजातीनामिन्द्रियसम्बन्द्धविशेषणविशेषणतया, शब्दस्य इन्द्रियविशेषणतया शब्दत्वस्य इन्द्रियविशेषणविशेषणतया प्रत्यक्षत्वमुपपादयितुं शक्यते~। तथा च तदर्थमतिरिक्तसम्बन्धकल्पनमनुचितमित्यर्थान्तरम्~।

		ननु 'तन्तुषु पटः', 'नीलो घटः' इत्याद्याः प्रतीतयः याः अनुभूयन्ते ताः एव अवयवावयविनोः‌ गुणगुणिनोः क्रियाक्रियावतोः जातिव्यक्त्योः समवायाख्यसम्बन्धसाधकाः इति चेन्न~। अन्योन्याश्रयात्~। तादृशप्रतीत्या समवायसिद्धिः, समवायसिद्धौ च तादृशविशिष्टप्रतीतयः इति अन्योन्याश्रयः~। किञ्च समवायाख्यस्य सम्बन्धस्य विशेषणविशेष्यापेक्षया अतिरिक्तत्वे तस्यापि कश्चन सम्बन्धः कल्पनीयः~। तस्यापि समवायत्वे अनवस्थाप्रसङ्गः~। तत्र अनवस्थाभिया तादात्म्यमेव सम्बन्धः कल्प्यते इति चेत् समवायस्थाने तादात्म्यसम्बन्धकल्पने एव लाघवात् समवायाख्यातिरिक्तसम्बन्धकल्पनं गुरुभूतम्~। तदुक्तम्~- {\fontsize{11.7}{0}\selectfont\s अवयवावयविनोः गुणगुणिनोः जातिजातिमतोः क्रियाक्रियावतोश्च परस्परं तादात्म्यमेव सम्बन्धः\footnote{मा.मे. २७५}} इति~। मैवम्~।

		'अवयविगुणक्रियाजातिविशिष्टबुद्धयः विशेषणविशेष्यसम्बन्धविषयाः विशिष्टबुद्धित्वात् दण्डीतिबुद्धिवत्'~,  गुणक्रियाजातिविशिष्टबुद्धयो विशेषणसम्बन्धनिमित्तिकाः सत्यत्वे सति विशिष्टबुद्धित्वात्~, दण्डीति बुद्धिवत् इति वा अनुमानं  प्रमाणम्~। तत्रादौ प्रथमानुमाने 'नीलो घटः' इत्यादिविशिष्टबुद्धौ विशेषणसम्बन्धविषयकत्वं सिध्यति, तस्याः विशिष्टबुद्धित्वात्~। न च 'अघटं भूतलम्' इत्यादौ व्यभिचारः इति वाच्यम्~। तत्रापि विशिष्टबुद्धौ स्वरूपसम्बन्धस्य विषयत्वात्~। न च स्वरूपसम्बन्धमादाय अनाकांक्षितार्थाभिधानादर्थान्तरम्~। स्वरूपसम्बन्धेनैव प्रकृतपक्षेऽपि विशिष्टव्यवहारसम्भवादिति वाच्यम्~। गुणक्रियाजातिविशिष्टबुद्धीनां विशिष्टबुद्धित्वात् विषयतया कश्चित्सम्बन्धः सिध्यति~। कल्प्यमानश्च सः लाघवज्ञानसहकृतपक्षधर्मताज्ञानेन एक एव सिद्ध्यति~। स च स्वरूपादतिरिक्तः~। अन्यथा अनन्तानां स्वरूपाणां सम्बन्धकल्पने आनन्त्यरूपगौरवप्रसङ्गः~।

		द्वितीयानुमानेऽपि विशेषणसम्बन्धः विशिष्टव्यवहारनिमित्तत्वेन सिध्यन् लाघवादेक एव सिध्यति~। तेन अनुगतानां तादृशविशिष्टबुद्धीनां अनुगतसम्बन्धजन्यत्वमपि उपपद्यते~। अस्मिन्ननुमाने हेतुकुक्षौ सत्यत्वदलं निवेशनीयम्~। अन्यथा भ्रमात्मकविशिष्टज्ञाने विशेषणसम्बन्धनिमित्तकत्वाभावात् व्यभिचारः स्यात्~। 

		न चेदमुभयमप्यप्रयोजकम्~। विशिष्टसाक्षात्कारस्य सम्बन्धविषयकत्वतज्जन्यत्वनियमात्~। अन्यथा गवाश्वादावपि विशिष्टबुद्धिप्रसङ्गः इति~।

		नव्यास्तु 'गुणक्रियादिविशिष्टबुद्धिः सम्बन्धिभिन्नविशेषणसम्बन्धविषया निर्विषयकभावविशेषणकविशिष्टबुद्धित्वात् दण्डीतिबुद्धिवत्'~, 'गुणक्रियादिविशिष्टबुद्धिः सम्बन्धिभिन्नविशेषणसम्बन्धविषया इतरनिरूपणानिरूप्यविशेषणकविशिष्टबुद्धित्वात् दण्डीतिबुद्धिवत्' इति वा अनुमानं समवायं साधयति~। अत्र सम्बन्धिभिन्नविशेषणसम्बन्धविषयकत्वाभाववति 'भूतलं घटाभाववत्'~, 'भूतलं ज्ञातम्' इत्यादिज्ञानेषु विशिष्टबुद्धित्वसत्त्वाद्व्यभिचारः~। अतस्तद्वारणाय प्रथमानुमाने हेतुकुक्षौ भावविशेषणकत्वस्य निर्विषयकत्वस्य च निवेशः~। 'भूतलं घटाभाववत्' इति बुद्धेः‌ अभावविशेषणकत्वात् 'भूतलं ज्ञातम्' इति बुद्धेश्च सविषयकपदार्थविशेषणकत्वाच्च हेत्वभावान्न व्यभिचारः~। द्वितीयानुमाने तु घटाभावस्य घटरूपप्रतियोगिनिरूपितत्वात् ज्ञानस्य तु विषयनिरूपितत्वात् इतरनिरूपणनिरूप्यविशेषणकत्वमेव इति न व्यभिचारः~। न चेदमप्रयोजकम्~। उपाधिसत्त्वे एव अस्य अप्रयोजकत्वसिद्धिः~। प्रकृते उपाधेरेवाभावात् नेदमप्रयोजकम् इति~।

		अथवा 'गुणक्रियाजातिविषयकविशिष्टसाक्षात्कारः इन्द्रियसम्बन्धसाध्यः जन्यप्रत्यक्षत्वात् दण्डिज्ञानवत्' इत्यनुमानेन इन्द्रियसम्बन्धघटकतया समवायः सिध्यति~। न चात्रापि स्वरूपेणार्थान्तरम् दोषः~। 'सिध्यतः सम्बन्धस्य एकत्वे नित्यत्वे च लाघवम्' इति लाघवज्ञानसहकृतपरामर्शात् कल्पनीयश्च सम्बन्धः एकः नित्यश्च कल्प्यते~। स्वरूपसम्बन्धस्य तु तत्तत्स्वरूपाणामनन्तत्वात् अनित्यत्वाच्च न तेन अर्थान्तरम्~। एवं जात्यादिसाक्षात्कारे इन्द्रियसम्बद्धविशेषणविशेषणतायाः सन्निकर्षत्वस्वीकारे महद्गौरवं तदपेक्षया समवायघटितसन्निकर्षे एव लाघवमिति वदन्ति~।

		नन्वत्र लाघवज्ञानसहकृतपरामर्शजन्यानुमानेन समवायः सिध्यति इति सर्वत्र प्रतिपादितम्~। किन्तु अभावस्थले विशिष्टबुद्धिनिर्वाहाय अवश्यं स्वरूपसम्बन्धः अभ्युपेयः~। तथा च स्वरूपसम्बन्धस्य क्लृप्तत्वात् समवायस्य तु कल्पनीयत्वात् समवायकल्पने एव गौरवम् इति चेन्न~। गुणक्रियाजातिविषयकविशिष्टबुद्धिनियामकस्य स्वरूपसम्बन्धत्वे गुणजात्यादिशून्यस्य द्रव्यस्य कदाचित्प्रत्यक्षत्वप्रसङ्गः~। तथा हि~- अभावविशिष्टस्य ज्ञानविशिष्टस्य च भूतलादेः यथा 'नीलवद्भूतलम्' इत्यादौ तदविशेषणतया प्रतीतिः सम्भवति तद्वत् गुणादिशून्यस्यापि द्रव्यादीनां कदाचित् ग्रहणप्रसङ्गः~। कदाचिदपि तादृशप्रतीतेरभावात् तत्र स्वरूपादतिरिक्त एव सम्बन्धः तादृशप्रतीतिनियामकः इत्युच्यते~।

		ननु गुणादीनां द्रव्याणाञ्च पृथक्तया प्रतीतेरभावात् तत्र तादात्म्यसम्बन्ध एव भवतु~। अन्यथा समवायस्यापि सम्बन्धिभेदसाधनार्थमन्यः तादात्म्यसम्बन्धः वक्तव्यः इति चेन्न~। गुणादीनां द्रव्यादिना सह तादात्म्यसम्बन्धाभ्युपगमे द्रव्याद्यपेक्षया गुणादीनामतिरिक्तत्वं न सिध्यति~। तथा च आत्मनि ज्ञानादीनां क्षणिकत्वानुभवात् तदा आत्मनः अपि नाशप्रसङ्गः~। तस्माद्द्रव्याद्यपेक्षया गुणादीनामतिरिक्तत्वमवश्यमभ्युपगन्तव्यमेव~। अतः तन्निर्वाहाय तादात्म्यादतिरिक्त एव सम्बन्धः स्वीकार्यः~। स च समवाय इत्युच्यते~। तस्योत्पत्तिविनाशाकल्पनाप्रयुक्तलाघवात्तस्य नित्यत्वमभ्युपगम्यते~।

		इदन्तु बोध्यम्~- अनवस्थाभिया समवायस्य स्वानुयोगिना सह स्वरूपसम्बन्धाभ्युपगमात् अभावसाक्षात्कारप्रयोजकस्य विशेष्यविशेषणभावसन्निकर्षस्य स्वरूपेणस्थितपदार्थग्राहकत्वाच्च समवायः प्रत्यक्ष एव~। न च सम्बन्धप्रत्यक्षं प्रति यावत्सम्बन्धिप्रत्यक्षस्य कारणत्वात्~, समवायसम्बन्धिनां परमाण्वाकाशादीनामप्रत्यक्षत्वात् न तस्य प्रत्यक्षत्वमिति वाच्यम्~। आकाशादिसंयोगस्य अप्रत्यक्षत्वादेवायं नियमो स्वीकृतः~। तथा च संयोगप्रत्यक्षं प्रत्येव यावत्सम्बन्धिप्रत्यक्षत्वमभ्युपगन्तव्यम्~। यदि सम्बन्धमात्रे स नियमः कल्प्यते इत्युच्यते तदा समवायस्य एकत्वे अप्रत्यक्षत्वं युज्यते~। तस्य नानात्वे तु तत्प्रत्यक्षत्वापलापो नैव सम्भवति~। समवायस्य प्रत्यक्षत्वे कारिकापि श्रूयते~- 'प्रत्यक्षं समवायस्य विशेषणतया भवेत्'\footnote{कारिका. ६१} इति~।

		\subsection{समवायस्यैकत्वविमर्शः}

		आनुमानप्रमाणेन यः सम्बन्धः गुणादीनां कल्प्यते तस्य नानात्वे प्रत्यक्षादेः बाधकस्य असत्वात् लाघवज्ञानसहकृतपरामर्शकार्यत्वाच्च एकः समवायः इति सिध्यति~। ननु तस्य एकत्वे तेन सम्बन्धेन द्रव्यत्वस्य द्रव्यवृत्तित्वम्~, न तु गुणवृत्तित्वमिति कथं सिध्यति~। द्रव्यत्वसमवायस्य गुणेऽपि सत्त्वात्~, गुणस्य समवायानुयोगित्वात्~, समवायस्य च एकत्वात्~। तथा च द्रव्यत्वादीनां सङ्करप्रसङ्गः~। किञ्च वायौ स्पर्शसमवायस्य सत्त्वात् पृथिव्यादिषु रूपसमवायस्य प्रसिद्धत्वात् वायोरपि रूपाधिकरणत्वप्रसङ्गः~। तथा च वायोरपि चाक्षुषत्वापत्तिः इति चेन्न~। अधाराधेयभावनियमात् द्रव्यत्वसमवायस्य द्रव्यवृत्तित्वमेवाभ्युपगम्यते, न तु गुणादिवृत्तित्वम्~। एवं गुणत्वादीनां तत्तदधिकरणवृत्तित्वमेव तेन सम्बन्धेन उच्यते, न त्वन्यत्र~। तथा च वायौ समवायसत्त्वेऽपि रूपाधिकरणत्वाभावात् न तत्र रूपोपलब्धिः, नापि तस्य चाक्षुषत्वम्~। ननु आधाराधेयभावनियामकः कः~? इति चेत् अन्वयव्यतिरेकसहचारावेव~। सर्वदा सर्वत्र द्रव्येष्वेव द्रव्यत्वदर्शनात् गुणादौ तददर्शनात् द्रव्यस्यैव द्रव्यत्वाधारत्वमभ्युपगन्तव्यम्~। एवमेव अन्यत्रापि~। अन्यथा 'भूतलं घटवत्' इत्यत्रापि भूतलघटयोर्मध्ये विद्यमानसंयोगस्य एकत्वात् भूतलस्यैव आधारत्वं घटस्यैव आधेयत्वमिति न सिध्येत्~। तथा च 'घटः भूतलवान्' इत्यपि तत्र प्रतीतिः स्यात्~। किन्तु तत्र भूतलस्य घटाधेयत्वं यथा अन्वयव्यतिरेकाभ्यां नाभ्युपगम्यते तद्वदत्रापि वक्तव्यम् इति चेत्~।

		अत्रोच्यते~- यद्यपि भूतलघटयोः आधाराधेयभावः अन्वयव्यतिरेकाभ्यां सिध्येत तथापि तन्न समवायैकत्वं साधयति~। अपि तु द्रव्येष्वेव द्रव्यत्वम्~, न तु द्रव्यत्वे द्रव्यमिति विपरीताधाराधेयभावं निवारयति~। विपरीताधाराधेयभावनिवृत्त्या सम्बन्धैकत्वं नैव सिध्यति~। अन्यथा संयोगस्यापि एकत्वगप्रसङ्गः~। न च तत्र प्रत्यक्षं बाधकम्~। अत्र तु समवायस्य अतीन्द्रियत्वान्न प्रत्यक्षं बाधकमिति वाच्यम्~। बाधकान्तरसत्त्वात्~। न हि एककालावच्छेदेन सम्बन्धप्रतियोग्यनुयोगिनोः सत्त्वे सम्बन्धस्य च एकत्वे तादृशानुयोगिनि प्रतियोगिप्रकारकबुध्युत्पत्तौ बाधकमस्ति~। अन्वयव्यतिरेकसहचारस्तु न बाधकम्~। तस्य आधाराधेयभावनियामकत्वात्~, न अन्यप्रतियोगिनः अन्यानुयोगिवृत्तित्वनिषेधकत्वम्~। तथा च एकस्मिन् क्षणे वायुसत्त्वे पृथिव्यादिगतरूपादीनाञ्च तस्मिन्नेव क्षणे सत्त्वे तयोः इन्द्रियसन्निकर्षे जाते तदुत्तरक्षणे वायौ रूपवत्ताबुद्धिप्रसङ्गः~। किञ्च आधाराधेयभावनियामकसम्बन्धाः संयोगादयः द्विनिष्ठाः एव, न तु तदधिकवृत्तयः इति लोके प्रसिद्धाः~। यत्र घटपटभूतलानां संयोगः तत्र स च संयोगः नैकः, घटस्य भूतलेन सह संयोगः पटेन सह संयोगश्च अन्य एव इत्यनुभवः~। न च 'भूतलं घटपटोभयाभाववत्' इत्यत्र भूतलस्य घटाभावपटाभावयोः साकमेकमेव सम्बन्धः~। सैव तत्राधाराधेयभावनियामकः इति वाच्यम्~। अभावस्थले तथा दर्शनेऽपि भावस्थले तथा अदर्शनात्~। एवञ्च भावयोः आधाराधेयभावसम्बन्धः द्विनिष्ठ एव~। तथा च एतादृशनियमानुपपत्तिः वाय्वादौ रूपवत्तप्रतीत्यापत्तिश्च समवायस्यैकत्वे बाधिका~। एवञ्च बाधकस्यैव सत्त्वात् समवायस्य नानात्वमभ्युपगन्तव्यमेव~।

		किञ्च आधाराधेयभावनियामकत्वं सम्बन्धस्यैव स्वीकर्तव्यम्~। अत एव वृत्तिनियामकः वृत्त्यनियामकश्चेति सम्बन्धानां द्विधा व्यवहारो शास्त्रप्रपञ्चे श्रूयते~। एवञ्च वाय्वादौ रूपाधारत्वं रूपसमवायासत्त्वे हि सिध्यति~। रूपसमवायाभावश्च समवायनानात्वपक्षे एव उपपद्यते इति~।

		न च समवायस्य नानात्वे लाघवाभावात् स्वरूपापेक्षया अतिरिक्तसम्बन्धकल्पनमयुक्तमिति वाच्यम्~। स्वरूपसम्बन्धो हि युतसिद्धानामेव सम्भवति~। समवायः यत्रोच्यते तत्र सम्बन्धिनोः अयुतसिद्धत्वमेव~। अयुतसिद्धयोः सम्बन्धिनोर्मध्ये प्रतियोगिभूतं अविनश्यदवस्थापन्नमनुयोगिनमेव आश्रियते~। न च अभावस्थले तथेति सम्बन्धिनोः वैलक्षण्यात् सम्बन्धस्यापि भेद इति~। किञ्च समवायस्य नानात्वेऽपि तेषां नित्यत्वात् लाघवमस्त्येव~। तत्तदधिकरणस्वरूपाणान्तु भूतलादीनां नाशादनित्यत्वमेवेति~।
		
		\subsection{समवायस्यैन्द्रियकत्वविमर्शः}

		परे तु~- विशेषणविशेष्यभावसन्निकर्षेण समवायाभावयोर्ग्रहणमिति वदन्ति~। तथा चानुमानम्~- 'समवायः प्रत्यक्षः प्रत्यक्षप्रयोजकसम्बन्धत्वात् संयोगवत्' इति~। प्रत्यक्षप्रयोजकेन्द्रियसम्बन्धानां संयोगादीनां प्रत्यक्षत्वात् समवायस्यापि प्रत्यक्षप्रयोजकत्वात् तस्यापि प्रत्यक्षत्वं सिध्यति इति~। तन्न~। 

		अनुमानस्य व्यभिचारदोषकबळितत्वात्~। तथा हि~- इन्द्रियद्रव्यसंयोगः प्रत्यक्षप्रयोजकस्तावदप्रत्यक्ष एव, सम्बन्धप्रत्यक्षं प्रति यावत्सम्बन्धिप्रत्यक्षस्य कारणत्वात्~, इन्द्रियस्य चाप्रत्यक्षत्वात्~। तथा च प्रत्यक्षत्वाभाववति तादृशसंयोगे प्रत्यक्षकारणत्वस्य सत्त्वात् व्यभिचारः~। न च समवायेन साकम् इन्द्रियसन्निकर्षस्य कारणस्य सत्त्वात् तस्य प्रत्यक्षत्वमिति वाच्यम्~। समवायो हि स्वानुयोगिनि तादात्म्येन वर्तते~। तादात्म्यञ्च स्वापेक्षया नातिरिक्तः सम्बन्धः~। तस्मात् प्रतियोगिना सह इन्द्रियसम्बन्धाभावात् न समवायस्य इन्द्रियैः ग्रहणमिति~। अभावस्थले तु  स्वरूपसम्बन्धः अनुयोग्यात्मकः, न तु प्रतियोग्यात्मकः~। अनुयोगिना सह इन्द्रियसंयोगादेः सम्भवात् तत्र विशेषणतया अभावादेर्भानं सम्भवति~। किञ्च 'समवायं साक्षात्करोमि' इत्यननुभवात् तस्य अप्रत्यक्षत्वं सिध्यति~।

		इदन्तु चिन्त्यम्~- सम्बन्धप्रत्यक्षं प्रति अन्वयव्यतिरेकाभ्यां यावतां सम्बन्धिनां प्रत्यक्षस्य कारणत्वं सिद्धम्~। अत एव घटाकाशयोः संयोगः आकाशस्याप्रत्यक्षत्वात् नैव इन्द्रियग्राह्यः~। तथा च समवायस्थले तस्यैकत्वे परमाण्वाकाशादीनां समवायसम्बन्धिनामप्रत्यक्षत्वात् तस्याप्रत्यक्षत्वमेव~। तस्य नानात्वे तु तन्तुपटादीनां रूपघटादीनां पतनफलादीनां गोत्वगवादीनाञ्च समवायसम्बन्धिनां प्रत्यक्षत्वात् तस्यापि प्रत्यक्षत्वमभ्युपगन्तव्यम्~। न च तत्र इन्द्रियसन्निकर्षरूपकारणान्तरविरहः~। स्वरूपसम्बन्धस्यापि इन्द्रियसन्निकर्षघटकत्वात्~। न च 'समवायं साक्षात्करोमि' इत्यननुभवः~। क्वचित् रूपघटयोः सम्बन्धविषयकज्ञानस्याप्यनुभवात् इति~।

		\subsection{शास्त्रान्तरेषु समवायविमर्शः}

		पूर्वोत्तरमीमांसकास्तु समवायाख्यं सम्बन्धं नाङ्गीकुर्वन्ति~। अयुतसिद्धत्वञ्च अभिन्नयोरेव इति तेषामाशयः~। अभिन्नपदार्थयोः अतिरिक्तसम्बन्धासम्भवात् न समवायः कल्पनीयः इति तेषामाशयः~। एतेषां नये गुणद्रव्यादीनामभेद एव~। तथा हि~- 

		{\fontsize{11.7}{0}\selectfont\s गुणानां द्रव्याधीनत्वं द्रव्यगुणयोरयुतसिद्धत्वादिति यदुच्यते, तत्पुनरयुतसिद्धत्वमपृथग्देशत्वं वा स्यात्~, अपृथक्कालत्वं वा, अपृथक्स्वभावत्वं वा~? सर्वथापि नोपपद्यते — अपृथग्देशत्वे तावत्स्वाभ्युपगमो विरुध्येत~। कथम्~? तन्त्वारब्धो हि पटस्तन्तुदेशोऽभ्युपगम्यते, न पटदेशः~; पटस्य तु गुणाः शुक्लत्वादयः पटदेशा अभ्युपगम्यन्ते, न तन्तुदेशाः~; तथा चाहुः~— ‘द्रव्याणि द्रव्यान्तरमारभन्ते गुणाश्च गुणान्तरम्’\footnote{वै. सू. १-१-१०} इति~; तन्तवो हि कारणद्रव्याणि कार्यद्रव्यं पटमारभन्ते, तन्तुगताश्च गुणाः शुक्लादयः कार्यद्रव्ये पटे शुक्लादिगुणान्तरमारभन्ते~— इति हि तेऽभ्युपगच्छन्ति~; सोऽभ्युपगमो द्रव्यगुणयोरपृथग्देशत्वेऽभ्युपगम्यमाने बाध्येत~। अथ अपृथक्कालत्वमयुतसिद्धत्वमुच्येत, सव्यदक्षिणयोरपि गोविषाणयोरयुतसिद्धत्वं प्रसज्येत~। तथा अपृथक्स्वभावत्वे त्वयुतसिद्धत्वे, न द्रव्यगुणयोरात्मभेदः सम्भवति, तस्य तादात्म्येनैव प्रतीयमानत्वात्~॥ युतसिद्धयोः सम्बन्धः संयोगः, अयुतसिद्धयोस्तु समवायः~— इत्ययमभ्युपगमो मृषैव तेषाम्~, प्राक्सिद्धस्य कार्यात्कारणस्यायुतसिद्धत्वानुपपत्तेः~।\footnote{ब्र.सू.शां.भा}}

		अत्रोच्यते~- द्रव्येषु रूपादिगुणानां परिवर्तनस्य अनुभवसिद्धत्वात् न तत्र रूपादीनां द्रव्येण साकमभेदः सम्भवति~। अभेदाभावे तु तयोः अतिरिक्तः सम्बन्धो अवश्यमभुपेयः~। सैव समवाय इत्यलं विस्तरेण~।


\begin{center}\begin{small}॥ इति प्राचीनन्याय-वैशेषिक-नव्यन्यायशास्त्रेषु तत्तद्व्याख्याकाराणां सैद्धान्तिकमतभेदानां विमर्शात्मकमध्ययनमिति प्रबन्धे  समवायप्रपञ्चे सैद्धान्तिकमतभेदाः इत्याख्यः चतुर्थोऽध्यायः~॥\end{small}\end{center}



\titleformat {\chapter}[display]{\normalfont\Large} % format
{अथ पञ्चमोऽध्यायः\\[1mm]} % label
{-3.8ex}{ \rule{\textwidth}{1pt}\vspace{-5ex}
\centering
} % before-code
[
\vspace{-6.7ex}%
\rule{\textwidth}{1pt}
]
\titlespacing*{\chapter} {10pt}{-60pt}{50pt}


\chapter{अभावप्रपञ्चे सैद्धान्तिकमतभेदाः}

'भूतले घटो नास्ति', 'वायौ रूपं नास्ति' इत्यादिप्रतीतिसिद्धः‌ अभावाख्यः पदार्थः~। 'अभावस्य भूयसि प्रमेये लोकप्रसिद्धे वैयात्यादुच्यते 'नाभावः प्रामाण्यं प्रमेयासिद्धेः' इति'\footnote{न्या. भा. २.२.७}~। इति वदन् भाष्यकारः सार्वजनीनप्रतीतिसिद्धः अभावाख्यः प्रमेयः नापलापमर्हतीति सूचयति~। षट्पदार्थप्रतिपादकेनापि कणादमुनिना 'क्रियागुणव्यपदेशाभावात् प्रागसत्'\footnote{वै. सू. ९.१.१}, 'सदसत्'\footnote{वै. सू. ९.१.२} इत्यादिसूत्राभ्यां भावातिरिक्तत्वेनाभावाख्यः प्रमेयो न्यरूपि~। इत्थञ्च प्रत्यक्षप्रमाणसिद्धोऽभावो प्रागभावप्रध्वंसश्चेति द्विविधः, अत्यन्ताभावेतरेतराभावयोरपि प्रमाणसिद्धत्वात् चतुर्विधः इति च कथयन्ति युक्तिकोविदाः~। तत्रापि प्रागभावध्वंसयोरधिकरणे अत्यन्ताभावः नास्तीति केचन, न तेषु विरोधः इत्यपरे वदन्ति~। तदत्र विविच्यते~-

	\section{अभावो द्विविधः}

	'लक्षितेष्वलक्षणलक्षितत्वादलक्षितानां तत्प्रमेयसिद्धिः'\footnote{न्या. सू. २.२.८} इति सूत्रेण, केषाञ्चिद्वाससां लक्षणानि कानिचिदुक्त्वा एतानि लक्षणानि येषु वासस्सु न सन्ति तान्यानयेति प्रेरितः कश्चित् येषु वस्त्रेषु लक्षणानि न भवन्ति तानि लक्षणाभावेन प्रतिपद्यते, प्रतिपद्य च आनयति~। तत्र लक्षणाभावोऽपि वासोविशेषप्रमितिहेतुत्वात् प्रमाणमिति प्रतिपादितम्~। लक्षणाभावो हि लक्षणध्वंसः, स च लक्षणशून्येषु वासस्सु कथं तदनन्तरं भवितुमर्हति, ध्वंसस्य प्राप्तिपूर्वकत्वात् इत्याक्षेपः~- 'असत्यर्थे नाभाव इति चेत्'\footnote{न्या. सू. २.२.९} इति सूत्रखण्डेनोत्थापितः~। तत्र समाधानकथनावसरे 'प्रागुत्पत्तेरभावोपपत्तेश्च'\footnote{न्या. सू. २.२.१२} इति सूत्रम्~। तेषु वासस्सु लक्षणध्वंसरूपस्याभावस्यासत्त्वेऽपि लक्षणप्रागभावरूपोभावः सम्भवति~। स एव च प्रत्यायकः इति तत्सूत्रस्य आशयः~। अनेन च प्रागभावध्वंसौ द्वावेवाभावौ सूत्रकारसम्मतौ इत्यवगम्यते~। एतदेव प्रकटीकृतं भाष्ये~-

	{\fontsize{11.7}{0}\selectfont\s अभावद्वैतं खलु भवति, प्राक्चोत्पत्तेरविद्यमानता~। उत्पन्नस्य चात्मनो हानादविद्यमानता~। तत्रालक्षितेषु वासस्सु प्रागुत्पत्तेरविद्यमानतालक्षणो लक्षणानामभावो नेतरः इति~।\footnote{न्या. भा. १७१}} इति~।

		\subsection{इतरेतराभावत्यन्ताभावौ प्रागभावान्नातिरिक्तौ}

		न्यायमञ्जरीकास्तु प्रागभावः प्रध्वंसाभावश्चेति अभावो द्विविधः~। अन्योन्याभावात्यन्ताभावौ प्रागभावान्नातिरिक्तौ~। यदा प्रागभावः स्वाधिकरणान्यानुयोगिसम्बन्धप्रतियोगिको भवति तदा तस्य अन्योन्याभावत्वेन व्यवहारः~। यदा तु तत्प्रतियोग्युत्पत्तिर्न ज्ञाता तदा तस्यैव अत्यन्ताभावत्वेन व्यवहारः~। इति कथयति~। तथा हि~-

		{\fontsize{11.7}{0}\selectfont\s स च द्विविधः, प्रागभावः प्रध्वंसाभावश्चेति~। चतुर्विध इत्यन्ये, इतरेतराभावः अत्यन्ताभावश्च~। ..... तत्र च~-\\ \begin{center}प्रागात्मालाभान्नास्तित्वं प्रागभावोऽभिधीयते~।\\ उत्पन्नस्य हानन्तु प्रध्वंस इति कथ्यते~॥\\ न प्रागभावादन्ये तु भिद्यन्ते परमार्थतः~।\\ स हि वस्त्वन्तरोपाधिरन्योन्याभाव उच्यते~॥\\ स एवावधिशून्यत्वादत्यन्ताभावतां गतः~।\footnote{न्या. मं. ५९}\end{center}}

		\subsection{अभावश्चतुर्विधः}

		प्रमाणसिद्धोप्यभावः किमर्थं नोद्दिष्टः~? इत्याशङ्कायाम्~, अभावज्ञानस्य स्वप्रतियोगिज्ञानाधीनत्वात् प्रतियोगिनां ज्ञाने सति तस्य सुलभेनावगमात् भावपदार्थाः एव‌ मुनिना भाष्यकारेण च निरूपिताः~। अभावस्य नित्यत्वानित्यत्वशङ्कायां प्रागभावध्वंसयोरनित्यत्वम्~, अत्यन्ताभावेतरेतराभावयोश्च नित्यत्वमित्युक्तम्~। एवञ्च प्रागभावादतिरिक्तानाम् अत्यन्ताभावान्योन्याभावानाञ्च सत्त्वात् अभावश्चतुर्विध‌ इति ज्ञायते~। तदुक्तं किरणावल्याम्~- 

		{\fontsize{11.7}{0}\selectfont\s अभावस्तु स्वरूपवानपि पृथङ्निर्दिष्टः~। प्रतियोगिनिरूपणाधीननिरूपणत्वात्~, न तु तुच्छत्वम्~। उत्पत्तिविनाशचिन्तायां प्रागभावप्रध्वंसाभावयोः, वैधर्म्ये चेतरेतराभावात्यन्ताभावयोस्तत्र तत्र प्रदर्शयिष्यमाणत्वात्~।\footnote{किरणा. ६}}

		\subsection{अत्यन्तासतः प्रतिषेधोऽत्यन्ताभावः}

		न्यायकन्दलीकारोऽपि चतुर्धा विभक्तस्याभावस्य अस्तित्वं मन्यमानः‌ अत्यन्तासतः शशविषाणादेः अभाव एवात्यन्ताभाव इति न्यनूपयत् '। अयं च तद्ग्रन्थः~-

		{\fontsize{11.7}{0}\selectfont\s अत्यन्ताभावस्तु सर्वथा असद्भूतस्यैव बुद्धावारोपितस्य देशकालानवच्छिन्नः प्रतिषेधः, यथा षट्पदार्थेभ्यो नान्यत् प्रमेयमस्ति~।\footnote{न्या. कं. २३०}}

		\subsection{प्रागभावध्वंसात्यन्ताभावानां वैरुध्यम्}

		वाचस्पतिमिश्रैस्तु प्रागभावध्वंसात्यन्ताभावानां परस्परविरोधः कथितः~। तत्प्रत्यपादि स्वकीयग्रन्थे न्यायवार्तिकतात्पर्यटीकायाम्~-

		{\fontsize{11.7}{0}\selectfont\s अनुत्पन्नो विनष्टोऽत्यन्तासन्नित्यत्राप्यभावत्रये विवक्षितविपरीतमापद्यत इति मन्तव्यम्~। .....  अत्राप्यनुत्पादो विनाशोऽत्यन्ताभाव इति विरुद्धं स्यादिति मन्तव्यम्~।\footnote{न्या. वा. ता. टी.}}

		नव्यनैयायिकास्तु प्रागभावध्वंसयोरत्यन्ताभावेन सह विरोधं न सहन्ते~। घटध्वंसानन्तरमपि तादृशघटप्रतियोगिकात्यान्ताभावविषयिण्याः 'घटो नास्ति' इति प्रतीतेः तादृशध्वंसाधिकरणे सम्भवात् तत्रापि अत्यन्ताभावोभ्युपेयः~।

		\subsection{प्रागभावध्वंसयोरत्यन्ताभावेन सह न विरोधः}

		यत्र घटे आदौ रक्तरूपं पाकवशान्नष्टं ततः श्यामरूपमुत्पन्नं ततः पाकवशात् श्यामनाशे रक्तरूपमुत्पन्नं तत्र श्यामरूपकाले रक्तप्रागभावध्वंसयोरदिकरणे 'रक्तरूपं नास्ति' इति प्रतीतिबलात् रक्तात्यन्ताभावः अवश्यं स्वीकर्तव्यः~। न चास्य रक्तध्वंसः तत्प्रागभावो वा विषयः, रक्तप्रागभावे सति रक्तध्वंसात्पूर्वम्~, उत्तरकाले रक्तोत्पत्त्यनन्तरं रक्तध्वंसस्य सत्त्वा 'रक्तो नास्ति' इति प्रतीत्यापत्तिः~। न चेदृशस्थले ध्वंसप्रागभावयोः रक्तत्वावच्छिन्नप्रतियोगिताकत्वमुपगन्तव्यम्~, पूर्वं रक्तप्रागभावकाले रक्तरूपस्यैव सत्त्वात् न रक्तत्वावच्छिन्नप्रतियोगिताकत्वं रक्तप्रागभावस्य एवमग्रेऽपीति, तथा च अन्तराश्यामे रक्तप्रागभावध्वंसयोरसत्त्वात्तत्र अत्यन्ताभावप्रतीतिरिति वाच्यम्~। विनिगमकाभावात् पदार्थान्तरेऽपि ध्वंसप्रागभावयोरीदृशीस्वरूपकल्पना करणीया इति गौरवम्~। तस्मात् तयोर्विरोध एव न स्वीक्रियत इति~। तदुक्तं दिनकर्याम्~-

		{\fontsize{11.7}{0}\selectfont\s कथमन्यथा पूर्वमग्रेऽपि रक्ते मध्ये श्यामे च घटे रक्तं रूपं नास्तीत्याकारकप्रत्ययः, तस्य सामान्याभावावगाहित्वात्~। न चेदृशप्रत्ययो रक्तिमध्वंसाद्यवगाही पूर्वापररक्तिमध्वंसप्रागभाववति रक्तेऽपि तादृशाकारप्रत्ययप्रसङ्गादिति~। न च रक्तत्वावच्छिन्नप्रतियोगिताकत्वं ध्वंसप्रागभावयोरव्याप्यवृत्ति, तेनान्तराश्यामे रक्तं रूपं नास्तीति प्रत्ययस्तदा ध्वंसप्रागभावयोः सामान्यावच्छिन्नप्रतियोगिताकत्वस्वीकारात् न तु पूर्वापररक्तिमध्वंसप्रागभाववति रक्ते तत्र तदा तदनभ्युपगमादिति वाच्यम्~। विनिगमनाविरहेण पदार्थान्तरध्वंसप्रागभावयोरपि तत्स्वीकारप्रसङ्गेन गौरवात्~, तदपेक्षया अत्यन्ताभावस्यान्तराश्यामादौ सम्बन्धकल्पनेनैव लाघवादित्यन्यत्र विस्तरः~।\footnote{दिनकरी ९९-१००}}


	\section{विमर्शः}

		\subsection{ध्वंसप्रागभावयोरत्यन्ताभावेन सह विरोधे प्राचां न्ययविदामाशयः}

		प्राचीनन्याये तु ध्वंसः प्रागभावश्चेति अभावद्वैतमेव सिद्धान्तितमिति भाति~। वस्तुतस्तु~- सूत्रकारैरपि 'असत्यर्थे नाभाव इति चेत्' इति पूर्वपक्षिणा ध्वंसो न सम्भवतीति चोदिते, न खलु ध्वंस एक एवाभावः, अस्ति च प्रागभावोऽपि~। सोऽत्र सम्भवतीति सिद्धान्तितम्~। तदुपपादयंश्च भाष्यकारः 'अभावद्वैतं खलु भवति' इति वदन् ध्वंसमात्रस्याभावत्वमिति मतिनिरासे तात्पर्यं बिभर्ति~। तावता अभावान्तरं अन्योन्याभावादिरूपं नानुमन्यते इति न युक्तम्~। अत एव तात्पर्यटीकाकारः~-

		{\fontsize{11.7}{0}\selectfont\s अभावद्वैतमिति प्रकृतापेक्षम्~। प्रकृतं प्रध्वंसमात्रवादिनं प्रति प्रागभावप्रतिपादनम्~। परमार्थतस्तु प्रथममभावेद्वैतं तादात्म्याभावः संसर्गाभावश्चेति~। संसर्गाभावोऽपि प्राक्प्रध्वंसात्यन्ताभावभेदेन त्रिविध इति चतस्रो विधा अभावस्य~।\footnote{१.१.१२}} इति अभावचतष्टयमपि न्यायदर्शनसम्मतमिति प्रत्यपादयत्~। 

		तत्र भेदापरपर्यायः अन्योन्याभावोऽपि सूत्रकारस्यानुमतः~। तथा हि~- शब्दानित्यत्वस्थापनप्रकरणे 'दशकृत्वोऽनुवाकोऽधीतः', 'विंशतिकृत्वोऽधीतः' इत्येवमभ्यासाच्छब्दो नित्यमिति पूर्वपक्षं समुत्थाप्य, 'द्विरनृत्यत्', 'त्रिरनृत्यत्' इत्यादौ भेदेऽप्यभ्यासस्य दर्शनात् भेदे सिद्धे अनित्यत्वं सिध्यतीत्याशयेन 'नान्यत्वेऽप्यभ्यासस्योपचारात्'\footnote{न्या. सू. २.२.३०} इति सूत्रेण सिद्धान्तः प्रदर्शितः~। ततः अन्यः इति शब्दप्रयोगोऽनुपपन्नः~। अन्यत्वेनाभिमतस्यापि स्वस्मादन्यत्वात् अन्यत्वस्यैव अभावादित्याशङ्क्य, 'अन्यत् स्वस्मादन्यत्' इति वाक्येन अन्यस्मिन् स्वस्मादन्यत्वाभावो बोध्यते~। तत्रान्यत्वस्याभावे कथमन्यशब्दप्रयोगः सङ्गच्छते इति सिद्धान्तितं 'तदभावे नास्त्यनन्यता, तयोरितरेतरापेक्षसिद्धेः\footnote{न्या. सू. २.२.३२} इत्यस्मिन् सूत्रे~। न च स्वस्मादन्यत् परस्मादप्यनन्यत् भवितुमर्हति~। न हि नीलं स्वस्मादन्यत् पीतादप्यनन्यद्भवति~। अन्यदेव नीलं पीतादिति सूत्रकारस्याशयः~। एवं स्थापयतः सूत्रकारस्य अन्योन्याभावोऽप्यनुमत इति स्पष्टमवबुध्यते~।

		एवं त्रिकालविषयं भवत्यनुमानमित्युक्त्वा वर्तमानस्यैवाभावात् कथं त्रैल्यग्रहणं सम्भवति इत्याक्षिपति 'वर्तमानाभावः पततः पतितपतितव्यकालोपपत्तेः'\footnote{न्या. सू. २.१.३९} इत्यनेन सूत्रेण~। तथा हि वृन्तात् विभक्तस्य फलस्य विभागादिकालः भूतः पतितकलः, अध्वा तत्संयुक्तः कालः भविष्यः पतितव्यकालः, एतदपेक्षया अन्यः वर्तमानकालः न विद्यते एव इति आक्षोप्य, 'तयोरप्यभावो वर्तमानाभावे तदपेक्षत्वात्'\footnote{न्या. सू. २.१.४०} इति सूत्रेण वर्तमानकालाभावे भूतभविष्ययोरपि कालयोरभाव एव इति सिद्धान्तयति~। अत्रायमाशयः~- पतितकालः इति यदुक्तं तत्कास्मात्पतितकालः इति अवध्याकाङ्क्षायां वर्तमानाकालादित्येव वक्तव्यम्~। एव पतितव्यकालेऽपि अवध्याकाङ्क्षानिवर्तकः वर्तमानकालः इति वर्तमानकालस्य अस्तित्वं साधयति~। अत्र वर्तमानाभावे तयोरप्यभाव इति कथनेन अत्यन्ताभावोऽपि सूत्रकारभाष्यकारयोः सम्मत इति ज्ञायते~।

		एतेन प्रागभावध्वंसयोः अत्यन्ताभावेन विरोधो न सूत्रकारादिसम्मतः इत्यपि सिध्यति~। तथा हि~- अत्यन्ताभावं स्वीकुर्वन्नपि सूत्रकारः केषाञ्चिद्वासस्सु लक्षणानुपपत्तौ तत्र लक्षणध्वंसानुपपत्तौ लक्षणप्रागभावं कथयति~। यद्यपि असाधारणधर्मस्य कालान्तरोत्पत्तिज्ञाने सति प्रागभावोपपद्यते~। वस्तूत्पत्तेः पूर्वमेव तत्र प्रागभावनिश्चयो न कर्तुं शक्यते~। तस्मात् तत्रात्यान्ताभावे सत्यपि प्रागभावव्यवहारात् तयोर्विरोधो नाभिमत एव~।

		एवं न्यायमञ्जरीकारोऽपि अभावद्वैतं प्रतिपादयन् प्रागभावस्यैव अत्यन्ताभावत्वं प्रतिपादयति~। एतेनापि प्रागभावाधिकरणे अत्यन्ताभावसत्त्वे न तयोर्विरोध इति तस्याशय इति ज्ञायते~। तथा च प्राचीननैयायिकानां ध्वंसप्रागभावयोरधिकारणे अत्यन्ताभावसत्त्वं सम्मतमेव~।

		\subsection{तत्रैव वैशेषिकानामाशयः}

		'क्रियागुणव्यपदेशाभावात् प्रागसत्'\footnote{वै. सू. ९.१.१} इत्यादिसूत्रेण अभावचतुष्टयं सिद्धान्तितं कणादमुनिना~। वैशेषिकभाष्यकारैः प्रशस्तपादाचार्यैस्तु अभावनिरूपणं न कृतम्~, तथापि तत्तत्सन्दर्भे तैरभावाश्चत्वारोऽपि व्यवहृता इति तेषां तत्र सम्मतिमनुमिमीमहे~। तथा हि~- चतुर्णां महाभूतानां सृष्टिसंहारप्रकारनिरूपणादेव प्रागभावप्रध्वंसौ सम्मतौ~। तयोरभावे सृष्टिसंहारयोर्दुरुपपादत्वात्~। एवं साध्यर्म्यवैधर्म्यप्रकरणे 'अभिन्नत्वञ्चान्यत्रनित्यद्रव्येभ्यः' इत्यादि व्यवहरता अन्योन्याभावो स्वीकृतः~। एवं 'सर्वत्र साध्यर्म्यं विपर्ययाद्वैधर्म्यम्' इत्यादिना अत्यन्ताभावोऽपि सूचितः~। एवमेव किरणावल्यादिग्रन्थेष्वपि चत्वार्यभावाः सिद्धान्तिताः~।

		किन्त्वत्यन्ताभावनिरूपणावसरे कन्दल्यादौ अत्यन्तासतः शशविषाणादेरभाव एव अत्यन्ताभाव इति सिद्धान्तितम्~। कुमारिलभट्टैरपि~-\\ \begin{center}शशशृङ्गादिरूपेण सोऽत्यन्ताभाव इष्यते~।\\ न च स्याद्व्यवहारोऽयं कारणादिविभागतः~॥\footnote{श्लो. वा. ४१०}\\\end{center} इति अत्यन्तासतः एवात्यन्ताभावत्वं प्रत्यपादि~। तथा च~- शशविषाणादेः अभावाधिकरणे तेषां प्रागभावध्वंसयोरसम्भवः, प्रतियोग्यप्रसिद्धत्वात्~। एवञ्च ये अत्यन्तासतः अभावः अत्यन्ताभाव इति वदन्ति तैः प्रागभावप्रध्वंसयोरत्यन्ताभावेन सह विरोध एव स्वीकृत इति सिध्यति~।

		इदन्तु बोध्यम्~- अत्यन्तासतः शशविषाणादिरेव अत्यन्ताभाव उत तेषामभावः~? नाद्यः प्रतियोगिज्ञानं विना अभावबुद्ध्यनुदयात्~। शशविषाणस्य सप्रतियोगिकत्वाभावाच्च न तस्यात्यन्ताभावत्वम्~। अत एव न द्वितीयः~। किञ्च शशविषाणादेः अप्रसिद्धत्वात्~, अप्रसिद्धस्य वस्तुनः निषेधो न कर्तुं शक्यः~। कलञ्जभक्षणादिक्रियाप्राप्त्यनन्तरं निषेधः सम्भवति~। शशविषाणादेरप्रसिद्ध्या अप्राप्तानां तेषां निषेधायोगात् न तेषामभावोऽपि सम्भवति~। तस्मादप्रसिद्धस्य अत्यन्तासतः प्रमेयत्वाभावाच्च नाभावत्वम्~, न तेषामभावोऽपीति~।

		किञ्च घटध्वंसोत्तरमपि तदधिकरणे 'घटो नास्ति' इति व्यवहारात्~, घटोत्पत्तेः पूर्वमपि मृदादौ 'घटो नास्ति' इति व्यवहाराच्च न प्रत्यक्षप्रमाणादेव प्रागभावध्वंसात्यन्तभावयोः ऐकाधिकरण्यं सिध्यति इति~।


\begin{center}\begin{small}॥ इति प्राचीनन्याय-वैशेषिक-नव्यन्यायशास्त्रेषु तत्तद्व्याख्याकाराणां सैद्धान्तिकमतभेदानां विमर्शात्मकमध्ययनमिति प्रबन्धे  अभावप्रपञ्चे सैद्धान्तिकमतभेदाः इत्याख्यः पञ्चमोऽध्यायः~॥\end{small}\end{center}



\titleformat {\chapter}[display]{\normalfont\Large} % format
{अथ षष्ठोऽध्यायः\\[1mm]} % label
{-3.8ex}{ \rule{\textwidth}{1pt}\vspace{-5ex}
\centering
} % before-code
[
\vspace{-6.7ex}%
\rule{\textwidth}{1pt}
]
\titlespacing*{\chapter} {10pt}{-60pt}{50pt}


\chapter{परिशिष्टम्}

	\section{अन्यथाख्यातौ भासमानसंसर्गस्य सत्त्वमसत्त्वं वा}

	इदानीं शुकौ 'इदं रजतम्' इत्याद्याकारकस्य भ्रमात्मकज्ञानस्य ख्यातिरिति व्यपदेशः दर्शनप्रपञ्चे प्रसिद्धः~। लोके यद्यज्ज्ञानमुत्पद्यते तत्सर्वं प्रमेति वक्तुं न शक्यते~। सर्वेषां ज्ञानां प्रमात्वे अनिष्टे रजतादौ प्रवृत्तिरेव न स्यात्~। अप्रमात्वे तु इष्टे रजतादावपि प्रवृत्तिरनुपपन्नैव~। तस्मात् किञ्चिज्ज्ञानं प्रमा किञ्चिदप्रमेति विभागः अवश्यं सर्वैरभ्युपगन्तव्यः~। तत्राप्रमात्मके ज्ञाने अन्यस्य शुक्त्यादेः अन्यथाभानं रजतत्वादिना भानात्मिकामन्यथाख्यातिमुररीकुर्वन्ति नैयायिकादयः~। तत्र हि भ्रमात्मके इन्द्रियसन्निकर्षजन्ये रजतज्ञाने इन्द्रियासन्निकृष्टस्य आपणस्थस्य रजतस्य कथं भानम्~? रजतत्वादिना भासमानशुक्त्यादौ रजतत्वादेरसत्सम्बन्धो भासते वा~? इत्याद्याः विप्रतिपत्तयः युक्तिकोविदां ग्रन्थेषु दरीदृश्यन्ते~। तदत्र विचार्यते~।

		\subsection{अन्यथाख्यातौ न्यायभाष्यसम्मतिः}

		'इन्द्रियार्थसन्निकर्षोत्पन्नं ज्ञानमव्यपदेश्यमव्यभिचारि व्यवसायात्मकं प्रत्यक्षम्'\footnote{न्या. सू. १.१.४} इति सूत्रभाष्ये अव्यपदेश्यादिविशेषणानां सार्थक्यं वदन् वात्स्यायनो ग्रीष्मकाले दूरस्थदेशे विद्यमानेन ऊष्मणा सह सन्निकृष्टानीन्द्रियाणि उदकमित्यतस्मिंस्तदाकारकं ज्ञानमुत्पादयन्ति इत्यन्यथाख्यातिप्रतिपादकं भाष्यं रचयति~। तथा हि~-

		{\fontsize{11.7}{0}\selectfont\s ग्रीष्मे मरीचयो भौमेनोष्मणा संसृष्टाः स्पन्दमाना दूरस्थस्य चक्षुषा सन्निकृष्यन्ते, तत्रेन्द्रियार्थसन्निकर्षात् उदकमिति ज्ञानमुत्पद्यते, तच्च प्रत्यक्षं प्रसज्यते इत्यत आह~- अव्यभिचारीति~। यदतस्मिन् तदिति तद्व्यभिचारि~। यत्तु तस्मिन् तदिति तदव्यभिचारि प्रत्यक्षमिति~।\footnote{न्या. भा. २२}}

		ननु पुरोवर्तिशुक्तिना सह सन्निकर्षे सति असन्निकृष्टस्य रजतादेः भानं कथं भवितुमर्हति~? सन्निकृष्टपदार्थग्रहणे एव इन्द्रियाणां सामर्थ्यादिति चेत्~- अत्र अन्यथाख्यातिवादिनां मध्येऽपि मतभेदो दृश्यते~।

		\subsection{अधर्मादसन्निकृष्टवस्तुनः स्मरणाद्विपर्ययः}

		सास्नादिना केसरादिना च पूर्वमवगताः गवाश्वादयः‌~। तयोर्मध्ये कदाचिद्गवा सह सन्निकर्षे जाते तत्र पित्तकफानिलरूपदोषविशिष्टेनिन्द्रियेण सास्नादेरग्रहणात् अश्वस्य च स्मरणात् अश्वः इति प्रतीतिरुदेति~। अत्रात्ममनस्संयोगस्य अश्वस्मरणजनकत्वमेव कुतः इति चेदधर्मवशात्~। अन्यथा दोषवशात् सर्वत्र सर्वमवभासेत~। तदाह प्रशस्तपादः~-

		{\fontsize{11.7}{0}\selectfont\s प्रत्यक्षविषये तावत् प्रसिद्धानेकविशेषयोः पित्तकफानिलोपहतेन्द्रियस्यायथार्थालोचनादसन्निहितविषयज्ञानजसंस्कारादात्ममनसोः संयोगादधर्माच्च अतस्मिंस्तदिति प्रत्ययो विपर्ययः~। यथा गव्येवाश्व इति |\footnote{प्र. भा. १४७} इति~।}

		\subsection{रजतसादृश्यज्ञानं संस्कारश्च रजतभाने हेतुः}

		पुरोवर्तिशुक्तिना सह संयुक्तमिन्द्रियं पित्तादिदोषवशात् शुक्तित्वमज्ञापयदपि रजतसादृश्यविशिष्टतया ज्ञापयति~। तेन च सादृश्यज्ञानेन रजतविषयकसंस्कारः प्रबोध्यते~। स च प्रबुद्धो रजतस्मरणजनने मनोदोषात् इन्द्रियसंयुक्तायां शुक्तौ अनुभवरूपं रजतज्ञानं जनयति~। तथा च पूर्वानुभवजनितसंस्कारः रजतभाने हेतुरिति~। तदुक्तं कन्दल्याम्~- 

		{\fontsize{11.7}{0}\selectfont\s अस्ति च शुक्तिकादेशे रजतार्थिनः प्रवृत्तिः, अस्ति च सामानाधिकरण्यप्रत्ययो रजतमेतदिति, अस्ति च बाधकप्रत्यय इदन्ताधिकरणस्य रजतात्मतानिषेधपरः~। तेनावगच्छामः शुक्तिसंयुक्तेनेन्द्रियेण दोषसहकारिणा रजतसंस्कारसचिवेन सादृश्यमनुरुन्धता शुक्तिकाविषयो रजताध्यवसायः कृतः~।\footnote{न्या. कं. ४२५}}

		\subsection{रजतस्मृतिसहिताद्दुष्टादिन्द्रियात् रजतभानम्}

		दोषसहितमिन्द्रियं पुरोवर्तिशुक्तिना सह सन्निकृष्टं सत् तद्गतविशेषधर्मान् शुक्तित्वत्रिकोणत्वादीन् नावभासयति~। सामान्यधर्मज्ञानं तु उद्बोधकविधया रजतस्मृतिं जनयति~। एवञ्च रजतस्मृतिसहिताद्दुष्टादिन्द्रियात् रजतस्य भानम्~। तदुक्तं न्यायमञ्जर्याम्~-

		{\fontsize{11.7}{0}\selectfont\s तस्माद्दोषकलुषितादिन्द्रियात् पुरोवस्थितधर्मिगतत्रिकोणत्वादिविशेषावमर्शकौशलशून्यात् सामान्यधर्मसहचरितपदार्थान्तरगतविशेषस्मरणोपकृताद्भवति विपरीतप्रत्ययः, सम्यग्ज्ञानापेक्षया च तद्दुष्टमुच्यते, स्वकार्ये तु विपर्ययज्ञाने तत्कारणमेव न दुष्टम्~, तस्माद्रजमित्यनुभव एव, न प्रमुषितस्मृतिः~।\footnote{न्या. मं. १६८}}

		{\fontsize{11.7}{0}\selectfont\s उक्तमत्र सदृशपदार्थदर्शनोद्भूतस्मृत्युपस्थापितस्य रजतस्यात्र प्रतिभासनमिति~।\footnote{न्या. मं. १७०}} इति~।

		\subsection{दोषाद् ज्ञानव्यापाराच्चान्यथाख्यातिः}

		प्रत्यक्षं प्रति विशेष्यसन्निकर्ष एव कारणम्~, न विशेषणेनेन्द्रियसन्निकर्षोऽपि इति वदन् मणिकारः विशेषणसन्निकर्षाभ्युपगमेऽपि भ्रमनिर्वाहो भवत्येव इति वदति~। तथा हि~- पुरोवर्तिशुक्त्या सह दोषयुक्तेन्द्रियस्य सन्निकर्षे जाते रजतत्वस्मृतौ भ्रमो जायते~। तत्र रजतत्वेन सन्निकर्षस्तु अलौकिकः रजतत्वज्ञानमेव इति~। तदुक्तम्~-

		{\fontsize{11.7}{0}\selectfont\s वस्तुतस्तु दोषाद्भ्रम इत्युभयसिद्धम्~।\footnote{त.म. ८५} अपि च दोषस्य विशिष्टमेव ज्ञानं व्यापारः प्रवृत्तौ तद्धेतुत्वस्य क्लृप्तत्वात्~।\footnote{त. म. ८६}}

		प्राचीननैयायिकास्तावत् अन्यथाख्यातौ संसर्गविधया असतः भानमङ्गीकुर्वन्ति~। 'इदं रजतम्' इति तादात्म्यारोपे विशेष्यप्रकारयोः सतोः शुक्तिरजतयोः असत्तादात्म्यं संसर्गतया भासते~। तथा च इन्द्रियजन्यत्वेऽपि असत्संसर्गकतयास्य प्रमाद्वैलक्षण्यमिति~।

		\subsection{अन्यथाख्यातेः असत्संसर्गकत्वम्}

		न च भ्रमस्य असत्संसर्गकत्वे असत्ख्यात्यभ्युपगमात् अपसिद्धान्त इति वाच्यम्~। असतः ख्यातेः अनभ्युपगमात्~। असत्ख्यातिमते तु संसर्गस्येव प्रकारविशेष्ययोरपि असत्त्वोपगमेन, अत्यन्तासतः प्रत्यक्षस्यैव असत्ख्यातित्वाभ्युपगमात्~। न चैषा प्राचीनन्यायसम्मता~। सन्निकर्षव्याप्तिज्ञानादीनां क्लृप्तानां ज्ञानकारणानां विरहेण असतो रजतादेः शशशृङ्गस्येव भानासम्भवात्~। न चैवं सति निरुक्तसामग्रीविरहाद् असत्संसर्गस्यापि भानमनुपपन्नम्~। सतोः विशिष्टस्योत्पादकसामग्र्याः एव दोषरूपकारणघटितायाः सद्विशेष्यकसत्प्रकारकासत्संसर्गकभ्रमोत्पादसम्भवात्~। तदुक्तं वाचस्पतिमैश्रैः~-

		{\fontsize{11.7}{0}\selectfont\s अत्रेदमालोचनीयम्~- किमेतन्मिथ्याज्ञानमसत् सदात्मना गृह्णातीत्यसद्विषयमुपेयते~? आहोस्वित्सदेव सदन्तरात्मना गह्णाति सतश्च सदन्तरात्मत्वेन असत्त्वादसद्विषयमुच्यते~? न तावत्पूर्वः कल्पः, रजतात्मना चेदसदालम्बेत न सती शुक्तिकाम्~, कथं पुनरसौ रजातर्थो शुक्तौ प्रवर्तते न पुना रजताभावे~। कस्माच्चेदमिति पुरोवर्ति द्रव्यमङ्गुल्या निर्दिश्य तस्य रजतत्वं निषेधति नेदं रजतमिति, यदि तत्र न प्रसञ्जितं रजतत्वं पूर्वविज्ञानेन~? अथ शुक्तिरेव रजतात्मना असतीति तदाकारतया तामालम्बमानं मिथ्याज्ञानमसदालम्बनमुच्यते~? तत्रानुज्ञया वर्तामहे~। न खल्वन्यथाख्यातिवादिनोऽपि सदन्तरं सदन्तरात्मना सदभ्युपगच्छन्ति~। तथा सत्यन्यथेत्येव न स्यात्~। यथाहुरन्यथाख्यातिवादिनः~-\\ तस्माद्यदन्यथा सन्तमन्यथा प्रतिपद्यते~।\\ तन्निरालम्बनं ज्ञानमसदालम्बनञ्च तत्~॥\footnote{न्या. वा. ता. टी.}}

		नवीनास्तु अन्यथाख्यातेः असत्संसर्गकत्वमपि नाङ्गीकुर्वन्ति~। दोषसहितेन्द्रियजन्यत्वादेव अस्य प्रमापेक्षया वैलक्षण्यमिति वदन्ति~।

		\subsection{भ्रमस्यासदविषयकत्वम्}

		ननु 'इदं रजतम्' इति ज्ञाने सतः आपणस्थस्य रजतस्य पुरोवर्तिशुक्तेश्च भ्रमे विषयत्वात् असद्विषयकत्वमिति भ्रमस्वरूपमसम्भवि~। न च तदुभयसंसर्गः तादात्म्यमसद् अतः अस्यासद्विषयकत्वाद् भ्रमत्वम् इति वाच्यम्~। तादात्म्यस्यानुयोगिस्वरूपत्वात् अनुयोगिनश्च शुक्त्यादेः असत्त्वाभावात्~। न च शशविषाणवद् असदेव शुक्तिरजतयोस्तादात्म्यमिति वाच्यम्~। तथा सति सम्बन्धिनश्चासत्त्वाभ्युपगमात् असत्ख्यातौ पर्यवसानाद् अपसिद्धान्तापत्तेः~। एवं ह्रदादौ 'अयं धूमवान्' इति संसर्गारोपेऽपि अयथार्थत्वं न असद्विषयकत्वम्~। ह्रदधूमसंयोगानां प्रत्येकं प्रसिद्धत्वात्~।

		अत्रोच्यते~- तदभाववति तत्प्रकारकत्वं प्रकारव्यधिकरणविषयताकत्वं वा अयथार्थत्वम्~। प्रमायामिव भ्रमेऽपि सदेव वस्तु भासते~। न तु क्वचिदपि शशविषाणवत् असद्भासते~। इयांस्तु विशेषः~- प्रमायां धर्मिणिभासमानः विशेषणसम्बन्धः तस्मिन् धर्मिणि अस्त्येव~। भ्रमे तु धर्मिणि भासमानः विशेषणसम्बन्धः तस्मिन् धर्मिणि नास्ति, अन्यत्र धर्मिणि प्रसिध्यति इति~। तदुक्तं मणौ~-

		{\fontsize{11.7}{0}\selectfont\s अथेदं रजतमिति ज्ञानं कथमयथार्थम्~? इदंरजतयोः सत्त्वात्~। न च तदुभयतादात्म्यमसत् उभयं तादात्म्यञ्चेति अतोऽधिकस्याभावात्~। तयोश्च सत्त्वात्~। असत्त्वे वा असत्ख्यातिः, एवं संसर्गारोपोऽपीति~। उच्यते~- सन्मात्रविषयत्वेऽपि तदभाववति तत्प्रकारकत्वं प्रकारव्यधिकरणविषयताकत्वं वा अयथार्थत्वम्~।\footnote{त. चि. ९७}}

	\section{विमर्शः}

	अस्ति तावत् रजतार्थिनः पुरोवर्तिनि द्रव्ये शुक्तिकादौ प्रवृत्तिः~। सा च न भेदाग्रहसहितेदंपदार्थग्रहणरजतस्मरणरूपज्ञानद्व्ययेन भवति~। न हि रजतप्रतिपादकार्थस्मरणमात्रस्य प्रवर्तकत्वं युज्यते~। इदं रजतमिति ज्ञानानन्तरमेव प्रवृत्तेरनुभवात्~, तस्य च पुरोवर्तिपदार्थेन सन्निकर्षानन्तरमेव उत्पन्नत्वात्~। तस्मात् इदम्पदार्थशुक्तिविशेष्यकरजतत्वप्रकारकज्ञानरूपात् अन्यथाख्यातिशब्दिताद्विशिष्टज्ञानादेव प्रवृत्तिरिति अन्यथाख्यातेरपलापो न कर्तुं शक्यते~।

		\subsubsection{भ्रमे असन्निकृष्टवस्तुनः भाने हेतवः}

		'इदं रजतम्' इत्यादिभ्रमे आपणादिस्थस्य रजतस्य इन्द्रियसन्निकर्षाभावात्कथं भानं भवितुमर्हति, अतः अन्यत्रस्थितस्यन्यत्र भानरूपा अन्यथाख्यातिर्नयुज्यत इत्याक्षेपः अन्यथाख्यातिवादोपरि सर्वैः क्रियते~। तत्र हि पित्तदोषवत एव 'पीतः शङ्ख' इति भ्रमसम्भवादन्वयव्यतिरेकाभ्यां दोषस्य भ्रमजनकत्वं सिद्धमेव~। तादृशदोषसहितादिन्द्रियात् शुक्त्यादिपुरोवर्तिपदार्थेन सह सन्निकर्षे जाते ततः अधर्मात् रजताद्यनुभवजन्यसंस्कारेण सह मनस्सम्बन्धो जायते, तेन च भ्रमो जायते इति~। तत्र हि अधर्मरूपदोषविशिष्टस्य मनसः कारणादेव भ्रमे रजतादीनां भानमिति केचित्~। अपरे तु पुरोवर्तिनमिन्द्रियसन्निकृष्टमपि शुक्तिं शुक्तित्वेनानवभासयद्रजतसादृश्यम् उद्बोधकविधया रजतं स्मारयति~। तथा च रजतसादृश्यवशादुद्बुद्धः रजतसंस्कारः दोषविशिष्टेन्द्रियेण सह सहितः भ्रमे रजतं भासयति इति~।  

		जयन्तभट्टादयस्तु रजतस्मृतिसहितात् दुष्टादिन्द्रियाद्रजतस्य भानमिति वदन्ति~।

		इदन्तु बोध्यम्~- दूरस्थचन्दनखण्डं पश्यन् तस्य 'सुरभि चन्दनम्' इति गन्धविषयकमपि चाक्षुषं ज्ञानमुत्पद्यते~। तत्र हि चन्दनेन सन्निकर्षे जाते उद्बुद्धः संस्कारः सुरभिस्मृतिं जनयति, तदेव ज्ञानलक्षणप्रत्यसत्त्या सुरभिविषयकं प्रत्यक्षं जायते~। तद्वदेव भ्रमस्थलेऽपि चाकचक्यादिदोषवशात् उद्बुद्धः संस्कारः रजतत्वादिस्मृतिं जनयति, तच्च स्मरणं सन्निकर्षविधया 'इदं रजतम्' इत्यादिभ्रमेषु रजतत्वादीन् भासयति इति~। एवञ्च असन्निकृष्टस्यापि रजतत्वस्य ज्ञानलक्षणप्रत्यासत्त्या प्रत्यक्षविषयत्वमुपपद्यत एव इति~।

		\subsubsection{भ्रमस्य असत्संसर्गकत्वविमर्शः}

		अयथार्थानुभवे प्रकारस्य विशेष्येण सह भासमानस्संसर्ग असदिति प्राञ्चः~। न च तस्य असद्विषयकत्वात् असत्ख्यात्यापत्तिः~। प्रकारविशेष्ययोरसत्त्वे हि असत्ख्यातिः सम्भवति~। भ्रमे च शुक्तिरजतत्वादीनाम् असत्त्वाभावात् न तस्यासत्ख्यातित्वमिति~। किञ्च प्रमायां भ्रमे च सत्संसर्गस्य भाने तयोर्वैलक्षण्यमेव न स्यात्~। तस्मात् दोषवशात् भ्रमस्यासत्संसर्गकत्वमभ्युपगन्तव्यमिति~। एतदेव शास्त्रदीपिकायामप्येवं न्यरुपि~-

		{\fontsize{11.7}{0}\selectfont\s सर्वत्र संसर्गमात्रमसदेवावभासते~। संसर्गिणस्तु सन्त एव~। सेयं विपरीतख्यातिरित्युच्यते मीमांसकैः~। शुक्तिरजतवेदनेऽपि विद्यमानैव, रजतत्वजातिर्विद्यमानस्यैव शुक्तिशकलस्य अनात्मभूतैवात्मतया अवगम्यते~।\footnote{शा. दी. ५८}} इति सन्दर्भेण नैयायिकसम्मता अन्यथाख्यातिरेव समर्थिता विपरीतख्यातिशब्देन व्यवहृता च~।

		नव्यास्तु भ्रमस्य सत्संसर्गकत्वमेव उररीकुर्वन्ति~। असत्संसर्गकत्वं भ्रमत्वमिति भ्रमस्वरूपं निराकृत्य मणिकारेण तद्वति तत्प्रकारकत्वं प्रकारव्यधिकरणविषयताकत्वं वा भ्रमत्वमिति न्यरूपि~। अन्यत्र रजतादौ प्रसिद्धस्य रजतत्वादेः संसर्गस्य पुरोवर्तिनि तदनधिकरणे शुक्त्यादौ भाने भ्रम इति व्यवह्रियते, तस्यैव स्वाधिकरणे रजतादौ भाने प्रमेति व्यवहारः~। एतदेव भ्रमप्रमयोः वैलक्षण्यमिति~।

		किञ्च असद्भूतस्य वस्तुनः‌ इन्द्रियादिना अग्रहणात् असत्संसर्गकज्ञानस्यापि इन्द्रियादिजन्यत्वं न युक्तियुतम्~। अन्यथा संसर्गस्यैव असत्वं न तु प्रकारविशेष्ययोः इति विनिगमना न स्यात्~। तथा च विनिगमकाभावात् प्रकारादीनां सर्वेषामसत्त्वं सत्त्वं वा उपगन्तव्यम्~। असत्पदार्थस्य इन्द्रियाद्यजन्यत्वस्य अन्वयव्यतिरेकाभ्यां सिद्धत्वात् प्रकारादीनां सत्त्वमेव अभ्युपगम्यते~। तथा च सत्संसर्गक एव भ्रम इति~।



	\section{मोक्षस्वरूपविचारः}

	इदानीं दर्शनानां परमफलत्वेन प्रसिद्धस्य परमपुरुषार्थस्य मोक्षस्य स्वरूपमत्र निरूप्यते~। तत्र हि आत्यन्तिकं सुखं मोक्ष इति, अशेषविशेषगुणोच्छेदो मोक्ष इति, आत्यन्तिकदुःखनिवृत्तिर्मोक्ष इति च त्रयः पक्षाः न्यायवैशेषिकयोः~। तदत्र निरूप्यते~- 

		\subsection{आत्यन्तिकं सुखमेव मोक्षः}

		कः पुनरयं मोक्षः~? न च अशेषविशेषगुणोच्छेदे सति आकाशवदात्मन उपस्थितिरेव मोक्षः~।‌ सुखदुःखादीनां सद्भावे पुरुषस्य विवेकासम्भवात्~, तेन पुनः पुनः कर्मकरणेन पुनः पुनः जन्म इति तेषां सर्वेषां हानो युक्तः~। न हि सुखार्थमेव लोकानां प्रवृत्तिः~। कण्टकादिजनितदुःखनिवृत्तावपि प्रवृत्तिदर्शनात्~। तस्मात् विशेषगुणानामत्यन्तनाशे सति आत्मन उपस्थितिरेव मोक्ष इति चेन्न~। श्रुतिविरोधात्~। ’आनन्दं ब्रह्मणो रूपं तच्च मोक्षेऽभिव्यज्यते’ इत्यादिश्रुतयो हि आत्यन्तिकसुखस्वरूपत्वमात्मनः मोक्षकाले ब्रुवन्ति~। किञ्च किञ्च कण्टकजनितदुःखादिनिवृत्तौ प्रवृत्तिस्तु तदुत्तरसुखोपभोगार्थमेवेति~। एवं तादृशस्यात्मनः मोक्षस्वरूपत्वे तस्य मुर्छावस्थातुल्यत्वात् तत्र प्रवृत्तिरेव न स्यात्~। न हि कश्चित् मुर्छावस्थामिच्छति~। तस्मादागमप्रमाणसिद्धमात्यन्तिकसुखमेव मुक्तिरिति~। तदुक्तं न्यायसारे~-

		{\fontsize{11.7}{0}\selectfont\s कः पुनरयं मोक्ष इति~। एके तावद्वर्णयन्ति – समस्तविशेषगुणोच्छेदे सति संहारावस्थायामाकाशवदात्मनोऽत्यन्तावस्थानं मोक्ष इति~। कस्मात्~। सुखदुःखयोरविनाभावित्वेन विवेकहानानुपपत्तेः~। न च सुखार्थैव प्रेक्षावतां प्रवृत्तिः, कण्टकादिजनितदुःखपरिहारार्थत्वेनापि प्रवृत्तेरुपलम्भात्~। मोहावस्थात्वान्मूर्छाद्यवस्थावदत्र विवेकिनां प्रवृत्तिर्न युक्तेत्याहुरन्ये~। दुःखे सति सुखोपभोगासम्भवात्~। कण्टकादिजनितदुःखपरिहारोऽपि सुखोपभोगार्थ एवेत्यसमो दृष्टान्तः~। कुतो मुक्तस्य सुखोपभोगसिद्धिरिति चेद् आगमात्~। उक्तं हि~-\\ ’सुखमात्यन्तिकं यत्तद्बुद्धिग्राह्यमतीन्द्रियम्~।\\ तं वै मोक्षं विजानीयाद् दुष्प्रापमकृतात्मभिः”~॥\\ तथा~- \\ ’आनन्दं ब्रह्मणो रूपं तच्च मोक्षेऽभिव्यज्यते’~। ’विज्ञानमानन्दं ब्रह्म’ इति च~। मुख्यार्थे बाधकाभावान्नोपचारकल्पना~।\footnote{न्या.सा.१४२-१४४}}

		\subsection{नित्यस्यापि सुखस्य संसारे अग्रहः}

		न च नित्यसुखस्य सत्त्वे संसारदशायामपि तत्सम्बन्धात् तत्साक्षात्कारापत्तिः~।‌ तस्य नित्यत्वेन संसारदशायामपि सत्त्वादिति वाच्यम्~। चक्षुर्घटयोर्मध्ये कुड्यादेः सत्त्वे विद्यमानस्यापि घटस्य यथा प्रत्यक्षत्वं न भवति तद्वत् अधर्मेणान्तरितं सुखं संसारदशायां नानुभूयते~। अधर्मनाशे सति प्रतिबन्धकाभावात् मोक्षदशायां भवति सुखसाक्षात्कारः~। 

		ननु अधर्मविरहप्रक्तसुखसम्बन्धः नश्यति कृतकत्वात् घटवदित्यनया युक्त्या मुक्तस्यापि पुनः संसारापत्तिरिति चेन्न प्रध्वंसस्य नाशाप्रतियोगित्वेऽपि कृतकत्वात् व्यभिचारः~। न च भावत्वे सतीति विशेषणात् व्यभिचारवारणसम्भवः~। तथा चापत्तिस्तदवस्थैवेति वाच्यम्~। सुखसम्बन्धस्य तादृशस्य भावत्वाभावात्~। न हि सर्वेषां सम्बन्धानां भावत्वनियमः~। तस्मात् कृतकत्वस्यापि नित्यसुखसम्बन्धस्य ध्वंसवन्नाशासम्भवात् न पुनः संसारापत्तिरिति~। तदुक्तं न्यायसारे~-

		{\fontsize{11.7}{0}\selectfont\s सुखसंवेदनयोर्नित्यत्वान्मुक्तसंसारिणोरविशेषप्रसङ्ग इति चेत्~। न~। चक्षुर्घटयोः कुड्यादेरिव सुखतत्संवेदनयोर्विषयविषयिसम्बन्धप्रत्यनीकस्याधर्माद् दुःखादेः संसारावस्थायां सद्भावात्~। तन्नाशे मुक्तावस्थायां भवति सुखसंवेदनयोः विषयविषयिसम्बन्ध इत्यतो नाविशेषः~। तस्य सम्बन्धस्य कृतकत्वेन कदाचिद् विनाशप्रसङ्ग इति चेत्~। न~। प्रध्वंसेनानैकान्तिकत्वात्~। वस्तुत्वे सतीति चेत्~। न~। द्रव्यादिष्वनन्तर्भावेण तदसिद्धत्वात्~। अन्तर्भावे वा समवायादिभिः सह तत्संवेदनस्य सम्बन्धो न स्यात्~। अदृष्टादिवशात् कर्मकारकं विषयः तज्जनितं ज्ञानं विषयीति चेत्~। न~। ईश्वरज्ञानस्य नित्यस्यार्थैः सह सम्बन्धाभावप्रसङ्गात्~। तस्मात् कृतकत्वेऽपि नित्यसुखसंवेदनसम्बन्धस्य विनाशे कारणाभावान्नित्यत्वं स्थितम्~। तत्सिद्धमेतन्नित्यसंवेद्यमानसुखेन विशिष्टात्यन्तिकी दुःखनिवृत्तिः पुरुषस्य मोक्ष इति~।\footnote{न्या.सा.१४४-१४६}}

		\subsection{शरीरानुत्पादो मुक्तिः}

		ज्ञानपूर्वको हि मोक्ष इत्यसन्दिग्धमेव~। तस्मात् 'स्वरूपतश्चाहमुदासीनो बाह्याध्यात्मिकाश्चविषयाः सर्व एवैते दुःखसाधनम्' इति यस्य साक्षात्कारो जातः विषयसुखान्निवृत्तः स निवृत्तिसाधनत्वेन प्रसिद्धं नित्यं कर्म करोति~। तेन तस्य विशुद्धे कुले जन्म भवति~। श्रद्धासम्पादनार्थं हि विशुद्धकुले जन्म~। श्रद्धावान् हि जिज्ञासते~। विशुद्धकुले जातः स प्रत्यहं दुःखैरभिहन्यमानः तद्विमोकार्थं मार्गमन्विषन्नाचार्यमुपगच्छति~। आचार्योपदेशादुत्पन्नषट्पदार्थतत्त्वज्ञानः स मननादिकं संसारनाशकमुपायमनुतिष्ठते~। तेन तस्मिन् परमात्मदर्शनजं शरीरपरिच्छेदं सुखमुत्पाद्य अदृष्टं नश्यति~। ततः जीवनादृष्टस्य नाशे शरीरनाशो भवति~। दग्धेन्धनस्य पुनरनलोत्पत्तिसामर्थ्याभावात् यथा तत्र अनलस्यानुत्पत्तिः तद्वत् जन्मकारणीभूतस्य अदृष्टस्यैवाभावात् पुनर्जन्म न भवति~। तस्माद् अन्तिमशरीरादेः नाश एव परमपुरुषार्थ इति~। तदुक्तं प्रशस्तपादभाष्ये~-

		{\fontsize{11.7}{0}\selectfont\s ज्ञानपूर्वकात्तु कृतसङ्कल्पितफलाद् विशुद्धे कुले जातस्य दुःखविगमोपायजिज्ञासोराचार्यमुपसङ्गम्योत्पन्नषट्पदार्थतत्त्वज्ञानस्याज्ञाननिवृत्तौ विरक्तस्य रागद्वेषाद्यभावात् तज्जयोर्धर्माधर्मयोरनुत्पत्तौ पूर्वसञ्चितयोः धर्माधर्मयोर्निरोधे सन्तोषसुखं शरीरपरिच्छेदं चोत्पाद्य रागादिनिवृत्तौ निवृत्तिलक्षणः केवलो धर्मः परमार्थदर्शनजं सुखं कृत्वा निवर्त्तते~। तदा निरोधाद् निर्बीजस्यात्मनः शरीरादिनिवृत्तिः, पुनः शरीराद्यनुत्पत्तौ दग्धेन्धनानलवदुपशमो मोक्ष इति~।\footnote{प्र.भा. ६८०,६८२}}

		\subsection{अशेषविशेषगुणोच्छेदोऽपवर्गः}

		मोक्षकाले आत्मा किं स्वरूपः~? न चानन्दस्वरूपत्वं तस्य, तदानन्दानुभवासम्भवात्~, तदनुभावकशरीरादेरभावात्~। न हि तदानन्दो नुभूयते~। किन्तु आनन्दप्राप्तिरेव मोक्ष इति चेत् तदानन्दस्य संवेदनादिफलाभावात् तत्कल्पनासम्भवः~। किन्तु बुध्यादिषट्कमदृष्टं भावना चेति नवानामात्मविशेषगुणानामुच्छेदे सति तच्छून्यस्यात्मनः स्वस्वरूपेणोपस्थितिरेव तदिति~। तदुक्तं न्यायकन्दल्याम्~- 

		{\fontsize{11.7}{0}\selectfont\s किन्तु समस्तात्मगुणोच्छेदोपलक्षिता स्वरूपस्थितिरेव~। यथा चायं पुरुषार्थस्तथोपपादितम्~।\footnote{न्या. कं. ६९२}}

		\subsection{दुःखनिवृत्तिः मोक्षः} 

		'तदत्यन्तविमोक्षोऽपवर्ग'\footnote{न्या. सू. १.१.२२} इति गौतमसूत्रम्~। तत्र तत्पदेन 'बाधनालक्षणं दुःखम्'\footnote{न्या. सू. १.१.२१} इति पूर्वस्मिन् सूत्रे प्रतिपादितः दुःखमेवावमृश्यते~। तच्च दुःखं बाधनालक्षणमेकविंशतिधा विभक्तम्~। 'एतदेव शरीरादिबाधनानुषङ्गात् दुःखमित्युच्यते, स्वभावतस्तु दुःखमेव दुःखम्'\footnote{न्या. वा. } इति वार्तिकाचार्यैरपि प्रतिपादितम्~। तस्मात्तदित्यनेन शरीरम्~, चक्षुर्जिह्वाघ्राणत्वक्श्रवणमनांसि षडिन्द्रियाणि, रूपरसगन्धस्पर्शशब्दाः षड्विषयाः, चाक्षुषरासनघ्राणजत्वाचश्रावणमानसाश्चेति षड्बुद्धयः, सुखम्~, दुःखञ्चेति एकविंशतिदुःखान्यवमृश्यन्ते~।

		ननु प्रलयकालेऽपि एतेषां नाशो भवतीति तदैव सर्वेषां मोक्षापत्तिरिति चेन्न~। संसारवृक्षबीजभूतयोः धर्माधर्मयोः प्रलयकालेऽपि सत्त्वात् ताभ्यां पुनरेतेषामुत्पत्तिः सर्गकाले भवत्येव~। तस्मादत्यन्तविमोक्ष इत्युक्तम्~। दुःखनिवृत्तेरात्यन्तिकत्वञ्च तेषामन्तिमध्वंस एव~। स च धर्माधर्मयोरत्यन्त नाशे एव सम्भवतीति न पुनर्दुखस्योत्पत्तिः~। एवञ्च दुःखशब्देन शरीरादीनामपि ग्रहणात् बुध्यादीनां नवानामात्मविशेषगुणानामुच्छेदे सति आत्मनः स्वस्वरूपमात्रावस्थितिरेवापवर्ग इति~। तदुक्तं न्यायमञ्जर्याम्~-

		{\fontsize{11.7}{0}\selectfont\s तदिति प्रक्रान्तस्य दुःखस्यावमर्शः, न च मुख्यमेव दुःखं बाधनास्वभावमवमृश्यते, किन्तु तत्साधनं तदनुषक्तं च सर्वमेव, तेन दुःखेन वियोगोऽपवर्गः, अस्ति प्रलयवेलायामात्मनो दुःखवियोगः, सत्वपवर्गो न भवति, सर्गसमये पुनरक्षीणकर्माशयानुरूपशरीरादिसम्बन्धे सति दुःखसम्भवादतस्तद्व्यावृत्त्यर्थमत्यन्तग्रहणम्~, आत्यन्तिकीदुःखनिवृत्तिरपवर्गो न सावधिका, द्विविधदुःखावमर्शिना सर्वनाम्ना सर्वेषामात्मगुणानां दुःखवदवमर्शादत्यन्तग्रहणेन च सर्वात्मना तद्वियोगाभिधानान्नवानामात्मगुणानां बुद्धिसुखदुःखेच्छाप्रयत्नधर्माधर्मसंस्काराणां निर्मूलोच्छेदोऽपवर्ग इत्युक्तं भवति~।}

		{\fontsize{11.7}{0}\selectfont\s \begin{center}यावदात्मगुणाः सर्वे नोच्छिन्ना वासनादयः~।\\ तावदात्यन्तिकीदुःखव्यावृत्तिर्नावकल्पते~॥\\ धर्माधर्मनिमित्तो हि सम्भवः सुखदुःखयोः~।\\ मूलभूतौ च तावेव स्तम्भौ संसारसद्मनः~॥\\ तदुच्छेदे तु तत्कार्यशरीराद्यनुपप्लवात्~।\\ नात्मनः सुखदुःखे स्त इत्यसौ मुक्त उच्यते~॥\\ इच्छाद्वेषप्रयत्नादि भोगायतनबन्धनम्~।\\ उच्छिन्नभोगायतनो नात्मा तैरपि युज्यते~॥\\ प्राणस्य क्षुत्पिपासे द्वे लोभमोहौ च चेतसः~।\\ शीतातपौ शरीरस्य षडूर्मिरहितः शिवः~॥\end{center}}

		{\fontsize{11.7}{0}\selectfont\s तदेवं नवानामात्मगुणानां निर्मूलोच्छेदोपवर्ग इति यदुच्यते तदेवेदमुक्तं भवति तदत्यन्तवियोगोपवर्ग इति~।}

		{\fontsize{11.7}{0}\selectfont\s \begin{center}ननु तस्यामवस्थायां कीदृगात्मावशिष्यते~।\\ स्वरूपैकप्रतिष्ठानः परित्यक्तोऽखिलैर्गुणैः~॥\\ ऊर्मिषट्कातिगं रूपं तदस्याहुर्मनीषिणः~।\\ संसारबन्धनाधीनदुःखक्लेशाद्यदूशितम्~॥\footnote{न्या. मं. ७७}\end{center}}

		\subsection{मोक्षकाले आत्मनः अचेतनत्वापत्तिवारणम्}

		ननु घटाद्यचेतनपदार्थापेक्षया आत्मनि वैलक्षण्यं दृश्यते~। 'आत्मा चेतनः' इत्याप्तानां व्यवहारोऽपि आत्मनि चैतन्यमेव प्रतिपादयति~। चैतन्यञ्च घटादावविद्यमानमात्ममात्रवृत्ति ज्ञानमेव~। तच्च अर्थसम्बन्धाज्जायमानं सामग्र्यधीनं कादाचित्कमेव~। मोक्षे तु शरीरादिसामग्र्यभावात् ज्ञानं नोत्पद्यते~। तथा च तदात्मनः घटादिवदचेतनत्वापत्तिरिति चेन्न~। ज्ञानप्रसवसामर्थ्यमेव आत्मनश्चैतन्यम्~। तच्च ज्ञानोत्पत्तिस्वरूपयोग्यत्वम्~। यथारण्यस्थदण्डे घटोत्पत्तिस्वरूपयोग्यत्वस्य सत्त्वेऽपि मृदादीनामभावात् घटो न जायते तद्वन्मोक्षदशायामात्मनि ज्ञानोत्पत्तिस्वरूपयोग्यत्वे सत्त्वेऽपि तत्कारणीभूतानां शरीरादीनामसत्त्वात् न तत्र ज्ञानमुत्पद्यते~। घटादौ तु ज्ञानोत्पत्तिस्वरूपयोग्यत्वस्याप्यसत्त्वात् तेषामचेतनत्वम्~। तदुक्तं न्यायमञ्जर्याम्~-

		{\fontsize{11.7}{0}\selectfont\s \begin{center}सचेतनश्चितायोगात्तदभावादचेतनः~।\\ चितिर्नामात्मविज्ञानं कादाचित्कन्तु तस्य तत्~॥\\ नार्थसंवेदनादन्यत् चैतन्यं नाम विद्यते~।\\ तच्च सामग्र्यधीनत्वात् कथं मोक्षे भविष्यति~॥\\ जाग्रतः स्वप्नवृत्तेर्वा सुषुप्तस्यापि वात्मनः~।\\ ज्ञानमुत्पद्यतेऽन्या तु चतुर्थी नास्ति तद्दशा~॥\\ जाग्रद्दशायां स्वप्ने च बुद्धेः प्रत्यात्मवेद्यता~।\\ सुखं सुप्तोहमेवेति पश्चात्पत्यवमर्शनात्~॥\\ तदात्वचेत्यमानापि सुषुप्ते धीः प्रकल्प्यते~।\\ तुर्यावस्था तु संवित्तिशून्यस्य स्थितिरात्मनः~॥\\ तुर्यावस्थातिगं रूपं यदाहुः केचिदात्मनः~।\\ प्रमाणागोचरत्वेन कल्पनामात्रमेव तत्~॥\\ संवित्प्रसवसामर्थ्यं सामग्रीसन्निधानतः~।\\ यदि नामात्मनोस्त्यस्य तावता न चिदात्मता~॥\footnote{न्या. मं. ८१}\end{center}}

		\subsection{दुःखनिवृत्तेः पुरुषार्थत्वसमर्थनम्}

		ननु दुःखाभावस्य पुरुषार्थत्वमयुक्तम्~। तदा तस्याननुभूयमानत्वात्~। अनुभूयमानेऽर्थे हि लोकाः प्रवर्तन्ते~। किञ्च विषादिजन्यमार्छाद्यवस्थासु दुःखहानेः सत्त्वेऽपि न तत्र लोकाः प्रवर्तन्ते~। तस्मान्न दुःखनाशस्य पुरुषार्थत्वं युक्तमिति चेन्न~। दुःखध्वंसादौ लोकानां प्रवृत्तिः वर्तत एव~। अत एव पुत्रादिवियोगं शास्त्रान्तरेण ज्ञात्वा तादृशदुःखानुत्पत्तै विषशस्त्रादिप्रयोगे प्रवृत्तिः लोकानाम्~। एवं क्वचित् सुखेऽपि महानर्थभिया प्रवृत्तिरपि न दृश्यते~। यथा परदारादिषु प्रेक्षावतामप्रवृत्तिः~। एवमेव विषादिप्रयोगेऽपि अत्यल्पसुखसम्भवेऽपि शास्त्रनिषेधादधर्मभिया न प्रेक्षावतां प्रवृत्तिः~। सद्यः फलप्रापकत्वेऽपि लोकशास्त्रविरोध एव हि प्रेक्षा~। लोकशास्त्रनिषिद्धाचरणे तु कालान्तरे दुःखप्राप्तिरेव हि फलम्~। तस्माद्दुःखस्य सर्वदा नेच्छन्तः सर्वदा तस्य हानार्थं प्रवर्तन्ते~। अत एव व्याधिना पीडिताः तज्जन्यदुःखमसहमानः तदपरिहारे प्रयागवटप्रपातानशनादिना शरीरं त्यजन्ति~। न तेषां दुःखानाशानुभवेच्छा~। तस्मात् अननुभूयमानेऽपि दुःखनाशे प्रेक्षावतां लोकानाञ्च प्रवृत्तिदर्शनात् तस्य पुरुषार्थत्वं सम्भवत्येव~।

		{\fontsize{11.7}{0}\selectfont\s अन्यस्त्वाह~- भवत्येव हि दुःखहानिः पुरुषार्थः, किं त्वनुभूयमानतया~। न हि विषादिजन्यमोहावस्थायां दुःखनिवृत्तिरिति तदर्थं प्रेक्षावन्तः प्रवर्तन्ते~। तस्मात् मोक्षे दुःखहानेरननुभूयमानत्वान्न पुरुषार्थत्वमिति~। तदुक्तम्~-\\ दुःखाभावोऽपि नावेद्यः पुरुषार्थतयेष्यते~।\\ न हि मूर्च्छाद्यवस्थार्थं प्रवृत्तो दृश्यते सुधीः~॥ इति~।}

		{\fontsize{11.7}{0}\selectfont\s तदप्यनुपपन्नम्~, पुत्रादिवियोगजन्यदुःखहानिमिच्छतो केषाञ्चिद्विषशस्त्रोदून्धनदावपि प्रवृत्तिदर्शनात्~। प्रेक्षावन्तो नैवं कुर्वन्तीति चेन्न, पुरुषार्थत्वाविरोधात्~। न हि परदारेषु शास्त्रनिषेधाङ्कुशनिवारिताः प्रेक्षावन्तो न प्रवर्तन्त इति~। न तत्र कामः पुरुषार्थः, अपि त्वल्पीयः सुखं महाननर्थ इति निवृत्तिः~। तथा विषादावपि प्रवृत्तस्य शास्त्रगर्हितत्वादल्पीयानर्थो निवर्तते~। महीयान् प्रवर्तत इति न ते प्रवर्तन्त इति~। इयमेव हि प्रेक्षा यत्पुरुषार्थत्वेऽपि लोकशास्त्रविरोधपरामर्शः, न तु अपुरुषार्थत्वादेव स्वरूपेणेष्यमाणतामात्रनिबन्धनत्वात्पुरुषर्थतायाः~। अत एव न यत्र लोकशास्त्रविरोधः, तत्र व्याध्यादिपरिपीडिताः प्रयागवटप्रपातानशनादिनापि देहं त्यजन्तः प्रेक्षावन्तो दृश्यन्ते~। न हि कश्चिद्दुःखनिवृत्तिमनुभविष्यामीति तत्साधने प्रवर्तते, अपि तु दुःख हास्यामीति~। अपि च दुःखनिवृत्तेरनुभूयमानतामात्रं विवक्षितम्~।\footnote{न्या. वा. ता. प}}


	\section{विमर्शः}

	धर्मार्थकाममोक्षाख्येषु चतुर्विधपुरुषार्थेषु दर्शनेषु प्रधानतया प्रतिपादितः परमपुरुषार्थेति प्रसिद्धो मोक्षः~। लोकानामन्तिमफलत्वेन प्रसिद्धस्य तस्य स्वरूपं किमिति जिज्ञासायां सुखमिति दुःखनिवृत्तिरिति च तत्र तत्र ग्रन्थेषु दृश्यते~। सुखार्थं दुखनिवृत्यर्थञ्च लोकानां प्रवृत्तिरनुभूयते~। तत्र चरमकाले किं प्राप्यते इति चादात्यन्तिकसुखमेव इति केचित्~।

		\subsection{आत्यन्तिकसुखस्य परमपुरुषार्थत्वविमर्शः}

		'आनन्दं ब्रह्मणो रूपं तच्च मोक्षेऽभिव्यज्यते', 'विज्ञानमानन्दब्रह्म' इत्यादिश्रुतिभ्यः\\ \begin{center}'सुखमात्यन्तिकं यत्तद्बुद्धिग्राह्यमतीन्द्रियम्~।\\ तं वै मोक्षं विजानीयाद् दुष्प्रापमकृतात्मभिः~॥'\\\end{center} इत्यादि स्मृतिभ्यश्च आत्यन्तिकसुखमेव मोक्ष इति सिध्यति~। न च तस्यानित्यत्वम्~। मुक्तानां पुनः संसाराभावात्~। न च तस्य नित्यत्वे संसारकालेऽपि तत्प्रतीतिरिति वाच्यम्~। धर्माधर्मादिप्रतिबन्धकवशादेव संसारदशायां तदग्रहणम्~। तेषां नाशे च तद्ग्रहणान्मुक्तिरिति~। तदेव ब्रह्मसाक्षात्कारादविद्याहेतुभूतकर्मनिवृत्तौ अविद्यानिवृत्त्या ब्रह्मानन्दानुभव इति वदन्ति ब्रह्मविद्यायाम्~- {\fontsize{11.7}{0}\selectfont\s तस्मादविद्याकामकर्मोपादानहेतुनिवृत्तौ स्वात्मन्यवस्थानं मोक्ष इति~। स्वयं चात्मा ब्रह्म~। तद्विज्ञानादविद्यानिवृत्तिरिति~।\footnote{तै.शां.}}

		न च तद्ग्राहकस्येन्द्रियस्याभावः~। मुक्तनन्तरमपि मनसः सत्त्वात् तेन सुखसाक्षात्कारः~। आनन्दश्रुतिबलादेव तदा बहिरिन्द्रियाणामुच्छेदेऽपि मनस्सत्त्वं कल्प्यते तस्मात्तेन आनन्दानुभव इति~। तदुक्तं शास्त्रदीपिकायाम्~- {\fontsize{11.7}{0}\selectfont\s  न च मुक्तस्येन्द्रियाणि सम्भवन्तीति कथमानन्दानुभवः स्यात्~? उच्यते~। बाह्येन्द्रियाण्येव मुक्तस्य निवर्तन्ते~। मनस्तु तस्यामवस्थायामनुवर्तत इत्यानन्दश्रुतिबलादेवाध्यवसीयते, एवं ज्ञानञ्च~। 'न हि विज्ञातुर्विज्ञातेर्विपरिलोपो वर्तत' इति श्रुतेः, विज्ञानघनश्रुतेश्च~। तस्मान्मुक्त्यवस्थायां मानसप्रत्यक्षेण परमानन्दमनुभवन्नात्मावतिष्ठते, तदुक्तं~-\\ 'निजं यत्त्वात्मचैतन्यमानन्दश्चेष्यते यः~।\\ यच्च नित्यविभुत्वादि तैरात्मा नैव मुच्यते~॥ इति~।\footnote{शा.दी. १०५}}

		 न च एतादृशसुखकल्पनापेक्षया दुःखध्वंसस्यैव परमपुरुषार्थत्वमस्तु इति वाच्यम्~। दुःखनिवृत्तेः पुरुषार्थत्वासम्भवात्~। न केवलं कण्टकादिजनितदुःखनिवृत्तौ लोकानां प्रवृत्तिः~। अपि तु ततः परमुत्पद्यमानसखार्थमेव~। तस्मान्नित्यसुखानुभवरूपा आत्यन्तिकसुखप्राप्तिरेव मुक्तिरिति~। तदसत्~। गौरवात् दुःखसम्बधाच्च न मोक्षस्य सुखस्वरूपत्वम्~। लोके हि सुखस्य क्षणिकत्वमनुभूयते~। तथा च प्रत्यक्षप्रमाणसिद्धस्य सुखस्य क्षणिकत्वात् तदपेक्षयातिरिक्तसुखकल्पने गौरवम्~। किञ्च सुखोत्पत्त्यनन्तरं कालान्तरे दुःखोत्पत्तिरपि लोकानभवसिद्धैव~। न च दुःखानन्तरमेव सुखमिति चेत् नियामकाभावात् उभयोरपि परस्परोत्पत्तिः नाशश्च अङ्गीकार्यः~। तथा च मोक्षानन्तरमपि दुःखोत्पत्तिप्रसङ्गः~। न च श्रुतिविरोधः~। तत्र लक्षणया आनन्द्ादिशब्दानां दुःखाभावबोधकत्वमेव~। तर्हि किमर्थमानन्दशब्दप्रयोगः इति चेत् लोकानां प्रवृत्यर्थमेव~। मोक्षकाले किमपि नास्तीत्युक्ते लोकानां तथा प्रवृत्तिः न स्यात् यथा तदा सुखमस्तीत्युक्ते प्रवृत्तिः~। तस्मान्न सुखस्य मोक्षस्वरूपत्वं युक्तम्~।

		\subsection{काणादमतविमर्शः}

		अशेषविशेषगुणोच्छेदः आत्यन्तिकशरीरनिवृत्तिर्वा मोक्ष इति काणादाः~। अत्रापि आत्यन्तिकदुःखनिवृत्तेः सत्त्वान्न विप्रतिपत्तिः~। किन्तु विशुद्धकुले जातस्यैव श्रद्धा इति तु अयुक्तम्~। विशुद्धकुले जातस्य क्वचित् कौरवादिवदश्रद्धापि स्यात्~। हीनकुले जातस्य श्रद्धपि क्वचित्स्यादेव~। श्रद्धां प्रति न विशुद्धकुलजन्म कारणम्~। तस्मात् निवृत्तिकर्मवशादधर्मादीनां नाशे श्रद्धा ततः तत्त्वजिज्ञासया श्रवणमननादिना दुःखनिवृत्तिरित्येव सुवचम्~।

		\subsection{तत्र गौतममतम्}

		'तदत्यन्तविमोक्षोऽपवर्ग'\footnote{न्या. सू. १.१.२२} इति गौतमसूत्रतात्पर्यभूतं आत्यन्तिकदुःखनिवृत्तिरेवापवर्गः~। अत्र शरीराद्येकविंशतिदुःखान्येव दुःखपदेन विवक्षितम्~। न च प्रलयादौ तेषां नाशात् तदानीं सर्वेषां मोक्ष इति वाच्यम्~। अत एव अत्यन्तविमोक्ष इत्युक्तम्~। प्रलयानन्तरं पुनरुत्पत्तिर्यदा भवति सर्गादौ तदा पुनरेतेषां दुःखानामुद्भवात् न तदा तेषामत्यन्तविमोक्षः~। आत्यन्तिकत्वञ्च स्वसमानाधिकरणदुःखप्रागभावासमानकालीनत्वमेव~। प्रलयादौ स्वसमानाधिकरणदुःखप्रागभावासमानकालिकत्वाभावान्न तस्य आत्यन्तिकत्वम्~, सर्गादौ पुनरुत्पत्तिदर्शनात् तद्धेतुभूतदुःखप्रागभावस्य प्रलयेऽपि सत्त्वात्~। मोक्षे तु ततः परं दुःखानुत्पत्त्या दुःखप्रागभावस्य तदासत्त्वात् स्वसमानाधिरकणदुःखप्रागभावः पूर्वं विद्यमान एव ग्राह्यः, तदसमानकालीत्वस्य दुःखध्वंसे सत्त्वात्समन्वयः~।

		\subsection{अपवर्गप्राप्त्युपायः}

		एतादृशापवर्गस्तु श्रवणादिना सिध्यति~। 'आत्मावारे दृष्टव्यः श्रोतव्यो मन्तव्यो निदिध्यानितव्य' इति श्रुत्या आत्मदर्शन हेतुत्वेन श्रवणमनननिदिध्यासनान्युक्तानि~। गुरुमुखादात्मविषयकोपनिषद्वाक्यानि श्रुत्वा आत्मविषयकं मननं कर्तव्यम्~। मननञ्च युक्तिभिरनुचिन्तनमेव~। तदर्थं पदार्थतत्त्वज्ञानमपेक्षितम्~। तत आत्मनः ध्याने सति तद्दर्शनं सञ्जायते~। एतादृशात्मतत्त्वज्ञानेन मिथ्याज्ञानादिनिवृत्तावपवर्गः सिध्यति~। तदुक्तं गौतममहर्षिभिः 'दुःखजन्मप्रवृत्तिदोषमित्याज्ञानानामुत्तरोत्तरापाये तदनन्तरापायादपवर्ग'\footnote{न्या. सू. १.१.२} इति~। आत्मतत्त्वदर्शनेन इदं शरीरमेवाहमिति शरीरादिषु आत्मत्वभ्रान्तिः न श्यति~। एषा भ्रान्तिरेव मिथ्याज्ञानमित्युच्यते~। तदपाये शरीरोपभोगार्थं ये प्रवृत्तिजनकदोषाः ये सन्ति रागद्वेषमोहाः ते नश्यन्ति~। ततः प्रवृत्त्यनुत्पत्तौ धर्माधर्मौ नोत्पद्येते~। कर्मजन्यौ हि तौ~। कर्मण अभावान्न जायन्ते~। तदपाये जन्म न भवति~। अदृष्टवशाद्धि आद्यशरीरप्राणसंयोगरूपं जन्म~। तदपाये शरीराभावाद्दुःखं नैव भवति इति~।

		\subsection{मोक्षकाले आत्मन अचेतनत्वापत्तिवारणम्}

		नन्वेतादृशमोक्षकाले आत्मनि ज्ञानाद्यभावे तस्य चेतनत्वमेवानुपपन्नम्~। पदार्थेषु जडचेतनभेदो दृश्यते~। घटादिषु ज्ञानोत्पत्तेरभावात् तेषामात्मापेक्षया वैलक्षण्यम्~। तथा च ज्ञानमेव हि जडचेतनव्यवस्थायां‌ प्रयोजकम्~। आत्मनः चैतन्यं ज्ञानमेव~। मोक्षकाले शरीराभावात् ज्ञानाभावे, तस्यैवाभावात् आत्मनः घटवदचेतनत्वापत्तिरिति चेन्न~। ज्ञानस्वरूपयोग्यत्वमेव हि चैतन्यम्~। ज्ञानन्तु अमुक्तस्यापि क्षणिकमित्यनुभवसिद्धम्~। अत एव न जानामीत्यादि प्रतीतिः लोके~। तथा च तदानीमपि आत्मनः जडत्वापत्त्या ज्ञानस्वरूपयोग्यत्वमेव चैतन्यमिति वक्तव्यम्~। यथा अरण्यस्थ दण्डे घटकारणत्वसत्त्वेऽपि न तत्र घटोत्पत्तिर्भवति मृदादिकारणान्तरस्याभावात् तथैव प्रकृतेऽपि ज्ञानकारणत्वस्य आत्मनि सत्त्वेऽपि शरीराद्यभावात् न तत्र ज्ञानोत्पत्तिः~। घटादौ तु न कदापि ज्ञानोत्पत्तिसम्भवः~। तस्माज्ञानस्वरूपयोग्यत्वाभावादेव घटादीनामचेतनत्वम्~। तत्सरूपयोग्यत्वात् आत्मनः चेतनत्वम्~। तच्च मोक्षदशायामप्यात्मनि सम्भवात् न तदा तस्य चेतनत्वानुपपत्तिः~।

		\subsection{दुःखनिवृत्तेः पुरुषार्थत्वस्थापनम्}

		न च दुःखनिवृत्तेः परमपुरुषार्थत्वसम्भवः~। दुःखनिवृत्तौ  जन्तूनां प्रवृत्त्यभावात्~। न केवलं कण्टकादिनिवृत्तौ प्रवृत्तिर्लोकानामपि तु तदुत्तरोत्पत्स्यमानसुखार्थमिति चेन्न~। यथा सुखार्थं लोकानां तथैव दुःखनिवृत्त्यर्थमपि लोकानां प्रवृत्तिः दृष्टा~। पुनः पुनः दुःखैः पीड्यमानाः रोगादिना आवृताः विषसेवनप्रपातपतनानशनादिना शरीरत्यागं कुर्वन्ति~। न तत्र सुखप्राप्तिरुद्देश्या, अपि तु दुःखनिवृत्तिरेव~। न हि तत्र प्रेक्षवतां प्रवृत्तिरिति~। लोकानां प्रवर्तकत्वेन तस्य पुरुषार्थत्वसिद्धिः~। प्रेक्षावन्तोऽपि न परदारादिगमनं कुर्वन्ति~। न तत्र सुखं हेतुः, शास्त्रनिषेधात् तत्करणे कालान्तरे महादुःखप्राप्तिभिया न प्रवृत्तिः~। तस्माद्दुःखनिवृत्तावपि लोकानां प्रवृत्तिदर्शनात् तस्य पुरुषार्थत्वं सिद्धमेव~। किञ्च सुखस्य परमपुरुषार्थत्वे तादृशं किञ्चित् कल्पनीयम्~। दुःखध्वंसस्य तथात्वे तु तस्य अनन्तत्वात् न किञ्चित्कल्पनीयमिति लाघवम्~। तस्मादात्यन्तिकदुःखनिवृत्तिरेव परमपुरुषार्थ इति~।

\begin{center}\begin{small}॥ इति प्राचीनन्याय-वैशेषिक-नव्यन्यायशास्त्रेषु तत्तद्व्याख्याकाराणां सैद्धान्तिकमतभेदानां विमर्शात्मकमध्ययनमिति प्रबन्धे  परिशिष्टाख्यः षष्ठोऽध्यायः~॥\end{small}\end{center}
