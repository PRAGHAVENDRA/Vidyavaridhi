\chapter{प्रथमोऽध्यायः  व्यक्तिशक्तिवादविमर्शः}

	पदानां कार्यान्विते शक्तिरिति पक्षं सयुक्तिकं निराकृत्य मणिकारः पदानां व्यक्तौ शक्तिः\footnote{इदं पदमिममर्थं बोधयतु, अस्मात् पदादयमर्थो बोद्धव्यः इति वा ईश्वरेच्छासङ्केतः इति नैयायिकाः,[त. सं.]} उत जातौ इति विचाराय प्रक्रमते~।

	\section{जातिशक्तिवादमाश्रित्य पूर्वपक्षः}

	\begin{small}
		यत्तु तच्छक्तत्वेन ज्ञातादेव तदर्थबोधः शक्तिभ्रमादपि धीदर्शनात्~। तथा योग्यतादन्वयानुपपत्तौ विना पदादुपस्थितिः शक्तिसाध्येति व्यक्ति\footnote{व्यक्तिर्गुणविशेषाश्रयो मूर्तिः (गौ-२-२-६६) व्यज्यत इति व्यक्तिरिन्द्रियग्राह्या इति न सर्वं द्रव्यं व्यक्तिः। यो विशेषगुणानां स्पर्शांतानां गुरुत्वघनत्वद्रवत्वसंस्काराणामव्यापिनः परिणामस्याश्रयो यथा सम्भवम् तद् द्रव्यम् इति तत्र भाष्यम्।}रपि शक्येति~। तन्न~। अन्यलभ्ये शक्तेरकल्पनात्~। अन्यथा तवान्वयेऽपि शक्तिर्लक्षणाद्युच्छेदश्च~। ननु न जातिरर्थः व्यवहाराभावेन व्युत्पत्तेरसिद्धेः~। कारकोपरक्तक्रिया हि व्यवहारगोचरः~। जातिश्च न क्रिया, नित्यत्वात्~। कारकं न कर्त्रादि क्रियायास्तत्रासमवायात्~, परसमवेतक्रियाफलभागित्वाभावात्~, कर्त्रव्यापारानाश्रयत्वात्~, अचेतनत्वात्~, तया सह विभागाभावात्~, क्रियानाधारत्वादिति चेत्~, न, व्यक्तिव्यवहारादेव उक्तन्यायेन जातौ शक्तिग्रहात्~। जातिसविकल्पकादव्यावृत्ततया ज्ञातायां व्यक्तौ क्रियान्वयः~। सविकल्पकञ्चालोचनद्वारा जातिजन्यमिति परम्परया जातेरपि कारकत्वेनान्वयः~। यद्वा न केवलव्यक्तेः कारकत्वम्~। न हि गौर्गच्छतीत्यत्र व्यक्तिमात्रं यातीति कस्यचित्प्रतीतिः~। किन्तु जातिविशिष्टायाः~। तथा चोभयमपि कारकम्~।
	\end{small}

	जातेरिव व्यक्तेरपि पदादुपस्थितिः शक्तिसाध्येति व्यक्तिरपि पदश्क्या इति न च शङ्कनीयम्~। पदादुपस्थिते अनन्यलभ्ये एव शक्ति स्वीकारात्~। अन्यलभ्येऽर्थे शक्त्यकल्पनात्~। अन्यथा न्यायमते पदस्य अन्वयेऽपि शक्तिस्वीकारप्रसङ्गः~। गङ्गादिपदानां तिरादौ लक्षणायाः उच्छेदप्रसङ्गश्च~। तत्रापि शक्तिस्वीकारस्य दुर्वात्वात्~।

	जातेः पदवाच्यत्वे तत्र व्यवहारेण  शक्तिग्रहः दुरुपपादः~। कारकोपरक्तक्रिया हि व्यवहारगोचरो भवति~। जातिस्तावत् न क्रिया~, नापि कारकम्~। अतो न व्यवहारगोचर इति नशङ्कनीयम्~। यतो हि केवलव्यक्तेः न कारकत्वं सम्भवति~। न हि 'गौः गच्छति' इत्यत्र व्यक्तिमात्रं भातीति कस्यचित् प्रतीतिः~। किन्तु जातिविशिष्टया व्यक्तेः कारकत्वं~। तथा च जातेः व्यवहारगोचरत्वात् व्यवहारेण तत्र शक्तिग्रहः सूपपाद एव~।

		\subsection{व्यक्तिशक्तिवादे प्राभाकरेणापादितं दूषणम् }
			
			\begin{small}	

				तत्र प्राभाकराः\footnote{मीमांसकैकदेशिनः, गुरवः इत्यपराभिधेयम्~।} – यद्यप्यानयनादिव्यवहारादव्यक्तावेव शक्तिरुचिता तथाप्यानन्त्यव्यभिचाराभ्यां तत्र न शक्तिग्रहः~। समुच्चये\footnote{विरोधानवगाही कोटिद्वयसामानाधिकरण्यावगाही च निश्चयः~।}न शक्यत्वे गां दद्यादित्यादौ सर्वोपादानासामर्थ्यम्~। एकस्य शक्यत्वेऽनध्यवसायः~। न च गोव्यक्तिमात्रमर्थः~। मात्रशब्दस्य सर्वार्थत्वे उक्तदोषात्~। सामान्यार्थत्वे व्यक्तेरप्रतीतेः~। नानार्थत्वे च सर्वासां प्रत्येकं ज्ञातुमशक्यत्वं, शक्तौ शक्ये च गौरवम्~, अपूर्वगवि\footnote{अपूर्वगवि व्यवहाराभावश्च इति तत्र तेन रूपेण शक्तेरगृहीतत्वाद्व्यवहाराभावप्रसङ्ग इत्यर्थः, एतच्च समासान्तविधिरनित्य इति साधु इति मणिव्याख्याने~।} व्यवहाराभावश्च~। नापि गोत्वेनोपलक्षिता व्यक्तिः शक्या, धेनु\footnote{’धेनुः सद्यः प्रसूतिका’ इति चामरः। सद्यः प्रसूतवती सवत्सा गौरिति भावः~।}पदवत्~। गृहविशेषो गोत्वेन धानकर्मव्यक्तिविशेषः\footnote{दोहनकार्यसमर्थागौरिति भावः।} स्वतोविलक्षणः~। न तु काकाद्याकारेणैव तत्प्रतीतिः~। न च व्यक्तीनां जातिं विना रूपान्तरमेकमस्ति ज्ञायते वा~। गोपदाद्गौरित्येव प्रतीतेः~। गोत्वविशिष्टे च कार्यान्वयाद्गोत्वं विशेषणं, नोपलक्षणम्~। तदन्येन कार्यान्वये उपलक्षणम्\footnote{व्यावर्तकमित्यर्थः, कार्यान्वयिविशेषणं\footnote{विद्यमानं सत् व्यावर्तकम् , साक्षात्कार्यान्वयि इति यावत्~।} तदन्वयि उपलक्षणम् इति लक्षणमत्र विवक्षितम्~।}~। यथायं वासस्वी देवदत्तशब्दवाच्य इत्यत्र वासः ~। न च गोत्वैकत्वेन व्यक्तीनामैक्यं क्रियते, अशक्यत्वात्~। नापि ज्ञाप्यते, असत्त्वात्~। न च तदेकत्वमेव व्यक्तेरेकत्वम्~, असम्भवात्~। नापि व्यक्तिः शक्या गोत्वमवच्छेदकं कारणत्वे दण्डत्ववत्~, एवं हि गोपदात् न गोत्वविशिष्टबुद्धिः स्यात्~। शक्तिग्रहाहितसंस्कार---गोपदादेव धेनुपदादिव गोत्वविशिष्टज्ञानम्~। तदुद्बोधश्च तद्वदेव~। परम्परासम्बन्धादिति चेत्~, न तत्र हि धानकर्मव्यक्तिविशेषस्यावश्यकत्वेऽनुगमाय गोत्वमवच्छेदकमात्रं न तु वैपरीत्यं, गोत्वस्य वृषभेऽपि सत्त्वात्~। इह तु व्यक्तिरप्रसक्तेति जातिविशिष्टैव सा शक्या स्यात्~। वस्तुतस्तु जातेः शक्योपलक्षणत्वे शक्त्यवच्छेदकत्वे वावश्यकत्वात् लाघवाच्च जातिरेव शक्या स्यात्~, न तु व्यक्तिः~। धेनुपदे तु गोत्वं न तथा अतिप्रसङ्गात्~। अस्तु तर्हि जातिविशिष्टं शक्यं स्वव्यवहारेण च स्वहेतुतया जातिविशिष्टज्ञानस्यानुमितत्वादिति चेत्~, न विशेष्यभेदाद्विशिष्टानामनन्तत्वेन व्यक्तिवाच्यत्वे उक्तदोषग्रासात्~। विशेषणस्यैक्येन विशिष्टानामैक्यस्योपलक्षणपक्षवद्दूष्यत्वात्~। तस्माद्व्यक्तेरपदार्थत्वे जातिरेव पदार्थः~।

				वस्तुतस्तु व्यक्तौ जातिरनुगमिका विशेषिका चावश्यं वाच्येति नागृहीतविशेषणन्यायेन सैव वाच्या~। अथ जातावपि व्यक्तिरेव विशेषः, धर्मान्तराभावात्~। गवेतरवृत्तित्वे सति सकलगोवृत्तित्वादेरुपाधेरपि व्यक्तिघटितत्वात्~। नागृहीतविशेषणन्यायो व्यक्तावपीत्युभयमपि वाच्यमिति न~। जातेः स्वत एव व्यावृत्तत्वात्~। अन्यथा जातिव्यक्त्योर्वाव्यत्तत्वज्ञानादन्योन्याश्रयः~। स्वतोव्यावृत्तत्वञ्च न स्वयमेव स्वव्यावर्तकत्वं स्वस्मिन् स्वावृत्तेः~। नापि व्यावर्तकं विनैव व्यावृत्तत्वम्~, असम्भवात्~। किन्तु स्वाश्रयवत्स्वात्मनि व्यावृत्तधीजनकत्वभावत्वं परेषामन्त्यविशेषत्वात्~। व्यावर्तकधर्मेऽपि धर्मान्तरादेव व्यवृत्तबुद्धावनवस्था स्यात्~। -- एव किञ्चिद्धि वस्तु स्वत एव विलक्षणमित्याहुः~। अस्तु वा व्यक्त्या व्यावृत्ततया बोधिता जातिरेव पदार्थो लाघवात् न तु वैपरीत्यमुभयं वा गौरवात्~। 
			\end{small}

			गवादिपदानां\footnote{गवादिशब्दात् संस्थानरूपाकृतेरपि  बोधस्यानुभविकतया गोत्वादिजातिवत् सापि गवादिपदवाच्ये विशेषणम्~। तस्याश्च  वाच्यविशेषणत्वेऽपि न प्रवृत्तिनिमित्तता साक्षात्संबन्धेन  वाच्यवृत्तित्वाभावात् , अवयवसंयोगरूपायाः  तस्याः सामानाधिकरण्यसंबन्धेनै  गवादौ सत्त्वात्~।  अत्र शक्त्या जात्याकृत्योः एकतरविनिर्मोकेण  अपरभानविरहाल्लाघवाच्चोभयविशिष्टे  गवादिपदस्यैकेव शक्तिः  स्वीक्रियते, शक्यविशेषणभेदेऽपि पुष्पवन्तादिपदवत्  धेन्वादिपदवच्च  शक्तैक्यस्य  दुरपवादत्वात्~।
			
			यत्त्वेकविशिष्टापरावच्छिन्ने  शक्तिरिति ? तदसत्~। विशेष्यविशेषणभावे विनिगमनाविरहात्  गवाद्यंशे सक्षादुभयप्रकारकबोधस्यानुभवसिद्धस्य  दुरपवादत्वाच्च~।

			यत्र केवलाकृतिविशिष्टे  गवादिपदस्य  तात्पर्यं यथा- ”पिष्टकमय्यो गावः” इति न्यायसूत्रे बहुवचनमुपेक्ष्य  ”पदार्थः” इत्येकवचनांतं निर्दिष्टवतो महर्षेरप्यनुमतम् ॥गा-श-वा-वि-कां-१७४ पृ,~।} गोत्वादिजातौ शक्तिः~। यद्यपि 'गामानय' इत्यादिव्यवहारात् गवादिव्यक्तावेव शक्तिः गृह्यते, न तु गोत्वादिजातौ तथापि न व्यक्तौ शक्तिः स्वीकर्तुं शक्यते~। तथाहि - ‘गोपदस्य सकलगोव्यक्तिषु  शक्त्युपगमे श्क्यगोव्यक्तीनामानन्त्येन शक्त्यानन्त्यात् गौरवम्~।

			गोपदस्य यत्किञ्चिद्गोव्यक्तौ शक्त्युपगमे तद्विषयकोपस्थितिं प्रति तद्विषयकशक्तिज्ञानं कारणम् इति  कार्यकारणभावस्य व्यभिचारः\footnote{अनैकान्तिकत्व्ररूपो हेतुदोषः,यथा नित्यः शब्दः अस्पर्शत्वात्(वात्स्या. १-२-५.)एकत्र अव्यवस्था इति वा।}~। गोव्यक्त्यन्तरे शक्तिग्रहाभावेऽपि तदुपस्थितेः जननात्~।

			गोपदस्य मिलितासु सकलगोव्यक्तिषु शक्त्युपगमे ‘गां दद्यात्' इत्यादौ सकलगोव्यक्तिषु दानकर्मतायाः बोधनीयतया तस्याश्च बाधितत्वेन अप्रामाण्यप्रसङ्गः~।	गोपदस्य अन्यतमगोव्यक्तौ शक्त्युपगमे ‘गां दद्यात्' इत्यादिवाक्यात् ‘इयं गौः गोपदशक्या उत अन्या' इति सन्देहोदयेन निर्धारणाभावात्  अप्रवृत्तिप्रसङ्गः~।
	
			गोव्यक्तिसामान्यं गोपदशक्यमित्युपगमे व्यक्तेरशक्यतया प्रतीत्यनुपपत्तिः~। तत्तद्रूपेण सकलगोव्यक्तीनां गोपदशक्यत्वे सर्वेषां तेषां रूपाणां ज्ञातुमशक्यतया शक्तिग्रहानुपपत्तिः~। शक्यतावच्छेदकभेदात् शक्तिभेदाच्च गौरवम्~। अग्रहीतशक्तिकतया अज्ञातरूपावच्छिन्नासु गोव्यक्तिषु गोशब्दव्यवहाराभावप्रसङ्गश्च~।

			गोत्वोपलक्षितानां धानकर्मव्यक्तीनां धेनुपदशक्यत्वमिव गोत्वोपलक्षितानां सकलव्यक्तीनां गोपदशक्यत्वमित्युपगमस्तु न सम्भवति, गोपदशक्यव्यक्तिषु गोत्वस्य उपलक्षणत्वायोगात्~। यतो हि रूपान्तरेण विज्ञातमन्येन रुपेण उपलक्ष्यत इति नियमः~। यथा उत्तृणत्वादिना विज्ञातो गृहविशेषः काकवत्वेन उपलक्ष्यते~। गोव्यक्तिस्तु  गोत्वादन्येन सास्नादिमत्वादिना न हि पूर्वं विज्ञायते~। गोपदश्रवणेन ‘गौः' इत्येव प्रतीतेः~। किञ्च मद्विशिष्टे कार्यान्वयः तद्विशेषणम्, मदन्यविशिष्टे कार्यान्वयः तदुलक्षणमिति लक्षणानुसारं गोत्वस्य नोपलक्षणत्वम्~। ‘गां पश्य' इत्यादौ गोत्वविशिष्टे दर्शनादिकार्यस्य अन्वयात्~। अपि तु विशेषणत्वमेव~। उपलक्षणत्वन्तु ‘अयं वायसी देवदत्तशब्दवाच्य' इत्यादौ वाससः सम्भवति~। वासोविशिष्टे देवदत्तशब्दवाच्यत्वस्य अनन्वयात्~, वाससि तद्बाधात्~। 

			गोत्वविशिष्टानां व्यक्तीनां गोपदशक्यत्वम्\footnote{विषयता संबन्धेन शक्त्याश्रयत्वम्~। यथा गवादेरर्थस्य गोपदशक्यत्वम्~। [न्या. बो.४. पृ.१९].}~, शक्यतावच्छेदकस्य गोत्वस्यैक्यात् सकलगोव्यक्तिनिष्ठशक्तेरैक्यं निर्वहति इत्यपि न सम्भवति~। 

			दण्डनिष्ठायां घटकारतायाम् अवच्छेदकीभूतस्यापि दण्डत्वस्य अकारणत्वमिव गोव्यक्तिनिष्ठायां गोपदशक्यतायाम् अवच्छेदकस्य गोत्वस्य अशक्यत्वेन गोपदजन्यशाब्दबोधे तस्य प्रकारतया भानानुपत्तेः~। तत्प्रकारकशाब्दबोधे वृत्तिग्रहाधीनतदुपस्थितेः हेतुत्वात्~। गोव्यक्तिषु गोपदशक्तेः  ऐक्यानुपपत्तेश्च~। शक्यस्यैव शक्यानुगमकत्वनियमेन शक्यत्वे सति शक्यतावच्छेदकं यत् तदैक्यस्यैव शक्त्यैक्यप्रयोजकत्वात्~। 

			यथा धेनुपदस्य शक्यानां धानकर्मकव्यक्तिविशेषाणाम् अनुगमकं  गोत्वम्~, अशक्यमपि (शक्यतावच्चेदकम्) शक्यैक्यप्रयोजकमित्युपगतं गुरुमते तथा गोपदस्य शक्यानां गोव्यक्तीनां अनुगमकं गोत्वम् अशक्यं सदपि शक्त्यैक्यं प्रयोजयति~। शाब्दबोधे प्रकारतया गोत्वस्य भानं तु गोपदशक्तिग्रहप्रयोज्यगोत्वप्रकारकसंस्कारसहकृतात् गोपदात् सम्भवति इत्यपि न युक्तम्~। 

			यतो हि धेनुपदशक्यानां धानकर्मव्यक्तिविशेषाणाम् अनुगमकस्य गोत्वस्य शक्यत्वं न सम्भवति, अतिप्रसक्तत्वात्~। धेनुपदशक्यत्वाभाववति वृषभेऽपि गोत्वस्य सत्वात्~। गोपदशक्यानां गोव्यक्तीनां तु अनुगमकं गोत्वम् अनतिप्रसक्ततया शक्यं भवेदेव~। तथा च आवश्यकत्वात् लाघवाय गोत्वस्यैव शक्यत्वं न तु गोव्यक्तीनामिति~।

			अस्तु गोत्वमपि शक्यम्~। तथा च गोत्वविशिष्टं गोपदशक्यम् उपगन्तव्यम्~। न तु गोत्वमात्रं गोत्वविशिष्टे गोशब्दव्यवहारात्, गोशब्दश्रवणेन गोत्वविशिष्टस्य उपस्थितेश्च इति न युक्तम्~।

			विशेष्यभेदेन विशिष्टानां भेदात् विशेष्यीभूतव्यक्तीनामानन्त्येन विशिष्टानामानन्त्यात् विशिष्टस्य शक्यत्वे शक्त्यानन्त्यस्य तादवस्थ्यात्~।

			विशिष्टानां भेदेऽपि यावद्विशिष्टनिष्ठावच्छेद्यायाः ऐक्यं सम्भवति, अवच्छेदकैक्यस्य अवच्छेद्यैक्य प्रयोजकत्वात्~। अवच्छेदकस्य च विशेषणीभूत गोत्वादिजातेरैक्यात् इति न शक्त्यानन्त्यम् इति न च वाच्यम्~।

			व्यक्तौ जातेर्विशेषणत्वेन 'नागृहीतविशेषणाबुद्धिः विशेष्यमुपसङ्क्रामत' इति न्यायेन जातेः वाच्यतायाः अवश्यमुपगन्तव्यतया न व्यक्तेः वाच्यत्वं गौरवात्~।

			जातौ व्यक्तिः विशेषणम्~। गोत्वे गवेतरावृत्तित्वे सति सकलगोवृत्तित्वस्य विशेषणत्वात्~, तस्य च व्यक्तिघटितत्वात्~। तथा च नागृहीतविशेषणन्यायेन व्यक्तेरपि वाच्यत्वं सिध्यति इति न युज्यते~। जातेः स्वत एव व्यावृत्तत्वेन तत्र व्यक्तेः विशेषणत्वाभावात्~। अन्यथा जातिव्यक्त्योः ज्ञप्तौ अन्योन्याश्रयप्रसङ्गात्~। स्वतो व्यावृत्तत्वञ्च स्वाश्रयवत् स्वस्मिन् व्यावृत्तधीजनकस्वभाववत्वं, भवन्मते(नैयायिकमते) अन्त्यविशेषवत्~। जातेः स्वतो व्यावृत्तत्वानभ्युपगमे तद्व्यावर्तकस्य उपाधेः पुनरन्यो व्यावर्तकः, तस्यापि अन्य इत्येवम् अनवस्थाप्रसङ्गः इति~।

			जातेः स्वतो व्यावृत्तत्वेऽपि तत्र व्यक्तेः विशेषणतया भानमावश्यकम्~। अन्यथा तन्मात्रावगाहिनो व्यक्त्यनवगाहिनो ज्ञानस्य निर्विकल्पकतया जातौ शक्तिग्रहानुपपत्तिप्रसङ्गः~। तथा च नागृहीतविशेषणन्यायेन व्यक्तेरपि वाच्यत्वम् आवश्यकमिति न युज्यते~। जातिव्यक्त्योः द्वयोः वाच्यत्वे गौरवेण जातिमात्रस्य वाच्यत्वात्~। तथा च जातिमात्रं पदार्थः न तु व्यक्तिरपि~।

		\subsection{जातिशक्तिवादे व्यक्तिभानानुपपत्तेः समाधानम्}

			\begin{small}
	
				अन्ये तु जातिशक्तमे\footnote{शक्तिमत्; तच्च निरूपकता संबन्धेन शक्तिमत्; यथा घटपटादिपदम् कम्बुग्रीवादिमदर्थे शक्तम्। न्या-बो-४पृ-१९.}व पदं जातिव्यक्त्योः स्मारकमनुभावकञ्चेति व्यक्तेरपि शाब्दत्वम्~। न च वृत्तिं विनान्वयानुभवेऽप्रवेशान्न व्यक्तेः शाब्दत्वं, वृत्तिं विनापि एकवित्तिवेद्यत्वनियमेन जातिशक्तादेव व्यक्तेरनुभवात्~। अन्यथा जात्यन्वयोऽपि न स्यात्~। व्यक्तिं विनापि जातेरनुभवात्~। अत एव जातिशक्तिरेव व्यक्तिं बोधयतीति गुरवः~। किञ्च शक्त्योपस्थापितस्यान्वयानुभवं पदानां कारणत्वम्~। अतो जातिवदुपस्थापिताया व्यक्तेरनुभवः पदात्~। न तु तत्तच्छक्योपस्थापितस्य, गौरवात्~। न चैवमशक्यपरत्वे लक्षणा~। यथा ह्यन्यत एव ज्ञानान्न शक्तिस्तथा लक्षणापि न~। तत् किमशक्येऽपि मुख्यः प्रयोगः~। सत्यम्~। शक्त्या साक्षादुपस्थित एव तस्य मुख्यत्वात्~। स्वशक्येति त्वधिकम्~। वस्तुतस्तु जातिशक्तादेव व्यक्तिधीसम्भवान्न व्यक्तौ शक्तिः~। यदि च ततो न तद्धीस्तदा तत्र शक्तिरेव स्यात्~। अन्यथा तद्धीर्न स्यादेव~। 
			\end{small}

			\subsubsection{प्राभाकरोक्तसमाधानम्}
				\begin{small}

					कुतस्तर्हि व्यक्तिधीः~। जातिशक्तावेव कथमन्यशक्तादन्यधीः~। स्वभावात्~। तत्स्वभावत्वमेव व्यक्तिशक्तिं विना न निर्वहतीति चेत् न~। गोपदं हि नियमतो जातिव्यक्ती बोधयति~। तत्रास्य जातिशक्तिधीरेव सहकारिणी कल्प्यते लाघवादावश्यकत्वाच्च~। न तु व्यक्तिशक्तिधीरपि गौरवात्~। जातिशक्तिज्ञाने सति तां विना व्यक्तिबोधे विलम्बाभावात्~। यथा तव पदार्थशक्तादेवान्वयधीः~। यद्वा जातिशक्तमेव पदं व्यक्तिं बोधयति~। अशक्यत्वेऽपि जात्याश्रयत्वमेव नियामकं, यथा अशक्यमपि स्वार्थान्वयं बोधयति~। तत्र प्रयोजकत्वेन क्लृप्ता शक्तिरेवास्तु~। जात्याश्रयत्वस्य तथात्वकल्पने गौरवादिति चेत् न~। अन्यलभ्यस्यापदार्थत्वात्~। अन्यथा अन्वयोऽपि शक्यः स्यात्~, लक्षणाद्व्युच्छेदश्च~। अथवा जातिशक्तिज्ञानाज्जातिधीर्भवन्ती व्यक्तिमपि गोचरयति~। व्यक्तिं विना जातेरभावात्~। यो येन विना न भासते तद्धीहेतुस्तमवबोधयति~। यथा ज्ञानधीहेतुस्तद्विषयम्~। यथा वा तवाधिकरणसिद्धान्तो ज्ञानादिनित्वम्~। अन्यथा पदं जातिमपि न बोधयेत्~, केवलायाः अप्रतीतेः~। तथा च जातिशक्तिकल्पनावैयर्थ्यम्~। तस्मात् जातिज्ञानार्थं क्लृप्ताशक्तिर्व्यक्तिमपि बोधयति, एकवित्तिवेद्यत्वनियमात्~। एतेनैकवित्तिवेद्यतयैव व्यक्तिशक्तिं विना न स्यात् ज्ञापकाभावात्~। न हि व्यक्तिज्ञानमहेतुकं जातिहेतुकं वा, सदातनत्वप्रसङ्गात्~। नापि जातिधीहेतुकं, संविद्भोधापत्तेरिति निरस्तम्~। जातिशक्तेरेव व्यक्तिज्ञापकत्वात्~। 

					ननु जातिं विना प्रत्यक्षादिना व्यक्तिज्ञानादन्यैव व्यक्तिधीसामग्री~। जातिविशिष्टज्ञानञ्चोभयज्ञापकसामग्रीद्वयसमाजादार्थम्~। अत एव व्यक्तिं विना जातेरस्मरणेन जातिस्मरणस्य व्यक्तिविषयत्वनियमात् जातिज्ञापकमात्रमेव व्यक्तिज्ञापकं कल्प्यते~। सामान्यकल्पनायां बाधकाभावादिति परास्तम्~। जातिं विनापि व्यक्तिस्मरणात्तत्रान्यैव सामग्री~। जातिविशिष्टस्मरणञ्चोभयांशस्मारकसमाजादिति~।

					अत्र ब्रूमः~। जातिव्यक्तिप्रत्यक्षादिबोधो तथैव सामग्रीद्वयस्य पृथगन्वयव्यतिरेकग्रहात्~। शाब्दे तु व्यक्तिबोधे जातिशक्तिज्ञानमेव हेतुर्लाघवात् न सामग्र्यन्तरम्~। तत्सत्त्वे तेन विना विलम्बाभावात्~। एवञ्च जातिशक्तत्वेन ज्ञातं पदं जातिविशिष्टस्य स्मारकमनुभावकञ्च~।
				\end{small}

				व्यक्तेः अपदार्थत्वे व्यक्तिबोधानुपपत्तिः~। तद्विषयकशाब्दबोधे तद्विषयकशक्तिज्ञानस्य हेतुत्वात् इति न च चिन्तनीयम्~। जातिविशिष्टव्यक्तिविषयकशाब्दबोधं प्रति जातिशक्तिज्ञानं कार्यकारणभावस्वीकारात् जातिशक्तिज्ञानेन व्यक्तिशक्तिज्ञानं विनापि व्यक्तिबोध उपपद्यते~। यथा न्यायमते व्यक्तिशक्तिज्ञानेन संसर्गबोधः~। यद्वा जातिविषयकबोधं प्रति जातिशक्तिज्ञानं कारणं, व्यक्तिविषयकबोधं प्रति जातिशक्तपदं कारणम्~। यथा न्यायमते अन्वये अशक्तमपि गोपदं स्वार्थस्य अन्वयं बोधयति तथा मन्मते व्यक्तौ अशक्तमपि गोपदं स्वार्थस्य आश्रयं बोधयति~। अथवा जातिविषयकबोधे जातिशक्तिज्ञानं कारणं, व्यक्तिबोधं प्रति जातिज्ञानं कारणम्~। जातिव्यक्त्योः एकवित्तिवेद्यत्वनियमात्~। तथा हि यो येन विना न भासते तद्धीहेतुस्तम् अवबोधयति~। यथा ज्ञानं विषयं विना न भासते अतः ज्ञानधीहेतुः विषयं घटादिकम् अवभासयति~। यथा न्यायमते क्षित्यङ्कुरादिकारणीभूतस्य ज्ञानादेः साधकमनुमानमेव तस्य नित्यत्वं साधयति~। तथा जातिभासकसामग्री व्यक्तिमपि अवबोधयति व्यक्त्या विना जातेः अभानात्~।

				व्यक्तिज्ञानस्य जातिज्ञानहेतुकत्वे संविद्भेदात् तुल्यवित्तिवेद्यत्वानुपपत्तिः~। अहेतुकत्वे जातिहेतुकत्वे च नित्यप्रसङ्ग इति न च वाच्यम्~। व्यक्तिज्ञानं प्रति जातिव्यक्तेः कारणत्वोपयोगमात्~।

				जातिधीहेतुः व्यक्तिभासकः इति न सम्भवति~। जातिम् अविषयीकृत्य प्रत्यक्षादिना व्यक्तिज्ञानोत्पादात् व्यक्तिधीसामग्र्याः जातिज्ञानसामग्रितो भिन्नत्वात्~। जातिविशिष्टव्यक्तिभानं तु उभयज्ञापकसामग्रीद्वयसमाजात् आर्थम् इति न च वाच्यम्~। जातिव्यक्त्योः  प्रात्यक्षिकबोधे सामग्रीद्व्यस्य पृथक् अन्वयव्यतिरेकग्रहात् जातिप्रत्यक्षहेतुः व्यक्तिप्रत्यक्षजनक इति नियमासम्भवेऽपि तयोः शाब्दबोधे पृथक् सामग्रीद्वयं निर्युक्तिकत्वात् नाङ्गीक्रियते~। तथा च जातिशाब्दबोधजनिका जातिशक्तिज्ञानादिसामग्र्येव व्यक्तिबोधजनिका नान्यसामग्री~।

			\subsubsection{कुब्जशक्तिवादेन समाधानम्}
				\begin{small}
					एवञ्च  जातिव्यक्तिज्ञानजनकत्वादुभयत्रापि शद्बशक्तिः~। जात्यंशे सा ज्ञाता व्यक्त्यंशे स्वरूपसतीहेतुर्लाघवादिति कुब्जशक्तिवादः~। एवञ्च सैव तदैव तेनैव ज्ञाता अज्ञाता च वाचिका अवाचिका वेत्यत्र जातिशक्त्यवच्छेदकभेदेनाविरोधः~। त्वयाप्यन्वये कुब्जशक्तिस्वीकारात्~। व्यक्तेः शक्यत्वेऽपि ज्ञातशक्तिशब्दजनितज्ञानविषयत्वलक्षणं वाच्यत्वं नास्ति~। न चैवं परिभाषा~। शब्दजन्यज्ञानविषयत्वेन वाच्यत्वे लाक्षणिकादेरपि वाच्यत्वापत्तेः~। व्यक्तेः शक्यत्वेऽपि व्यक्तिशक्तिज्ञानं न कारणम्~। व्यक्तिशक्तिज्ञानत्वं नावच्छेदकमिति लाघवम्~।
				\end{small}

				गोपदात्जातिव्यक्त्युभयविषयकबोधजनात् तदुभयत्र गोपदस्य शक्तिरस्ति~। किन्तु जात्यंशे सा शक्तिः ज्ञाता उपयुज्यते, व्यक्त्यंशे स्वरूपसती उपयुज्यते~। अयमेय कुब्जशक्तिवादः~। नैयायिकेनापि अन्वयांशे कुब्जशक्तिः स्वीक्रियते~। इत्थञ्च व्यक्तेः शक्यत्वेऽपि वाच्यत्वं नास्ति, यतो हि ज्ञातशक्तिमच्छब्दजनितज्ञानविषयत्वरूपं वाच्यत्वम्~।

				शब्दजन्यज्ञानविषयत्वेन लाघवाद् अस्तु वाच्यत्वं, तथा च व्यक्तेरपि वाच्यत्वं       सिध्यतीति न वाच्यम्~। तथा सति लाक्षणिकगङ्गादिपदेरपि वाच्यत्वप्रसङ्गात्~। 

			\subsubsection{श्रीकरोक्तसमाधानम्}
				\begin{small}
					श्रीकरस्तु केवलजातिव्यक्त्योरकारकत्वात् क्रियान्वयो व्यक्तेराश्रयतया जातेरवच्छेदकतया आरुण्यादिवत्~। एवञ्च जातिशक्तपदात् जातेरनुभवः शाब्दः~। व्यक्तेरौपादानिकः अशक्यत्वादिति~। ननु पदजातिभ्यामप्येको जातिव्यक्त्यनुभवः क्रियते, तत्र जात्यंशे पदस्य व्यक्त्यंशे जातेरनुभावकत्वम्~, अयमेव उपादानार्थ इति चेत्~, 

					ननु पदजातिभ्यामप्येको जातिव्यक्त्यनुभवः क्रियते, तत्र जात्यंशे पदस्य व्यक्त्यंशे जातेरनुभावकत्वम्~, अयमेव उपादानार्थ इति चेत्~, 
				\end{small}

				'गौः गच्छति' इत्यादौ व्यक्तेः अनुभवः औपादानिकः, जात्यनुभवः शाब्दः~। यद्यपि जातिव्यक्त्योः एकोऽनुभवः जायते~। तथापि तत्र जातिभाने पदं प्रयोजकं व्यक्तिभाने जातिः प्रयोजिका~। जातेः व्यक्त्यनुभावकत्वमेव उपादानार्थः~।

				गोव्यक्तेः गमनादिक्रियया आश्रयतासम्बन्धेन अन्वयः, गोत्वस्य तु अवच्छेदकतया यथा 'अरुणया पिङ्गाक्ष्या एकहायन्या गवा सोमं क्रीणाति' इत्यत्र गोः करणत्वेन क्रयणक्रियान्वयः आरुण्यस्य अवच्छेदकतया तदन्वयः~। 

			\subsubsection{भाट्टसम्मतं समाधानम्}
				\begin{small}

					भाट्टमते तु जातिरेव शक्या लाघवात्~। व्यक्तिस्त्वाक्षेपलभ्या~। नन्वाक्षेपाद्व्यक्तिधीर्न गोत्वेन गोत्वस्य गोत्वविशिष्टाभेदेनाक्षेपाभावादिति चेन्न~। विशेषणविशेष्ययोर्भेदेनानुमानविरोधात्~। अत एव गोत्वं व्यक्त्याश्रितं जातित्वादिति पक्षधर्मताबलात् गोत्वाश्रयव्यक्तिसिद्धिः~। अर्थापत्तेर्वा तत्सिद्धिः~। ननु व्यक्त्या विना किमनुपपन्नं, व्यक्तिं विनापि गोत्वस्य तद्बुद्धेश्च सिद्धेः कथमर्थापत्तिरिति चेन्न~। व्यापकं विना व्याप्यस्यासिद्धेः~।

					भाट्टमते तु जातिरेव शक्या लाघवात्~। व्यक्तिस्त्वाक्षेपलभ्या~। 

					नन्वाक्षेपाद्व्यक्तिधीर्न गोत्वेन गोत्वस्य गोत्वविशिष्टाभेदेनाक्षेपाभावादिति चेन्न~। विशेषणविशेष्ययोर्भेदेनानुमानविरोधात्~। अत एव गोत्वं व्यक्त्याश्रितं जातित्वादिति पक्षधर्मताबलात् गोत्वाश्रयव्यक्तिसिद्धिः~। अर्थापत्तेर्वा तत्सिद्धिः~। ननु व्यक्त्या विना किमनुपपन्नं, व्यक्तिं विनापि गोत्वस्य तद्बुद्धेश्च सिद्धेः कथमर्थापत्तिरिति चेन्न~। व्यापकं विना व्याप्यस्यासिद्धेः~। 
				\end{small}

				गवादिपदात् जातिमात्रलाभः, व्यक्तिलाभस्तु आक्षेपात् इति भाट्टमतमत्र विचार्यते~। ननु गोपदात् गोत्वस्मरणे जाते ततः आक्षेपात् उत्पद्यमाने व्यक्तिज्ञाने व्यक्तिः व्यक्तित्वेन भासते उत गोत्वेन रूपान्तरेण~। न तावत् व्यक्तित्वेन नापि रूपान्तरेण~। 'गाम् आनय' इत्यादिशब्दप्रयोगोत्तरं गोत्वस्यैव प्रतीतेः तयोरप्रतीतेः~। नापि गोत्वेन व्यक्तेर्भानं भवितुमर्हति~। तथा सति आक्षेप्ये गोत्वविशिष्टे अन्तर्गततया गोत्वस्य आक्षेपकत्वसम्भवात् इति चेत् न~। गोत्वेन रूपेण व्यक्तेः आक्षेप्यत्वेऽपि गोत्वांशस्य नाक्षेप्यत्वम्~, अपि तु विशेष्यभूतव्यक्तिमात्रस्य आक्षेप्यत्वम्~। तथा च गोत्वं व्यक्त्याश्रितं जातित्वात् इत्यनुमानं फलति~। अत्र हेतोः पक्षधर्मताबलात् गोत्वाश्रयव्यक्तिः सिध्यति~। अथवा अर्थापत्या व्यक्तिः सिध्यति~। यथा हि दिवा अभुञ्जानत्वे सति पीनत्वस्य रात्रिभोजनमन्तरा अनुपपद्यमानतया देवदत्तस्य रात्रिभोजनं अर्थापत्त्या कल्प्यते तथैव व्यक्तिमन्तरा गोत्वादिजातेः अनुपपद्यमानतया व्यक्तिः कल्प्यते इति~। 

			\subsubsection{मण्डनोक्तं समाधानम्} 

				\begin{small}

					उच्यते~। गामानयेत्यतो गोत्वविशिष्टस्य क्रियान्वयबोधाद्गौरित्याकारकगोविशेष्यकबुद्धिः कारणम्~। सा च न शब्दं विना~। आक्षेपाद्व्यक्त्याश्रितं गोत्वमिति धीर्न तु गौरिति~। न चैवं व्यक्तेः क्रियान्वयोऽपि~। गोत्वाश्रिततया निराकाङ्क्षत्वात् राजपुरुषमानयेत्यत्रैव राज्ञः~। अन्वये वा व्यक्तिमानयेति धीः स्यात्~, न तु गामिति~। किञ्च गोत्वं न व्यक्तिव्याप्यम्~। न हि यत्र यदा वा गोत्वं तत्र तदा व्यक्तिर्यत् सामान्यं सा व्यक्तिरिति वा नियमः~। व्यभिचारात्~। नापि गोत्वं गवाश्रितं गोत्वादित्यनुमितिः, व्यप्तिग्रहशरीरत्वात्~। न च जातित्वं व्यक्त्याश्रितत्वे लिङ्गम्~। जातित्वस्य पदादनुपस्थितेः~। तथात्वे वा जातिवित्तिवेद्यैव व्यक्तिः~। अपि च लिङ्गं व्याप्यमनुपपन्नं स्वाश्रये व्यापकमुपपादकञ्च बोधयति~।

					चेह गोत्वाश्रये व्यक्तिबुद्धिः~।

					वयन्तु ब्रूमः – व्यक्तेरपदार्थत्वे विभक्त्यर्थसङ्ख्याकर्मत्वादेर्व्यक्तावन्वयः स्यात्~। सुविभक्तीनां प्रकृत्यर्थान्वयस्वार्थबोधकत्वस्य व्युत्पत्तिसिद्धत्वात् प्रकृतितात्पर्यविषये तदन्वयव्युत्पत्तौ लक्षणोच्छेदो गौरवञ्च~। आख्यातार्थसङ्ख्यापि नानुमितेनान्वेति~। किन्तु भावनान्वयिना शुद्धेन प्रथमान्तादुपस्थितेन
				\end{small}

				मण्डनप्रभृतयः गोशब्दस्य जातौ शक्तिः व्यक्तौ लक्षणा~। इत्थञ्च व्यक्तेःपदार्थत्वात् विभक्त्यर्थसङ्ख्याकर्मत्वादिना अन्वय उपपद्यते~। ” गौः नष्टा, गौरुत्पन्ना ” इत्यादिव्यवहारदर्शनात्, जातौ उत्पादविनाशयोः अन्वयानुपपत्त्या व्यक्तौ लक्षणा स्वीकर्तव्या इति वदन्ति~। 

			\subsubsection{जातिशक्तिवादे संस्कारेण व्यक्तिप्रतीत्युपपादनम्}

				\begin{small}

					एतेन जातिरेव शक्या लाघवात्~। शक्तिग्रहजन्यसंस्कारस्य व्यक्तिविषयत्वनियमेन पदात् जातिस्मरणमुत्पाद्यमानमवश्यं व्यक्तिविषयम्~। संस्कारस्यानियतोद्बोधकत्वेऽपि जात्यंशोद्बोधकादेव व्यक्त्यंशोद्बोधनियमकल्पना~। यथा पदेनोद्बुद्धसंस्कारादेव नियता शक्तिस्मृतिः पदं विनापि च सर्वा जातिस्मृतिर्व्यक्तिविषया~। अन्यथा केवलजातिमात्रस्मरणापत्तेः~। संस्कारसहितात् पदादेव जातिविशिष्टानुभवोऽपीन्द्रियादिव प्रत्यभिज्ञा~। अत एव भाष्यं ’संस्कारशब्दशक्तिभ्यां विशिष्टानुभव इत्युन्नीतमतमपास्तम्’~। 
				\end{small}

				पदानां जातावेव शक्तिः, व्यक्तिप्रतीतिस्तु संस्कारात्~। तथा हि - जातौ शक्तिमवगाहमानो ग्रहः व्यक्तिमपि विषयीकतोति~। अतः सोऽयं शक्तिग्रहः नियमेन व्यक्तिविषयकसंस्कारं जनयति~। तथा च गवादिपदात् जातिविषयकस्मरणमुत्पद्यमानं व्यक्तिमपि नियमेन अवगाहते~। न च संस्कारस्य नियोतोद्बोधकत्वाभावात् शाक्तिग्रहजन्यस्मरणस्य नियमेन व्यक्तिविषयकत्वमनुपपन्नम् इति वाच्यम्~। जात्यंशोद्बोधकस्यैव व्यक्त्यंशोद्बोधकत्वोपगमेन जातिस्मरणस्य नियमेन व्यक्तिविषयकत्वोपपत्तेः~। यथा संस्कारसहितात् इन्द्रियात् तत्तविशिष्टेदन्तावगाहिनी प्रत्यभिज्ञा समुदेति, तत्ताभाने संस्कारः इदन्ताभाने इन्द्रियं प्रयोजकं तथा संस्कारसहितात् पदादेव व्यक्तिविशिष्टजातेस्तु अवगाहमानो अनुभवो जायते, व्यक्तिभाने प्रयोजकः संस्कारः जातिभाने पदम्~।

			\subsubsection{ज्ञानशक्तिवादे व्यक्तिशक्तिनिराकरणम्}

				\begin{small}
					ननु ज्ञाने शक्तिः\footnote{यत्तु ज्ञाने पदानां शक्तिः, नत्वर्थे तथा च तज्ञानशक्तत्वं तद्वाचकत्वं  तच्च तद्विषयकशाब्दधीजनकतावच्छेदकधर्मवत्वम् , स च धर्मो वह्न्यादिनिष्ठदाहाद्यनुकूलशक्तिवत्  पदार्थान्तरमेव~। लाक्षणिकं च नानुभावकमेवेति न तत्र शक्तिः ॥
					
					घटादिपदस्य मितिमातृविषयकधीजनकत्वेपि न तद्वाचकत्वं तद्विषयकत्वावच्छिन्नज्ञानशक्तपदस्यैव  तद्वाचकत्वात् , मितिमातृविषयकत्वेन च ज्ञानस्य न  किञ्चित्पदशक्यताज्ञानसामान्यसामग्र्या एव मितिमातृभासकतया  पदान्तरस्य च तदर्थभासकतया  तद्विषयकत्वस्य  तत्पदजन्यतावच्छेदकत्वे  मानाभावात्~। ग-श-वा-सा,कां, पृ,सं,१२-१३.} शक्यत्वात्~। तथा च यस्य ज्ञाने शक्तिस्तच्छक्यं जातिव्यक्तिज्ञाने च शक्तिरिति जातिवत् व्यक्तिरपि शक्येति चेत् न~। यद्विषयतया हि ज्ञाता ज्ञाने शक्तिरुपयुज्यते तत् शक्तिज्ञाने विषयतया शक्यतावच्छेदकम्~। शक्यं जातिज्ञानञ्च तथेति जातिरेव शक्या न तु व्यक्तिज्ञानत्वमप्यवच्छेदकं गौरवात्~। न च शक्यज्ञाने नियतविषयत्वमेव शक्यत्वं व्यवहारानुमितशक्यानुभवविषयाणामन्वये तत्सम्बन्धिमितिमातॄणामपि शक्यतापत्तेः~। 	ज्ञानशक्तिवादे पदस्य यद्विषयकज्ञानजननशक्तिः तत् शक्यम्~। तथा च गोपदस्य गोव्यक्तिगोत्वजातिएतदुभयविषयकबोधजननशक्तिमत्वात् तदुभयं शक्यम्~। तथा च व्यक्तिशक्तिसिद्धिरिति न च वाच्यम्~। यद्विषयतानिरूपितशक्तिविषयता शाब्दबोधजनकतावच्छेदिका तस्यैव शक्यत्वोपगमात्~। जातिविषयतायाः जनकतावच्छेदककोटिप्रविष्टतया जातेः शक्यत्वसम्भवेऽपि व्यक्तिविषयतयाः लाघवात् जनकतावच्छेदककोटौ अप्रवेशेन व्यक्तेः अशक्यत्वात्~। शक्तिविषयतानियतविषयताशालित्वमेव शक्यत्वं भवतु, लाघवात्~। तथा च व्यक्तेः  शक्यत्वं सिध्यतीति न वाच्यम्~। तथा सति मिति-मातॄणामपि शक्यत्वापत्तेः~।
				\end{small}	

	\section{व्यक्तिशक्तिवादसमर्थनम्}

		\subsection{व्यक्तिशक्तिवादे आपादितदोषस्य निराकरणम् }

			\begin{small}
				अत्रोच्यते~। गोत्वज्ञाने गोज्ञाने वा शक्तं पदमित्याकारः शक्तिग्रहः~। तथा च शक्यज्ञाने विषयतया जातेरवच्छेदकत्वं व्यक्तिमादायैव प्रतीयते, न केवलायाः~। व्यक्तिं विना जातेरप्रतीतेः~। तथा च जातेरवच्छेदकत्वे नागृहीतविशेषणान्यायेनव्यक्तेरवच्छेदकत्वं वज्रलेपायितमिति शक्तिज्ञानेविषयतया अवच्छेदकत्वात् जातिवत्सापि शक्या~। तस्मात् परिहरैकवित्तिवेद्यत्वनियमं स्वीकुरु वा व्यक्तेर्वाच्यत्वम्~। अपि च यद्धर्मवत्तया ज्ञात एव यत्र यस्य ज्ञानं स तत्रावच्छेदकः~। व्यक्तिज्ञानत्वेन ज्ञात एव तत्र ज्ञाने शक्तिधीरिति व्यक्तिरपि शक्या~। नन्वाद्यव्युत्पत्तौ मितिमातृविषयत्वेन ज्ञात एव ज्ञाने प्रवर्तकत्वं ज्ञातम्~। न च तयोर्ज्ञानं प्रवर्तकमयो व्यभिचारः~। अन्यथा प्रवर्तकत्वेन मितिमातृज्ञानमपि प्रयोज्यस्यानुमाय बालस्तत्रापि शक्तिं गृहीयात्~। एवं तवापि कार्यान्वितज्ञाने एव शक्तिः स्यात्~। तत्त्वेन ज्ञात एव ज्ञाने शक्तिग्रहादिति चेत्~, मितिमातृज्ञानत्वं कार्यान्वितज्ञानत्वञ्च विनापि प्रवर्तकज्ञाने घटज्ञानत्वादिकं ज्ञातुं शक्यमिति तयोर्नावच्छेदकत्वम्~। किन्तु घटज्ञानत्वादिकमेव लाघवात् तयोरपि तत एव प्राप्तेश्च~। किञ्च पदं व्यक्तिज्ञानार्थं शक्यज्ञाने विषयतया व्यक्तेरवच्छेदकत्वमात्रं कल्पयति लाघवात्~। जातिविषयत्ववद्व्यक्तिविषयत्वस्य ज्ञानवित्तिवेद्यत्वेनावश्यं शीघ्रोपस्थितत्वात्~। न तु जातिशक्तिस्तद्बोधे कारणान्तरं वा तदवच्छेदकं गौरवाच्छक्तिग्रहकाले कल्पनीयोपस्थितिकत्वाच्च~।
	
				अन्ये तु प्रथमं व्यवहारानुमितव्यक्तिज्ञाने शब्दानुविधानात् पदं शक्तमित्यवधारयति, न तु जातिज्ञाने~। प्रथमं व्यवहारानुमितव्यक्तिज्ञाने शब्दानुविधानात् पदं शक्तमित्यवधारयति न तु जातिज्ञाने~। व्यवहाराजनकत्वेन तदा तस्यानुपस्थितेः~। पश्चाद्व्यक्तेर्वावृत्त्यर्थम् अनुगमार्थञ्च जातिरपि तद्विषय इति मानान्तरेण ज्ञात्वा जातिज्ञानेऽपि तत्पदस्य कारणतां प्रत्येति~। तथा च व्यक्तिशक्तिज्ञानमपि कारणं, न तु जातिशक्तिज्ञानेनान्यथासिद्धिः~। व्यक्तिज्ञानकारणतामुपजीव्य जातिज्ञाने कारणताग्रह इत्युपजीव्यविरोधात्~। 

		
				उक्तन्यायैर्जातिवद्व्यक्तेरपि शक्यत्वात्~। विशिष्टानामानन्त्येऽपि एकत्र विशिष्टे तत्त्वं विहाय गोत्वमादाय गोत्वविशिष्टं शक्यमिति शक्तिग्रहः~। यथा च क्वचिदेव धूमे धूमो वह्निव्याप्य इति बुद्धिः~। यथा वा तवैव कार्याणामानन्त्येऽपि क्वचित् कार्ये तत्त्वं तटस्थीकृत्य कृतिमादाय धर्मिणि कार्यं शक्यमिति लिङादेरपूर्वे शक्तिग्रहः~। अवच्छेदकैक्याच्छक्तेरेकत्वं तद्वदेव~। यथा वा व्यक्तिवाचकपश्वादिपदानाम्~। अथवा गोत्वेन सामान्यलक्षणया ज्ञाते सर्वत्र गवि गोत्वमादाय शक्तिग्रहः~। प्रमेयत्वेन च सर्वज्ञाने सार्वज्ञमिष्यत एव~। नेष्यते तु घटत्वादिसर्वप्रकारज्ञानवत्वेन~। सर्वैकोदासीनगोः शक्यत्वात् यत्किञ्चिदेकोपादानेऽपि गोरुपादानादन्योपादाना शक्यत्वम्\footnote{विषयता संबन्धेन शक्त्याश्रयत्वम्~। यथा गवादेरर्थस्य गोपदशक्यत्वम्~। [न्या. बो.४. पृ.१९].} एकोपादानेऽनध्यवसायो वा~। अन्यथा तवाप्येकवित्तिवेद्यतया सर्वैकपरत्वे उक्तदोषे का गतिः~। का वा गतिर्व्यक्तिवाचकपश्वादिपदानाम्~। 
			\end{small}

			\subsubsection{ज्ञानशक्तिवादे व्यक्तिशक्त्युपपादनम्}

				\begin{small}

					ननु नाक्षेप एकवित्तिवेद्यत्वात्तयोः~। समानानां हि भावः सामान्यम्~। तच्च व्यक्तिं विना न भासते इति चेन्न~। स्वरूपेण शक्या जातिः~। न च सामान्यत्वं तस्याः स्वरूपं, तद्धर्मत्वात्~। अन्यथा आलोचनेऽपि सा न भासेत~। कथं सामान्यत्वेनाप्रतीता जातिर्व्यक्तितो भिन्नतया शब्देनाभिधातव्येति चेन्न~। शब्देन व्यक्तितो भिन्नतया जातेरबोधनात्~। ननु व्यावृत्तजातिर्वाच्या~। व्यावृत्तबुद्धिं विना व्यक्तिविशेषानपेक्षात्~। व्यावर्तिका च तत्र व्यक्तिरेव, अनुगतत्वमप्यनुगम्यमानं विना न भासत इति जातिवित्तिवेद्यैव व्यक्तिरिति चेन्न~। स्वतो व्यावृत्तजातिस्वरूपस्य वाच्यत्वात्~। व्यक्तेर्धर्मान्तरस्य वा व्यावर्तकत्वेऽन्योन्याश्रयोऽनवस्था वा~।

					ननु गौरिति पदात् जातिव्यक्त्योर्युगपत्प्रतीतिः~। न च सूक्ष्मकालभेदाग्रहात् सा भ्रान्ता~। बाधकाभावात्~। तथा च गोपदात् गोत्वधीस्ततः क्रमेण व्यप्तिपक्षधर्मताज्ञानं ततो व्यक्त्यनुमितिरिति ज्ञानपरम्पराकल्पनाद्वरं जातिवित्तिवेद्यत्वं व्यक्तेरिति चेन्न~। व्युत्पत्त्यधीनं हि शब्दस्य बोधकत्वम्~। अतो व्युत्पत्तिपर्यालोचनया युगपज्ज्ञानमसिद्धम्~। अत एव ज्ञानपरम्पराकल्पनमपि युक्तम्~। अन्यथा कर्तुरप्याक्षेपो न स्यात्~। शब्दात् सकर्तृकाया एव क्रियाया अवगमात्~। न च जातिज्ञानत्वेन व्यक्तिविषयतानियमः~। प्रत्यक्षादौ तस्य व्यक्तिधीहेतुसमाजाधीनत्वात्~। अत एव न जातिधीहेतौ व्यक्तिधीहेतुसहकारितानियमः समाजस्यार्थसिद्धत्वात्~। गोत्वं गवाविषयप्रतीतिविषयः जातित्वात् गोभिन्नभावत्वाद्वेति जातिमात्रधीसिद्धेश्च~। अथ यत् यत्परतन्त्रं तत् तेनैकवित्तिवेद्यं यथार्थपरतन्त्रं ज्ञानमर्थेन जातिश्च परतन्त्रेति व्यक्तौ  भासमानायामेव भासत इति चेन्न~। परतन्त्रत्वं हि न परसमवेतत्वं गन्धादिना व्यभिचारात्~। न तद्धीनिरूप्यत्वमसिद्धेः~। नापि तस्मिन् भासमान एव भासमानत्वं साध्याविशेषात्~। नापि विशेषणत्वेनैव ज्ञानं गौरित्येव प्रतीतेः~। गवि गोत्वमिति कश्चित् प्रत्येतीति चेन्न~। आलोचने विशेषणत्वं विनापि स्वरूपतः प्रतीतेः~। जातिमात्रशक्तात् पदात् जातेः स्मरणमालोचनमेव जातिविशिष्टगोचरसंस्कारादेव पदेन जात्यंशोद्बोधे सति जातिमात्रस्मरणात्~। अत एव ततो जातिं विनापि कदाचिद्वक्तिस्मृतिः~। अस्तु वा गुरोरिवालोचनमपि संस्कारजनकं शब्दव्युत्पत्तिबलेन जातिमात्रस्मरणसिद्धेः~। न च स्मरणस्य विशिष्टज्ञानत्वमेव~। अनुभवस्यापि तथात्वेन निर्विकल्पकासिद्धिप्रसङ्गात्~। एकवित्तिवेद्यत्वेऽपि प्रथमदर्शनवत् शब्दाद्गोत्वस्मरणमालोचनमेव~। गोत्वे गोव्यक्तिवृत्तित्वादिवैशिष्ट्यस्याशक्यत्वेन तदविषयत्वात्~। न चालोचनस्येन्द्रियजन्यत्वात् न स्मृतित्वम्~। ज्ञानत्वसाक्षाद्व्याप्यधर्मत्वेन स्मृतित्वस्यालोचनवृत्तित्वात्~। 

					पदान्तरादुपस्थितिरेव तत्राक्षेपार्थः~। अत एव न व्यक्तेराक्षेपः~। किन्तु लक्षणया गोपदाद्गौरिति व्यक्तिधीरिति मण्डनः~। यदाह- “जातावस्तित्वनास्तित्वे न हि कश्चिद्विवक्षति~। नित्यत्वोल्लक्ष्यमाणाया व्यक्तेस्ते हि विशेषणे” इति~। उच्यते – स्वार्थादन्येन रूपेण ज्ञाते भवति लक्षणा~। तीरत्वेन ज्ञाते गङ्गापदस्येव~। न चेह गोत्वादन्येन रूपेण व्यक्तेरुपस्थेतिः, किन्तु गोत्वेनैव~। व्यक्तित्वेन सास्नादिमत्वेन चोपलक्ष्यत्वे गोपदादव्यक्तित्वादिरूपेण धीः स्यान्न तु गौरिति~। नापि गोत्वसम्बन्धिनि गोत्वविशिष्टे लक्षणा~। गोत्वे हि न साक्षादानयनाद्यन्वय इति व्यक्त्यवच्छेदकतया तस्यान्वयेऽमुख्यत्वम्~। लक्षणयापि गोत्वावच्छिन्नैव व्यक्तिः क्रियान्वयिनी प्रतीयते न केवला व्यक्तिरिति गोत्वविशिष्टस्य लक्ष्यत्वे युगपद्वृत्तिद्वयविरोधः~। गोत्वेऽपि वा लक्षणा~। अपि च जातिमात्रे न शक्तिर्न वा व्यक्तौ लक्षणा~। जातौ मुख्यप्रयोगाभावात्~। तयोस्तन्मूलकत्वात्~। प्रयोगो हि व्यवहारहेतुज्ञानार्थः~। न च जातिमात्रनिर्विकल्पकाद्व्यक्तिमनादाय केवलजातौ व्यवहारः~। तस्य विशिष्टज्ञानसाध्यत्वात्~। गां पश्य, गौरस्तीत्यादावपि गोत्वविशिष्टस्यैव ज्ञानं व्यवहारश्च~। 
				\end{small}

				'गोत्वज्ञाने शक्तं गोपदम्' 'गोज्ञाने शक्तं गोपदम्' इत्याकारको वा शक्तिग्रहः, गोव्यक्तिं विषयीकृत्यैव गोत्वजातिमवगाहते, जातिव्यक्त्योः तुल्यवित्तिवेद्यत्वात्~। तथा च शक्तिज्ञाननिष्ठकारणतायाः अवच्छेदककोटौ जातिविषयताया इव व्यक्तिविषयताया अपि प्रविष्टत्वात् व्यक्तेः शक्यत्वं दुर्वारम्~, नागृहीतविशेषणन्यायात्~। व्यक्तिमविषयीकृत्यापि शक्तिज्ञाने जातिभानस्योपगमे तुल्यवित्तिवेद्यत्वनियमस्य परिहृतत्वात् शब्दात् व्यक्ति प्रतीतिनिर्वाहाय तत्र पदवाच्यत्वस्वीकार आवश्यकः~। यद्धर्मवत्तया ज्ञानं एव यत्र यस्य ज्ञानं स धर्मः तन्निष्ठायाः तस्याः अवच्छेदक इति नियमः~। व्यक्तिविषयकत्वेन ज्ञात एव जातिज्ञाने गोपदशक्तेः ज्ञानोदयात् व्यक्तिविषयकत्वम् जातिज्ञाननिष्ठायाः गोपदशक्यतायाः अवच्छेदकम्~। 

				दर्शितनियमो व्यभिचरितः~। मितिमातृविषयकत्वेन ज्ञाते एव घटादिज्ञाने घटादिपदशक्यतायाः ग्रहणेऽपि मितिमातृविषयकत्वयोः घटादिपदशक्यतावच्छेदकत्वानभ्युपगमात्~। एवं न्यायनयेऽपि आद्य व्युत्पत्तिसमये कार्यान्वितत्वेन ज्ञाते एव घटादौ घटादिपदशक्यतायाः ग्रहणेऽपि कार्यान्वितत्वस्य घटादिपदशक्यतावच्छेदकत्वानभुपगमात् इति नाशङ्कनीयम्~। यतो हि मितिमातृविषयकत्वेन कार्यान्वितत्त्वविषयकत्वेन च आज्ञातेऽपि घटादिज्ञाने घटादिपदशक्यतायाः अवगाहनेन तयोः न शक्यतावच्छेदकत्वम्~।  व्यक्तिविषयकत्वेन तु रूपेणाज्ञाते घटत्वादिज्ञाने घटादिपदशक्यत्वस्य अवगाहनं नैव सम्भवति, व्यक्तिमविषयीकृत्य जातेः भानाभावात्~। 

				उक्तनियमानभ्युपगमेऽपि व्यक्तेः पदवाच्यत्वं सिध्यति~। तथा हि घटादिपदात् जातिव्यक्त्युभयावगाहिप्रतीतिस्तावत् सर्वसम्मता~। तथा हि घटादिपदमेव तदुभयप्रतीतिकारणत्वान्यथानुपपत्त्या तदुभयविषयकत्वस्य शक्यतावच्छेदकत्वं कल्पयति~। प्रयोगस्तु घटादिपदं जातिव्यक्त्युभयविषयकत्वावच्छिन्नज्ञाननिष्ठशक्यताकं जातिव्यक्त्युभयविषयकशाब्दबोधजनकत्वात् इति~। जात्याश्रयत्वादेः कारणान्तरस्य व्यक्तिभासकस्य कल्पनायां तु गौरवम्~। 

			\subsubsection{व्यक्तिशक्त्युपपादने युक्त्यन्तरम्}

				प्रथमं व्यवहारेण शक्तिग्रहकाले घटादिपदानां घटादिव्यक्तौ शक्तिः अवधार्यते, न तु जातौ~। तस्याः व्यवहारविषयत्वात्~। व्यक्तेरेव व्यवहारविषयत्वात्~। पश्चाच्च व्यक्तेः व्यावृत्त्यर्थम् अनुगमार्थं च जातेरपि व्यवहारविषयतां प्रमाणान्तरेण विनिश्चित्य तत्रापि पदस्य शक्तिः निर्णीयते~। तथा च जातिशक्तिज्ञानस्य व्यक्तिशक्तिज्ञानं प्रत्युपजीविकत्वात् जातिशक्तिज्ञानेन व्यक्तिशक्तिज्ञानस्य नान्यथासिद्धिः सम्भवति~। उपजीव्यविरोधात् इति~। 

				अत्रास्वरसबीजम्- व्यक्तेः व्यावृत्यर्थम् अनुगमार्थमेव जातेः शक्तिकल्पनम् इतुक्तम्, लाघवादिविनिगमकस्य जातौ शक्तिकल्पकस्य विद्यमानत्वात्~। किञ्च व्यक्तेः अन्यलभ्यत्वे प्रथमक्लृप्तापि व्यक्तिशक्तिधीस्त्याज्यैव~। व्यक्तेः अन्यलभ्यत्वे तु तत एव व्यक्तिशक्तिसिद्धौ उपजीव्योपजीवकभावकल्पनं व्यर्थमेवेति~। 

			\subsubsection{व्यक्तिशक्तिवादे शक्त्यैक्योपपादनम्}

				विशिष्टानाम् आनन्त्येऽपि विशेषणीभूतस्य अवच्छेदकधर्मस्य ऐक्ये अवच्छेद्यैक्यं सम्भवति~। यथा विशेषणीभूतधूमव्यक्तीनाम् आनन्त्येऽपि धूमत्वेन कस्यचिद् एकधूमव्यक्तौ वह्निव्याप्तिर्गृह्यते 'धूमो वह्निव्याप्य' इति~। यथा गुरुमते कार्याणाम् आनन्त्येऽपि 'कार्यं शक्यम्' इति लिङः क्वचित् एकापूर्वे कार्यत्वावच्छिन्ने शक्तिः गृह्यते, अवच्छेदकस्य कार्यत्वस्यैक्यात् शक्तेरैक्यम्~। यथा मीमांसकमतेऽपि पश्वादिपदानां लोमवल्लाङ्गूलयोगित्वलक्षणोपाधिमादाय क्वचित् पशु व्यक्तौ शक्तिग्रहः~। शक्यानां व्यक्तीनाम् आनन्त्येऽपि तत्तावच्छेदकस्य ऐक्यात् शक्तेरैक्यम्~। तथा गोपदस्य शक्यानां गोव्यक्तीनामानन्त्येऽपि शक्यतावच्छेदकगोत्वस्य ऐक्यात्, शक्तेरैक्यं सम्भवति~। [गोत्वेन रूपेण क्वचिद्गवि शक्तिर्गृह्यते 'गौःशक्या' इति] 

				अथवा गोत्वसामान्यलक्षणया उपस्थितासु सकलगोव्यक्तिषु शक्तिर्गृह्यते 'गौः शक्या' इति~। न चैवं प्रमेयत्वेन रूपेण सकलप्रमेयज्ञानसम्भवात् सार्वज्ञ्यापत्तिः इति वाच्यम्~। इष्टत्वात्~। घटत्वादियावत्  धर्मप्रकारेण ज्ञानं नेष्यते~।

		\subsection{जातिशक्तिवादे दूषणस्य स्थिरीकरणम्}

			\begin{small}

				तर्हि जातेः कारणत्वापेक्षया शक्तिज्ञाने उपस्थितव्यक्तेरवच्छेदकत्वमात्रकल्पनैव लघीयसी, जातेः प्रमाणान्तरत्वापातश्च~। अत एव व्रीहीनवहन्तीत्यत्र व्यक्तौ न लक्षणा~। तत्साध्योपस्थितेर्जातिशक्तित एव सिद्धेः~। अस्तु वा गवावच्छेदकत्वेनारुण्यादिवद्व्रीहित्वेऽप्यवघातान्वय इति~। 
			\end{small}

			\subsubsection{श्रीकरमतम्}

				इदन्तु न शोभनम्~। तथाहि जातिशक्त्यैव व्यक्तिधीसम्भवे व्यक्तिधियं प्रति जातेः कारणत्वस्य अक्लृप्तस्य कल्पनायां गौरवम्~। व्यक्तिलाभाय गौः गच्छति इत्यादौ जातिशक्त्यतिरिक्तप्रमाणाश्रयणे 'व्रीहीनवहन्ति' इत्यत्रापि लक्षणाश्रयणप्रसङ्गः~।

			\subsubsection{व्यक्तेः तुल्यवित्तिवेद्यत्वमिति पक्षस्य विमर्शः} 

				न च जातिव्यक्त्योः एकवित्तिवेद्यत्वात् न आक्षिप्य आक्षेपकभावः सम्भवति~। जातिः हि सामान्यधर्मः। सामान्यञ्च समानानां भावः~। स च व्यक्तिं विना न भासते~। अतस्तयोः तुल्यवित्तिवेद्यत्वमेवेति वाच्यम्~।

				जातेः स्वरूपत एव शक्त्योपगमात्~। तथा च व्यक्तिघटितसामान्यत्वेन रूपेण जातेः अभानात् व्यक्तेः न तुल्यवित्तिवेद्यत्वम्~। जातेः स्वरूपतो भानानङ्गीकारे निर्विकल्पेऽपि सा न भासेत~। 

				न च व्यक्तिघटितेन सामान्यत्वेन रूपेण जातेः वाच्यत्वमावश्यकम्~। तथा सत्येव व्यक्तितो भिन्नतया जातेर्लाभेन व्यक्त्याक्षेपकत्वसम्भवात्~। व्यावृत्तत्वबुद्धिमन्तरा व्यक्तिविशेषाक्षेपकत्वासम्भवात्~। तथा जातिवित्तिवेद्यैव व्यक्तिर्नाक्षेपलभ्येति वाच्यम्~। 

				स्वतो व्यावृत्तेरेव वाच्यत्वात्~। व्यक्तेः तद्घटितसामान्यत्वस्य वा व्यावर्तकत्वे अन्योन्याश्रयाद्~, अनवस्थानाद्वा~।

				न च व्यक्तेः आक्षेपापेक्षया जातिवित्तिवेद्यत्वमेव उचितम्~। तथा हि गोपदात् गोत्वधीः ततः क्रमेण व्याप्तिज्ञानं पक्षधर्मताज्ञानं ततो व्यक्त्यनुमितिः इति ज्ञानपरम्पराकल्पनायां गौरवम्~। 'गौः' इति प्रतीतौ जातिव्यक्त्योः युगपत्भानोपगमे लाघवम्~। तत्र सूक्ष्मकालभेदेन जातिव्यक्त्योः भानम्~, न तु युगपत् इति तु नाशङ्कनीयम्~। युगपत्भानोपगमे बाधकाभावत्~। इति वाच्यम्~।

				शब्दस्य बोधकतायाः शक्त्यधीनत्वात्~। लाघवात् जातेरेव पदशक्यतायाः सिद्धौ व्यक्तिलाभाय उक्तज्ञानपरम्परानुसरणस्य आवश्यकत्वात्~। अन्यथा (गवादिपदात् व्यक्तेः आक्षेपेण लाभो यदि नानुमन्यते तदा) 'चैत्रःपचति' इत्यादौ आख्यातादपि कर्तुः आक्षेपेन लाभो न स्यात्~। तत्रापि कृतिकर्त्रोः तुल्यवित्तिवद्यत्वसम्भवात्~। 

				न च गोशब्दजन्यप्रतीतिः व्यक्तिविषयिणी जातिज्ञानत्वात् 'इयं गौः' इति प्रतीतिवत् इत्यनुमानेन व्यक्तेः तुल्यवित्तिवेद्यत्वं सिध्यतीति वाच्यम्~। अप्रयोजकत्वात्~। जात्यवगाहिनः प्रत्यक्षप्रतीतेः व्यक्त्यवगाहितायाः व्यक्तिधीहेतुसमाजाधीनत्वात् जातिधीहेतुसमाजाधीनत्वभावात्~। व्यक्तिधीहेतोः संयोगादिसन्निकर्षस्य जातिधीहेतोः संयुक्तसमवायादिसन्निकर्षस्य च परस्परभेदात्~। गोत्वं गवाविषयप्रतीतिविषयः जातित्वात् गोभिन्नभावत्वाद्वा इत्यनुमानेन व्यक्त्यनवगाहिन्याः जात्यवगाहिप्रतीतेः सिद्धिसम्भवाच्च~। 

				न च व्यक्तिः जातिवित्तिवेद्या जातिपरतन्त्रकत्वात् यत्परतन्त्रकत्वं यत्र तत्र तेनैकवित्तिवेद्यत्वं यथा ज्ञानपरतन्त्रकत्वं विषये तत्र ज्ञानेनैकवित्तिवेद्यत्वम् इत्यनुमानसम्भवात् व्यक्तेः जत्या तुल्यवि- त्तिवेद्यत्वसिद्धिः निष्प्रत्यूहैवेति वाच्यम्~।

				परतन्त्रत्वस्य अत्यूनानतिप्रसक्तस्य दुर्निर्वचतया तद्घटितव्याप्त्यसिद्धेः~। तथा हि परतन्त्रत्वं तावत् न परसमवेतत्वम्, गन्धपरतन्त्रकत्ववत्यां पृथिव्यां गन्धेन एकवित्तिवेद्यत्वविरहेण व्यभिचारात्~। नापि परधीनिरुप्यत्वम्~, ज्ञानस्य परधीनिरुप्यत्वस्य असिध्या दृष्टान्तासिद्धेः~। नापि परस्मिन् भासमाने एव भासमानत्वम्~। तस्य साध्याविलक्षणतया स्वरूपासिद्धेः~। नापि विशेषणत्वेनैव भासमानत्वम्~। 'गौः' इति प्रतीतौ जातेः विशेषणतया भासमानत्वेऽपि 'गवि गोत्वम्' इति प्रतीतौ विशेष्यतया भासमानत्वेन स्वरूपासिद्धेः~। न च स्वनिष्ठविशेष्यताऽनिरूपकत्वे सति स्वनिष्ठविशेषणतानिरूपकं यज्ज्ञानं तद्विषयत्वमिति विवक्षणे नोक्तदोषः, 'गौः' इति प्रतीतिमात्रस्य तथात्वेन आदातव्यत्वात् इति वाच्यम्, जातौ विशेषणत्वस्य दुर्घटतया स्वरूपासिद्धितादवस्थ्यात्~। तथाहि - पदात् जायमानं जातिस्मरणं निर्विकल्पकमेव~। तत्र जातेः स्वरूपत एव भानात् न विशेषणत्वम्~। जातिविशिष्टव्यक्तिविषयकसंस्कारादेव गोपदेन जात्यंशोद्बोधे सति जायमानं स्मरणं जातिमात्रावगाही भवति~। तथा च जातिविशिष्टविषयकसंस्कारः जातिमात्रावगाहि स्मरणं हेतुरङ्गीक्रियते~। निर्विकल्पकस्य संस्काराजनकत्वेन जातिमात्रावगाहिसंस्कारस्य दौर्लभ्यात्~। अथवा निर्विकल्पकस्यापि संस्कारजनकत्वमुपगम्य जातिमात्रावगाहिसंस्कारो जातिमात्रावगाहिस्मरणहेतुराङ्गीकर्तव्यः~। न च स्मरणं विशिष्टज्ञानात्मकमेव न तु निर्विकल्पकमिति वाच्यम्~। तथा सति विशिष्टज्ञानात्मकस्मरणं प्रति हेतुभूतस्य अनुभवस्यापि विशिष्टज्ञानात्मकतायाः आवश्यकतया निष्प्रकारकज्ञानात्मकस्य निर्विकल्पस्य विलोपप्रसङ्गात्~। व्यक्तेः तुल्यवित्तिवेद्यत्वाभ्युपगमपक्षेऽपि शब्दजन्यस्य गोत्वादिस्मरणस्य निर्विकल्पत्वम् अवश्यम् अभ्युपगन्तव्यम्, गोव्यक्तिवृत्तित्वादिवैशिष्ट्यस्य अशक्यत्वेन गोत्वे विशेषणतया भानासम्भवात्~। न च निर्विकल्पकं न स्मृतिः इन्द्रियजन्यत्वात् यथा सविकल्पकम् इत्यनुमानसम्भवात् निर्विकल्पस्य स्मरणरूपत्वं न सिध्यतीति वाच्यम्~। स्मृतित्वं निर्विकल्पकवृत्ति ज्ञानत्वसाक्षात्व्याप्यधर्मत्वात् यथा अनुभवत्वम् इत्यनुमानेन निर्विकल्पकस्य स्मरणरूपस्य सिद्धेः~।

			\subsubsection{आक्षेपात् व्यक्तिलाभपक्षस्य निराकरणम्}

				आक्षेपात् अर्थापत्त्या वा व्यक्तिलाभोपगमे 'गामानय' इत्यादिवाक्यात् क्रियान्वयबोधो नोपपद्यते~। तत्र हि गोत्वविशिष्टस्य आनयनक्रिया बोध्यते~। तत्र च बोधे गोविशेष्यकगोत्वप्रकारकः 'गौः' इत्याकारको बोधः कारणम्~। स च गोशब्दादेव उपपादनीयः न तु आक्षेपात्~। आक्षेपातद् 'गोत्वं व्यक्त्याश्रितम्' इत्याकारकस्यैव बोधस्य जननात्~,  न चास्य बोधस्य क्रियान्वयबोधहेतुत्वमस्तीति वाच्यम्~। 'गामानय' इत्यत्र द्वितीयाप्रकृत्युपस्थापितस्यैव आनयनक्रियान्वये साकाङ्क्षत्वात्~। व्यक्तेः अतथात्वेन क्रियान्वये आकाङ्क्षा विरहात्~। यथा राजपुरुषमानय इत्यत्र  द्वितीयाप्रकृत्युपस्थापितस्य पुरुषस्यैव आननयनादिक्रियान्वये साकाङ्क्षता न तु पुरुषसम्बन्धितया उपस्थितस्य राज्ञः~। तथा प्रकृते गोपदात् उपस्थितस्य गोत्वस्यैव क्रियान्वयः स्यात् न तु गोत्वाश्रयतया उपस्थितयाः व्यक्तेः~।

				न च राज्ञः आनयनान्वयस्य बाधितत्वात् न तथा अन्वयबोधः~।  प्रकृते व्यक्तेस्तु आनयनादिना अन्वयस्य बाधनिश्चयाभावात्  भवति अन्वयबोधः  इति वाच्यम्~।  तथा सति ’गामानय’ इत्यादिवाक्यात् ’व्यक्तिकर्मकानयनम्’ इति बोधोदयप्रसङ्गात्~।  न चेष्टापत्तिः~।  तत्र ’गोकर्मकानयनम्’ इत्येव बोधोदयात्~। 

				किञ्च अर्थापत्त्या व्यक्तिलाभो दुर्घटः~।  तथा हि व्यक्तिं विनापि गोत्वजातेः तद्बुद्धेश्च सिद्धत्वात् व्यक्तिमन्तरा जातेरनुपपद्यमानत्वमेव नास्ति~।  न च व्यापकं विना व्याप्यं न सिध्यति~।  व्यक्तेः व्यापिकायाः विरहे जातेः व्याप्यभूतायाः असिद्धिरेव  इति वाच्यम्~।  जातिव्यक्त्योः  व्याप्य-व्यापकभावस्य दुर्घटत्वात्~।  यत्र यत्र गोत्वं तत्र तदा व्यक्तिः इति नियमः न सम्भवति~, व्यक्तौ व्यभिचारात्~।  यत् सामान्यं स व्यक्तिः इत्यपि न सम्भवति~, जातौ व्यभिचारात्~।  हेतुदुर्भिक्षमपि दुर्वारम्~।  तथा हि गोत्वं गवाश्रितं गोत्वादिति हेतुप्रयोगे सिद्धसाधनम्, पक्षहेत्वोः ऎक्येन  व्याप्तिग्रहस्यैव  साध्यनिश्चयरूपत्वात्~।  गोत्वं गवाश्रितं जातित्वात् इति हेतूपन्यासे  जातित्वरूपहेतोः ज्ञानस्य ’गामानय’  इत्यादिवाक्यप्रयोगस्थले विरहेण व्याप्तिग्रहानुपपत्तिः~।  न हि तत्र जातित्वं पदादुपस्थितम्~।  पदादेव जातित्वलाभोपगमे  तु तद्वित्तिवेद्या एव व्यक्तिः न आक्षेपात् अर्थापत्त्या वा लभ्येति युज्यते~।  जातिव्यक्त्योः कथञ्चित् व्याप्यव्यापकभावोपगमेऽपि अर्थापत्तेः दौर्घट्यम्~।  यतो हि लिङ्गं व्याप्यमनुपपन्नं स्वाश्रये व्यापकमुपपादकञ्च बोधयति इत्यर्थापत्तिस्थले दृष्टम्~।  यथा दिवाऽभुञ्जानत्वे सति पीनत्वं लिङ्गं व्याप्यमनुपपन्नं स्वाश्रये देवदत्ते व्यापकमुपपादकं रात्रिभोजनं कल्पयति~।  इह तु व्याप्यस्य गोत्वस्य आश्रये गवि न व्यक्तिबुद्धिः उदेति~। 
	
				किञ्च~, व्यक्तेः आक्षेपलभ्यत्वे अपदार्थत्वेन विभक्त्यर्थसंख्याकर्मत्वादीनाम् अन्वयानुपपत्तिः दुर्वारा~।  तथा हि –सुब्विभक्तीनां स्वप्रकृत्यर्थान्वितस्वार्थबोधकत्वमिति व्युत्पत्तिः~।  ’गामानय” इत्यत्र अम् प्रत्ययार्थकर्मत्व-एकत्वसङ्ख्ययोः गोपदार्थेन अन्वयो भवेत्~।  व्यक्तेः गोपदार्थत्वाभावे व्यक्त्या साकं कर्मत्व-एकत्वयोः अन्वयो न स्यात्~।  न च सुपां स्वप्रकृतितात्पर्यविषयान्वितस्वार्थबोधकत्वमिति परिष्करणात् नोक्तानुपपत्तिः~।  ’गामानय’ इत्यत्र व्यक्तेः गोपदार्थत्वाभावेऽपि गोपदतात्पर्यविषयत्वाक्षतेः इति वाच्यम्~। अपदार्थेऽपि पदार्थान्तरान्वयोपगमे ’शाब्दी ह्याकाङ्क्षा शब्देनैव प्रपूर्यते’ इति व्युत्पत्तिविरोधः~।  ’गङ्गायां घोषः” इत्यादौ गङ्गादिपदानां लक्षणास्वीकारवैयर्थ्यञ्च~। 

				एवं व्यक्तेः अपदार्थत्वे तत्र आख्यातार्थसङ्ख्यायाः अन्वयानुपपत्तिरपि दुर्वारा~।  तथा हि आख्यातार्थसङ्ख्या भावनान्वयिना प्रथमान्तादुपस्थितेन अन्वेति इति नियमः~।  यथा ’चैत्रः पचति’ इत्यत्र एकत्वं कृत्यन्वयिना चैत्रेण साकमन्वेति~।  ”गावस्तिष्ठन्ति” इत्यत्र गोव्यक्तेः अपदार्थत्वे तस्याः आख्यातार्थकृत्यन्वयित्वाभावेन च आख्यातार्थबहुत्वान्वयः न स्यात्~। 

			\subsubsection{लक्षणयाव्यक्तिलाभपक्षस्य विमर्शः}
			
				अत्र विमृश्यते- पदानां स्वशक्य-शक्यतावच्छेदकातिरिक्तरूपेणैव उपस्थिते अर्थे लक्षणा भवति~।  यथा ”गङ्गायां घोषः” इत्यत्र गङ्गापदस्य तीरत्वेनोपस्थिते तीरे लक्षणा~।  तदुक्तं ”स्वार्थादन्येन रूपेण  ज्ञाते भवति लक्षणा” इति~।  गोपदात्तु गोव्यक्तेरुपस्थितिः गोत्वेनैव भवति~।  तच्च गोपदस्य स्वार्थ एव~। ततोऽन्येन  रूपेण व्यक्तेरुपस्थितिविरहात्~।

				लक्षणा भवितुं नार्हति~। न च व्यक्तित्वेन सास्नादिमत्वेन वा स्वार्थादन्येन लक्षणा भवतीति वाच्यम्~।  तेनरूपेण शाब्दबोधप्रसङ्गात्~।  न चेष्टापत्तिः, गोत्वेनैव व्यक्तिबोधस्य आनुभविकत्वात्~।  न च स्वार्थादन्येनेत्यादिनियमो अयुक्तः~, विशिष्टवाचकानां धेन्वादिपदानां विशेष्ये लक्षणायाः असङ्ग्रहप्रसङ्गात्~।  तथा च गोत्ववाचकस्य  गोपदस्य गोत्वविशिष्टे लक्षणा भवितुमर्हति~।  गुणवाचकस्य नीलपदस्य नीलविशिष्टे लक्षणावत् इति वाच्यम्~। 

				एकत्र युगपद्वृत्तिद्वयस्य विरुद्धत्वात्~।  गोत्वे शक्तस्य गोपदस्य गोत्वान्तर्भावेण लक्षणाया अभ्युपगमासम्भवात्~।  व्यक्तिमात्रे लक्षणोपगमे तु  आनयनादिक्रियान्वयिन्यां व्यक्तौ गोत्वस्य अवच्छेदकतया साक्षाद् क्रियान्वयाभावेन मुख्यत्वहानिप्रसङ्गः~। 

				किञ्च गोशब्दस्य जातिमात्रे शक्तिः~, व्यक्तौ लक्षणा इति न सम्भवति~।  शक्तिलक्षणयोः प्रयोगमूलकत्वात्~।  जातौ गोशब्दस्य मुख्यप्रयोगाभावात्~।  न हि ”गामानय”~,” गौरुत्पन्ना ” इत्यादौ गोशब्दः\footnote{मणिकृतस्तु प्रयोगहेतुभूतार्थतत्त्वज्ञानजन्यः प्रमाणशब्दः इत्याहुः [त. चिं.शब्दखण्डः], आप्तोपदेशः शब्दः इति गौतमः [न्या.सू.], आप्तवाक्यं शब्दः इति अन्नंभट्टः [त. सं]} जातौ प्रयुज्यते~।  अपि तु व्यक्तावेव~।  अत एव ’गां पश्य’, ’गौरस्ति’ इत्यादौ अपि व्यक्तिपरः गोशब्दः उपगम्यते~, न तु जातिपरः~।  लक्षणा च मुख्यार्थान्वयानुपपत्त्या ततोऽन्यत्रार्थे शब्दप्रयोगात् सिद्ध्यति~।  व्यक्तेः मुख्यार्थादन्यत्वम् असिद्धम्~।  तस्मात् जातिविशिष्टव्यक्तिः शक्या~।  न च तर्हि अस्तु गोत्वे गवि च शक्तिः इति वाच्यम्~।  तथासति तृतीयायाः करणत्व-एकत्वयोः पृथक् शक्तिरिव गोगोत्वयोः पृथक्शक्त्युपगमात् गोपदात् गोत्वं, गोव्यक्तिः इति बोधापत्तिः~।  गौरितिबोधानुपपत्तिश्च~।  व्यक्तिजात्योः वैशिष्ट्यं च समवायाख्यसम्बन्धः~।


\chapter{द्वितीयोऽध्यायः पङ्कजपदस्य योगरूढिविमर्शः}

	\section{पङ्कजपदस्य रूढिं विना पद्मबोधकत्वमिति पूर्वपक्षः}

		\subsection{पङ्कजपदस्य रूढिनिराकरणे युक्तिप्रदर्शनम्} 

			\begin{small}
				
				अत्र मीमांसकाः – न तावत् स्मृत्यर्थं शक्तिः~। पङ्कजपद\footnote{शक्तं पदम्~। तच्चतुर्विधम्~। क्वचिद्यौगिकम् , क्वचिद्रूढम् , क्वचिद्योगरूढम् , क्वचिद्यौगिकरूढम्~। तथा हि यत्र अवयवार्थ एव बुध्यते तद्यौगिकम् , यथा पाचकादिपदम्~। यत्रावयवशक्तिनिरपेक्षया समुदायशक्त्या बुध्यते तद्रूढम्~। यथा गोमण्डलादिपदम्~।  यत्र तु अवयवशक्तिविषये समुदायशक्तिरप्यस्ति तद्योगरूढम् , यथा पङ्कजादिपदम्~। तथाहि पङ्कजपदम् अव्यवशक्त्या पङ्कजनिकर्तृरूपमर्थं बोधयति, समुदायशक्त्या च पद्मत्वेन रूपेण पद्मं बोधयति~। न च केवलया अवयवशक्त्या कुमुदे प्रयोगः स्यादिति वाच्यम्~। रूढिज्ञानस्य केवलयौगिकार्थबुद्धौ  प्रतिबन्धकत्वादिति प्राञ्चः~। वस्तुतस्तु समुदायशक्त्युपस्थितपद्मे अवयवार्थपङ्कजनिकर्तुरन्वयो भवति सान्निध्यात् ॥ यत्र तु रूढ्यर्थस्य बाधः प्रतिसन्धीयते तत्र लक्षणया कुमुदादेर्बोधः~। यत्र तु कुमुदत्वेन  रूपेण बोधे न तात्पर्यज्ञानं पद्मत्वस्य च बाधः तत्र अवयवशक्तिमात्रेण निर्वाह इत्यप्याहुः ॥ यत्र तु स्थलपद्मादौ अवयवार्थबाधः तत्र समुदायशक्त्या पद्मत्वेन रूपेण बोधः~। यदि तु स्थलपद्मं विजातीयमेव तदा लक्षणया एवेति ॥ यत्र तु यौगिकार्थरूढ्यर्थयोः स्वातन्त्र्येण बोधः तद्यौगिकरूढम् , यथा उद्भिदादिपदम्~। तत्र हि उद्भेदनकर्ता तरुगुल्मादिर्बुध्यते यागविशेषोऽपि इति ॥ न्या-सि-मु., शब्दखण्डः।}प्रयोगविषये नियतपद्मानुभवजनितसंस्कारात् स्मृतेरेवोपपत्तेः, स्मृतेस्तज्जन्यत्वनियमात्~। नाप्यनुभवार्थं, नियमतः स्मृतं पद्ममादाय व्यक्तिवचनन्यायेनावयवैः पङ्कजनिकर्तृ पद्ममित्यनुभवसम्भवात्~। स्मृतिश्च रूढ्या अन्यथा वेति न कश्चिद्विशेषः~। शक्तिं विना नियमतः प्रयोग एव कुत इति चेन्न~। पूर्वप्रयोगमपेक्ष्य अवयवानामुक्तन्यायेन पद्मानुभवजनकत्वनियमात्~। पूर्वप्रयोगोऽपि तत्पूर्वप्रयोगमपेक्ष्येत्यनादितैव~। अथानियतोद्बोधस्य संस्कारस्य शक्तिं विना नियतोद्बोधे हेत्वभावात् नियता स्मृतिरेव न स्यादिति चेन्न~। कदाचिच्छक्तितोऽपि उद्बोधाभावेन शक्यास्मरणात् शक्तिं विनापि नियमतः शक्तिस्मरणाच्च~। उद्बोधकञ्च न नियतं, सदृशपदशक्तिसम्बन्धिज्ञानानां प्रत्येकं व्यभिचारात्~। किन्तु स्मृतिः तत्र तत्कालोत्पन्नमनियतमेवोद्बोधकं कल्प्यते फलबलात्~। कार्योन्नेयधर्माणां यथाकार्यमुन्नयनात्~। 
			\end{small}

			मीमांसकास्तावत् पङ्कजपदस्य पद्मत्वावच्छिन्ने रूडिं नाभ्युपयन्ति~।  अयं तेषामाशयः –पङ्कजपदात् पद्मस्य स्मरणं शाब्दबोधश्च तत्ररूढिं विनापि उपपादयितुं शक्यते~।  तथा हि – पद्मस्मरणं प्रति हेतुत्वेन  क्लॄप्तस्य पद्मानुभवजन्यसंस्कारस्य पङ्कजपदप्रयोगस्थले नैयात्योपगमादेव पद्मस्मरणसम्भवेन न तदर्थे पद्मे शक्तिकल्पनमावश्यकम्~।  पङ्कजपदप्रयोगस्थले नियमेन पद्मस्मरणस्य सिद्धौ~, स्मृतेपद्मे योगेनोपस्थितस्य पङ्कजनिकर्तुः नियमेनान्वयोऽपि उपपद्यते~।  व्यक्तिवचनानां सन्निहितविशेषपरकत्वात्~।  

			न च तदर्थबोधतात्पर्येण नियमतः तत्पदप्रयोगं प्राति तदर्थे तत्पदस्य शक्तिः प्रयोजिका~। तथा च पद्मे पङ्कजपदस्य शक्तिविरहे पद्मबोधतात्पर्येण नियमतः पङ्कजपदप्रयोगो न स्यात् इति वाच्यम्~।  पङ्कजपदस्य पद्मानुभवजनकत्वनिश्चयादेव तस्य नियमतः पद्मबोधतात्पर्येण प्रयोगः~।  तत्पदस्य तदर्थबोधजनकत्वनिश्चय एव तत्पदस्य नियमतः तदर्थबोधतात्पर्येण प्रयोगं प्रति प्रयोजकः इत्यभ्युपगमात्~।

			पङ्कजपदस्य पद्मानुभवजनकत्वनिश्चयश्च पूर्वप्रयोगदर्शनादेव जायते~।  तत्र च पूर्वप्रयोगे प्रयोजकः  तदर्थबोधकत्वनिश्चयः तत्पूर्वतनप्रयोगदर्शनाद् जायते इत्यनादिता एव~। 

			न च पङ्कजपदस्य रूडिविरहे पङ्कजपदात् पद्मस्मरणस्य नैयत्यं न स्यात्~।  पद्मानुभवजनितसंस्कारस्य नियमेन सत्वेऽपि तत्सहकारिणः उद्बोधकस्य सत्वनियमाभावात्~।  उद्बोधकसहकृतस्यैव संस्कारस्य फलोपधायकत्वात् इति वाच्यम्~। 

			पद्मे पङ्कजपदस्य शक्त्युपगमेनापि नियतपद्मस्मरणस्योपपादयितुमशक्यत्वात्~।  तथा हि शक्तिग्रहेऽपि कदाचित् शक्यार्थस्मरणं नोदेति~।  शक्त्यग्रहेऽपि कदाचित् शक्यार्थस्मरणमुदेति~।  अतः शक्तिग्रहः अर्थस्मरणे न नियतोद्बोधकः~।  अपि तु अनियत एव सादृश्यज्ञानसम्बन्धिज्ञानादिवत्~।  एतेषाञ्च  शक्तिज्ञान-सादृश्यज्ञान-सम्बन्धिज्ञानादीनां संस्कारोद्बोधं प्रति न हेतुत्वं घटते~।  प्रत्येकं व्यभिचारात्, तस्मात् यत्र स्मरणमुत्पद्यते तत्र पूर्वकालोत्पन्नं किञ्चित् अनियतमपि उद्बोधकं कल्प्यते, फलबलात्~।  यथा कार्यं कारणस्य उन्नेयत्वात्~।  तथा च यत्र पद्मस्मरणमुत्पद्यते तत्र पूर्वकालोत्पन्नं किञ्चित् अनियतमपि पद्मसंस्कारस्य उद्बोधकं कल्पनीयमिति पद्मे शक्तिग्रहोपगमो व्यर्थः~। 

		\subsection{रूढिनिराकरणे प्रसक्तदोषाणां परिहारः}

			\begin{small}
				
				स्यादेतत्~। पद्मे नियतप्रयोगरूपसम्बन्धेन पङ्कजपदादेव पद्मस्मृतिरस्तु~। एवञ्च पद्मे विभक्त्यसक्तर्थान्वयः शाब्दानुभवप्रवेशश्चोपपद्यते पङ्कजपदप्रतिपाद्यत्वात्~। न च प्रतिपाद्यता वृत्यैव तदुपयोगिनौ, पाचकमानयेत्यादौ वाक्योपस्थाप्ये तदभावात्~। न च पदाद्वृत्त्यैव स्मृतिः~। किन्तु सम्बन्धिज्ञानाद्वृत्तेरपि सम्बन्धत्वेन स्मृत्युपयोगात्~। अन्यथा पदाच्छक्तेः स्मृतिर्न स्यात्~। गवादिपदे त्वेवं स्मृतिसम्भवेऽप्यन्वयानुभवार्थं शक्तिरित्युक्तम्~। 
			\end{small}
			
			पङ्कजपदात् नियतप्रयोगसम्बन्धेन पद्मस्मृतिरुपेयते~।  तथा च पद्मं पङ्कजपदप्रतिपाद्यमेव~। अतः ”पङ्कजमानय” इत्यादौ पद्मस्य विभक्त्यर्थकर्मत्वादिना अन्वयः उपपद्यते~।  प्रत्ययानां स्वप्रकृत्यर्थान्वितस्वार्थबोधकत्वमिति व्युत्पत्तौ प्रकृत्यर्थत्वस्य प्रकृतिप्रतिपाद्यत्वरूपत्वात्~। 

			न च प्रकृत्यर्थत्वं वृत्त्या प्रकृतिप्रतिपाद्यत्वमेव~।  तच्च ’पङ्कजमानय’ इत्यत्र पद्मे अनुपपन्नमिति वाच्यम्~।  तथा सति ’पङ्कजमानय’ इत्यत्र न्यायमते पाककर्तुः  वृत्त्या पदप्रतिपाद्यत्वाभावेन  विभक्त्यर्थकर्मत्वेन अन्वयो न स्यात्~।  पाककर्तुः पाचकवाक्यप्रतिपाद्यत्वात्, वाक्ये च वृत्तिविरहात् ~। 

			न च तथापि पद्मस्य प्रकृतिप्रतिपाद्यत्वमनुपपन्नम्~।  तद्धि प्रकृतिजन्यस्मृतिविषयत्वमेव~।  तत्र प्रकृतिः पद्मं वाक्यञ्च~।  तत्र पदजन्या स्मृतिः वृत्त्यैवेति नियमः~।  पङ्कजपदस्य वृत्तिविरहे ततः स्मृतिः अनुपपन्नेति वाच्यम्~।  पदात् वृत्त्यैव स्मृतिः इति नियमे मानाभावात्~।  पदस्य शक्तौ वृत्तिविरहेऽपि ततः तत्स्मरणोदयेन व्यभिचारात्~।
		
			\subsubsection{अ}
			
				\begin{small}
				
					न च पद्मत्ववत् तद्व्यापकादेरपि स्मृतिप्रसङ्गः~। स्मृतिबलेनोद्बोधकल्पनमिति न तत्र स्मृत्यभावेन तदुद्बोधाभावात्~। तस्मात् शक्तिं विना शक्तेरिव पद्मत्वस्य नियता स्मृतिः~। 
				
					उच्यते~। नियतपद्मस्मृतेरुक्तसम्बन्धेन पङ्कजपदसाध्यत्वे पद्मवत्तद्व्याप्यव्यापकयोरपि नियमतः स्मृतिप्रसङ्गः~। अथ पद्मे प्रयोग एव पद्म एव प्रयोग इति नियतसम्बन्धेन पद्मेतरव्यावृत्तेन पद्मस्यैव पदात् स्मृतिः~। 
				
					तर्हि ज्ञानस्यास्य हेतुत्वे सकलप्रयोगादर्शिनो बहुधा पद्म एव गृहीतप्रयोगस्य समव्याप्तिज्ञानाभावान्न नियता पद्मस्मृतिः स्यात्~। स्वरूपसतस्तस्थात्वे वा अगृहीतपद्मप्रयोगस्यापि ततो नियतपद्मस्मृतिप्रसङ्गः~। 
	
					अथ यादृशः प्रयोगस्तथा शक्तिग्राहकत्वेनाभिमतः स एव पद्मस्मारकोऽस्तु~। न च प्रयोगसमव्याप्तत्वेन गृहीतस्य सौरभादेरुपाधेस्ततः स्मृतिप्रसङ्गः~। तवापि शक्तिग्रहप्रसङ्गात्~। तयोस्तुल्यत्वेऽपि शक्तिग्राहकप्रमाणे लाघवादितर्कावतारान्नोपाधौ शक्तिरिति यदि तदा तर्कसहकृत एव शक्तिग्राहकत्वाभिमतः सम्बन्धः पद्मस्मृतिहेतुरस्तु~। तर्काद्यनवतारेऽपि बहुधा गृहीतप्रयोगस्य पद्मस्मृतिदर्शनान्न तथेति चेत्~, तर्हि तवापि तर्कं विना शक्तिग्राहकाभावात् कथं पद्मस्मृतिरिति तुल्यम्~। न चैवं तत एव शक्तिग्रहोऽपि स्यात्~। अनन्यलभ्यस्यैव शब्दार्थत्वादिति~। 
				\end{small}
				
				न च विना शक्तिं पङ्कजपदात् पद्मत्वस्य स्मृतिसम्भवे पद्मत्वव्यापकस्यापि स्मृतिप्रसङ्गः, शक्तिग्रहरूपोद्बोधकविरहस्य उभयत्र सामान्यादिति वाच्यम्~।  पङ्कजपदात् पद्मत्वव्यापकस्य स्मृत्यनुदयेन तदुद्बोधकाभावः निश्चीयते~।  पद्मत्वस्य तु स्मृत्युदयेन तत्पूर्वकालिकं  किञ्चित् उद्बोधकं कल्प्यते~।  स्मृतिबलेन उद्बोधककल्पनात्~।  तथा च यथा न्यायमते पङ्कजपदात् शक्तेः स्मरणं तत्र शक्तिविरहेण अङ्गीक्रियते तथा पद्मस्य स्मरणं तत्र शक्तिविरहेऽपि भवितुमर्हति~।

				न च पङ्कजपदस्य वृत्तिं निराकृत्य नियतप्रयोगरूपसम्बन्धेन पद्मस्मृतेः उपपादने पद्मव्याप्यस्य पद्मव्यापकस्य च पञ्कजपदात् तेन सम्बन्धेन नियमतः स्मृतिप्रसङ्गः इति वाच्यम्~।  पङ्कजपदनियतप्रयोगस्य पद्मे सत्वेऽपि  पद्मव्याप्ये पद्मव्यापके च विरहात्~।  पङ्कजपदस्य पद्मे प्रयोग एव~, पद्मे एव प्रयोगः  इत्येव पङ्कजपदनियतप्रयोग इत्यनेन विवक्षितत्वात्~। 
	
				न च पङ्कजपदात् पद्मस्य स्मृतौ प्रयोगसमनैयत्यरूपसम्बन्धस्य ज्ञानं हेतुरित्युपगमे  पङ्कजपदस्य सकलप्रयोगान् अदृष्टवतः बहुधा पद्मे एव गृहीतप्रयोगस्य पुरुषस्य प्रयोगसमनैयत्यज्ञानाभावात् पङ्कजपदात् नियतपद्मस्मृतिः न स्यात्  इति वाच्यम्~। प्रयोगसमव्याप्यत्वसम्बन्धस्य स्वरूपसत एव  हेतुत्वोपगमेन उक्तदोषानवकाशात्~। 

				न चैवं सति येन पुरुषेण पङ्कजपदस्य पद्मे प्रयोगो न गृहीतः तस्यापि पङ्कजपदात् नियतपद्मस्मृतिप्रसङ्गः इति वाच्यम्~।  न्यायमते शक्तिग्राहको यः प्रयोगः तस्यैव अस्माभिः पद्मस्मारकतायाः स्वीकारात्~।  उक्तस्थले च शक्तिग्राहकप्रयोगस्य  विरहेण स्मरणस्य आपादयितुमशक्यत्वात् न दोषः~। 

				न च प्रयोगसमव्याप्यत्वस्य पद्मत्वावच्छिन्ने इव सौरभावच्छिन्नेऽपि सत्वात् ततोऽपि पद्मस्मृत्युत्पत्तिप्रसङ्ग इति वाच्यम्~।  न्यायमते ततः शक्तिग्रहोत्पत्तिप्रसङ्गस्य तुल्यत्वात्~। 

				न च पद्मत्व-सौरभयोः द्वयोः पङ्कजपदप्रयोगविषयतासमव्याप्यत्वस्य अविशिष्टत्वेऽपि लाघवादितर्कसहकृतस्यैव प्रयोगसमनैयत्यग्रहस्य शक्तिग्राहकत्वोपगमात् सौरभादिरूपोपाधेः पद्मत्वापेक्षया गुरुत्वेन न तत्र शक्तिः गृह्यते  इति वाच्यम्~।  तथा सति तर्कसहकृतस्यैव शक्तिग्राहकत्वेनाभिमतस्य प्रयोगसमव्याप्यत्वादिसम्बन्धस्य पद्मस्मृतिहेतुत्वोपगमेन, सौरभादौ च प्रयोगसमव्याप्यत्वस्य सत्वेऽपि लाघवादितर्कसहकारविरहेण पद्मस्मृतेः वारणसम्भवात्~।
				
				न च बहुधा पद्मे गृहीतपङ्कजपदप्रयोगस्य पुरुषस्य  लाघवादितर्कविरहिणोऽपि पद्मस्मृतिदर्शनात्, पद्मस्मृतिं प्रति प्रयोगसमव्याप्यत्वस्य हेतुतायां लाघवादितर्कस्य सहकारित्वे नाभ्युपगन्तुं शक्यमिति वाच्यम्~। तथा सति न्यायमते लाघवादितर्कविरहिणोऽपि पुरुषस्य  पङ्कजपदशक्तिग्रहस्य  ततः पद्मस्मरणस्य च उत्पाददर्शनात्, शक्तिग्राहकत्वेन अभिमतसम्बन्धस्यापि  लाघवादितर्कसहकारितायाः अभ्युपगन्तुमशक्यत्वात्~।

			\subsubsection{इ}

				\begin{small}

					न चैवं गवादिपदेऽपि न शक्तिः स्यात्~। व्यवहारकालीनसंस्कारादेव गवादिस्मृतिसम्भवादिति वाच्यम्~। न हि तत्र स्मृत्यर्थं शक्तिः~। किन्तु अनुभवार्थं पदादन्यतो गवादेरनुभवासम्भवात् अव्युत्पन्नस्य ततोनुभवासम्भवाच्च~। पद्मानुभवश्च भोगादेवेति न समुदायो हेतुरन्यायसिद्धत्वात्~। अतो नानुभवबलात् समुदाये शक्तिकल्पनम्~। नन्वेवं गवादिपदानां प्रमेयत्वे शक्तिरस्तु~। गवादिस्मृतिः संस्कारादितिचेन्न~। गोव्यवहारेण स्वोपपादके गोज्ञाने पदस्य शक्तिकल्पनं, न त्वनुपपादके प्रमेयत्वेन गोज्ञाने गोपदात् प्रमेयो गौरित्यननुभवाच्च~। 
				\end{small}
					
				न च पङ्कजपदस्य पद्मे शक्तिविरहेऽपि तत्स्मरणोपगमे गोपदस्यापि गवि शक्तिं विना गोस्मरणाभ्युपगम्अप्रसङ्गः~। गवानुभवजनितसंस्कारादेव कुतश्चित् उद्बुद्धात् गोविषयकस्मरणस्य उत्पादसंभवात् इति वाच्यम्~। गोविषयकशब्दानुभवनिर्वाहाय गोशब्दस्य गवि शक्तिस्वीकारस्य आवश्यकत्वात्~।  गोपदशक्तिग्रहरहितस्य पुरुषस्य ततः गोविषयकशाब्दबोधानुदयेन तयोः शक्तिग्रहशाब्दबोधयोः कार्यकारणभावस्य स्वीकार्यत्वात्~। पङ्कजपदात् पद्मविषयकशाब्दबोधस्तु  अवयवशक्तिग्रहादौ असम्भवः~।  अतो न समुदायशक्तिग्रहस्य कारणत्वं कल्प्यते~, अन्यथासिद्धत्वात्~। 
	
				न चैवमपि गोपदस्य प्रमेयत्वे शक्तिप्रसङ्गः~।  गोपदाधीनशाब्दबोधस्य गोविषयकतायाः गोविषयकस्मरणेनैव उपपद्यमानत्वाच्च~।  तस्य च गवानुभवजनितसंस्कारादेव उपस्थितत्वाच्च~।  शब्दानुभवशक्तिग्रहयोस्तु समानविषयकत्वेन नास्ति कार्यकारणभाव इति वाच्यम्~। गविगोव्यवहारस्य उपपादकतया गवि एव गोशब्दस्य शक्तेः कल्पनात्~।  प्रमेयमात्रे गोव्यवहारविरहेण प्रमेयमात्रे गोशब्दस्य  शक्तिकल्पनासम्भवात्~।  गोशब्दाद् प्रमेयत्वेन बोधस्य अननुभवाच्च~। 
	
			\subsubsection{उ}
	
				\begin{small}
				
					अथैवं संस्कारादेव तीरादिस्मृतिसम्भवे गौणलाक्षणिकोच्छेदः~। तीराद्यनुभवार्थं हि न तत्कल्पनं तदनुभवस्येतरपदादेव सिद्धेः तयोरननुभावकत्वात्~। तस्मान्नियता स्मृतिः वृत्तिसाध्येति तयोः कल्पनात्~। तथा च नियतपद्मस्मृत्यर्थं पङ्कजपदेऽपि वृत्तित्वेन शक्तिकमावश्यकं लक्षणाद्यभावादिति चेन्न, गङ्गायामित्यादौ वृत्तिं विना तीरादेरपदार्थत्वे विभक्त्यर्थान्वयस्तत्र न स्यात्~। विभक्तीनां प्रकृत्यर्थगतस्वार्थान्वयबोधकत्वव्युत्पत्तेः~। पद्मस्य तु पङ्कजवाक्यप्रतिपाद्यत्वेन पाचकादेरिव विभक्त्यर्थान्वयोपपत्तिः~।
				\end{small}
			
				न च पङ्कजपदप्रयोगस्थले  वृत्तिं विना संस्कारादेव पद्मस्मरणोपगमे लाक्षणिकगङ्गादिपदप्रयोगस्थलेऽपि संस्कारादेव तीरादिस्मरणोपपादनसम्भवेन लक्षणाया उच्छेदप्रसङ्गः~।  लक्ष्यार्थशाब्दबोधं प्रति समभिव्याहृतशक्तपदान्तरस्यैव हेतुतया तीरादिविषयकशाब्दानुभावोपपादकतयापि गङ्गादिपदे लक्षणानुसरणसार्थकतायाः दुर्घटत्वात् इति वाच्यम्~। ’गङ्गायां घोष’ इत्यादौ तीरादेः विभक्त्यर्थवृत्तित्वादिना अन्वयनिर्वाहात् तीरे लक्षणायाः उपगन्तव्यत्वात्~।  अन्यथा तीरस्य अपदार्थत्वे पदविभक्त्यर्थान्वयानुपपत्तेः~।  प्रत्ययानां स्वप्रकृत्यर्थान्वितस्वार्थबोधकत्वम् इति व्युत्पत्तेः~।  ’पङ्कजमानय’ इत्यत्रतु पङ्कजवाक्यस्य रूडिविरहेऽपि योगेन लब्धं पद्मं विभक्त्यर्थेन कर्मत्वेन अन्वेति यथा ’पाचकमानय’ इत्यत्र~।
	
			\subsubsection{ऋ}
			
				\begin{small}
				
					यत्तु शब्दोपस्थित एव शाब्दान्वयबोधः~। अन्यथा प्रत्यक्षोपस्थिते कलाये पचतीत्यन्वयबोधः स्यात्~। तन्न~। पदस्य तत्र तात्पर्याग्रहात्~। तद्ग्रहे भवत्येव किं पचसीत्युक्ते वधूपदर्शितकलायादौ~। अन्यथा दैवात् श्रुतस्मृतकलायपदात् कुतो नान्वयबोधः~। द्वारमित्यादावपि न पिधेहिपदाध्याहारः~। किन्तु तदर्थस्यैव लाघवात्~। न च पद्ममानयेत्यादौ शाब्दानुभवे पद्मोपस्थापकपदजन्यत्वादन्यत्रापि तथेति वाच्यम्~। शाब्दपद्मानुभवे हि तदन्वयबोधतात्पर्यकपदत्वेन कारणता न तु पद्मोपस्थापकपदत्वेनापि गौरवात्~। 
				\end{small}
			
				न च शब्दादुपस्थित एवार्थे शाब्दान्वयबोधनियमः~।  अन्यथा प्रत्यक्षेणोपस्थिते  कलाये ’पचति’ इति शब्दादुपस्थितपाकस्य अन्वयबोधापत्तिः~।  तथा च ’पङ्कजमानय’ इत्यत्र  शब्दादनुपस्थिते संस्कारेण स्मृते पद्मे द्वितीयार्थकर्मत्वस्य शाब्दान्वयबोधो दुरुपपाद इति वाच्यम्~। प्रत्यक्षादिना उपस्थितेऽपि अर्थे तात्पर्यग्रहसत्वे शाब्दान्वयबोधोदयेन उक्तनियमस्य अप्रामाणिकत्वात्~। अत एव ’द्वारम्” इत्यादौ अर्थस्यैव  अध्याहारः, न तु ’ पिधेहि’ इति पदस्य~।
			
			\subsubsection{ऌ}

				\begin{small}
				
					कुमुदे लक्षणा न स्यात् योगार्थस्याबाधादिति चेन्न~। वक्ष्यते हि तत्र तस्यासाधुत्वम्~। 
	
					नव्यास्तु नियमतः स्मृताद्मस्यान्वयानुपपत्त्यनन्तरमेव कुमुदधीरिति न तत्र लक्षणव्यवहारः वस्तुतो मुख्यतैव~। तवापि पद्मत्वस्यायोग्यतया अनन्वये योगादेव कुमुदधीर्नलक्षणयेति~। 
				\end{small}

				न च पङ्कजपदस्य पद्मे शक्तिविरहे कुमुदादौ लक्षणा नोपपद्यते~।  शक्याप्रसिद्धौ शक्यसम्बन्धरूपलक्षणायाः अप्रसिद्धेः इति वाच्यम्~। इष्टत्वात्~।  पङ्कजपदस्य कुमुदे मुख्यत्वमेव न तु लक्षणा~।  यथा न्यायमते रूड्या स्मृतस्य पद्मस्य पङ्कजनिकर्तृत्वेन  अनुभूतस्यापि पदार्थान्तरान्वययोग्यताविरहनिश्चयेन अनन्वये सति पुनः योगादेव पङ्कजनिकर्तृत्वेन कुमुदस्य बोधः~, न तु लक्षणया~। 
		
		\subsection{रूढिस्वीकारे दोषोद्भावनम्}
					
			\subsubsection{अ}
			
				\begin{small}
				
					एतेन पङ्कजपदान्नियमतः पद्मज्ञानं न वृत्तिं विना~। अतो लक्षणाद्यभावे शक्तिरित्यपास्तम्~। ज्ञानं हि स्मृतिरनुभूतिश्च शक्तिं विनाप्युपपन्ना न तां कल्पयति अनन्यलभ्यस्यैव शब्दार्थत्वात्~। अन्यथा शक्तिं विना पदान्नोपस्थितिरित्यन्वयेऽपि शक्तिर्लक्षणोच्छेदश्च~। 
				\end{small}
					
					न च पङ्कजपदं पद्मे वृत्तिमत् नियतपद्मप्रतीतिजनकत्वात्, पङ्कजपदं  पद्मे शक्तं पद्मे वृत्तिमत्त्वे सति पद्मे लाक्षणिकत्वाभावात्~, इत्यनुमानाभ्यां शक्तिः सिध्यति इति वाच्यम्~। अप्रयोजकत्वात्~।  यतो हि प्रतीतिः ज्ञानम्~।  तद्धि स्मृतिरनुभवश्च वृत्तिमन्तराप्युपपन्नमिति  न वृत्तिं कल्पयति~।  तथा च इन्द्रियादौ व्यभिचारः~। हेतुकुक्षौ  पदत्वनिवेशे तु संसर्गांशे वृत्तिविरहेण पदे व्यभिचारः~।

			\subsubsection{इ}
			
				\begin{small}
				
					पङ्कजं पद्ममुच्यत इति प्रसिद्धार्थपदसामानाधिकरण्यात् पद्मस्य ज्ञापकं तत् सिध्यति न तु तच्छक्तम्~। 
				\end{small}
			
				 न च ’पङ्कजं पद्ममुच्यते’  इति प्रसिद्धपदसामानाधिकरण्यात् पङ्कजपदस्य पद्मपदसमानार्थकत्वेन  पद्मे शक्तिः सिध्यति~। यथा घटपदसामानाधिकरण्येन कलशपदस्य घटे शक्तिः इति वाच्यम्~। प्रसिद्धपदसामानाधिकरण्यस्य शक्तिग्राहकत्वानभ्युपगमात्~। 

			\subsubsection{उ}
			
				\begin{small}
				
					अत एव पङ्कजपदं पद्मशक्तं नियमतस्तत्स्मारकत्वात् पद्मपदवदिति पदवदिति निरस्तम्~। शक्तौ व्यभिचारात्~, अनन्यलभ्यत्वस्य पद्मानुभावकत्वस्य चोपाधित्वाच्चेति~। 
				
					न चैवं नियतप्रयोगादेव शक्तिसिद्धिः~। अनन्यलभ्यस्यैव पदार्थत्वादिति
				\end{small}
			
				न च पङ्कजपदं पद्मशक्तं नियमतः पद्मस्मारकत्वात् पद्मपदवत् इत्यनुमानेन शक्तिः सिध्यति इति वाच्यम्~।  शक्तौ व्यभिचारात्~।  अनन्यलभ्यार्थकत्वस्य  पद्मानुभावकत्वस्य चोपाधित्वात्~। पद्मपदे अनन्यलभ्यपद्मार्थकत्वस्य सत्वात् पङ्कजपदे तु तयोः असत्वात्~।  यद्यपि पद्मानुभावकत्वस्य विरहः पङ्कजपदे सन्दिग्धः तथापि संदिग्धोपाधिः भवितुमर्हति~। 	

				न चैवं पङ्कजपदं पद्मे शक्तं पद्मे नियतप्रयोगात् इत्यनुमानेन शक्तिः सिध्यति इति वाच्यम्~।  अनन्यलभ्यार्थकत्वस्य तत्रोपाधित्वात्~। 

	\section{पङ्कजपदस्य रूढ्या पद्मबोधकतायाः समर्थनम्}
		
		\subsection{रूढिस्वीकारे युक्तिप्रदर्शनम्} 

			\begin{small}
			
				एवं पद्मं पङ्कजपदशक्यं ततो नियमतः पङ्कजनिकर्तृपद्ममिति प्रतीतेः~। अवयवानां तत्रासामर्थ्यात् रूढिं विना योगमात्रात् कुमुदे प्रयोगधीप्रसङ्गाच्च~। ननु रूढावपि योगात् कुमुदे तौ कुतो न स्याताम्~। रूढ्या प्रतिबन्धादिति प्राञ्चः~। वयन्तु नियमतो रूढ्या स्मृतं पद्ममेव व्यक्तिवाचकडप्रत्ययेन पङ्कजनिकर्तृतयानुभाव्यते, बाधकं विना व्यक्तिवचनानां सन्निहितविशेषपरत्वनियमात्~। यथाग्नेयीति ढगन्तपदेन प्रकरणादिना सन्निहिता ढगभिहिता ऋग्व्यक्तिर्बोध्यते~। एवञ्च सर्वत्र पद्मानुभवसामग्र्येवेति न कुमुदे धीर्न वा तदर्थप्रयोगः~। नन्वेवं रूढिरेवास्तु तत एवोभयलाभात् किं योगरूढ्या~। न~। अवयवशक्तेः क्लृप्तत्वात् यौगिकार्थानुभवाच्च~। यदि च रूढ्यर्थ एव यौगिकार्थ एव वा अनुभूयेत तदा विवाद एव न स्यात्~, अनुभवेनैव तद्विच्छेदात्~। 
				
				उच्यते~। अस्ति व्युत्पन्नस्य पदज्ञानानन्तरंनियमतोऽन्वयप्रतियोगिस्मृतिरतस्तद्धेतुसंस्कारोद्बोधकं विना पदज्ञानमेव दृष्टानुविधानत्वादनुगतत्वाल्लाघुत्वाच्च~। न तु प्रतिपदार्थस्मृतितत्कालोत्पन्नमनन्तमहष्टचरमनिर्वचनीयं गौरवात्~। अतः पद्मस्मृतावपि गृहीतसम्बन्धं पङ्कजपदमपि तथेति सम्बन्धत्वेन शक्तिसिद्धिः~। 
			
				तस्मात् पदाधीना नियता स्मृतिः शक्तिसाध्या वा नियतसम्बन्धसाध्या वा वृत्तिसाध्या वा~। तत्र परिशेषादिह शक्तिसाध्यैव~। न च शक्तिस्मृतौ व्यभिचारः~। शक्तिस्मारकत्वाभिमताद्धि पदात् पदार्थस्यैवाहत्य स्मरणं न शक्तेर्हेत्वभावात् कल्पनागौरवाच्च सम्बन्धस्यास्मरणेन प्रथमं पदार्थस्मरणस्यावश्यकत्वाच्च~। पदार्थस्मृतेश्च शक्तिविषयत्वे मानाभावात्~। उद्बुद्धसंस्कारविषयत्वाद् यदि शक्तिरपि तद्विषयः तथापि पदार्थशक्तेनैव ज्ञातस्य पदस्य शक्तिस्मारकता न तु शक्तिसम्बन्धित्वेन अन्यथासिद्धत्वात्~। यद्वा पदान्नियमतः स्मृतिर्नियतसम्बन्धसाध्या शक्त्यापि समं पदस्य शक्त्याश्रयत्वमेव नियतः सम्बन्धः~। 
			
				अपि च कुमुदेऽवयवशक्तिप्रतिबन्धार्थं रूढिः~। न च नियतपद्मस्मृतिरेव प्रतिबन्धिका~। रूढिं विना नियतस्मृतेरभावात्~। अत एव न व्यक्तिवचनन्यायोऽपि~। 
			\end{small}
			
			पङ्कजपदं अवयवशक्त्या पङ्कजनिकर्तारं बोधयति, समुदायशक्त्या पद्मं बोधयति~।  तत्रावयवशक्तिः क्लृप्ता एव~। समुदायशक्त्युपगमस्तु यत् पदात् नियमतः यत् प्रतीयते तत् तत्पदशक्यमिति सार्वजनीननियमानुरोधात्~।  न च पद्मस्यापि पङ्कजनिकर्तृत्वात् पङ्कजपदस्य अवयवशक्त्यैव पद्मस्य पङ्कजनिकर्तृत्वेन बोधनिर्वाहसम्भवात्, तत्र अतिरिक्तायाः रूडेः स्वीकारो व्यर्थ इति वाच्यम्~।  पङ्कजपदस्य कुमुदे प्रतीतिप्रयोगापत्तिवारणाय पद्मे रूडिस्वीकारात्~।  पङ्कजपदस्य अवयवशक्तेः कुमुदावपि  अक्षतत्वात~। 

			न च पङ्कजपदस्य रूडिस्वीकारेऽपि उक्तापत्तितादवस्थ्यम्~।  पङ्कजपदस्य कुमुदे रूडिविरहेऽपि  अवयवशक्तेरक्षततया योगमात्रेण तत्र पङ्कजपदप्रयोगसम्भवत् इति वाच्यम्~।  केवलयौगिकार्थप्रतीतिं प्रति रूडिज्ञानस्य प्रतिबन्धकत्वाभ्युपगमात् नोक्तापत्तिरिति प्राञ्चः~।  नव्यास्तु प्रतिबन्धकत्वकल्पनं विनैव आपत्तिं वारयति~।

			पङ्कजपदात् योगेन पङ्कजनिकर्तुः रूड्या पद्मस्य च उपस्थितौ सत्यां पङ्कजनिकर्ता सन्निहिते पद्मे अन्वेति~। न च पङ्कजनिकर्तृत्वस्य कुमुदादिसाधारणतया तदवच्छिन्नस्य पद्मेऽन्वयः न नियन्तुं शक्य इति वाच्यम् व्यक्तिवचनानां संन्निहितविशेषपरत्वनियमात्~।  यथा हि ”आग्नेय्या आग्नीध्रमुपतिष्ठते ” इति श्रुतौ आग्नेयी इति डगन्तपदेन व्यक्तिवाचकेन प्रकरणात् सन्निहितः ऋग्व्यक्तिविशेषः बोध्यते, तथा पङ्कजमिति डप्रत्ययान्तेन व्यक्तिवाचकपदेन पङ्कजनिकर्तृव्यक्तिविशेषः सन्निहितः बोधनीयः~।  सान्न्निध्यञ्च बौन्द्धिकं पदस्यैव न कुमुदादेः~।  रूढ्या पदस्य उपस्थितत्वात्~।  तथा च रूढिज्ञानकाले पङ्कजपदस्य पङ्कजनिकर्तृत्वेन पद्मपरातानियमात् पद्मत्वावच्छिन्ने अन्वयनियमो युज्यते~।  रूढिज्ञानदशायां सर्वत्र पद्मानुभवसामग्र्या एव सत्वात् कुमुदतात्पर्येण प्रयोगान् आपादयितुं शक्यते~।  न च तर्हि पङ्कजपदस्य पद्मे रूडिरेव अस्तु, न योगः पङ्कजनिकर्तरि~।  तावतैव पङ्कजपदात् पद्मप्रतीतिकुमुदाप्रातीत्योः उअपादनसम्भवात् इति वाच्यम्~।  अवयवशक्तेः क्लृप्ततया निराकर्तुमशक्यत्वात्, यौगिकस्य पङ्कजनिकर्तुः अनुभूयमानत्वाच्च~। 
		
			लोके व्युत्पन्नस्य पुरुषस्य पङ्कजपदश्रवणानन्तरं नियमतः अन्वयप्रतियोगिनः स्मृतिः भवति~। तद्धेतुभूतसंस्कारस्य उद्बोधकं तावत् असति बाधके पदज्ञानमेव भवितुमर्हति~।  गवादिपदे दृष्टान्वयव्यतिरेकशालित्वात्, पदज्ञानत्वस्य अनुगतस्य कारणतावच्छेदकत्वे  लाघवात् क्लृप्तेन पदज्ञानत्वाच्छिन्नकारणत्वेनैव उपपत्तौ अतिरिक्तकारणताकल्पने गौरवाच्च~।  तथा च पङ्कजपदात् पद्मस्मृतौ हेतुभूतसंस्कारस्योद्बोधकं  पद्मेन गृहीतसम्बन्धस्य पङ्कजपदस्य ज्ञानमेवेति पद्मपङ्कजपदयोः सम्बन्धत्वेन शक्तिः सिध्यति~।  तथाचानुमानम्- पङ्कजपदश्रवणनियतान्वयप्रतियोगिविषयकस्मृतिजनकसंस्कारः पदज्ञानेन उद्बुद्धः पदश्रवणनियतान्वयप्रतियोगिविषयकस्मृतिजनकसंस्कारत्वात्~। पङ्कजपदज्ञानं पद्मसम्बन्धावगाहि पदश्रवणनियतपद्मस्मृतिजनकसंस्कारोद्बोधकत्वात् इति~। 

			किञ्च, पङ्कजपदाधीना नियतापद्मस्मृतिः शक्तिप्रयोज्या पदाधीननियतस्मृतित्वात्, गोपदाधीनगोस्मृतिवत् इत्यनुमानेन पङ्कजपदस्य  शक्तिः सिध्यति~। अत्र हेतुमति शक्तिस्मरणे शक्तिप्रयोज्यत्वविरहेण व्यभिचार इति नाशङ्कनीयम्~।एकसम्बन्धिज्ञानात्  अपरसम्बन्धिस्मरणस्य सम्बन्धविषयकत्वनियमे मानाभावात्~।पदज्ञानाधीनं पदार्थस्मरणमेव न तु शक्तिस्मरणम्~। न च पद्मज्ञानात् स्वातन्त्र्येण शक्तिस्मरणस्य अनुत्पत्तावपि जायमानं पदार्थस्मरणमेव  शक्तिविषयकं बभवति~। पदार्थस्येव शक्तेरपि उद्बुद्धसंस्कारविषयत्वात्~। तथा च व्यभिचारतादवस्थ्यम् इति वाच्यम्~। तथा सति पदार्थस्मरणं प्रति  पदज्ञानस्य पदार्थशक्तपदविषयकज्ञानत्वेनैव हेतुतायाः उपगन्तव्यतया शक्तिस्मरणस्यापि  शक्तिप्रयोज्यतायाः अक्षततया व्यभिचारनिरासात् ~।

			पङ्कजपदस्य अवयवशक्तिमात्रेण  कुमुदे प्रयोगापत्तिवारणाय रूडिस्वीकार आवश्यकः~। न च तत्र इष्टापत्तिः सम्भवति~। पद्मान्वयस्य  बाधनिश्चयमन्तरा पङ्कजपदात्  कुमुदबोधस्य अनुभवविरुद्धत्वात्~। रूडिस्वीकारे च एकत्ररूडिज्ञानं अन्यत्र अवयवश्क्त्यवगाहिबोधं प्रति प्रतिबन्धकमित्युपगमेन दर्शितापात्तिवारणं सम्भवति~।

			न च अवयवशक्त्या कुमुदबोधं प्रति नियतपद्मस्मृतेः प्रतिबन्धक्त्वोपगमादेव उक्तापत्तिवारणसम्भवात्  रूडिकल्पनं व्यर्थमिति वाच्यम्~। शक्तिं विना नियतपद्मस्मृतेः दुरुपपादत्वात्~।

		\subsection{रूढिस्वीकारे उद्भावितदोषाणां परिहारः}
		
			\subsubsection{अ}
		
				\begin{small}
			
					न चैकं पदमेकदैकेनैव रूपेण वर्तते~। अतो न तत्रावयवार्थक्रियान्वयेनाप्रतिबन्धान्नानार्थे च नानार्थानुभवात् नैकशक्त्यान्यशक्तिप्रतिबन्धः~। यत्तु अवयवशक्तिस्मृतिकाले समुदायशक्तिस्मृतिनियमे हेत्वभावान्न तया प्रतिबन्ध इति~। तन्न~। समुदायस्य तावदवयवरूपत्वेन सर्वावयवादेवोभयशक्तिस्मृतिसम्भवात्~, मण्डपपदे तथा दर्शनात्~।
				\end{small}
		
			न च एकं पदं एकदा एकेनैव रूपेण प्रवर्तते इति नियमः~। तथा च एकस्य पङ्कजपदस्य एकदा पङ्कजनिकर्तृत्व-पद्मत्वोभयरूपावच्छिन्नबोधकत्वं  न सम्भवतीति वाच्यम्~। पङ्कजसमुदायस्य पद्मत्वावच्छिन्नबोधकत्वं डप्रत्ययादीनां अवयवानां कर्तृत्वाद्यवच्छिन्नबोधकत्वमिति एकस्य पदस्य उभयरूपावच्छिन्नबोधकत्वविरहात्~। अवयवसमुदाययोः भेदात्~। 
  
			न च अवयवावयविनोः भेदेऽपि समुदायसमुदायिनोः अभेद एव~। प्रकृते डप्रत्ययादीनां पङ्कजशब्दस्य च समुदायसमुदायिभाव एव~। अतोऽभेदात् उक्तनियमविरोधः दुष्परिहर एवेति वाच्यम्~। लम्बकर्णादिपदे व्यभिचारेण उक्तनियमस्य अप्रामाणिकत्वात्~। तथा हि लम्बकर्णशब्दः योगेन लम्बकर्णं रूड्या लम्बकर्णवत्पुरुषं युगपत् बोधयति~।

			
			\subsubsection{इ}

				\begin{small}
				
					अत एव पङ्काद्यवयवैः पद्ममेवानुभाव्यत इति अवयवनियमस्य कल्पनान्न कुमुदे धीप्रयोगौ~। अन्यथा रूडावपि योगान्न कुतस्तौ कुमुदे~। यथा उद्भित्पदस्य योगात् खनित्रे रूड्या कर्मविशेषे प्रयोग इति निरस्तम्~। अगृहीतनियतपद्मप्रयोगस्य कुमुदसधारणबोधदर्शनात्~। रूडिवादे तु सैव प्रतिबन्धिका मण्डपपदवत्~। ज्ञातस्य चावयवनियमस्य प्रयोजकत्वे पद्मे शक्तिरेव पद्मज्ञानजनकत्वज्ञानस्य शक्तिनिर्वाह्यत्वात्~। उद्भित्पदे त्ववयवार्थः कर्मण्ययोग्यत्वादेव नान्वीयते न तु रूड्या प्रतिबन्धात्~। अवयवान्वययोग्ये  खनित्रेऽपि न प्रतिबन्धः योगेन खनित्रेऽपि प्रयोगात्~। एवञ्च रूडियोगाभ्यामुद्भित्पदस्य नानार्थतुल्यतैव  उभाभ्यामेकार्थानवगमात्~। 
				\end{small}
				
				न चैवं सति पङ्कजशब्दस्य रूडिस्वीकारो व्यर्थः~। तत्स्वीकारेऽपि लम्बकर्णपदवत्  पङ्कजशब्दात् उभयरूपावच्छिन्नबोधस्य वारयितुम् अशक्यत्वात्~। रूड्या पद्मत्वावच्छिन्नस्य योगेन पङ्कजनिकर्तृत्वावच्छिन्नकुमुदस्य च पङ्कजपदात्  युगपत् बोधे इष्टापत्तेः सम्भवात् इति वाच्यम्~।
 
				लम्बकर्णपदात् उपस्थितस्य रूड्यर्थस्य इव अवयवार्थकर्णस्यापि पदार्थान्तरेण आनयनादिक्रियया  अन्वयेऽपि  पङ्कजपदात् योगमात्रेण उपस्थितस्य पङ्कजनिकर्तृत्वावच्छिन्नकुमुदस्य रूड्यर्थपद्मेन समं आनयनादिक्रियया  अन्वयविरहात्~। तथा च इतरान्वयप्रतिबन्धकयोगरूड्योः अन्यतरप्रवृत्तिः इत्येव नियमः~। लम्बकर्णपदात् रूडिज्ञानं यौगिकार्थान्वयबोधं न प्रतिबध्नाति~। अतस्तत्र द्वयोः योगरूड्योः प्रवृत्तिः भवति~। पङ्कजपदात्तु  रूडिज्ञानं  यौगिकार्थकुमुदस्य अन्वयबोधं प्रतिबध्नाति~। अतस्तत्र न द्वयोः प्रवृत्तिः~। अपि तु रूडेरेव~। इतरान्वयबोधं प्रति रूडिज्ञानस्य प्रतिबन्धकत्वं तु अत्र अनुभवसिद्धमेव~। क्वचित्तु नानार्थकस्थले  ’सैन्धवमानय’ इत्यादौ युगपत् लवणाश्वोभयतात्पर्यग्रहसत्वे द्वयोरपि यौगिकार्थरूड्यर्थयोः अन्वयानुभवः दृश्यते~। तथा च तत्र योगरूड्योः इतरान्वयबोधप्रतिबन्धकत्वाभावात् द्वयोरपि तयोः प्रवृत्तिरेव~।

				न च रूडिज्ञानस्य यौगिकार्थान्वयबोधं प्रति प्रतिबन्धकत्वं न घटते~।अवयवशक्तिस्मरणकाले समुदायशक्तेः स्मरणनियमाभावात् इति वाच्यम्~। समुदायस्य तावदवयवरूपत्वेन अवयवशक्तेः स्मरणकाले  तावदवयवसमुदायशक्तेरपि स्मरणस्य सम्भवात्~।

			\subsubsection{उ}
			
				न च पङ्कजपदस्य पद्मे रूडिस्वीकारेऽपि योगेन कुमुदे रूड्या पद्मे प्रयोगो दुर्वारः यथा उद्भित्पदस्य योगेन खनित्रे रूड्या यागविशेषे  प्रयोगः~। अतः पङ्काद्यवयवानां पद्मविषयकबोधं प्रत्येव  प्रयोजकत्वम् इति नियमकल्पनेन निरुक्तप्रयोगो वारणीयः~। तथा च रूडिस्वीकारो व्यर्थ इति वाच्यम्~। उक्तप्रयोजकत्वकल्पने पङ्कजपदे अगृहीतनियतपद्मप्रयोगस्य  पुरुषस्य ततः कुमुदसाधारणबोधस्यानुपपत्तेः~।

				न च ज्ञातस्यैव पङ्काद्यवयवे पद्मबोधकतानियमस्य पद्मविषयकानुभवे प्रयोजकत्वोपगमात् नियमग्रहदशायां पद्मेतरकुमुदादिबोध उपपद्यत इति वाच्यम्~। एवं सति पद्मे शक्तिसिद्धेः निष्प्रत्यूहत्वात्~। पदे तद्विषयकशाब्दबोधजनकत्वावगाहिज्ञानस्य  तच्छक्तिनिर्वाह्यत्वेन पङ्कजपदे पद्मविषयकशाब्दबोधजनकत्वावगाहिज्ञानस्य पङ्कजपदे पद्मशक्तिमन्तरा निर्वाहसम्भवात्~।

				न च पङ्कजपदस्थले इव उद्भित्स्थलेऽपि  रूडिज्ञानस्य यौगिकार्थबुद्धौ प्रतिबन्धकत्वापत्तिः, अविशेषात् इति वाच्यम्~। प्रयोजनविरहेण तदकल्पनात्~। तथा हि –योगरूड्यर्थयोः विशिष्टप्रतीतिविरहस्य निर्वाहाय~, रूड्या एकत्र प्रयुक्तपदस्य योगेन अन्यत्र प्रयोगाभावस्य वा निर्वाहाय रूडिज्ञानस्य  यौगिकार्थबुद्धौ प्रतिबन्धकत्वं कल्पनीयम्~। तत्राद्यस्य निर्वाहस्तावत्  योग्यतानिश्चयरूपक्लृप्तशाब्दबोधकारणविरहादेव भवति~। रूड्यर्थयागे यौगिकार्थखनित्रस्यान्वयबाधात्~। द्वितीयस्तु असिद्ध एव~। रूड्या यागे प्रयुज्यमानस्य उद्भित्पदस्य  योगेन ततोऽन्यत्र खनित्रादौ प्रयोगस्य विद्यमानत्वात्~। न च एवं सति उद्भित्पदस्य हर्यादिपदवत्  नानार्थत्वं स्यादिति वाच्यम्~। इष्टत्वात्~। 

		\subsection{रूढिनिराकरणपक्षे दोषोद्भावनम्}

			\subsubsection{अ}
			
				\begin{small}
				
					एवञ्च रथकारशब्देऽपि जातिविशेषे रूढिर्न स्यात्~। संस्कारादेव तदुपस्थितिसम्भवात्~। तथा च ’वर्षासु रथकारोऽग्नीनादधीते’त्यत्र रूढ्यर्थस्य बलवत्वेन शीघ्रमुपस्थितत्वात् जातिविशेषवत एवाधानं विधीयत इति राद्धान्तव्याघातः~। यथा च शाब्दः सन्निधिरन्वयबोधाङ्गं तथोक्तमासत्तिप्रस्तावे~।
			
					मैवं, एवं गवादिपदेऽपि शक्तिग्राहकत्वाभिमतादेवंविधप्रयोगादेव ज्ञाताद्गवादेः स्मृत्यनुभवौ स्यातां किं शक्त्या~। 
				\end{small}
			
				पङ्कजादिपदस्य रूडिनिराकरणे सति  ”वर्षासु रथकारोऽग्नीनादधीत्”  इत्यत्र रथकारशब्देन रूड्या उपस्थापितस्य सङ्करजातिविशेषस्यैव पुरुषस्य आधानं विधीयत इति मीमांसासिद्धान्तो व्याहन्येत~। तत्रापि रथकारशब्देन योगेन उपस्थापितस्य रथकर्तुरेव विवक्षणीयत्वात्~।
			
				पङ्कजपदे शक्तिग्राहकत्वेन अभिमतात् नियतप्रयोगादेव पद्मस्मरणशाब्दबोधयोः उपगमे तुल्ययुक्त्या गोपदे शक्तिग्राहकत्वेन अभिमतात् नियतप्रयोगादेव गोस्मरणशाब्दबोधयोः सम्भवात् गवादिपदानां शक्त्युच्छेदप्रसङ्गः~। 
			
			\subsubsection{इ}
			
				\begin{small}
					
					अथ पङ्कजोत्तरजपदं प्रतिबन्धकं गोबलीवर्दवत् पङ्कजनिपदप्रयोगे उप्रत्ययस्य कर्तृविशेषपद्मपरत्वं वा जनेः पद्मजनिविशेषपरत्वं वा स्वभावादेव कुमुदबोधाजनकत्वं वा चक्षुष इव रसे कुमुदे अवयवशक्तिकुण्ठनं वा कल्प्यताम्~। तेन कुमुदे न धीरिति चेत्~, तर्हि अज्ञातपद्मप्रयोगस्य कुमुदसाधारणबोधो न स्यात्~। न भवत्येव इति चेत्~, किमवयवव्युत्पन्नस्य~। ततोऽर्थप्रत्यय एव न भवति~। यद्वापङ्कजत्वेन पद्ममेवानुभूयते~। आद्ये मन्दुरजादिपदादिव सामग्रीसत्वे कथं नानुभवः~। द्वितीये कुमुदमप्यनुभूयेत अविशेषात्~। न चैवं पङ्कजशब्दस्य  सामान्यशब्दत्वात् तात्पर्यवशेन विशेषतः पद्मकुमुदधीसम्भवात् न रूढिः न वा लक्षणा व स्यादिति वाच्यम्~। न हि प्रयोजनक्षतिभयेन सामग्री नानुभावयति~। यदि च पद्मकुमुदयोस्तुल्यता तदा अगृहीतपद्मप्रयोगस्य पद्मान्वयानुपपत्तिपुरःसरैव कुमुदधीर्न स्यात्~। स्याच्च कदाचित् वैपरीत्यम्~। अतः पद्मे रूढिरेव~। कुमुदे यथा लक्षणा तथा वक्ष्यामः~। कुण्ठनञ्च शक्तिवृत्तिधर्मान्तरं वा शक्तेरनुद्भवो वा अभिभवो वा तत्रकारणान्तरं वा अदृष्टचरं कल्प्यम्~। रूडेः प्रयोगप्रतिबन्धकत्वं मण्डपादौ दृष्टमेव~। 
				\end{small}
			
				न च पङ्कजपदस्य पद्मे रूढिं विनापि कुमुदबोधापत्तिः वारयितुं शक्यते~। तथा हि –पङ्कजोत्तरजपदं कुमुदबोधं प्रति प्रतिबन्धकमास्तां, डप्रत्ययस्य कर्तृसामान्यपरत्वेऽपि पङ्कजनिपूर्वकस्य  डप्रत्ययस्य  पद्मरूपकर्तृविशेषपरत्वं गोबलीवर्दन्यायेन आस्थीयतां, तेनैव न्यायेन  डप्रत्ययपूर्वस्य पङ्कपदोत्तरस्य जनेः पद्मजनिरूपविशेषपरत्वमास्ताम्, चक्षुषः स्वभावादेव रासनप्रत्यक्षाजनकत्वमिव  पङ्कजशब्दस्य स्वभावादेव कुमुदबोधाजनकत्वमास्ताम्, पङ्कजशब्दस्य कुमुदे अवयवशक्तिकुण्ठनं कल्प्यताम्~, तावतैव पङ्कजपदात्  कुमुदबोधवारणसम्भवात्  रूढिकल्पनमयुक्तमिति वाच्यम्~। तथा सति पङ्कजशब्दस्य  पद्मे प्रयोगो येन पुरुषेण  न ज्ञातः तस्य पुरुषस्य पङ्कजपदात् कुमुदसाधारणपङ्कजनिकर्तृत्वावच्छिन्नबोधो न स्यात्~।
 
				न च तत्र कुमुदसाधारणबोधस्यानुदय  इष्टमेवेति वाच्यम्~। अवयवव्युत्पत्तिमतः  पुरुषस्य पङ्कजशब्दात् निरुक्तबोधस्य सामग्रीसत्वेन अवश्यमुत्पादात्~। न च तत्र पङ्कजनिकर्तृत्वेन पद्ममेव अनुभूयते  न कुमुदमिति वाच्यम्~। पङ्कजनिकर्तृत्वस्य कुमुदसाधारणतया तेन रूपेण पद्मस्येव कुमुदस्यापि
	
				बोधस्य अवश्यमेवोत्पादात्~। न चैवं योगेन पद्मकुमुदोभयसाधारणरूपावच्छिन्नबोधकत्वे पङ्कजपदस्य सामान्यधर्मावच्छिन्नवाचकत्वात् तात्पर्यग्रहसहकारादेव क्वचित् विशिष्य पद्मस्य क्वचिच्च विशिष्य कुमुदस्य धीसम्भवात् न पद्मे रूढिः न च कुमुदे लक्षणा आवश्यिका इति वाच्यम्~। 

				उभयसाधारणरूपावच्छिन्ने शक्तिग्रहरूपसामग्रीसत्वे तेन रूपेण तदुभयबोधस्य वारयितुमशक्यत्वात्~। पद्मे समुदायशक्तेः अस्वीकारे तु अवयवव्युत्प्त्तिमतः अगृहीतपद्मप्रयोगस्य पुरुषस्य पङ्कजशब्दात् पद्मान्वयानुपपत्तिपूर्वकत्वं  कुमुदबोधस्य नोपपद्येत~। कदाचित् कुमुदान्वयानुपपत्तिपूर्वकत्वं च पद्मबोधस्य आपद्येत~। पङ्ककुमुदयोः अवयवशक्तिमत्तायाः समुदायशक्तिराहित्यस्य च  तुल्यत्वात्~।

				किञ्च~, अवयवशक्तिकुन्ठनादेः अक्लृप्तस्य कल्पनापेक्षया क्लृप्तस्य रूडेः  प्रयोगप्रतिबन्धकत्वस्य मण्डपादिपदे दृष्टस्यात्र पङ्कजपदस्थले  स्वीकारो लघुः~।
			
			\subsubsection{उ}

				\begin{small}

					ननु गृहीतपद्मप्रयोगस्य अवयवैः सह प्रयोगे पद्ममेवानुभाव्यत इति स्वभावक्ल्पनमस्तु~। न चैवं व्युत्पत्त्यन्तरम्~। स्वरूपसत एव प्रयोगगृहस्य शक्तिगहस्य इव उपसन्धानत्वात् उपसर्गस्येव धातोः प्रकर्षादिबोधकत्वे, उपसन्धानं विना कुमुदबोधो भवत्येव~। 
					
					मैवम्~। न हि पदानां स्वभावाधीनं बोधकत्वम्~। किन्तु शक्तिज्ञानाधीनम् अज्ञातशक्तेरबोधकत्वात्~, शक्तिभ्रमेण भोधकत्वाच्च~। तथा च बोधकशक्त्यभावादेव अबोधकत्वं कार्याभावे हेत्वभावस्यैव तन्त्रत्वात्~। धातोश्च उपसर्गोपसन्धानात् प्रकर्षादौ शक्तिरेव~। 
				\end{small}

				न च यथा क्रियायां शक्तात् धातोः प्रकर्षाविशिष्टक्रियाबोधने उपसर्गः सहकारी तथा पङ्कजनिकर्तरिशक्तात् पङ्कजपदात् पद्मस्य बोधने  पङ्कजपदस्य पद्मे प्रयोगग्रहः सहकारी~। तद्विरहदशायां पङ्कजपदात् कुमुदबोधो भवत्येव~। तत्सत्वे तु कुमुदबोधो न भवति~, अपि तु पद्मबोध एव~। तन्नियमनाय गृहीतपद्मप्रयोगस्य पङ्कजपदस्य अवयवैः सह प्रयोगे पद्मानुभावकत्वस्वभावः कल्प्यते~। न तु रूढिः इति वाच्यम्~।

				पदानां बोधकतायाः स्वभावाधीनत्वाभावेन तथा स्वभावकल्पनस्य अप्रामाणिकत्वात्~। प्रत्युत अज्ञातशक्तिकपदस्य अबोधकत्वं शक्तिभ्रमेण च बोधकत्वम् इत्यन्वयव्यतिरेकाभ्यां  शक्तिज्ञानाधीनत्वं बोधकतायाः सिध्यति~। धातोश्च क्रियायमिव प्रकर्षादौ अपि शक्तिरस्तीति विषमो दृष्टान्तः~।

			\subsubsection{ऋ}

				\begin{small}

					स्यादेतत्~। यथा सर्वनामत्वमहंपदेषु बुद्धिस्थत्वसम्बोध्यत्वोच्चारयितृत्वानि प्रयोगोपाधयः~। तेन बुद्धिस्तत्वादिकमेव तद्बोधयति~। तत्रैव प्रयुज्यते च न तु बुद्धिस्तत्वादिकं शक्यं तेन रूपेण ज्ञानाभावात्~। तथा पद्मत्वमपि प्रयोगोपाधिरिति चेन्न~। तेन हि नावयवशक्तिर्विच्छेद्यते पद्मे तत्सत्वात्~। नापि कुमुदे तात्पर्यमपोद्यते, बाधकं विना तस्य औत्सर्गिकत्वात्~। नाप्यवयवाधीनज्ञानं प्रतिबध्यते~। शक्त्याकाङ्क्षादितात्पर्याणां सत्वे अन्वयज्ञानस्य आवश्यकत्वात् उपाधेः अन्वयबोधप्रतिबन्धाहेतुत्वात्~। नापि प्रयोगो वार्यते, इष्टकुमुदधीहेतुतया इष्टसाधनताज्ञानस्य तल्लिङ्गककार्यताज्ञानस्य वा प्रयोगहेतोः सत्त्वे अप्रयोगस्याहेतुत्वात्~। अहमादिपदे आत्ममात्रशक्ते शक्तिग्रहसहकारितया स्वरूपसदेव उच्चारयितृत्वादिकं धीनियामकं स्वभावान्नियमतः तस्यैव बोधनादिति स्वरूपसत एव उपाधित्वम्~। पद्मत्वं तु न तथा, कुमुदसाधारणबोधदर्शनात्~। 
				\end{small}

				न च यथा सर्वनाम-त्वम्-अहम्पदानां बुद्धिस्थत्व-सम्बोध्यत्व-उच्चारयितृत्वानि प्रयोगोपाधयः~, न शक्यानि शक्यतावच्छेदकानि च तथा पङ्कजपदस्य पद्मत्वं प्रयोगोपाधिरास्ताम्~। अतः पद्मत्वशून्यस्य कुमुदादेः बोधापत्तिः वार्यते~। तथा च रूढिस्वीकारो नावश्यक इति वाच्यम्~। तथा सति पङ्कजपदात् पद्मत्वेन पद्मानुभवस्य अनुपपत्तिप्रसङ्गात्~। बुद्धिथत्ववत् अप्रकारीभूतस्यैव प्रयोगोपाधित्वात्~। एवं~, पङ्कजपदस्य कुमुदे प्रयोगहेतोः इष्टसाधनताज्ञानादेः सत्वे कुमुदे प्रयोगस्य वारणासम्भवाच्च~। यथा त्वम्- पदस्य उच्चारयितृत्वं स्वरूपसदेव उपाधिः~, तथा पङ्कजपदस्य पद्मत्वं स्वरूपसत् उपाधिः न भवितुमर्हति~, क्वचित् कुमुदसाधारणबोधस्य पङ्कजपदात् जननात्~। 

			\subsubsection{ऌ}

				\begin{small}

					ननु कुमुदेपूर्वेषामप्रयोगात् आधुनिकानामप्रयोगः~। न च प्रयोगमात्रे को हेतुरिति वाच्यम्~। प्रयोगाणां हि न मेलनकार्यता~। किन्तु प्रत्येकम्~। तत्रोत्तरस्य पूर्वो हेतुः~। यथा पूर्वगौरुत्तरस्य~। अन्यथा गोमात्रे को हेतुरित्यत्र किमुत्तरं तवेति चेत् न ~। प्रयोगहेतोः सत्त्वात्~, पूर्वप्रयोगस्य च अहेतुत्वात्~, पूर्वप्रयोगमज्ञात्वापि अवयवव्युत्पन्नेन पाचकादिपदानां नवकाव्यानां गौणलाक्षणिकदेवदत्तादिपदानाञ्च प्रयोगात्~। 
				\end{small}
			
				न च पङ्कजपदस्य कुमुदे अधुना अप्रयोगः पूर्वेषां तत्र अप्रयोगात्~। तत्र च हेतुः तत्पूर्वेषां तत्र अप्रयोगः~। उत्तरप्रयोगे  पूर्वप्रयोगस्य हेतुत्वात्~। अप्रयोगे अप्रयोगो हेतुः~। पङ्कजपदप्रयोगसामान्यं प्रति तु न किञ्चित् कारणम्, यथा न्यायमतेऽपि  आधुनिकानां गवि गोशब्दप्रयोगे पूर्वेषां तत्र गोशब्दप्रयोगः हेतुः~। तथा च पङ्कजपदस्य विना रूढिं  कुमुदेऽप्रयोगः उपपद्यते इति वाच्यम्~। शब्दप्रयोगं प्रति शक्त्याकाङ्क्षातात्पर्यज्ञानादीनां क्लृप्तकारणानां सत्वात् पूर्वप्रयोगस्य अहेतुत्वात्~। (पङ्कजपदस्य कुमुदे अप्रयोगः क्लृप्तकारणस्य शक्तिज्ञानादेः विरहादेव उपपादनीयः~।) पूर्वप्रयोगमज्ञात्वाऽपि अवयवव्युत्पत्तिमता पुरुषेण पाचकादिपदानां नवकाव्यानां गौण्लाक्षणिकपदानां च प्रयोगात् व्यभिचारेण पूर्वप्रयोगस्य कारणतायाः कल्पयितुमशक्यत्वात्~।
			
\chapter{तृतीयोऽध्यायः साध्वपभ्रंशपदविचारः}

	\section{पदसाधुत्वविचारः}
	
		\subsection{साधुत्वज्ञानस्य शाब्दबोधहेतुता विमर्शः}
	
			\subsubsection{साधुत्वज्ञानस्य अन्वयबोधहेतुता विमर्शः}
			
				\begin{small}
					
					स्यादेतत्~। ग्रामगामादिपदे अन्वयबोधाभावात् साधुत्वज्ञानमपि हेतुः~। न चैवं तदेव हेतुरस्तु~। किं शक्त्या~। गौणलाक्षणिकविवेकश्च सोपाधित्वानुपाधित्वाभ्यामेवास्तु इति वाच्यम्~। प्रथमं व्यवहारेण शब्दस्य ज्ञानकारणतावगमात् तत्रैव शक्तिकल्पनं ततो ग्रामगामादौ व्यभिचारादाकाङ्क्षादेरिव साधुत्वज्ञानस्य हेतुत्वकल्पनात्~। 
				\end{small}
			
				ग्रामगामादिशब्दप्रयोगस्थले शक्तिज्ञानादिक्लृप्तकारणानां सत्वेऽपि अन्वयबोधानुदयात् साधुत्वज्ञानमपि अन्वयबोधहेतुरित्युपगन्तव्यः~। एवञ्च ’पङ्कजमानय’ इत्यत्र पङ्कजशब्दस्य कुमुदे साधुत्वज्ञानविरहात् न तत्र कुमुदबोधापत्तिः~।
  
				न च साधुत्वज्ञानस्य अन्वयबोधहेतुत्वोपगमे शक्त्युच्छेदप्रसङ्गः~। गोशब्दस्य अश्वादौ साधुत्वाभावेनैव तस्य गोशब्दात् बोधापत्तिवारणसम्भवात् गोशब्दस्य गवि शक्तिस्वीकारे प्रयोजनाभावात्~।शब्दानां मुख्यगौणभेदः शक्तिमन्तराऽपि सिद्ध्यति~। तथा हि यस्मात् शब्दात् इतारन्वयानुपपत्तिप्रतिसन्धानानन्तरमेव अन्वयबोधो जायते स मुख्यः इति वाच्यम्~। 
	
				साधुत्वज्ञानेन शक्तिज्ञानस्य अनन्यथासिद्धेः~। सत्यपि साधुत्वज्ञाने शक्तिज्ञानशून्यपुरुषस्य शाब्दबोधानुदयेन शक्तिज्ञानस्य कारणत्वोपगमः आवश्यकः~। एवं शक्तिज्ञाने सत्यपि विना साधुत्वज्ञानं बोधानुदयेन तस्यापि कारणत्वमावश्यकम्~। 
			
			\subsubsection{साधुत्वनिर्वचनम्}
			
				\begin{small}
			
					साधुत्वं च न पूर्वप्रयोगप्रवाहाविच्छेदः नवकाव्यादौ तदभावात्, किन्तु भ्रमादिजन्यत्वाभावः~। कुमुदे च तत्प्रयोगो बोधकत्वभ्रमात्~।  न चैवं भ्रमजन्यत्वादबोधकत्वमबोधकत्वे च भ्रमजन्यत्वमित्यन्योन्याश्रयः, पूर्वभ्रमजन्यत्वेनोत्तरं प्रति बोधकत्वभ्रमात् उत्तरस्य प्रयोगो भ्रमजन्यः, एवं पूर्वस्यापीति तत्र भ्रमपरम्परैव मूलम्~, एवं ग्रामगामादेरपि~। कथमिदं ज्ञातव्यमिति चेत् ? ग्रामगमनकर्तरि  ग्रामगामपदस्येव प्रामाणिकानां कुमुदे तस्य निरुपाधिप्रयोगाभावात्~। अत एव साधुत्वभ्रमात् कुमुदसाधारणो बोधोऽपि~।  कुमुदेह्यसाधुत्वात् समुदायेन तत्र पङ्कजन्मकर्तृत्वं लक्ष्यते~। न चासाधुत्वे ग्रामगामपदवत् लक्षणापि न स्यात्, पद्मे तस्य साधुत्वात्~। उच्यते, पङ्कजन्मकर्तरि तत्प्रयोगस्य भ्रमाद्यजन्यत्वेन साधुत्वम्~। कुमुदे न तथेति चेत्? न, पुण्डरीकवन्नवपद्मवच्च तस्य व्यक्तिस्थानीयत्वात्, प्रतिवाक्यार्थं च साधुत्वज्ञानस्य हेतुत्वेऽभिनवव्यक्तौ भ्रमाद्यजन्यत्वस्य ग्रहीतुमशक्यतया पाचकादिपदप्रयोगो न स्यात्~, पाककर्तृजातीये भ्रमाद्यजन्यत्वस्य सुग्रहत्ववत् पङ्कजन्मकर्तृजातीयेऽपि सुग्रहत्वात्, येन रूपेण अन्वयबोधस्तस्य कुमुदवृत्तित्वात्~।           
   
					अथ पङ्कजपदं कुमुदे वृत्यन्तरं विना न साधु वृत्यन्तरं विना वृद्धैः तत्र अप्रयुज्यमानत्वात्, यः शब्दो यत्र वृत्यन्तरं विना वृद्धैः न प्रयुज्यते स तत्र वृत्यन्तरं विना न साधुः यथा गङ्गापद्ं तीरे, यथा वा ग्रामगामपदं ग्रामगमनकर्तरि, न प्रयुज्यते च वृत्यन्तरं विना पङ्कजपदं कुमुदे~। तस्माद् वृत्यन्तरं विना न तत्र साध्विति 
				\end{small}
			
				साधुत्वञ्च न तावत् पूर्वप्रयोगप्रवाहाविच्छेदः~। नवकाव्यादौ प्रयुक्तपदानां पूर्वप्रयोगाभावेन साधुत्वानुपपत्तिप्रसङ्गात्~। अपि तु भ्रमजन्यत्वाभाव एव साधुत्वम्~। स च तत्र शब्दप्रयोगेऽक्षत एव~। कुमुदे पङ्कजशब्दप्रयोगस्तु बोधकत्वभ्रमादेव इति तत्र निरुक्तसाधुत्वज्ञानविरहात् न शाब्दबोध उदेति~। न च प्रतिवाक्यार्थं साधुत्वज्ञानस्य हेतुत्वे अभिनवे पाककर्तृव्यक्तिविशेषे शब्दप्रयोगस्य भ्रमाद्यजन्यत्वरूपसाधुतायाः गृहीतुमशक्यत्वात् पाचकादिशब्दप्रयेगे बोधो न स्यादिति वाच्यम्~। पाककर्तृसजातीये निरुक्तसाधुत्वज्ञानसम्भवेन बोधोपपत्तेः~। 
	
				न चैवं सति पङ्कजनिकर्तृसजातीयेऽपि कुमुदे पङ्कजशब्दप्रयोगस्य निरुक्तसाधुतायाः गृहीतुं शक्यत्वात् पङ्कजशब्दात् कुमुदबोधापत्तिः अपरिहार्यैव इति वाच्यम्~। पङ्कजशब्दस्य लक्षणादिवृत्त्यन्तरं विना कुमुदे असाधुत्वमिति निर्णयेन साधुत्वज्ञानस्य प्रतिबन्धात् नोक्तकुमुदबोधापत्तिः~। असाधुत्वन्तु अनुमानेन साध्यते~। तथा हि पङ्कजपदं कुमुदे वृत्त्यन्तरं विना न साधु वृत्त्यन्तरं विना वृद्धैः तत्राप्रयुज्यमानत्वात् यः शब्दो यत्र वृत्त्यन्तरं विना वृद्धैः न प्रयुज्यते स तत्र वृत्त्यन्तरं विना न साधुः, यथा गङ्गापदं तीरे~। न प्रयुज्यते च वृत्त्यन्तरं विना पङ्कजपदं कुमुदे~। तस्मात् वृत्त्यन्तरं विना पङ्कजपदं कुमुदे न साधु इति~। 
			
			\subsubsection{असाधुत्वनिश्चयस्य अन्वयबोधप्रतिबन्धकतापक्षः}
			
				\begin{small}
			
					न, व्यर्थविशेषणत्वात् तत्राप्रयुज्यमानत्वस्यैव व्याप्यत्वात् आकाङ्क्षादिमद्वाक्यासाधुत्वे अवयवार्थान्वयाबोधकत्वस्य व्याकरणस्मृत्यनुपगृहीतत्वस्य पक्षधर्मावच्छिन्नसाध्यव्यापकस्योपाधित्वाच्च~। पक्षे उभयाभावेन साधनाव्यापकत्वात्~। पङ्कजशब्दः कुमुदे साधुः साधुत्वे सति कुमुदबोधप्रयोजकाकाङ्क्षादिमद्वाक्यत्वात्, साधुत्वे सति यद्वाक्यं यद्बोधप्रयोजकाङ्क्षादिमद्भवति तत्तत्र साधु,यथा तरुप्रसूनशब्दः कुसुमे, यथा वा पाचकशब्दः पाककर्तरि, तथा पङ्कजपदं कुमुदे, तस्मात्तथेति सत्प्रतिपक्षत्वाच्च~। वस्तुतस्तु साधुत्वज्ञानं न हेतुः अवयवादिव्युत्पन्नेन भ्रमाद्यजन्यं पूर्वओरयोगमज्ञात्वापि नवकाव्यादिप्रयोगात्~। कुतस्तर्हि ग्रमगामादौ नान्वयबोधः,असाधुत्वनिश्चयात् इतिब्रूमः~। 
	
					साधोरप्यसाधुत्वनिश्चयेन अबोधकत्वात् अभिनवकाव्यादौ नासाधुत्वनिश्चयः, 
				\end{small}
			
				शक्तिज्ञानादिक्लृप्तकारणकलापे सत्यपि ग्रामगामादिशब्दात् अन्वयबोधानुदयात् असाधुत्वनिश्चयस्य अन्वयबोधप्रतिबन्धकत्वं कल्प्यते~। न तु साधुत्वज्ञानस्य कारणत्वम्~। अभिनवकाव्यादौ साधुत्वज्ञानविरहेण शाब्दबोधानुदयप्रसङ्गात्~। अवयवशक्तिग्रहवता पुरुषेण भ्रमाद्यजन्यं पूर्वप्रयोगम् अज्ञात्वापि नवकाव्यादिप्रयोगात्~। पङ्कजपदस्य कुमुदे साधुतायाः दुर्वारत्वाच्च~। तथा हि – पङ्कजपदस्य कुमुदे असाधुतासाधनाय  उक्तमनुमानम् उपाधिग्रस्तं सत्प्रतिपक्षदूषितम्~।
	
				पङ्कजपदं कुमुदे वृत्त्यन्तरं विना न साधु वृत्त्यन्तरं विना वृद्धैः तत्राप्रयुज्यमानत्वात् इत्यत्र हेतुकुक्षौ तत्राप्रयुज्यमानत्वस्यैव व्याप्यत्वसम्भवेन अधिकविशेषणोपादानात् व्यर्थविशेषणत्वम्~। तावन्मात्रोपादाने तु स्वरूपासिद्धिः~। अवयवार्थान्वयाबोधकत्वे सति व्याकरणत्वस्मृत्यनुपगृहीतत्वस्य आकाङ्क्षादिमद्वाक्यत्वरूपपक्षधर्मावच्छिन्नसाध्यं प्रति पाचकादिशब्दान्तर्भावेण व्यापकतया, निरुक्तपक्षधर्मावच्छिन्नसाधनं प्रति पङ्कजशब्दान्तर्भावेण अव्यापकतया उपाधित्वात् हेतोः व्याप्यत्वासिद्धिश्च~। पङ्कजशब्दः कुमुदे साधुः साधुत्वे सति कुमुदबोधप्रयोजकाङ्क्षादिमद्वाक्यत्वात् इति सत्प्रतिपक्षोऽपि~। अत्र सामान्यव्याप्तिः सम्भवति~। साधुत्वे सति यद्वाक्यं यद्बोधप्रयोजकाङ्क्षादिमद्भवति तत्तत्र साधु, यथा तरुप्रसूनशब्दः कुसुमे, यथा वा पाचकशब्दः पाककर्तरि इति~।
			
			\subsubsection{साधुत्वासाधुत्वनिर्वचनम्}
			
				\begin{small}
			
					असाधुत्वञ्च  नभ्रमादिजन्यत्वमनाप्तोक्तेऽसाधुत्वापत्तेः शुकाद्युदीरिते ग्रामादौ भ्रमाद्यजन्यत्वाच्च~। किन्तु महाजनपरिगृहीतव्याकरणस्मृतिनिषिद्धत्वं तदपरिगृहीतत्वं वेति वक्ष्यते~। पदस्य साधुत्वं वृत्तिरेव, वृत्तिश्च शाब्दबोधहेतुपदार्थोपस्तित्यनुकूलपदपदार्थयोः सम्बन्धः~। वाक्यसाधुत्वञ्चाकाङ्क्षादिमत्स्वार्थान्वयबोधकत्वे सति व्यकरणस्मृत्यनिषिद्धत्वं, 
				\end{small}
			
				असाधुत्वञ्च वाक्यगतं न तावत् भ्रमादिजन्यत्वम् , अनाप्तोक्ते पाचकादिवाक्ये असाधुत्वप्रसङ्गात्~। शुकाद्युच्चरिते ग्रामगामादिवाक्ये असाधुत्वानुपपत्तेश्च~। किन्तु महाजनपरिगृहीतव्याकरणस्मृतिनिषिद्धत्वं महाजनापरिगृहीतत्वं वा असाधुत्वम्~। अतो नोक्तदोषः~। साधुत्वम् आकाङ्क्षादिमत्स्वार्थान्वयबोधकत्वे व्याकरणस्मृत्यनिषिद्धत्वं, वाक्यस्य महाजनपरिगृहीतत्वं वा~। पदस्य साधुत्वं वृत्तिमत्वमेव~। वृत्तिश्च शाब्दबोधहितुभूतपदार्थोपस्थित्यनुकूलः पदपदार्थयोः सम्बन्धः~। एवञ्च पङ्कजशब्दस्य कुमुदे प्रयोगं प्रति प्रतिबन्धकतया समुदायशक्तिः सिद्ध्यति~। न च लाघवात् पङ्कजसमुदायान्तर्गतस्य जशब्दस्यैव शक्तिरस्तु~। न समुदाये इति वाच्यम्~। तथा सति मन्दुरजशब्दात् पद्मबोधप्रसङ्गात्~। पङ्कजोत्तरजपदत्वेन शक्तिः इति स्वीकारे सर्वत्र तत्तद्वर्णोत्तरान्त्यवर्णे एव शक्तिस्वीकारसम्भवेन समुदायशक्तेः उच्छेदप्रसङ्गः~। 
			
			\subsubsection{ग्रामगामादौ अन्वयबोधविचारे मतभेदः}
			
				\begin{small}
			
					अन्ये तु ततोऽन्वयबोधो भवत्येव, अपभ्रंशादिव यथा तव साधुत्वभ्रमात्, साधुत्वं तस्य नास्ति तद्वदेव~। अपरे तु अत्र आकाङ्क्षा एव नास्ति घटः कर्मत्वमानयनमित्यत्रेव~।
 
					यत्तु गमाद्यनुत्तराण्त्वेन गमादिपूर्वेतराणत्वेन वा शक्तिः काशपूर्वेतरकुशत्वेनेवेति, तन्न, कर्मोपपदधातूत्तराणत्वेनैव शक्तेर्न तु तत्तद्धात्वनुत्तराणत्वेन गौरवात्, तत्तद्धातूनां विशिष्याज्ञाने शक्तिग्रहानुपपत्तेः कर्मण्यणिति सूत्रार्थमजानतः कुम्भकारादिपदप्रयोगानुपपत्तेश्च शक्तिसंदेहात्~।    
				\end{small}
			
				ग्रामगामशब्दात् अन्वयबोधविषये पञ्चपक्षाः मणौ उपस्थापिताः~। यथा अपभ्रंशात् अन्वयबोधो भवति तथा ग्रामगामशब्दादपि ’ग्रामगमनकर्ता इत्यन्वयबोधो भवति, किन्तु तस्य शब्दस्य साधुत्वं नास्तीति एकः पक्षः~। ’घटः कर्मत्वम् आनयनं कृतिः’ इति वाक्यप्रयोगे इव ग्रामगामवाक्यप्रयोगे अपि आकाङ्क्षाविरहात् अन्वयबोधो न भवति इति द्वितीयः पक्षः~। ’अण्’ प्रत्ययस्य गमादितत्तद्धातु-अनुत्तराण्त्वेन शक्यत्वोपगमात् ग्रामगामेत्यत्र अणः तद्रूपानाक्रान्तत्वेन शक्तिज्ञानरूपकारणविरहात् नान्वयबोधो जायते इति तृतीयः पक्षः~। अयं च पक्षः मणिकृतैव दूषितः~। अन्वयबोधं प्रति आकाङ्क्षादेरिव साधुत्वज्ञानस्यापि हेतुत्वात् ग्रामगामशब्दे साधुत्वज्ञानरूपकारणस्य विरहात् नान्वयबोधो जायते इति चतुर्थः पक्षः~। अयमपि पक्षो दूषितः~। अन्वयबोधं प्रति असाधुत्वनिश्चयस्य प्रतिबन्धकत्वात्~। 'ग्रामगाम'इत्यत्र असाधुत्वनिश्चयसत्त्वेन अन्वयबोध नोदेतीति पञ्चमः पक्षः~।
			
		\subsection{पदसाधुत्वस्य निर्वचनम्}
	
			\subsubsection{पूर्वपक्षविमर्शः}
			
				पदानां साधुत्वं हि न तावत् प्रतिपादकत्वम्~, अपभ्रंशे अतिव्याप्तेः~। 

				नापि यस्य शब्दस्य यत्र अनादिः प्रयोगः स तत्र साधुरिति, ‘हेऽरयो हेऽलय इतिवदन्तः असुराः पराबभूवुः’ इति वेदे श्रवणात्  अलिशब्दे अपभ्रंशे अतिव्याप्तेः~। अनादित्वस्य सजातीयप्रयोगपूर्वकत्वरूपतया अपभ्रंशे सर्वत्र नियमतः तथात्वेन अतिव्याप्तेः, संज्ञाशब्दे अव्याप्तेः, लाक्षणिकेषु अव्याप्तेश्च~। नापि भ्रमाद्यजन्यत्वम्~, अपभ्रंशे अतिव्याप्तेः~। तस्यापि बोधकत्वेन बोधकत्वभ्रमजन्यत्वाभावात्~। नापि गत्वादिव्यापकः व्याकरणसंस्कृतश्रोत्रग्राह्यो जातिविशेषः, जातेः सप्रतियोगिकत्वानुपपत्तौ सर्वस्य सर्वत्र साधुतापत्तेः~। न च इष्टापत्तिः~। गोशब्दस्य गवि साधुत्वेऽपि अश्वे साधुत्वाभावात्~। प्रतिशब्दं भिन्नतद्विषयजातिस्वीकारे च अननुगमः~। न च जातेः सप्रतियोगिकत्वाभावनियमे तारत्वस्य सप्रतियोगिकतया जातित्वानुपपत्तिरिति वाच्यम्~। तारत्वगतस्य उत्कर्षादेरेव सप्रतियोगिकत्वात्~। तारत्वस्य निष्प्रतियोगिकतया जातित्वोपपत्तेः~। नापि यज्ञप्रयोगार्हत्वं साधुत्वम्~, अर्हत्वस्य दुर्निरूप्यत्वात्~। चाण्डालाद्युच्चार्यमाणस्यापि साधुत्वेन तत्राव्याप्तेः~। यज्ञेषु अप्रयुक्तेऽपि शब्दे साधुत्व-व्यवहारेण तत्राव्याप्तेश्च~। नापि    धर्मजनकशब्दत्वम्~। अपभ्रंशेऽपि आर्तत्राणाद्यर्थमुच्चारितेऽतिव्याप्तेः, पतितोच्चार्यमाणेऽव्याप्तेश्च~। नापि वेदस्थशब्दत्वम्~, अपभ्रंशे अलिशब्देऽतिव्याप्तेः~। संज्ञाशव्दे अव्याप्तेः~। भाषामात्रप्रयुक्तसाधुशब्देषु अव्याप्तेः~। नापि व्याकरणव्युत्पाद्यशब्दत्वम्~। संज्ञाशब्दे अव्याप्तेः~। व्याकरणत्वस्य साधुत्वान्वाख्यापकत्वरूपतया अन्योन्याश्रयाच्च~।

	
			\subsubsection{प्राचीनानां सिद्धान्तः}
			
				प्राचां सिद्धान्तः - न च साधुशब्दानामिव अपभ्रंशानामपि बोधकत्वसाम्ये ईश्वरसङ्केतित्व-तदसङ्केतित्वाभ्यां वैषम्यं निर्युक्तिकमिति वाच्यम्~। यथाद् या काचिदोषधी नकुलदंष्ट्रा स्पृष्टा सा विषं हन्ति तथा ईश्वरसङ्केतिता एव शब्दाः धर्मोपयोगिनः स्वभावादुपगम्यते~। ईश्वरसङ्केतज्ञानञ्च व्यवहारात्, उपमानात्, प्रसिद्धार्थपदसामानाधिकरण्यात्, आप्तोपदेशात् यववराहेन्द्रादिशब्देषु वाक्यशेषात् कोशात् तदभियुक्तेन्द्रपाणिन्यादिप्रणीतशब्दानुशासनाच्च तेषां साध्वसाधुविभाग एवाधिकारात्~। तदुक्तं तत्र तत्वमभियोगात् स्यादिति~। ’अभियुक्ताश्च ये यत्र यन्निबद्धप्रयोजनाः~। ते तत्र गुणदोषाणां ज्ञाने चाधिकृता मताः’इति~। न च साधुत्वे सिद्धे शब्दानुशासनं तस्मिंश्च सति तद्धीरिति अन्योन्याश्रयः, पूर्वपूर्वशब्दान्वाख्यानेन साधुत्वमवगम्योत्तरोत्तरानुशासनप्रवृत्तेः~। सर्गादौ भगवत एव तत्प्रणीतव्याकरणाद्वा साधुत्वनिश्चयेन उत्तरव्याकरणप्रवृत्तिः~। न चैवं सति विश्रामविश्रमशब्दयोः साधुत्वासाधुत्वनिश्चयो दुर्घटः | तत्र पाणिनिचन्द्रगोमिव्याकरणयोः विरोधात् इति वाच्यम् | ’वौ श्रमेर्विकल्प इष्यत’ इति महाभाष्यकारवचनादुभयमपि साधु~। ननु मैत्रादिशब्दानामीश्वरसङ्केताभावादसाधुत्वे यज्ञादौ न प्रयोगः स्यादिति चेत्? न, द्वादशेऽहनि पिता नाम कुर्यादित्यनेन सामान्यतः तेषामपीश्वरसङ्केतविषयत्वात्~। 
	
				अत्रोच्यते ईश्वरसङ्केतितत्वं साधुत्वम्~। न च तस्य निष्प्रतियोगिकत्वेन सर्वत्र सर्वस्य साधुतापत्तेरिति वाच्यम्~। यः शब्दः यत्रेश्वरेण सङ्केतितः स तत्र साधुः इत्युपगमेन अदोषात्~। 

				व्याकरणव्युत्पाद्यत्वं साधुत्वम्~। यः शब्धः यस्मिन्नर्थे व्याकरणव्युत्पादितः स तत्र साधुः~। गावीशब्दो व्याकरणोक्तव्युत्पत्त्या यमर्थं प्रतिपादयति स तत्र साधुः, असाधुरन्यत्र~। नामसु नाव्याप्तिः, ‘उणादयो बहुलम्’ इत्यादिना तेषां व्युत्पादनात्~। न चान्योन्याश्रयः, शब्दगुणदोषविद्भिः अभियुक्तैः प्रकृतिप्रत्ययादिकल्पनया शब्दान्वाख्यायकस्य व्याकरणत्वात्~। पूर्वपूर्वव्याकरणतः साधुत्वमवगम्य उत्तरोत्तरव्याकरणेन व्युत्पादनात्~।
	
	\section{अपभ्रंशविचारः}

		\subsection{अपभ्रंशानामपि शक्तत्वमिति पूर्वपक्षः}
		
			\begin{small}
			
				नन्वेवं पङ्कजपदस्येवापभ्रंशानामपि शक्तिस्ततो नियमेनार्थप्रतीतेः व्यवहाराधीनव्युत्पत्तेर्विशेषात्, अन्यथा असति वृत्त्यन्तरे तेभ्योऽर्थधीर्न स्यात्~। न च लक्षणा, मुख्यार्थाबधात्~। न चापभ्रंशेन स्मारितसाधुशब्दादन्वयबोधः, साधुशब्दमजानतामपि पामराणां ततोऽर्थप्रतीतेः तदर्थकसंस्कृतेनार्थप्रतीतेश्च~। शक्त्यारोपात्ततोऽर्थप्रत्यय इति चेत्?, न, मानाभावात् व्यवहारादिनावृत्तशक्तिग्रहे बाधाकाभावाच्च्~। न च शक्तिसाध्य्ं शक्त्यारोपाद् भवितुमर्हति, आरोपितादपि दहनाद्दाहापत्तेः~।
  
				अथ ज्ञायमानकरणे प्रयोजकरूपवत्तया ज्ञायमानादेव फलं बाष्पे धूमारोपाद्यनुमितिवदिति चेत्? न, तर्हि तद्वदेव फलस्यायतार्थ्तत्वापत्तिः~। अन्यथा अपभ्रंश एव शक्तिः साधुशब्द एव तथेति किं न स्यात्~। न च सादित्वात् देवतादिशब्दवदाधुनिका एव तत्र सङ्केताः गवादिपदे तु अनादित्वात् शक्तिरिति वाच्यं, सजातीयप्रयोगजन्यप्रयोगत्वस्य भ्रमाद्यजन्यत्वस्य वा अनादित्वस्यापभ्रंशेऽपि सत्वात्~। तत्तद्देशीयापभ्रंशेऽपि सङ्केतयितृणामस्मरणाच्च्~। न चापभ्रंशानां देशभेदेन एकत्रार्थे बहुत्वात् संस्कृतस्यैकत्वाल्लाघवेन तत्रैव शक्तिः, प्रमाणवतो गौरवस्यापीष्टत्वात्~। अन्यथा नानार्थोच्छेदः प्रतीत्यन्यथानुपपत्तिस्तुल्यैव~। ननूभयोः शक्तत्वे साध्वसाधुविभागाभावात् तद्व्यवहारविरोधः ”साधुभिर्भाषितव्यं नापभ्रंशितवै न म्लेच्छितवै” इत्यादिकवैदिकविधिनिषेधानुपपत्तिश्चेति चेत्? तर्हि शक्तत्वाविशेषेऽपि तदभियुक्तेन्द्रपाणिन्यादिप्रणीतव्याकरणोपगृहीतानामेव संस्कृतानां साधुत्वमस्तु, न ह्यपभ्रंशितैः साधुत्वं स्मर्यते, तद्विषया एव साधुभिर्भाषितव्यमित्यादिविधयः तेषामेव धर्मोपयोगित्वं स्वभावात् बीजस्येवाङ्कुरे, तस्मादपभ्रंशोऽप्यर्थप्रत्यायकत्वाच्छ्क्त् इति~।
			\end{small}

			पङ्कजपदवत् अपभ्रंशस्यापि शक्तिः स्यात्~। ततः नियमेन अर्थप्रतीतेः~। न चापभ्रंशात् लक्षणया अर्थप्रतीतिः, मुख्यार्थस्याबाधात्~। न चापभ्रंशेन स्मारितसाधुशब्दात् अन्वयबोधः, साधुशब्दम् अजानतामपि पामराणां ततो अर्थप्रतीतेः, तदर्थकेन साधुशब्देन अर्थप्रतीतेश्च~। न च शक्त्यारोपात् ततः अर्थप्रतीतिः, आरोपितादपि दहनात्  दाहापत्तेः~। न च धूमारोपात् वह्न्यनुमितिवत् सम्भवति, तद्वदेव फलस्य शाब्दबोधस्य अयथार्थत्वापत्तेः~। न च तत्रानुमितेः विषयबाधात् अयाथार्थ्यम्~। अत्र शाब्दबोधस्य तु विषयाबाधात् याथार्थ्यमिति वाच्यम्~। एवमपि अपभ्रंशे एव शक्तिः न साधुशब्दे इत्यत्र विनिगमनाविरहप्रसङ्गात्~। न चापभ्रंशस्य देवदत्तादिशब्दवत् सादित्वात् आधुनिकसङ्केत एव~। न तु शक्तिः~। गवादिपदस्य तु अनादित्वात् शक्तिरिति विनिगमनासम्भवतीति वाच्यम्~। अनादित्वस्य सजातीयप्रयोगजन्यप्रयोगविषयत्वरूपस्य भ्रमाद्यजन्यप्रयोगविषयतारूपस्य वा अपभ्रंशेऽपि अक्षतत्वात्~। न च ----सम्भवतीति वाच्यम्~। शक्तिग्राहकमानस्य अपभ्रंशेऽक्षततया गौरवस्य अकिञ्चित्करत्वात्~। अन्यथा नानार्थमात्रोच्छेदप्रसङ्गः~।  न च उभयोः शक्तत्वे साध्वसाधुविभागाभावात् तद्व्यवहारविरोधः, ‘साधुभिर्भाषितव्यं नापभ्रंशितवै न म्लेच्छितवै’ इत्यादिविधिनिषेधानुपपत्तिश्च इति वाच्यम्~। उभयोः शक्तत्वाविशेषेऽपि तदभियुक्तेन्द्रपाणिन्यादिप्रणीतव्याकरणोपगृहीतानामेव संस्कृतानां साधुत्वमिति नियमसम्भवात्~। न हि अपभ्रंशे तैः साधुत्वं स्मर्यते~। तद्विषया एव साधुभिर्भाषितव्यम् इत्यादिविधयः~। तेषामेव धर्मोपयोगित्वं बीजस्येवाङ्कुरे~। तस्मात् अपभ्रंशा अपि अर्थप्रत्यायकत्वात् शक्ता एवेति~। 

		\subsection{अपभ्रंशानां न शक्तिरिति सिद्धान्तः}

			\begin{small}

				उच्यते, एकत्र शक्त्याप्यन्यत्र तदारोपात् तदर्थप्रतीत्युपपत्तावेकत्रैव शक्तिर्लाघवात्, अनन्यलभ्यस्यैव शब्दार्थत्वात्~। अन्यथा वृत्यन्तरोच्छेदः~। तदाह भगवान् जैमिनिः  ”अन्यायश्चानेकशब्दत्वमिति” ~। सा च शक्तिः संस्कृत एव सर्वदेशे तस्यैकत्वात् नापभ्रंशेषु तेषां प्रतिदेशमेकत्रार्थे भिन्नभिन्नरूपाणां तावच्छक्तिकल्पने गौरवात् पर्यायबहुतरत्वञ्चोभयत्रापि~। न च देशभेदेऽपि प्राकृतस्यैकरूपत्वात् तत्रैव शक्तिः, संस्कृतप्रभवतत्समदेशिभेदेन तस्याप्यनेकत्वात्~। एवमेकत्र शक्त्यारोपादर्थप्रत्ययोपपत्तौ नापभ्रंशे शक्तिः~। ननु म्लेच्छादीनां संस्कृतमजानतां कथं तच्छ्क्त्यारोपः? उच्यते, केनचिद् गौरिति  शब्दे प्रयोक्तव्ये प्रमादाद् गावीशब्दे प्रयुक्ते व्युत्पन्नस्तेन गोशब्दमुन्नीय ततो गां प्रतीत्य व्यवहृतवान्, यथाहुः-”अम्बाम्बेति यदा बालः शिक्ष्यमाणः प्रभाषते~। अव्यक्तं तद्विदां तेन व्यक्ते भवति निर्णयः॥’ इतिपार्श्वस्थश्च व्युत्पित्सुःगावीशब्दादेव अयम् गां प्रतीतवान् इत्यवगम्य गावीशब्दमेव  गोशक्तत्वेन प्रतीत्य  अन्येषां व्युत्पादको  बभूव इति ततः प्रभृत्यपभ्रंशे शक्तत्वभ्रमः~। एवञ्च प्रथमव्युत्पन्नस्य अपभ्रंशादेव  स्मारितसाधुशब्दादर्थप्रत्ययस्तन्मूलकश्चाननुसंहितसाधुशब्दानामपि शक्तिभ्रमात् शक्तिमत्तया ज्ञायमानस्यैव शाब्दज्ञानहेतुत्वात्~। 
				
				यत्तु जनकज्ञानस्य  भ्रमत्वे  शब्दज्ञानायथार्थत्वं लिङ्गभ्रमजन्यानुमितिवदिति,  तन्न, न हि शक्तिज्ञानयथार्थत्वं   शाब्दप्रमायां प्रयोजकं, अनाप्तोक्ते  व्यभिचारात्~। नापि जनकज्ञानभ्रमत्वेन अयथार्थत्वं  भ्रमानुव्यवसाये  व्यभिचारात्~। प्रमाजनकत्वञ्चापभ्रंशस्यावश्यकशाब्दप्रमाप्रयोजकयोग्यतादेर्यथार्थतद्बोधाद्व  न तु शक्तिज्ञानयथार्थत्वं तन्त्रं  गौरवात्~। 
			\end{small}

			उच्यते – अपभ्रंशानां शक्तिविरहेऽपि शक्त्यारोपेणार्थप्रतीत्युपपत्तौ न तस्य शक्तिः स्वीक्रियते~। अनन्यलभ्यस्यैव शब्दार्थत्वात्~। अन्यथा वृत्त्यन्तरोच्छेदप्रसङ्गात्~। तथा हि गोशब्दात् गावीशब्दाच्च गोविषयकप्रतीत्युदयात् गवि तयोः द्वयोरपि शब्दयोः शक्तिस्वीकारे तीरशब्दात् गङ्गाशब्दाच्च तीरविषयकप्रतीत्युदयात् तीरे तयोः द्वयोरपि शक्तिसिद्ध्या लक्षणायाः उच्छेदप्रसङ्गः~। तदाह ----~। सा च शक्तिः संस्कृतशब्दे एव~, सर्वदेशे तस्य एकत्वात्~। नापभ्रंशेषु, तेषां प्रतिदेशं एकत्रार्थे भिन्नभिन्नरूपाणां तावच्छक्तिकल्पने गौरवात्~। न च देशभेदेऽपि प्राकृतस्य ----~। न च म्लेच्छादीनां संस्कृतमजानतां शक्त्यारोप एव दुर्घट इति आशङ्कनीयम्~। संस्कृतमजानतामपि शक्त्यारोपेण बोधसम्भवात्~।  

			तथा हि केनचित् गौरिति शब्दे प्रयोक्तव्ये प्रमादात् गावी इति प्रयुक्ते व्युत्पन्नः तेन गोशब्दमुन्नीय ततो गां प्रतीत्य व्यवहरति~। यथाहुः – ‘अम्बाम्बेति यदा बालः शिक्ष्यमाणः प्रभाषते~। अव्यक्तं तद्विदां तेन व्यक्ते भवति निर्णयः’ ॥ इति~। पार्श्वस्थश्च व्युत्पित्सुः गावीशब्दादेव अयं गां प्रतीतवान् इत्यवगम्य गावीशब्दमेव गोशक्तत्वेन प्रतीत्य अन्येषां व्युत्पादको बभूव इति ततः प्रभृति अपभृंशे शक्तत्वभ्रमः~। एवञ्च प्रथमव्युत्पन्नस्य अपभ्रंशादेव स्मारितसाधुशब्दात् अर्थप्रतीतिः जायते~। तन्मूलकश्च अननुसंहितसाधुशब्दानामपि म्लेच्छप्रभृतीनां शक्तिभ्रमात् अर्थप्रतीतिः जायते~। न च यथा व्याप्यवत्ताभ्रमजन्यायाः अनुमितेः भ्रमत्वं तथा शक्तिमत्ताभ्रमजन्यशाब्दबोधस्यापि भ्रमत्वं स्यात्~, शाब्दप्रमात्वे शक्तिग्रहप्रमात्वस्य प्रयोजकत्वादिति वाच्यम्~। अनाप्तोक्तवाक्यजन्यबोधे शक्तिप्रमाजन्येऽपि प्रमात्वविरहेण व्यभिचारात्~। न च जनकज्ञानभ्रमत्वं शाब्दबोधायथार्थत्वे प्रयोजकमिति वाच्यं, भ्रमानुव्यवसाये व्यभिचारात्~। योग्यताज्ञानयाथार्थ्येन शाब्दयार्थार्थ्यात्~। शक्तिभ्रमाधीनस्यापि  बोधस्य यथार्थयोग्यज्ञानजन्यत्वेन यथार्थत्वोपपत्तेः~। 

\chapter{लक्षणाविचारः}

	\section{शक्तिज्ञानस्य शाब्दबोधहेतुताया विमर्शः}

		एतावतापि ------ ईश्वरसङ्केतो वेति~।
		
		ननु पदात् अर्थप्रतीतौ तद्धीजनकत्वज्ञानमेव हेतुरास्ताम्~। शक्तिज्ञानस्य हेतुतायाः निर्युक्तिकत्वात् अत एव लौकिकानां ईश्वरसङ्केताज्ञानेऽपि शक्तेः अज्ञाने वृद्धव्यवहारात् अर्थज्ञानं प्रति शब्दस्य जनकत्वमवधार्य अग्रे शब्दादर्थप्रत्ययः उपपद्यते~। न चैवं सति अर्थप्रत्यायकतावच्छेदकशक्तेरसिद्ध्या तद्घटितत्वेन लक्षणाया अपि उच्छेदात् गङ्गापदात् मुख्यार्थप्रतीत्यपेक्षया लक्ष्यार्थतीरादिप्रतीतेः विलम्बोऽनुभूयमानः नोपपद्यते~। उभयत्र जनकत्वज्ञानरूपकारणसाम्यात् इति वाच्यम्~। तद्विषयकप्रतीतेः तद्विषयकबोधजनकत्वज्ञानजन्यत्वे अविलम्बः, तदान्यविषयकबोधजनकत्वज्ञानजन्यत्वे तु विलम्बः इति व्यवस्थायाः सम्भवात्~। तीरप्रतीतेः तीरान्यप्रवाहविषयकबोधजनकत्वज्ञानजन्यत्वेन विलम्बोपपत्तेः~। 

		न च पदज्ञाननिष्ठा अर्थबोधजनकता ज्ञाप्यसम्बन्धाधीना ज्ञानजनकतात्वात् इन्द्रियनिष्ठप्रत्यक्षजनकतावत् इत्यनुमानेन ज्ञाप्यार्थसम्बन्धतया शक्तिः सिद्ध्यति~। यथा प्रत्यक्षविषयसम्बन्धतया इन्द्रियस्य सन्निकर्षः इति वाच्यम्~। उच्यते – देवदत्तादिपदात् अर्थप्रतीतौ सङ्केतज्ञानस्य कारणत्वावधारणात् अन्यत्रापि गवादिपदेषु तत्कल्पनम्~। न च सादिदेवदत्तादिपदे सङ्केतावगाहिज्ञानस्य अर्थप्रत्यायकत्वेऽपि अनादिगवादिपदे न तथेति वाच्यम्~। अर्थप्रत्ययं प्रति लाघवात् पदत्वस्यैव प्रयोजकत्वसम्भवेन सादित्वस्य प्रयोजककुक्षौ अनिवेशात्~। 

	\section{लक्षणाविचारः} 

		वृत्तिश्च शाब्दबोधहेतुपदार्थोपस्थित्यनुकूलः पदपदार्थयोः सम्बन्धः~। सङ्केत एव मुख्या वृत्तिः~। तात्पर्यनिर्वाहकत्वात्~। लक्षणा च वृत्त्यन्तरम्~। यत्र वाच्यार्थान्वयानुपपत्त्या वाच्यसम्बन्धोपस्थापिते वाक्यार्थान्वयः~। यथा ‘गङ्गायां घोष’ इत्यत्र गङ्गापदस्य तीरे~। तदुच्यते – ‘वाच्यस्यार्थस्य वाक्यार्थे सम्बन्धानुपपत्तितः~। तत्सम्बन्धवशप्राप्तस्यान्वयात् लक्षणोच्यते’~॥ इति~। न च तीरे शक्तिरेव, शक्तिग्राहकव्यवहारस्य मुख्यलक्ष्यसाधारणत्वात्~। शाब्दत्वस्य तात्पर्यस्य च शक्तिनियतत्वात्~। तीरे यदि तात्पर्यमवधृतं तदा तन्निर्वाहकत्वेन क्लृप्ततया शक्तेरेव कल्पनीयत्वात्~। न तु अक्लृप्तवृत्त्यन्तरं कल्पयितुम् उचितम् इति वाच्यम्~। गङ्गापदस्य लक्षणीयतीरादिसहस्रेषु प्रत्येकं शक्तिकल्पनायां तद्ग्रहार्थं प्रत्येकं वृद्धव्यवहारकल्पनायाञ्च गौरवम्~, अननुभवश्च~। लक्षणा तु शक्यसम्बन्धरूपा, वृद्धव्यवहारान्तरानपेक्षा चेति लाघवम्~। न च गौरवमपि न्याय्यमेव, प्रामाणिकत्वात्~। गङ्गायां घोष इत्यादौ तीराद्यन्वयबोधस्य तीरे शक्तिग्रहं विना अनुपपत्तेः~। तीरादिसहस्रेषु शक्तिकल्पनेन गौरवस्य प्रामाणिकत्वात् इति वाच्यम्~। शक्तिं विनापि तीराद्यन्वयबोधोपपत्तेः~। तथा हि ---- न शक्तिः~। 

\chapter{पञ्चमोऽध्यायः परिशिष्टविचारः}

	\section{आकृतेः पदशक्यत्वम्} 
	
		गोपदात् जात्याकृतिविशिष्टस्यैव अनुभवात् आकृतावपि शक्तिः स्वीक्रियते~। आकृतिश्च अवयवसंयोगः~।  तथा च गोपदस्य अवयवसंयोगः, गोत्वजातिः एतदुभयविशिष्टव्यक्तौ शक्तिरिति फलितम्~।  ”पिष्टकमय्यो गावः” इत्यत्र तु गोशब्दस्य गवाकृतिसदृशाकृतौ लक्षणा~। एवं जात्याकृतिव्यक्तीनां प्रत्येकमात्रस्य बोधनेऽपि लक्षणा~।  यथा ”गौर्नित्या” इत्यत्र जातिमात्रपरत्वं गोपदस्य~।  यथा च गुरुमते कार्यशक्तायाः लिङः लोके  कार्यत्वमात्रपरत्वे~।  समुदायशक्तस्य पदस्य प्रत्येकस्मिन् लाक्षणिकत्वमिति नियमात्~। 

		जात्याकृतिव्यक्तिषु मिलितासु एकैव शक्तिः सूत्रकृता ”व्यक्त्याकृतिजातयस्तु पदार्थः” इत्येकवचनप्रयोगेण सूच्यते~।  यत्पदात् नियमतो यत् प्रतीयते तस्य तत्पदशक्यत्वम् इति नियमात् गोपदात् जात्याकृतिव्यक्तीनां तिसृणामपि प्रतीतेः तत्त्रितयस्य गोपदशक्यत्वम् आवश्यकम्~। 

		\begin{small}

			तस्मादेकवित्तिवेद्यत्वनियमात् जातिविशिष्टं शक्यम्~। यदि च तृतीयायाः करणैकत्व इव गोत्वे शक्ये तदा गोत्वं गोव्यक्तिश्चेति धीः स्यात् न तु गौरिति~। वैशिष्ट्यञ्च सम्बन्धो वा ज्ञातोघट इत्यत्रेव विशेषणताविशेषोऽर्थान्तरं वेत्यन्यदेतत्~। जातिविशेषवदवयवसंयोगरूपाकृतिरपि पदशक्या गोपदात् जात्याकृतिविशिष्टस्यैवानुभवात्~। पिष्टकमय्यो गाव इत्यादौ गवाकृतिसदृशाकृतौ लक्षणा पिष्टकसंयोगविशेषस्याशक्यत्वात्~। जात्याकृतिव्यक्तीनां प्रत्येकमात्रपरत्वे लक्षणैव~। प्रत्येकस्य जात्याकृतिविशिष्टादन्यत्वात्~। यथा गुरूणां कार्यशक्ताया लिङ्गो लोके कार्यत्वपरत्वे~। अत एव ’व्यक्त्याकृतिजातयस्तु पदार्थ’ इति पारमर्षसूत्रम्~। एकयैव शक्त्या एकवित्तिवेद्यत्वसूचनाय पदार्थ इत्येकवचनम्~। 
		\end{small}

\chapter{उपसंहारः}
	
	अवयवसंयोगरूपाकृतिरपि पदशक्या~। गोपदात् जात्याकृतिविशिष्टस्यैव अनुभवात्~। 

	अत्र चिन्त्यते – व्यक्तिवत् अननुगतानां अवयवसंयोगरूपाकृतीनां शक्यत्वे शक्यानन्त्यं दुर्वारम्~। न च जातिविशेषवान् यो अवयवसंयोगः संस्थानविशेषः तद्रूपाकृतीनां शक्यत्वम्~। तथा च जातिविशेषस्यैव अनुगमकतया नोक्तदोष इति वाच्यम्~। अन्यतरकर्मजत्वादिना सङ्करेण तादृशजात्यप्रसिद्धेः~। न च उपाधिरेवानुगतः सुलभः इति वाच्यम्~। तादृशोपाधेः अनिरुक्तेः~। न च ‘जातिविशेषवदवयवसंयोगः’ इति मणौ जातिविशेषवत्त्वं अवयवविशेषणम्~। तथा च कपालसंयोगादिकमेव तथेति वाच्यम्~। गवादिपदे सास्नाद्यवयवस्याप्यननुगतत्वेन दोषतादवस्थ्यात्~। 

	अत्रोच्यते – जातिविशेषवतः गवादेः अवयवसंयोगस्य आकृतेः शक्यत्वम्~। विशेषपदोपादानेन पृथिवीपदादौ नाकृतिः शक्या इति लब्धम्~। तथा च गवावयवसंयोगत्वादिकमेव अनुगतम् इति न शक्त्यानन्त्यम्~। न चैवं तत्प्रकारकधीः स्यादिति वाच्यम्~। इष्टापत्तेः~। गवादौ आकृतिवैशिष्ट्यञ्च परम्परासम्बन्धेन~। ‘पिष्टकमय्यो गावः’ इत्यत्र तु गोपदस्य गवाकृतिसदृशाकृतौ लक्षणा~। पिष्टकसंयोगविशेषस्याशक्यत्वात्~। एवं जात्याकृतिव्यक्तीनां प्रत्येकमात्रपरत्वे लक्षणैव~। समुदायशक्तस्य पदस्य प्रत्येके लाक्षणिकत्वात्~। न च ‘गौरुत्पन्नः’ इत्यत्र व्यक्तिमात्रपरत्वेऽपि न लक्षणेति वाच्यम्~। तत्र व्यक्तिमात्रपरत्वाभावात्~। तद्व्यक्तित्वादिना तदुपस्थितौ यत्र तात्पर्यं तत्र लक्षणा इष्यत एव~। ‘व्यक्त्याकृतिजातयस्तु पदार्थः’1 इति पारमर्षं सूत्रम्~। तत्र च एकयैव शक्त्या व्यक्त्याकृतिजातीनां बोध इति सूचनाय पदार्थ इत्येकवचनम्~।

\end{document}
