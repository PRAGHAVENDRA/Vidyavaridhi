\section{पाकाधिकरणविषये मतवैलक्षण्यम्}

पीलुपाकवादः पिठरपाकवादः इति इमौ दर्शनप्रपञ्चे प्रसिद्धावेव | नवशरावादौ अग्निसंयोगात् पूर्वरूपादीनां नाशः रूपान्तरस्य उत्पत्तिः लोके दृष्टः | तच्च परिवर्तनं पीलावेव भवति इति पीलुपाकवादिनो वैशेषिकाः | 

तथा हि - तेजस्संयोगात् अवयविनाशात् स्वतन्त्रेषु परमाणुषु पूर्वरूपादीनां नाशः पुनस्तेजस्संयोगात् रूपातन्तरोत्पत्तिरिति वैशेषिकानामाशयः | तत्र रूपातन्तरोत्पत्तौ पञ्चषडष्टनवदशादिक्षणाः अपेक्षिताः इति नानाप्रक्रियाः वैशेषिकग्रन्थेषु उपलभ्यन्ते | तेष्वष्टसु क्षणेषु रूपान्तरोत्पत्तिः अवयविनि जायते इति प्रसिद्धं वर्तते | अग्निसंयोगवशात् परमाणौ द्व्यणुकनाशानुकूलक्रिया, ततः पूर्वदेशविभागः तदुत्तरं संयोगनाशः, तस्मात् असमवायिकारणनाशात् द्व्यणुकनाशो जायते | इदानीं स्वतन्त्रेषु परमाणुषु प्रथमक्षणे पाकवशात् श्यामरूपस्य नाशो जायते | ततः द्वितीयक्षणे पाकवशात् रक्तोत्पत्तिर्भवति | ततः तृतीयक्षणे अदृष्टवदात्मसंयोगात् परमाणुषु द्व्यणुकारम्भकक्रिया उत्पद्यते | ततः‌ चतुर्थे पूर्वदेशविभागो जायते | ततः पञ्चमे पूर्वसंयोगनाशो भवति | ततः षष्ठे परमाणुद्वयसंयोगो उत्पद्यते | ततः सप्तमे असमवायिकारणवशात् द्व्यणुकमुत्पद्यते | अष्टमे कारणगुणाः कार्यगुणानारभन्ते इति न्यायानुसारं द्व्यणुके परमाणुगतरक्तरूपवशात् रक्तोत्पत्तिर्भवति इति वैशेषिकसिद्धान्तः |

\subsection{पार्थिवपरमाणुषु पाकजानामुत्पत्तिप्रदर्शनम्}

तत्र हि घटादिगतपरमाणौ द्व्यणुकारम्भकसंयोगनाशकविभागजनिका क्रिया उत्पद्यते | अस्यां क्रियां प्रति अग्न्यभिघातः\ नोदनं वा कारणम् | न च केवलावयविषु अग्निसंयोगस्यादौ दर्शनात् तत्रैव तादृशाग्निसंयोगात् रूपनाशः रूपान्तरोत्पत्तिश्च जायतामिति वाच्यम् | अतितीक्ष्णस्याग्नेः क्षणाभ्यन्तरे सर्वावयवेन सह संयोगस्य अभ्युपगमात् केवलावयविरूपस्य परावर्तकत्वस्याग्नेः कल्पनासम्भवात् | अवयविनि अपि रूपपरावर्तनं भवतु इत्यपि नाशङ्कनीयम् | अवयवसंयोगादग्नेः अवयवेषु परमाणुषु क्रियोत्पत्त्या द्व्यणुकादिनाशक्रमेण अवयविनः एव नाशात् आश्रयाभावादेव तत्र न रूपपरावृत्तिः जायते | तस्मात् अवयविनाशः अवश्यमभ्युपगन्तव्यः | ततः स्वतन्त्रेषु परमाणुषु अन्यस्मादग्निसंयोगात् पूर्वरूपस्य श्यामादेः नाशो जायते | ततः अन्यस्मादग्निसंयोगात् रूपान्तरं रक्तादिकमुत्पद्यते | अत्र श्यामादि नाशकः एव अग्निसंयोगः न रूपाद्युत्पादकः‌ | तथा सति उत्पादकस्यैव नाशकत्वप्रसङ्गः | ?? ततः द्व्यणुकारम्भकसंयोगजनिका क्रिया परमाणुषु उत्पद्यन्ते | तत्र अदृष्टवदात्मसंयोग एव कारणम् | केवलात्मसंयोगस्य तद्धेतुत्वे यदा कदापि परमाणुषु क्रियोत्पत्तिप्रसङ्गः | आत्मनः विभुत्वात् तेन सह मूर्तानां परमाणूनां संयोगः सर्वदा वर्तत एव | अतः अदृष्टविशिष्टस्यात्मनः इत्युक्तम् | केवलादृष्टस्य तु क्रियासाधारणकारणत्वाभावात् आत्मसंयोगः अवश्यमभ्युपगन्तव्यः | ततः विभागादिक्रमेण द्व्यणुकोत्पत्तौ तत्र परमाणुगत रक्तरूपादिना रूपाद्युत्पत्तिरिति | अवयविगतरूपं प्रति अवयवगतरूपस्य स्वसमवायिसमवेतत्त्वसम्बन्धेन कारणत्वाभ्युपगमात् पटे स्वसमवायिसमवेतत्त्वसम्बन्धेन तन्तुरूपस्य श्वेतादेः विद्यवानत्वात् पटेऽपि समवायेन श्वेतोत्पत्तिदर्शनाच्च परमाणुगतरूपस्यैव द्व्यणुकगतरूपं प्रति असमवायिकारणत्वं स्वीक्रियते | न तु पाकस्य | अत एव 'कारणगुणाः कार्यगुणानारभन्ते' इति न्यायोऽपि सङ्गच्छते |

{\fontsize{11.7}{0}\selectfont\s पार्थिवपरमाणुरूपादीनां पाकजोत्पत्तिविधानम् । घटादेः आमद्रव्यस्याग्निना सम्बद्धस्याग्न्यभिघातान्नोदनाद्वा तदारम्भकेष्वणुषु कर्माण्युत्पद्यन्ते । तेभ्यो विभागा , विभागेभ्यः संयोगविनाशाः , संयोगविनाशेभ्यश्च कार्यद्रव्यं विनश्यति । तस्मिन् विनष्टे स्वतन्त्रेषु परमाणुष्वाग्निसंयोगादौष्ण्यापेक्षाच्छ्यामादीनां विनाशः , पुनरन्यस्मादग्निसंयोगादौष्ण्यापेक्षात् पाकजा जायन्ते । तदनन्तरं भोगिनामदृष्टापेक्षादात्माणुसंयोगादुत्पन्नपाकजेष्वणुषु कर्मोत्पत्तौ तेषां परस्परसंयोगाद् द्व्यणुकादिक्रमेण कार्यद्रव्यमुत्पद्यते । तत्र च कारणगुणप्रक्रमेण रूपाद्युत्पत्तिः ।  न च कार्यद्रव्य एव रूपाद्युत्पत्तिर्विनाशो वा सम्भवति , सर्वावयवेष्वन्तर्बहिश्च वर्तमानस्याग्निना व्याप्त्यभावात् । अणुप्रवेशादपि च व्याप्तिर्न सम्भवति , कार्यद्रव्यविनाशादिति ।\footnote{प्र.भा.}}

\subsection{अवयविषु पाकोत्पत्तिनिराकरणपूर्वकं स्वमतस्थापनम्}

नन्वेवमपि अवयविनि अग्नोसंयोगस्य सम्भवात् तत्रापि पाकादेव रूपपरावर्तनं भवतु | यतः दृश्यन्ते च अग्नौ प्रक्षिप्तस्यामघटस्य चेदनाभावः एवं पाकोत्तरमपि 'स एवायं घटः' इति प्रत्यभिज्ञानदर्शनात् अवयविन्यपि पाकाद्रूपपरावृत्तिर्भवति इति चेन्न | अवयवित्वं तावत् घटस्यान्तःप्रदेशेऽपि विद्यते, तत्प्रदेशस्यापि अवयवसमवेतत्वात् | तत्र बहिः विद्यमानस्याग्नेः संयोगासम्भवात् अवयविनि सर्वत्र अग्निसंयोगो न व्याप्तः | तस्मात् कारणाभावादेव अवयविषु पाकानुत्पत्तिः सिद्ध एव | 

ननु अवयव्यवष्टब्धेष्वेवावयवेषु पाकाद्रूपोत्पत्त्या अवयविनाशस्तु न कल्पनीयः | अवयवसंयोगस्य एकदेशे विद्यमानत्वेऽदेशान्तरावच्छेदेन अग्निसंयोगजननसम्भवात् अवयव्यवष्टब्धेष्वेव अवयवेषु अग्निसंयोगात् रूपादिपरावृत्तिसम्भवात् अवयविनाशो न कल्पनीय इति वाच्यम् | यद्यपि स्थूलद्रव्यावयवेषु एतादृशस्यान्तरस्य अप्रतिभानात् परमाण्वादिष्वेव तत्कल्पनीयम् | तत्र च एतादृशातिरिक्तांशकल्पनापेक्षया अवयविनाशकल्पने एव लाघवम् | एवं क्रियाविशिष्टाग्निसंयोगात् अवयवेषु क्रियानुत्पत्तिरपि अनुभवविरुद्धं कल्पनीयम् | तदपेक्षया अवयविनाशकल्पनमेव वरमिति स्वतन्त्रेष्वेव परमाणुषु पाकाद्रूपपरावृत्तिरिति |

{\fontsize{11.7}{0}\selectfont\s अथ कथं कार्यद्रव्ये एव रूपादीनामग्निसंयोगादुत्पादविनाशौ कल्प्येते ? प्रतीयन्ते हि पाकार्थमुपक्षिप्ता घटादयः सर्वावस्थासु प्रत्यक्षाश्छिद्रविनिवेशितदृशाः , प्रत्यभिज्ञायन्ते च पाकोत्तरकालमपि त एवामी घटादय इति चेत् । तत्रोच्यते अन्तर्बहिश्च सर्वेष्वयवेषु वर्तमानस्य समवेतस्यावयविनो बाह्ये वर्तमानेन  वह्निना व्याप्तेर्वापकस्य संयोगस्याभावात् कार्यरूपादीनामुत्पत्तिविनाशयोरक्लृप्तेरन्तर्वर्तिनामपाकप्रसङ्गादिति भावः । सच्छिद्राण्येवावयविद्रव्याणि । न तावत्परमाणवः सान्तराः , निर्भागत्वात् । द्व्यणुकस्य सान्तरत्वे चानुत्पत्तिरेव , तस्य परमाण्वोरसंयोगात् । संयुक्तौ चेदिमौ निरन्तरावेव । सभाग्ययोर्हि वस्तुनोः केनचिदंशेन संयोगात् केनचिदसंयोगात् सान्तरः संयोगः । निर्भागयोस्तु नायं विधिरवकल्पते । स्थूलद्रव्येषु प्रतीयमानेष्वन्तरं न प्रतिभात्येव , त्रणुकेष्वेवान्तरम् । तच्चानुपलब्धियोग्यत्वान्न प्रतीयत इति गुर्वीयं कल्पना । तस्मान्निरन्तरा एव घटादयः । तेषामन्तस्तावदग्निप्रवेशो नास्ति यावत्पार्थिवावयवानां व्याप्ति भेदो न स्यात् । स्पर्शवति द्रव्ये तथाभूतस्य द्रव्यान्तरस्य प्रतिघाताद् व्यतिभिद्यमानेषु चावयवेषु क्रियाविभागादिन्यायेन द्रव्यारम्भकसंयोगविनाशादवश्यं द्रव्यविनाश इति कुतस्तस्याणुप्रवेशादभिव्यक्तिः । तस्मात् परमाणुष्वेव पाकोत्पत्तिरिति ।\footnote{न्या.कं.}}


पाकजाः रूपादयः पृथिव्यवयविषु अपि जायन्ते | अवयवनाशो नानुभवसिद्धः | पाकजरूपाद्युत्पत्त्यनन्तरमपि पूर्वघटसादृश्यस्य तत्र भानात् | तस्मात् अवयविनाशादिकं कल्पनीयम् | अनन्तावयवनाशतदुत्पत्त्यादीनां कल्पनापेक्षया अवयविष्वपि पाकजरूपाद्युत्पत्तिकल्पनमेव लघुभूतमिति न्यायविदामाशयः |

\subsection{अवयविनि अवयवे च पाकजरूपाद्युत्पत्तिकथनम्}

समानविषयेषु कल्पनापेक्षया प्रत्यक्षस्य प्रबलप्रमाणत्वात् अग्निनिक्षिप्तामघटादेः रूपादिपरावर्तनकाले नाशादर्शनात् रूपपरावर्तनानन्तरमपि 'स एवायं घटः' इति प्रत्यभिज्ञादर्शनात् अवयव्यवष्टब्धेष्वपि अवयवेषु पाकाद्रूपादिपरावर्तनम् अभ्युपगन्तव्यम् | न च तदा सर्वत्रावयविषु तदवष्टब्धावयवेषु च अग्निसंयोगाभावदर्शनात् सर्वत्र रूपपादिपरावर्तनं न स्यादिति वाच्यम् | शरावान्तःप्रवशिष्टजलादिगतशीतस्पर्शस्य शरावबहिर्भागेऽपि त्वाचप्रत्यक्षत्वात् अवयव्यवष्टब्धावयववैरपि द्रव्यान्तरसंयोगो भवतीति सिद्धम् | अन्यथा अत्रापि बहिः शीतस्पर्शोपलब्ध्युपपादनाय शरावनाशः कल्पनीयः स्यात् | तस्मात् अवयव्यवष्टब्धेष्ववयवेषु अवयवीषु च पाकाद्रूपादिपरावर्तनमङ्गीकरणीयमेव | क्वचित् अग्निसंयोगात् अवयविनाशोऽपि दृष्टः | यथा अग्निसंयोगात् पटादीनां नाशः | यथा वा अग्निसंयुक्तघटादीनां काष्टाभिघातात् नाशः | तत्र तु अवयवान्तरोत्पत्तिरपि अनुभवविरुद्ध एव | तत्रापि पटादीनां नाशः‌ अग्निसंयोगात् भवति चेदपि घटादीनां नाशस्तु नाग्निसंयोगात् अपि तु काष्टाभिघातादेव |

{\fontsize{11.7}{0}\selectfont\s अपरे पुनः प्रत्यक्षबलवत्तया घटादेरविनाशमेव पच्यमानस्य मन्यन्ते | सुषिरद्रव्यारम्भाच्च अन्तर्बहिश्च पाकोऽप्युपपत्स्यते | दृश्यते च पक्वेऽपि कलशे निषिक्तानामपां बहिः शीतस्पर्शग्रहणम् | अतश्च पाककाले ज्वलदनलशिखाकलापानुप्रवेशकृतविनाशवत् तदपि शिशिरतरनीरकणनिकरानुप्रवेशकृतविनाशप्रसङ्गः | न चेदृशी प्रमाणदृष्टिः | अतः प्रकृतिशुषिरतयैव कार्यद्रव्यस्य घटादेरारम्भात् अन्तरान्तरा तेजः कणानुप्रवेशकृतपाकोपपत्तेरलं विनाशकल्पनया | पिठरपाकपक्ष एव पेशलः ||\\ यादृतेव हि निक्षिप्तः घटः पाकाय कन्दुके |\\ पाकेऽपि तादृगेवासौ उद्धतो दृश्यते ततः ||\\ 

क्वचित्तु सन्निवेशान्तरदर्शनं काष्ठाद्यभिघातकृतमुपपत्स्यते | पावकसम्पर्ककारित्वे तु सर्वत्र तथाभावः स्यात् | तस्मादविनष्टा एव घटादयः पच्यन्ते |\footnote{न्या.म. २८७,२८८}}

\subsubsection{विमर्शः}

पृथिव्यां रूपादीनां पाकजगुणानामुत्पत्तिक्रमः विमृश्यते | तत्र हि वेगवदग्न्यभिघातात् परमाणौ द्व्यणुकारम्भकसंयोगप्रतिद्वन्द्विक्रिया जायते | ततः विभागसंयोगनाशक्रमेण द्व्यणुकादीनां माहावयविपर्यन्तानां नाशे जाते स्वतन्त्रेषु परमाणुषु अग्निसंयोगात् श्यामादिरूपानां नाशः पाुनरन्यस्मादग्निसंयोगात् रक्तादीनामुत्पत्तिश्च जायते | ततः भोगिनामदृष्टसहकृतात्मसंयोगवशात् परमाणुषु द्व्यणुकारम्भकक्रिया उत्पद्यते | ततः विभागादिक्रमेण द्व्यणुकमुत्पद्यते | तत्र च कारणगुणप्रक्रमेण रक्तरूपमुत्पद्यते | एवं रीत्या महावयविन्यपि रक्तोत्पत्तिर्भवति इति वैशेषिकानां सिद्धान्तः |

न च पृथिव्यां पाकजरूपादिकं परमाणुष्वेव उत्पद्यन्ते, न तु घटादिषु इत्यत्र किं प्रमाणम् ? किञ्च घटादीनां नाशोत्पत्तौ बाधकमस्ति | तथा हि - अवयव्यवष्टब्धेषु परमाणुषु क्रियोत्पत्तौ न किञ्चित्प्रमाणमस्ति | एवं पाकोत्तरमपि 'स एवायं घटः' इति प्रत्यभिज्ञोपलब्धेः | किञ्च यदि उत्पन्नपाकजाः परमाणवः द्व्यणुकादिक्रमेण घाटादिकामारभेरन् तदा तत्र तत्सङ्ख्याकतत्परिमाणकघटादीनामेवोत्पत्तिरिति कथं स्यात् ? तथा नियमाभावात् | किन्तु तस्मिन्नेव देशे तावत्परिमाणकतत्सङ्ख्याक एव घटः पाकानन्तरमपि दृष्टः | तस्मात् पाकाद्रूपान्तरोत्पत्तौ न द्रव्यनाशो कल्पनीयः इति वाच्यम् | यथा घटोपरि दण्डादिना ताडने सति तन्नश्यति, तत्र घटावष्टब्धेष्वेव कपालेषु क्रियोत्पत्या विभागाघटनाशो भवति तथैव द्व्यणुकावष्टब्धेष्वपि परमाणुषु क्रियाभवितुमर्हति | तदेवात्र कल्प्यते | न च प्रतिभिज्ञापूर्वोत्तरपदार्थयोरभेदं सर्वत्र साधयति | अन्यथा तदेवौषधमित्यादिप्रत्यभिज्ञानमेव न स्यात् | तस्मात् साजात्यस्थलेऽपि प्रत्यभिज्ञायाः दर्शनात् प्रकृतेऽपि प्रत्यभिज्ञायाः साजात्यबोधकत्वमेव | एवं वेगवतः तेजसः अभिघातात् अवयव्यवष्टब्धेष्वपि परमाणुषु क्रियोत्पत्तेः स्वीकारात् ततः अवश्यः द्व्यणुकादिक्रमेणावयविनाशो भवत्येवेति सिद्धम् | पाकानन्तरं तद्देशे तत्सङ्ख्याकस्तत्परिमाणकश्च घटो अनुभूयते | तस्मात् प्रत्यभिज्ञानाच्च तत्र तादृशस्यैव घटस्योत्पत्तिः कल्प्यते |

न च पाकनिक्षिप्तेषु तृणादिना आवृतेष्वपि घटादिषु छिद्रप्रदेशेन नीरीक्षमाणस्य कुलालस्य अक्षजप्रतीतिः 'घटः पच्यत' इति | न तु 'परमाणुः पच्यत' इति | एवं तस्य घाटनाशविषयकप्रतीतेरप्यनुदयात् अवयविन्यपि पाको अनुभवबलादभ्युपेयः इति वाच्यम् | घटस्तावत् साक्षात् परमाणुसमूहादारब्धः येन परमाणुगतक्रियया विभागादिना तस्य अविलम्बेन नाशो भवति | अपि तु द्व्यणुकाक्तमेणैव आरब्धः | तस्मात् द्व्यणुकनाशात् तत्स्यंयोगनाशे त्र्यणुकनाश इत्येवं क्रमेण असङ्ख्येयद्रव्यादिनाशपरम्परया घटनाशो विलम्बेनैव भवति | एवमेकत्र अवयवाः विनश्यन्ति, अन्यत्र तस्मिन्नेव देशे पक्तावयवात् संयोगादिना अवयवान्तरमुत्पद्यते इत्यप्यनुभवः | अत एव एकस्मिन्नेव घटे देशान्तरावच्छेदेन पक्वता अन्यदेशावच्छेदेन अपक्वता च कदाचिद्दृश्यते | तस्मात् अवयवनाशात् घटभेदेऽपि घटान्तरस्य तत्र सम्भवात् तेन सह सन्निकर्षाच्च 'घटः पच्यत' इत्यादिव्यवहाराः उपपद्यन्ते | एवमनयारीत्या पाकोत्तेः पूर्वं घटारम्भकावयवाः यावन्तः तावन्त एव पाकानन्तरमपि घटमुत्पद्यन्ते इत्यतः तस्य समानपरिमाणकत्वं समानसङ्ख्याकत्वञ्च उपपद्यते |

किञ्च गुणवति तत्सजातीयगुणान्तरानुत्पत्तेः रूपान्तरोत्पत्तौ पूर्वरूपनाशो अवश्यं वक्तव्यः | तच्च द्रव्यनाशादेव भवति | आश्रयनाशादेव रूपादीनां नाशः कारणगुणादेव रूपादीनाञ्च उत्पादः लोके दृष्टः | यथा दण्डाद्यभिघाते सति घटस्य नाशात् तद्गतरूपादीनां नाशो भवति | एवं घटोत्पत्त्यनन्तरमेव घटे रूपमुत्पद्यते | तस्मात् घटादीरूपाणां नाशं प्रति घटादीनां नाशः पटादीनां रूपादिकं प्रति तन्त्वादिगतरूपादीनां च कारणत्वमभ्युपगम्यते | एवं घटाद्यवयविषु अग्निसंयोगेन रूपादिपरावर्तनमप्यनुपपन्नमेव | तद्यथा घटरूपावयवी स्वावयवेषु सर्वत्र अन्तः बहिश्च वर्तते | पाकानन्तरं रूपोपलब्धिरपि घटे सर्वत्र भवति | एतच्च तदा सम्भवति यदा तादृशावयविनि सर्वत्र अग्निसंयोगः स्यात् | किन्तु घटस्य बहिः अग्निसंयोगदर्शनात् कथमन्तः पाकजरूपाद्युत्पत्तिरिति | न च तेजसः अतिसूक्ष्मत्वात् तेषामवयव्यवष्टेब्धेष्वपि अवयवेषु प्रवेशः सम्भवति | अतः सर्वत्र पाकाद्रूपपरावर्तनमिति वाच्यम् | तेजसः अणुदेशप्रवेशस्तु कार्यद्रव्यनाशं विना अनुपपन्नम् | तथा हि वेगवद्द्रव्यसंयोगस्य अन्यत्र क्रियाहेतुत्वदर्शनात् अत्रापि तादृशतेजसः परमाणुना सह संयोगे सति क्रिया उत्पद्यते | तथा च नाशकस्य सत्त्वात् विभागादिक्रमेण अवयविनाशो भवत्येव | तथा च स्वतन्त्रेषु परमाणुष्वेव पादाद्रूपोत्पत्तिरिति | मैवम् |



गौरवान्मानाभावाच्च घटाद्यवयवीनां नाशो नैव कल्प्यते | यदुक्तं वेगवद्द्रव्यसम्बन्धस्य क्रियाजनकत्वमिति तन्न सार्वत्रिकम् | अत एव वेगवतः कन्दुकादेः भित्यासह संयोगे सति तत्र तदवयवे वा क्रिया उत्पद्यते | तथा च यत्र तादृशसंयोगे क्रिया उत्पद्यते तत्रैव  वेगवद्द्रव्यसम्बन्धस्य क्रियाजनकत्वं स्वीकर्तव्यम् | प्रकृते 'घटः पच्यते', 'स एवायं घटः' इत्यादिप्रतीतीनामुपपादनाय वेगवतः तेजसः परमाणुना सह संयोगे सत्यपि क्रिया नोत्पद्यते इत्युच्यते | 

न च प्रकारान्तरेण तादृशप्रतीतीनामुपपत्तिः सम्भवति | यथा परमाणुनाशोत्तरक्षणे एव घटनाशस्य अभावात् , क्वचिदवयवान्तरनाशः क्वचिदवयवान्तरोत्पात्तिरिति स्वीकारादिति वाच्यम् | तथापि नाशानन्तरमुत्पन्नस्य घटस्य तत्परिमाणकत्वसत्त्वेऽपि तदाकारकत्वमनुपपन्नम् | समानपरिमाणकत्वं न समानाकारकत्वे तन्त्रम् | अन्यथा समानसङ्ख्याकतन्तुभिरारब्धानां पटानां सामानाकारकत्वं स्यात् | किन्तु तथा लोके न दृश्यते | क्वचित् अवयवसंयोगविशेषात् आकारविशेषोऽपि दृश्यते | समानाकारकत्वानुपपत्तौ च प्रत्यभिज्ञानुपपत्तिः तदवस्थैव | नापि समानाकारकत्वं कल्पयितुं शक्यम् | सर्वत्र पाकजस्थले तथाकल्पनापेक्षया अवयविनि पाकजरूपकल्पनमेव लघुभूतम् |

यदुक्तं गुणनाशं प्रति आश्रयीभूतद्रव्यस्य नाशः‌ कारणमिति, तदपि व्यभिचरितमेव | द्रव्यनाशाभावेऽपि संयोगविभागयोः नाशाभ्युपगमात् | न च रूपादीनामेव तथा इति वाच्यम् | परमाणौ व्यभिचारात् |‌ परमाणूनां नित्यत्वात् तेषां नाशासम्भवेऽपि तत्र रूपादिनाशस्तु भवतापि स्वीकृत इति व्यभिचारः | 

न च अवयविनि पाकाद्रूपोत्पत्तौ अवयवगतरूपं प्रतिबन्धकम् | तथा हि - तन्तौ नीलरूपसत्त्वे पटेऽपि नीलरूपमेव उत्पद्यते | न तु रूपान्तरमिति लोके दृष्टम् | तथा च रूपान्तरोत्पत्तौ अवयवगतरूपस्य प्रतिबन्धकत्वं कल्प्यते | अत एव नानरूपविशिष्टतन्तुभिरारब्धे पटे चित्ररूपं स्वीक्रियते इति वाच्यम् | अवयव्यवष्टब्धावयवेष्वपि रूपान्तरोत्पत्तिः कल्प्यते | 

न चैवमपि सर्वत्रावयविनि अग्निसंयोगस्यासम्भवात् सर्वत्र घटे पाकाद्रूपादिपरावर्तनं न स्यादिति वाच्यम् | तेजसः अतिसूक्ष्मत्वात् तेषां तादृशसामर्थ्यं कल्प्यते |‌ दृष्टञ्च लोके जलपूरितघटस्य बहिरपि शीतस्पर्शानुभवः | त च्च जलस्य घटावयवान्तःप्रवेशं विना नैव सम्भवति | यदि तत्रापि वेगवज्जलस्य सम्भवात् परमाणुषु क्रिया स्याति तर्हि घटनाशः स्यात् | नष्टे च घटे जलाधारत्वासम्भवात् जलस्य स्यन्दनं स्यात् | तस्मात् तत्र घटनाशासम्भवात् प्रकृतेऽपि अवयव्यवष्टब्धेषु घटपरमाणुषु तेजस्संयोगात् तत्रापि रूपपरावृत्तिः भवत्येव |‌ एवं द्व्यणुकादिक्रमेण सर्वत्र रूपान्तरोत्पत्तिरिति |

न च अवयविनि पाकाजगुणोत्पत्तिपक्षेऽपि  अवयव्यवष्टब्धेषु परमाण्वादिषु अग्निसंयोगः कल्पनीयः | एवं द्व्यणुकादिषु आश्रयनाशं विना रूपादीनां नाशः, कारणगुणं विना गुणोत्पतिः कल्पनीयः इति महद्गौरवमिति वाच्यम् | पीलुपाकवादिनापि परमाणावग्निसंयोगात् द्रव्यारम्भकसंयोगनाशकक्रिया, ततः विभागः, ततः संयोगनाशः, ततः अवयविनाशः इत्येवं महावयविनाशपर्यन्तं कल्पनीयम् | पुनरुत्पत्तिः, उत्पन्नस्य पूर्वद्रव्यसमानाकारकत्वं समानसङ्ख्याकत्वं, समानपरिमाणकत्वं च कल्पनीयमिति महद्गौरवम् | तस्मादवयविनि अवयव्यवष्टेब्धेषु अवयवेषु च पाकाद्रूपपरिवर्तनं भवति इति सिद्धम् |

इदन्तु बोध्यम् - अग्निसंयोगस्य क्वचित् क्रियाजनकत्वमपि स्वीकर्तव्यम् | यत्र उष्णस्पर्शाधिक्यं तत्र वेगवतः तादृशतेजसः संयोगे सति अवयवादिषु क्रिया जायते | तत्र च द्रव्यं न श्यति | किन्तु तत्र पुनः तद्द्रव्यस्योत्पत्तिः न भवति | अत एव पटादिना सह तेजस्संयोगे सति ते भस्म भवन्ति | क्वचित्तेजस्संयोगे रूपादिगुणानां परिवर्तनमात्रं भवति | तथा च यत्र द्रव्यनाश एव भवति तत्र तेजस्संयोगस्य क्रियाजनकत्वम् , यत्र तु रूपादिपरिवर्तनं तत्र तेजस्सम्बन्धस्य  न क्रियाजनकत्वमिति स्वीकर्तव्यम् | एतेन कार्यभेदेन कारणभेदमप्युपपद्यते इति |
