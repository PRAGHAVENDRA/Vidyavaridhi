\section{परमाणुः}

अथास्य परिदृश्यमानस्य पृथिव्याद्यात्मकस्य जगतः आरम्भकः कः ? इति जिज्ञासायां युक्त्या परमाणुरिति वदन्ति वैशेषिकाः न्यायविदश्च | सर्वेषां परिदृश्यमानानामवयविनां अवयवारम्भकत्वात् त्रसरेण्वादीनामपि तदबाधितम् | तस्य द्व्यणुकस्यापि महदारमभकत्वं सावयवत्वे एवोपपद्यते इत्यतः द्व्यणुकावयवत्वेन परमाणुः सिध्यति इति | यद्यपि प्रमाणपदार्थस्य विशदतया प्रतिपादने तत्पराः नव्यनैयायिकाः अस्मिन् विषये तथा विचारं न कृतवन्तः, तथापि श्रीरघुनाथशिरोपणिप्रभृतिभिः परमाणुः निराकृतो वर्तते | एवं परमाणुवृत्तिसंयोगस्य अव्याप्यवृत्तित्वविषये च मतभेदो दृश्यते | अतस्तदत्र निरूप्यते |

तत्र वैशेषिकास्तावत् परमाणूनामनुमानगम्यत्वमनेकत्वन्निरवयवत्वञ्चाभ्युपगच्छन्ति |

\subsection{परणुसत्त्वे प्रमाणम्}

किं तावत्परमाणु सत्त्वे प्रमाणमिति चेत् , अनुमानम् | तथा हि - अणुपरिमाणतारतम्यं क्वचिद्विश्रान्तं परिमाणतारतम्यत्वात् महत्परिमाणतारतम्यवत्' इति अनुमानस्वरूपम् | घटपटादिपरिमाणे तन्तुकपालादिपरिमाणापेक्षया औत्कट्यं दृश्यते | एवं परिमाणतारतम्यं यत्र यत्र द्रव्येषु दृष्टं तत्र सर्वत्रापि महावयविनि घटादौ तस्य विश्रान्तिरपि दृश्यते | यथा कपालाद्यवयवधारा उत्कृष्टपरिमाणवति घटे विश्रान्ता, तन्त्वाद्यवयधारा पटे विश्रान्ता | एवमेव अणुपरिमाणदारापि क्वचिद्विश्रान्ता, यत्र सा विश्रान्ता स परमाणुः | न च अणुपरिमाणस्य तद्वतश्च सद्भावे किं मानमिति चेत् , येषां घटादीनां साक्षात्कारो भवति ते सर्वेऽपि सावयवाः, अतिसूक्ष्मोपि त्रसरेणुः यो दृश्यते सोऽपि सावयव एव परिदृश्यमानत्वात् | तच्च न महत्परिमाणकम् , तस्य प्रत्यक्षत्वाभावात् | नापि परममहत्परिमाणकं अविभुत्वात् | तस्मात्तदतिरिक्तपरिमाणकमेव | तच्च परिमाणमणुपरिमाणमिति | न च त्रसरेण्ववयवाः एव परमाणवः | तस्यापि अवयवस्य सत्त्वात् | अन्यथा महद्रव्यारम्भकत्वमनुपपन्नम् | तस्मात्तस्यावयवाः परमाणवः इति | तदुक्तं न्यायकन्दल्याम् - 

{\fontsize{11.7}{0}\selectfont\s परमाणुस्वभावायाः पृथिव्याः सत्त्वे किं प्रमाणम् ? अनुमानम् | अणुपरिमाणतारतम्यं क्वचिद्विश्रान्तं परिमाणतारतम्यत्वात् महत्परिमाणतारतम्यवत् , यत्रेदं विश्रान्तं यतः परमाणुर्नास्ति सः परमाणुः | अत एव नित्यो द्रव्यत्वे सत्यनवयवत्वात् आकाशवत् अथायं सावयवो न तर्हि परमाणुः, कार्यपरिमाणापेक्षया तदवयवपरिमाणस्य लोकेऽल्पीयस्त्वप्रतीतेः | यश्च तस्यावयः सः परमाणुर्भविष्यति |\footnote{न्या.कं. ९४}}

\subsection{परमाणूनां सावयवत्वनिरासः}

अथ द्व्यणुकावयवत्वेन सिद्धस्यापि न परमाणुत्वम् , तस्यापि सावयवत्वादिति चेत् , तदा अवयवधारायाः विश्रान्त्यभावात् अनवस्था स्यात् | न चेष्टापत्तिः | लोके  अवयवसङ्ख्या हि अवयविनि परिमाणतारतम्यं जनयति | न्यूनसङ्ख्याकतन्तुभिर्जन्यः पटः अल्पः, अधिकसङ्ख्याकतन्तुभिर्जन्यः बृहदिति लोके व्यवहारः | तर्हि अवयवधारायाः अनन्तत्वात् अनन्तावयवकत्वं सर्वेषां कार्यद्रव्याणां सिध्यति | तथा च परिमाणतारतम्यानुपपत्तिः | अस्ति च लोके परिमाणतारतम्यम् , अतो अवयवधारापि क्वचिद्विश्रान्ता | यत्र सा विश्रान्त स एव परमाणुः | निरवयवत्वात्सः नित्यः | तदुक्तं न्यायकन्दल्याम् - 

{\fontsize{11.7}{0}\selectfont\s अथ सोऽपि न भवति अवयवान्तरसद्भावात् ? एवं तर्ह्यनवस्था, ततश्चावयविनामल्पतरतमादिभावो न स्यात् , सर्वेषामनन्तकारणजन्यत्वाविशेषेण परिमाणप्रकर्षाप्रकर्षहेत्वोः कारणसङ्ख्याभूयस्त्वयोरसम्भवात् | अस्ति तावदयं परिमाणभेदः, तस्मादणुपरिमाणं क्वचिन्निरतिशयमिति सिद्धो नित्यः परमाणुः |\footnote{न्या.कं. ९४}}



न्यायनयेऽपि परमाणुपदार्थः जगदारम्भकत्वेन प्रसिद्धः | तस्य सत्त्वे प्रमाणन्तावदनुमानमेव | तथा हि -

\subsection{परमाणुसाधकमनुमानम्}

'त्रसरेणुः\footnote{जालसूरमरीचिस्थं यत् सूक्ष्मं रजः तत्त्रसरेणुः} सावयवः दृश्यत्वात् घटवत्' इत्यनेनानुमानेन त्रसरेणोरवयवत्वेन द्व्यणुकं सिध्यति | 'तदपि सावयवः महदारम्भकत्वात् कपालवत्' इत्यनुमानेन द्व्यणुकावयवत्वेन परमाणुः सिध्यति | अस्य पुनरवयवाभावादस्य परमाणुत्वमुपपद्यते | एवं अवयवसंयोगनाशादवयविनां नाशो भवति | परिदृश्यमानं सर्वमपि नाशधर्मकमित्यतः त्रसरेणुरपि नश्यति | तस्य नाशार्थं तदवयवसंयोगनाशो अपेक्षितः | तादृशावयवसंयोगप्रतियोगितया‌ किञ्चिद्द्रव्यं सिध्यति | त्रसरेणुनाशकसंयोगनाशार्थं पुनस्तदाश्रयनाशापेक्षा, तन्नाशस्तु तदवयवसंयोगनाशात् , तत्संयोगाश्रयतया सिद्धं द्रव्यमेव परमाणुः‌ | तस्य पुनरवयवो नास्ति | अन्यथानवस्था स्यात् |

अत्र विकल्पप्रदर्शनपुरस्सरं परमाणुं व्यवस्थापयति | तथा हि - घटादयः महावयविनः निरवयवाः अनन्तावयवकाः परमाण्वन्तावयवकाः वा ? घटपटादीनां कपालतन्त्वाद्यवयवकत्वं प्रत्यक्षप्रमाणेनोपलब्धमित्यतः तेषामनित्यत्वाच्च प्रथमकल्पोऽवसितः | द्वीतीयकल्पोऽपि न युक्तः | लोके अवयवसङ्ख्यातरतम्यात् परिमाणतारतम्यो दृष्टः | अनन्तावयवकत्वे परिमाणतारतम्यानुपपत्त्या मेरुसर्षपयोः समानपरिमाणकत्वप्रसङ्गः | किन्तु तेषामसमानपरिमाणकत्वं प्रत्यक्षम् | तस्मात्परमाण्वन्तावयवका एव | तदुक्तं न्यायमञ्जर्याम् - 

{\fontsize{11.7}{0}\selectfont\s तथा हि पार्थिवमाप्यं तैजसं वायवीयमिति चतुर्विधं कार्यं स्वावयवाश्रितमुपलभ्यते, तत्र यथा घटः सावयवः कपालेश्वाश्रित  एवं कपालान्यपि सावयवत्वत्तदवयवेषु तदवयवा अपि तदवयवान्तरेष्वित्येवं तावद्यावत्परमाणवो निरवयवा इति, यत्र यावतः कार्यजातस्य स्वावयवाश्रितस्य प्रत्यक्षेण ग्रहणं तत्र तदेव प्रमाणम् , तदपि हि कार्यं स्वावयवत्वात् , परिदृश्यमानकार्यवत् | निरवयवत्वे तु तस्य परमाणुत्वमेव | परमाणुषु च सावयवत्वस्य च हेतोरसिद्धत्वान्नावयवान्तरकल्पना | तेषां हि सावयवत्वे तदवयवाः परमाणवो भवेयुः | न ते उत्पत्तिक्रमवत् विनाशक्रमेणापि परमाणवोऽनुमीयन्ते | लोष्टस्य प्रविभज्यमानस्य भागाः, तद्भागानां च भागान्तराणीत्येवं तावत् यावदशक्यभङ्गत्वमदर्शनविषयत्वं च भवति | तद्यतः परमवयवविभागो न सम्भवति, ते परमाणव उच्यन्ते | तेष्वपि हि विभज्यमानेषु तदवयवाः परमाणवो भवेयुर्न ते | तदेतदेवं उत्पत्तिक्रमवत् विनाशक्रमस्येदृशो दर्शनात् सन्ति परमाणवः |

अत्र हि त्रयी गतिः | अस्य घटादेः कार्यस्य निरवयवत्वमेव वा, अवयवानन्त्यं वा, परमाण्वन्तता वा ? तत्र निरवयवत्वमनुपपन्नम् , अवयवानां पटे तन्तूनां घटे च कपालानां प्रत्यक्षमुपलम्भात् | अनन्तावयवयोगित्वमपि न युक्तम् , मेरुसर्षपयोरनन्तावयवयोगित्वाविशेषेण तुल्यपरिमाणत्वप्रसङ्गात् | तस्मात्परमाण्वन्ततयैव युक्तिमती |\footnote{न्या.मं. ४२०}}

\subsection{निरवयवः परमाणुः}

लोष्टादिसाधनैः द्रव्यपरिमाणानामल्पतराल्पतमविभागो दृश्यते | तेषु यदल्पतमः तन्निरवयः परमाणुः | अयम्भावः - सावयवस्य तु सर्वापेक्षयाल्पतमत्वं न सम्भवति, तदवयवस्य ततोऽप्यल्पत्वात् | निरवयवस्य तत्सम्भवति, तस्यावयवाभावात् | तस्माद्यदपेक्षयाल्पतरपरिमाणकं द्रव्यं नास्ति सः निरवयवः परमाणूरिति |

तस्य निरवयवत्वन्तु अवयवाणामानन्त्यापत्तिवारणायकल्पितम् | अन्यथा समानावयवत्वात्सर्वेषां समानपरिमाणकत्वापत्तिः | तस्मात्परमाणुः निरवयव एव कल्प्यते | तदुक्तं न्यायभाष्ये - 

{\fontsize{11.7}{0}\selectfont\s निरवयवत्वं खलु परमाणोः, विभागैरल्पतरप्रसङ्गस्य यतः नाल्पीयस्तत्रावस्थानात् | लोष्टस्य खलु प्रविज्यमानावयवस्याल्पतरमल्पतममुत्तरमुत्तरं भवति, स चायमल्पतरप्रसङ्गः, यल्मान्नाल्पतरमस्ति यः परमोऽल्पस्तत्र निवर्तते | यतश्च नाल्पीयोस्ति तं परमाणुं प्रचक्ष्महे इति |\footnote{न्या.भा. ४.२.१६} अवयवविभागस्यानवस्थानाद् द्रव्याणामसङ्ख्येयत्वात् त्रुटित्वनिवृत्तिरिति |\footnote{न्या.भा. ४.२.१७}}

\subsection{तत्राव्याप्यवृत्तिसंयोगसाधनम्}

 ननु परमाणूनां निरवयवत्वं न युक्तम् | निरवयवस्य हि द्रव्यस्य दिगवच्छेदेन भेदो न दृष्टः, सर्वगतत्वात् गगनावत् | परमाणूनान्तु सर्वगतत्वाभावात् दिगवच्छेदेन तस्य सावयवत्वमेव | तथा च षड्दिगुपादिभिः विशिष्टस्य परमाणोः पुनः षडवयवाः सिध्यन्ति इति चेन्न | दिशः सर्वगतत्वादेकत्वात्तस्य षट्सङ्ख्याकत्वमनुपपन्नम् | षडुपाधीनां विरहादेव परमाणूनामपि न षडवयवकत्वम् | अस्तु वा 'अयमस्माद्विप्रकृष्टः' इत्यादिव्यवहारनिर्वाहार्थं सुमेरुपर्वताद्युपाधिभेदेन दिशः षट्त्वम् | तेन परमाणूनां न सावयवत्वं सिध्यति | तद्यथा एकस्यापि निरवयवस्य नानासंयोगाः भवितुमर्हन्ति | यथा देशभेदेन घटपटादिना सह गगनस्य संयोगः | तद्वत् परमाणूनामपि सर्वत्र परितः दिशस्संयोगो वर्तते | दिशः उपाधिभेदेन षटसङ्ख्याकत्वात् षटसंयोगाः भवितुमर्हन्ति | एतेन परमाणुवृत्तिसंयोगः अव्याप्यवृत्तिरिति सिद्धम् | एवञ्च निरवयवस्य परमाणोः अव्याप्यवृत्तिसंयोगासम्भवात् द्व्यणुकाद्यनुपपत्तिशङ्कापि निरस्ता | तदुक्तं तात्पर्यटीकायाम् -

{\fontsize{11.7}{0}\selectfont\s यत्पुनरुक्तं दिग्देशभेदो यस्यास्ति तस्यैकत्वं न युक्तमिति । परमाणोः किल भवदभिमतस्यैकस्य दिग्भागाः षट्, न चैकस्य दिग्भागे भेदोऽस्तीति षडेव परमाणवः । एतद्दूषयति क एवमाह दिग्देशभेदो यस्यास्तीति । स्वरूपेणैका दिक्सर्वगता च नास्या भेदोऽस्तीत्यर्थः । यद्येकैव दिक्क्व तर्हि परमाणावस्मादयं परमाणुः पूर्वोऽयं पश्चिम इत्यादयो बुद्धिव्यपदेशभेदा इत्यत आह । दिग्देशभेदाश्च दिशः संयोगा एकत्वेऽपि दिश आदित्योदयदेशप्रत्यासन्नदेशसंयुक्तो यः सैतरस्माद्विप्रकृष्टदेशसंयोगात्परमाणोः पूर्वःप एवमादित्यास्तमयदेशप्रत्यासन्नदेशसंयुक्तो यः स इततस्माद्विप्रकृष्टदेशसंयोगात्परमाणोः पश्चिमः तौ च पूर्वपश्चिमौ परमाणू अपेक्ष्य यः सूर्योदयास्तमयदेशविप्रकृष्टदेशसंयोगः स मध्ववतीम् । एवमेतयोर्यौ तिर्यग्देशसम्बन्धिनौ मध्यस्य आर्जवेन व्यवस्थितौ पार्श्ववर्तिनौ तौ दक्षिणोत्तरौ परमाणू एवं मध्यन्दिनवर्तिसूर्यसन्निकर्षविप्रकर्षौ पूर्वसंख्यावच्छिन्नत्वं चाल्पत्वं परसंख्यावच्छिन्नत्वं च भूयस्त्वम् । तस्मादेकस्यापि परमाणोः परमाण्वन्तरसंयोगा अव्याप्यवृत्तय एव भागाः ।\footnote{न्या.वा.ता.टी. ४.२.२५}}

\subsection{परमाणुद्वयसंयोगो व्याप्यवृत्तिः}

निरवयवानां परमाणूनामव्याप्यवृत्तिसंयोगानुपपत्तिशङ्कायां तस्य व्याप्यवृत्तिसंयोगः कल्प्यते | न च संयोगः सर्वोऽप्यव्याप्यवृत्तिरिति नियमविरोधः | तस्य दैशिकाव्याप्यवृत्तित्वाभावेऽपि कालिकाव्याप्यवृत्तित्वमुपपद्यते, अनित्यत्वात् | तेनैव नियममप्युपपद्यते | न च संयोगत्वस्वाश्रयसमवायिदेशवृत्तित्वयोः व्याप्तिर्विद्यते | येन परमाणुसंयोगः स्वाश्रयसमवायिदेशवृत्तिः संयोगत्वादित्यनुमानेन परमाणूनां सावयवत्वसिद्धिः | तत्र हि सावयववृत्तिसंयोगत्वमुपाधिः | सावयववृत्तिसंयोगत्वं च स्वाश्रयसमवायिदेशवृत्तिसंयोगे सर्वत्र वर्ततेत्यतः तस्य साध्यव्यापकत्वम् | विवादाध्यासिते परमाणुसंयोगे अन्यत्र गगनसंयोगे वा संयोगत्वस्य सत्त्वेऽपि सावयववृत्तिसंयोगत्वस्य विरहात् साधनाव्यापकत्वम् | न च संयोगानां सर्वेषां स्वाश्रयसमवायिदेशमवच्छेदकम् | तथा सति सर्वेषां सावयवत्वात् परिमाणतारतम्याभावात् जगदारम्भकत्वेन परमाणुकल्पनाया एवानुपपत्तेः | तस्मात्परमाणावपि संयोगो जायत एव | स च व्याप्यवृत्तिरिति आरम्भवादप्रणेतॄणां बदरीनाथशुक्लमहोदयानामाशयः - 

{\fontsize{11.7}{0}\selectfont\s न खलु परमाण्वोः संयोगासम्भवः सम्भावनीयः, तयोर्व्याप्यवृत्तिसंयोगाङ्गीकारात् , संयोगत्वावच्छेदेनैव अव्याप्यवृत्तित्वनियमस्य परमाणुसंयोगे कालिकाव्याप्यवृत्तित्वमादायापि निर्वाहसम्भवात् | यो यः संयोगः स सर्वः स्वाश्रयसमवायिदेशावच्छिन्नो भवति इति नियमस्तु नास्त्येव, सावयववृत्तिसंयोगत्वस्य उपाधित्वात् , यत्र यत्र संयोगे स्वाश्रयसमवायिदेशावच्छिन्नत्वं प्रमाणसिद्धं तत्र सर्वत्र सावयववृत्तिसंयोगत्वस्य सत्त्वेन साध्यव्यापकत्वात् , विवादाध्यासिते परमाणुसंयोगे संयोतत्वरूपसाधनाव्यापकत्वाच्च | संयोगत्वावच्छेदेन स्वाश्रयसमवायिदेशावच्छिन्नत्वनियमस्वीकारे च परमाणुकल्पनाया एव वैयर्थ्यप्रसङ्गात् | यैः परमोदाप्रतिभैः परमाणवः स्वीक्रियन्ते तैर्व्याप्यवृत्तयः तेषां संयोगाः स्वीक्रियन्त एवेति तदीयमभिप्रायमनवगच्छतो भवतो न खेदमाददाति |\footnote{आरम्भवादः १०}}

\subsection{द्व्यणुकसाधनम्}

अस्तु तावन्महदारम्भकत्वेन अणुः | तथापि पुनस्तदवयवकल्पनमयुक्तमिति चेत् , त्रुट्यवयवभूतानामणूनाम् अन्त्यावयवत्वे तेषामेव परमाणुस्वरूपत्वमिति वक्तव्यम् | परमाणु परिमाणस्य च न महदारम्भकत्वम् | परिमाणस्य स्वसजातीयस्वोत्कृष्टपरिमाणारम्भकत्वनियमात् | तस्मात् त्रसरेणुगतमहत्वं प्रति बहुत्वं कारणमिति वक्तव्यम् | न च परमाणुत्रसंयोगात् महत्वोत्पत्तिः, परमाणुत्वे सति बहुत्वसङ्ख्यायुक्तत्वात् | अन्यथा घटादीनां नाशेऽपि कपालचूर्णादिक्रमो न स्यात् | तस्मात् नाशक्रमानुरोधेन महत्वोत्पत्तिकारणत्वेनच त्रुट्यवयवः द्व्यणुकमिति कल्प्यते | तस्यापि कार्यत्वं प्रकल्प्य तदवयवत्वेन परमाणुमभ्युपगच्छामः | तदुक्तं तात्पर्यटीकायाम् - 

{\fontsize{11.7}{0}\selectfont\s परमाणूनां बहूनामनारम्भकत्वात् । तथा हि त्रयः परमाणवो न कार्यद्रव्यमारभन्ते परमाणुत्वे सति बहुत्वसंख्यायुक्तत्वाद् घटोपगृहीतपरमाणुप्रचयवत् । आरम्भकत्वे तेषां घटोपगृहीतानां कपालशर्कराचूर्णक्रमो घटनाशे नोपलभ्येत द्व्यणुके च विजातीयानारम्भकत्वे सिद्धे तेनैव दृष्टान्तेनान्यत्रापि विजातीयेनारम्भो निषेध्यः ।\footnote{न्या.वा.ता.टी. ३.१.३०}}



\subsection{परमाणुसत्त्वे प्रमाणाभावः}

दीधितिकारेति प्रसिद्धाः श्रीरघुनाथशिरोमणयः परमाणुं निराकुर्वन्ति | त्रुटेः चाक्षुषत्वादिना, तदवयवस्य च महदारम्भकत्वादिना च सावयवत्वकल्पनमप्रयोजकशङ्काकलङ्कितम् | अन्यथा अनन्तावयवधाराया अपि कल्पनमापद्येत | न चानवस्थाभयान्नकल्प्यते इति वाच्यम् | तर्हि तदर्थं त्रुटावेवावयवधारायाः विश्रान्तिः कल्प्यताम् | तस्यैव निरवयवत्वं नित्यत्वञ्च स्वीक्रियताम् | तस्मात् प्रमाणाभावान्नास्त्येव परमाणुः | तदुक्तं पदार्थतत्त्वनिरूपणे - 

{\fontsize{11.7}{0}\selectfont\s परमाणुद्व्यणुकयोश्च मानाभावः त्रुटावेव विश्रमात् | त्रुटिः समवेता चाक्षुषद्रव्यत्वात् घटवत् , ते च समवायिनः समवेताः चाक्षुषद्रव्यसमवायित्वादिति चाप्रयोजकम् | अन्यथा तादृशसमवायिसमवायित्वादिभिरनवस्थिततत्समवायिपरम्परासिद्धिप्रसङ्गात् |\footnote{प.त.नि २२}}

\subsection{परमाणुव्यवहारोपपादनम्}

न च अणुरिति प्रामाणिकानां व्यवहारानुपपत्तिरिति वाच्यम् | कपालाद्यपेक्षया घटादौ परिमाणाधिक्यात् यथा 'महत्तम' इति व्यवहारः तद्वत् परिमाणाल्पत्वे अणुरिति व्यवहारोऽपि उपपद्यते |

{\fontsize{11.7}{0}\selectfont\s अणुव्यवहारश्चापकृष्टपरिमाणनिबन्धनो महत्यपि महत्तमादणुव्यवहारात् |\footnote{प.त.नि २२}}


\subsection{विमर्शः}

\subsubsection{परमाणुसत्त्वे प्रमाणविमर्शः}
परमाणुसत्त्वे प्रमाणन्तु न प्रत्यक्षम् , अपि तु अनुमानादिकमेव | किं तावदनुमानमिति चेत् 'अणुपरिमाणतारतम्यं‌ क्वचिद्विश्रान्तं परिमाणतारतम्यत्वात् महत्परिमाणतारतम्यवत्' इति केचन वदन्ति | 'त्रसरेणुः कार्यः स्वावयवत्वात् परिदृश्यमानकार्यवत्' इत्यपरे | 'त्रुटिः सावयवः चाक्षुषत्वात् घटवत्', 'तदवयवः सावयवः महदारम्भकत्वात् कपालवत्' इत्यनुमानाभ्यां क्रमेण द्व्यणुकपरमाण्वोः सिद्धिरित्यन्ये | अत्र प्रथमानुमाने पक्षकुक्षौ प्रविष्टस्याणुपरिमाणस्य सद्भावे तश्रयस्य नानात्वे च प्रमाणान्तरं वक्तव्यम् | तथा हि अणुरिति व्यवहार एव अणुपरिमाणसत्त्वे प्रमाणम् | तदाश्रितस्यैकत्वे च त्रुट्यनुपपत्तिः | अनेकैरवयवैरेव कार्यद्रव्योत्पत्तिः | तस्मादणुपरिमाणाश्रयः अनेकः त्र्युट्यारम्भकत्वात् पटवत् इत्यनुमानेन तस्यानेकत्वसिद्धिः | महत्परिमाणतारतम्यस्य यथात्युत्कृष्टमहति घटादौ विश्रमात्तेषाम् अणुतारतम्यस्यापि अत्युत्कृष्टाणावेव विश्रमः कल्प्यते | सैव परमाणुरित्युच्यते |

द्वीतीयानुमानन्तु 'न प्रलयः परमाणुसद्भावात्'\footnote{न्या.सू. ४.२.१६} इति सूत्रभाष्यावलोकनेन ज्ञायते | अत्र हि कार्याणां सर्वेषां सावयवत्वनियमात् त्रुटेरपि कार्यत्वात् तस्यापि सावयवत्वसिद्धिः | किन्त्वत्र यतः अवयवविभागो न सम्भवति सः परमाणुरित्युक्तत्वात् तस्यापि महदारम्भकत्वेन सावयवत्वं कल्पनीयम् | सैव परमाणुरिति |

तृतीयानुमाने तु चाक्षुषद्रव्यत्वेन हेतुना त्रुट्यवयवः द्व्यणुकं संसाध्य तस्य च महदारम्भकत्वेन सावयवत्वं साध्यते इति विशेषः | अत्र निरवयवस्य महदारम्भकत्वं न सम्भवतीत्यंशोऽपि भासते |

एवं विनाशक्रमेणापि परमाणवः अनुमीयन्ते | तथा हि - कपालादिस्वावयवसंयोगनाशात् घटादिकार्याणां नाशो लोके दृष्टः, द्रव्यनाशं प्रत्यसमवायिकारणनाशस्य कारणत्वात् | एवञ्च 'त्रुटिनाशोऽपि स्वावयवसंयोगनाशजः कार्यद्रव्यत्वात् घटवत्' इत्यनुमानेन त्रुटिनाशकारणीभूतनाशप्रतियोगिसंयोगाश्रयतया किञ्चिद्द्रव्यं सिध्यति | तदेव द्व्यणुकमिति वदामः | महदारम्भकत्वाच्च तस्य कार्यत्वे सिद्धे तन्नाशकारणीभूतनशप्रतियोगिसंयोगाश्रयतया परमाणुः सिध्यति | न च तस्यापि महदारम्भकद्रव्यारम्भकत्वात् कार्यत्वम् | अप्रामाणिकानन्तावयवधाराकल्पनारूपानवस्थाप्रसङ्गात् | तस्मात् सः अकार्यत्वान्निरवयवत्वमेव | सैव परमाणुरिति |

\subsubsection{परमाणूनां नित्यत्वविमर्शः}

कार्यातिरिक्तानां गगनादिपदार्थानां नित्यत्वमिति क्लृप्तम् | न च तेषां नित्यत्वे विभुत्वमेव प्रयोजकमिति वाच्यम् | अविभुनः मनसोऽपि नित्यत्वात् | तस्मान्नित्यत्वं निरवयवप्रयुक्तमेव गगनादीनां मनसश्च निरवयवत्वात् नित्यत्वम् | परमाणूनामपि निरवयवत्वान्नित्यत्वं सिद्धमेव | निरवयवत्वञ्चास्याप्रामाणिकानन्तावयवधारायाः कल्पना स्यादिति भिया न स्वीक्रियते |  द्व्यणुकस्यावयवत्वेन परमाणूनां साधनात्तेषामनेकत्वमिति | 

\subsubsection{परमाणूवृत्तिसंयोगस्य व्याप्यवृत्तित्वविमर्शः}

एतादृशपरमाणुषु अदृष्टवदात्मसंयोगात् क्रिया उत्पद्यते | ततः विभागादिक्रमेण परमाणुद्वयसंयोगो जायते | ततः कार्यद्रव्यं द्व्यणुकमुत्पद्यते | एवं रीत्या महावयविपर्यन्तमुत्पत्तिः इति सृष्टिक्रमः | ननु परमाणौ संयोगो अनुपपन्नः | तथा हि - {\fontsize{11.7}{0}\selectfont\s 'तदेवं नियतस्य कस्यचित्कर्मनिमित्तस्याभावान्नाणुष्वाद्यं कर्म स्यात् ; कर्माभावात्तन्निबन्धनः संयोगो न स्यात् ; संयोगाभावाच्च तन्निबन्धनं द्व्यणुकादि कार्यजातं न स्यात् । संयोगश्चाणोरण्वन्तरेण सर्वात्मना वा स्यात् एकदेशेन वा ? सर्वात्मना चेत्, उपचयानुपपत्तेरणुमात्रत्वप्रसङ्गः, दृष्टविपर्ययप्रसङ्गश्च, प्रदेशवतो द्रव्यस्य प्रदेशवता द्रव्यान्तरेण संयोगस्य दृष्टत्वात् ; एकदेशेन चेत्, सावयवत्वप्रसङ्गः ; परमाणूनां कल्पिताः प्रदेशाः स्युरिति चेत्, कल्पितानामवस्तुत्वादवस्त्वेव संयोग इति वस्तुनः कार्यस्यासमवायिकारणं न स्यात् ; असति चासमवायिकारणे द्व्यणुकादिकार्यद्रव्यं नोत्पद्येत'\footnote{ब्र.शां.भा. २.२.१२}} इति | तन्न | अदृष्टस्य जिवात्मवृत्तिविशेषगुणस्य सत्त्वात् तद्विशिष्टात्मसंयोग परमाणौ द्रव्यारम्भकसंयोगजनकक्रियानकत्वात् |‌ न च अदृष्टमचेतनगुणः | तथा सति उत्पन्नस्य शिशोः भुभुक्षादिनिवृत्तौ प्रवृत्तिरेव न स्यात् | शरीररूपजडपदार्थस्य इदमिदानीमुत्पन्नत्वात् , घटादीनाञ्च सामानाधिकरण्याभावात् | तस्मात् परमाणौ क्रियाजनकसामग्रीसत्त्वात् क्रिया उपपद्यते | यत्पुनरुक्तं परमाणुषु संयोगोऽनुपपन्न इति | तदप्यसारम् | यद्यपि तत्र निरवयवत्वात् किञ्चिद्देशावच्छेदेन प्रसिद्धः अव्याप्यवृत्तिसंयोगोऽनुपपन्नः | तथापि व्याप्यवृत्तिसंयोगस्सम्भवति | संयोगस्याव्याप्यवृत्तित्वनियमस्तु कालिकाव्याप्यवृत्तित्वमादायापि उपपद्यते | अथवा परमाणूनां नानात्वात् परमाणौ यथा भिन्नदेशवृत्तित्वं प्रलयकाले स्वीक्रियते तत्र यथा दिशः अवच्छेदकत्वं क्लृप्तमवश्यं वक्तव्यञ्च तथैव प्रकृतेऽपि उपाधिभेदात् दिशामपि भेदं प्रकल्प्य तत्तद्दिक्संयोगस्याव्याप्यवृत्तित्वं स्वीकृत्य यत्किञ्चिद्दिगवच्छेदेन परमाण्वन्तरसंयोगो कल्प्यते | स च संयोगो दैशिकाव्याप्यवृत्तिरिति |

अत्रेदं चिन्त्यते -  त्रुटिकारणत्वेन परमाणुः न सिध्यति, तद्गतमहत्वं प्रति बहुत्वस्य कारणत्वात् इत्युक्तम् | न च परमाणुत्रयसंयोगः कल्प्यतामिति चेन्न | महत्वाश्रयद्रव्यस्य अव्याप्यवृत्तिसंयोगजन्यत्वनियमात् | घटपटादिषु तथैव दृष्टत्वात् | परमाणूनान्तु निरवयवत्वात् तत्र व्याप्यवृत्तिसंयोग एव जायते | तस्मादवान्तरद्रव्यं किञ्चित्कल्पनीयमिति युक्तम् | न च व्यप्यवृत्तिसंयोगो असिद्ध इति वाच्यम् | तूलकादौ तूलकान्तरस्य व्याप्यवृत्तिसंयोगस्य कथञ्चिदनुभवात् , अत एव यदा तत्र व्याप्यवृत्तिसंयोगः तदा तत्र परिमाणे उत्कृष्टता नानुभूयते | अस्मिन् पक्षे परमाणुद्व्यणुकयोः परिमाणे तथा तारतम्यं न भवितुमर्हति व्याप्यवृत्तिसंयोगजन्यद्रव्ये परिमाणसाम्यात् | यदि तत्र तारतम्यमपि एष्टव्यमित्याग्रहः तदा अव्याप्यवृत्तिसंयोग एव आदरणीयः इत्यलं विस्तरेण |

\subsubsection{परमाणुसत्त्वे प्रमाणाभावो विमृश्यते}

ननु परमाणुरिति पदार्थो नास्त्येव, तत्सत्त्वे प्रमाणाभावात् | न च 'त्रुटिः सावयवः चाक्षुषत्वात् घटवत्', 'द्व्यणुकं सावयवं महदारम्भकत्वात् कपालवत्' इत्यनुमानेन परमाणुः सिध्यति | अप्रयोजकत्वात् | एवं तर्हि परमाणूनामप्यवयवः कल्प्यताम् | न च तत्रानास्थाभिया न कल्प्यते इति वाच्यम् | तर्हि त्रुटीनामेव निरवयवत्वं कल्प्यताम् | तस्मात्त्रुटिरेव निरवयवो नित्यः इति केचित् | तदेव शास्त्रान्तरेऽपि वर्णितं यथा - {\fontsize{11.7}{0}\selectfont\s  जालरन्ध्रविसरद्रवितेजोजालभासुरपदार्थविशेषान् |\\ अल्पकानिह पुनः परमाणून् कल्पयन्ति हि कुमारिलशिष्याः ||\footnote{मा.मे.} इति |

अत्रेदं चिन्त्यते - त्रुटीनां निरवयवत्वस्वीकारे तद्वृत्तिद्रव्यान्तरारम्भकसंयोगः अव्याप्यवृत्तिर्न वा | नाद्यः निरवयवमूर्तस्य अव्याप्यवृत्तित्वसंयोगासम्भवात् | न द्वीतीयः तज्जन्यद्रव्यस्य तदपेक्षया उत्कृष्टपरिमाणत्वानुपपत्तिः | ननु  तत्राप्यव्याप्यवृत्तिसंयोगः प्रत्यक्षसिद्धः इति चेत् , तस्य सावयवत्वमपि युक्त्या सिद्धमेव | यदुक्तमप्रयोजकमिति तदपि न | यदि त्रुटिः सावयवं न स्यात् तर्हि तस्य चाक्षुषत्वमपि न स्यात् , चाक्षुषद्रव्यस्य अवयवजन्यत्वनियमादिति तर्क एव अप्रयोजकशङ्कानिवर्तकः | न च तथापि परमाणुस्तावत्सिध्यति इति वाच्यम् | द्वितीयानुमानेऽपि यदि द्व्यणुकं सावयवं न स्यात् तर्हि तस्य महदारम्भकत्वमपि न स्यादिति तर्केण द्वितीयानुमानेऽपि अप्रयोजकशङ्कावारणात् | तस्मात् प्रमाणतः प्रतिपन्न एव परमाणुरिति |
