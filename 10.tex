\chapter{मोक्षस्वरूपविचारः}

\section{न्या. सा.}कः पुनरयं मोक्ष इति । एके तावद्वर्णयन्ति – समस्तविशेषगुणोच्छेदे सति संहारावस्थायामाकाशवदात्मनोऽत्यन्तावस्थानं मोक्ष इति । कस्मात् । सुखदुःखयोरविनाभावित्वेन विवेकहानानुपपत्तेः। न च सुखार्थैव प्रेक्षावतां प्रवृत्तिः, कण्टकादिजनितदुःखपरिहारार्थत्वेनापि प्रवृत्तेरुपलम्भात् । मोहावस्थात्वान्मूर्छाद्यवस्थावदत्र विवेकिनां प्रवृत्तिर्न युक्तेत्याहुरन्ये । दुःखे सति सुखोपभोगासम्भवात् । कण्टकादिजनितदुःखपरिहारोऽपि सुखोपभोगार्थ एवेत्यसमो दृष्टान्तः । कुतो मुक्तस्य सुखोपभोगसिद्धिरिति चेद् आगमात् । उक्तं हि -
’सुखमात्यन्तिकं यत्तद्बुद्धिग्राह्यमतीन्द्रियम् ।
तं वै मोक्षं विजानीयाद् दुष्प्रापमकृतात्मभिः” ॥
तथा
’आनन्दं ब्रह्मणो रूपं तच्च मोक्षेऽभिव्यज्यते’ । ’विज्ञानमानन्दं ब्रह्म’ इति च ।
मुख्यार्थे बाधकाभावान्नोपचारकल्पना । सुखसंवेदनयोर्नित्यत्वान्मुक्तसंसारिणोरविशेषप्रसङ्ग इति चेत् । न । चक्षुर्घटयोः कुड्यादेरिव सुखतत्संवेदनयोर्विषयविषयिसम्बन्धप्रत्यनीकस्याधर्माद् दुःखादेः संसारावस्थायां सद्भावात् । तन्नाशे मुक्तावस्थायां भवति सुखसंवेदनयोः विषयविषयिसम्बन्ध इत्यतो नाविशेषः । तस्य सम्बन्धस्य कृतकत्वेन कदाचिद् विनाशप्रसङ्ग इति चेत् । न । प्रध्वंसेनानैकान्तिकत्वात् । वस्तुत्वे सतीति चेत् । न । द्रव्यादिष्वनन्तर्भावेण तदसिद्धत्वात् । अन्तर्भावे वा समवायादिभिः सह तत्संवेदनस्य सम्बन्धो न स्यात् । अदृष्टादिवशात् कर्मकारकं विषयः तज्जनितं ज्ञानं विषयीति चेत् । न । ईश्वरज्ञानस्य नित्यस्यार्थेः सह सम्बन्धाभावप्रसङ्गात् । तस्मात् कृतकत्वेऽपि नित्यसुखसंवेदनसम्बन्धस्य विनाशे कारणाभावान्नित्यत्वं स्थितम् । तत्सिद्धमेतन्नित्यसंवेद्यमानसुखेन विशिष्टात्यन्तिकी दुःखनिवृत्तिः पुरुषस्य मोक्ष इति ।\footnote{न्या.सा.१४२-१४६}

\section{प्रशस्तपादः}(६८०,६८२) - ज्ञानपूर्वकात्तु कृतसङ्कल्पितफलाद् विशुद्धे कुले जातस्य दुःखविगमोपायजिज्ञासोराचार्यमुपसङ्गम्योत्पन्नषट्पदार्थतत्त्वज्ञानस्याज्ञाननिवृत्तौ विरक्तस्य रागद्वेषाद्यभावात् तज्जयोर्धर्माधर्मयोरनुत्पत्तौ पूर्वसञ्चितयोः धर्माधर्मयोर्निरोधे सन्तोषसुखं शरीरपरिच्छेदं चोत्पाद्य रागादिनिवृत्तौ निवृत्तिलक्षणः केवलो धर्मः परमार्थदर्शनजं सुखं कृत्वा निवर्त्तते | तदा निरोधाद् निर्बीजस्यात्मनः शरीरादिनिवृत्तिः, पुनः शरीराद्यनुत्पत्तौ दग्धेन्धनानलवदुपशमो मोक्ष इति |\footnote{प्र.भा. ६८०,६८२}

\section{कन्दली} किन्तु समस्तात्मगुणोच्छेदोपलक्षिता स्वरूपस्थितिरेव | यथा चायं पुरुषार्थस्तथोपपादितम् |\footnote{न्या. कं. ६९२}

\section{न्यायमञ्जर्यान्तु} अपि च मोक्षे सुखमस्ति न वेति विचार एष न प्रामाणिकजनोचितः | स्वरूपेण व्यवस्थानमात्मनो मोक्ष इति मोक्षविदः | तत्र आत्मस्वरूपमेव कीदृशमिति चिन्त्यम् , न पृथङ्मोक्षस्वरूपम् | आत्मनश्च सुखदुःखबुध्यादय आगन्तुका गुणाः, न महत्त्ववत् सांसिद्धिका इति निर्णीतमेतदात्मलक्षणे, सुखादिकार्येण चात्मनोऽनुमानादिति |\footnote{न्या. मं. }

\section{शास्त्रान्तरीयविचारः}

\subsection{तत्र मीमांसकाः} न च मुक्तस्येन्द्रियाणि सम्भवन्तीति कथमानन्दानुभवः स्यात् ? उच्यते | बाह्येन्द्रियाण्येव मुक्तस्य निवर्तन्ते | मनस्तु तस्यामवस्थायामनुवर्तत इत्यानन्दश्रुतिबलादेवाध्यवसीयते, एवं ज्ञानञ्च | 'न हि विज्ञातुर्विज्ञातेर्विपरिलोपो वर्तत' इति श्रुतेः, विज्ञानघनश्रुतेश्च | तस्मान्मुक्त्यवस्थायां मानसप्रत्यक्षेण परमानन्दमनुभवन्नात्मावतिष्ठते, तदुक्तं -\\ 'निजं यत्त्वात्मचैतन्यमानन्दश्चेष्यते यः |\\ यच्च नित्यविभुत्वादि तैरात्मा नैव मुच्यते || इति |\footnote{शा.दी. १०५}

\subsection{तत्र वेदान्तिनः} तस्मादविद्याकामकर्मोपादानहेतुनिवृत्तौ स्वात्मन्यवस्थानं मोक्ष इति । स्वयं चात्मा ब्रह्म । तद्विज्ञानादविद्यानिवृत्तिरिति ।\footnote{तै.शां.}
