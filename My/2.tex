\chapter{वायुप्रत्यक्षत्वविचारः}

{\san 'वायुर्वाति', 'शीतो वायुः' इत्याद्यनुभवाः लोके जायन्ते | किमत्र विषयीभूतो वायुः इन्द्रियग्राह्यः ? यदि इन्द्रियग्राह्यः स्यात् तर्हि किं तदिन्द्रियम् ? न तावत् चक्षुः, रूपवद्द्रव्यस्यैव चक्षुषा ग्रहणात् | नापि श्रोत्रजिह्वाघ्राणरसनानि, द्रव्यग्राहकत्वाभावात् | नापि मनः, मनसा द्रव्येषु आत्ममात्र ग्रहणात् | तस्मातदवशिष्टेन त्वगिन्द्रियेणैव वायुविषयकप्रतीतिः इति वक्तव्यम् | अन्वयव्यतिरेकसहचाराभ्यामपि आपाततः वायुविषयकग्रहजनकत्वं त्वगिन्द्रियस्य  सम्भवति | तथाप्यत्र वैशेषिकास्तावत् विवदन्ते |

तथा हि प्रशस्तपादाचार्यैः भाष्यग्रन्थे 'विषयस्तूपलभ्यमानस्पर्शाधिष्ठानभूतः' इति कथनात् वायुगतानुष्णाशीतस्पर्शाश्रयत्वेन वायुः सिध्यति इत्यवगम्यते | अत एव अग्रे 'स्पर्शशब्दधृतिकम्पलिङ्गः' इत्यपि वायुसद्भावे लिङ्गानि प्रदर्शितानि | तथा च वायुविषयकपरोक्षज्ञानजनकसामग्रीणामेव प्रदर्शनात् वायुविषयकापरोक्षज्ञानं नोदेतीति भाष्यकाराणामाशयः प्रतिभाति |



ननु वायुः प्रत्यक्षः प्रत्यक्षस्पर्शाश्रयत्वात् इत्यनुमानेन वायौ प्रत्यक्षत्वमेव सिद्ध्यति | न च साधनावच्छिन्नसाध्यव्यापकमुद्भूतरूपवत्वमुपाधिः | स्पार्शनप्रत्यक्षत्वे तस्याप्रयोजकत्वान्महत्वे सत्युद्भूतस्पर्शस्यैव तन्त्रत्वात् , न तूद्भूतरूपस्यापि गौरवात् | ऊष्मादेश्च स्पार्शनत्वान्न व्यभिचारः | न च वायोः प्रत्यक्षत्वे तद्गतसङ्ख्यादिप्रत्यक्षत्वापत्तिः व्यक्तौ व्यभिचारात् | पृष्टे संलग्नस्य वस्त्रादिव्यक्तेः सङ्ख्याद्यग्रहेऽपि त्वचाग्रहात् तज्जातीये च फूत्कारादौ सङ्ख्यादेर्ग्रहात् | न च प्रत्यक्षविषयत्ववत् बहिरिन्द्रियजन्यतद्विषयत्वमपि नैकरूपावच्छिन्नम् , सति सम्भवे बाधकाभावे च त्यागायोगात् तथापि नोद्भूतस्पर्शवत्त्वं तथा चन्द्रमहसि तदभावात् | नापि जन्येन्द्रियग्राह्यविशेषगुणवत्त्वं कार्यमहत्त्वसमानाधिकरणविशेषगुणवत्त्वं वा रसनगतपित्तद्रव्यस्य उद्भूततिक्तरसस्य वायूपनीतसुरभिभागस्य च प्रत्यक्षत्वापत्तेः | वायुर्यदि स्पार्शनप्रत्यक्षविषयद्रव्यं स्यात् तदा उद्भूतरूपः स्यात् यथा घटः, न चैवं तस्मान्न तथेति तादृशद्रव्यस्य न चक्षुरितरेन्द्रियग्राह्यत्वमपि इति चेन्न | घटो यदि चक्षुरितरेन्द्रियग्राह्यद्रव्यं स्यादरूपमचाक्षुषं स्यात् , न चैवं, तस्मान्न तथेति घटस्याप्यस्पार्शनत्वापत्तेः | तस्मात्स्पर्शवद्द्रव्यस्य त्वाचत्वमङ्गीकरणीयमेव | मैवम् | 

अत्रोच्यते | चलति पक्षिकाण्डादौ पूर्वदेशविभागोत्तरदेशसंयोगयोरप्रत्यक्षत्वे चलतीति धीस्तल्लिङ्गिकापि न स्यात् | पूर्वदेशोत्तरदेशयोः प्रभामण्डलयोरनुद्भूतस्पर्शत्वेनाप्रत्यक्षत्वात् | न च पूर्वोत्तरदेशानुपलम्भोपलम्भाभ्यां तयोरनुमानं पूर्वोत्तरदेशयोरप्रत्यक्षत्वेन लिङ्गासिद्धेरित्युद्भूतस्पर्शवत्त्वं तत्राप्रयोजकम् | 

न च शीतो वायुरित्याद्यारोपो न स्पार्शनः विशेष्यस्यायोग्यत्वात् | न चानुमानोपनीतवायौ मनसा शीतारोपः | तदा शीतस्पर्शानाश्रयत्वेन प्रमिते वायौ तदनारोपात् | न च स्वप्नविभ्रमवन्मानसोऽयं भ्रमः बहिरिन्द्रियव्यापारानुविधानादिति वाच्यम् | यथा हि वायुवैधर्म्येण जलवृत्तितया गृहीतस्य शीतस्पर्शस्य वायावारोपः, दिङ्मोहवद्दोषमाहात्म्यात् , तथा शीतस्पर्शे वायुलिङ्गत्वारोपाद्वायवीयत्वानुमानम् | न चैवं शीतोऽयं वायवीय इति धीः स्यात् तदनन्तरं शीतो वायुरिति मानसारोपादिति सङ्क्षेपः | 


वायुप्रत्यक्षत्वे तद्गतसङ्ख्यादीनां प्रत्यक्षत्वापत्तिः इति चेन्न | यज्जातीयं द्रव्यं प्रत्यक्षं तज्जातीयस्य सङ्ख्यादेः ग्रहणं वा उत या व्यक्तिः प्रत्यक्षा तदीयसङ्ख्यादेः ग्रहणमिति वा | आद्ये वायुविशेषे फूत्कारे सङ्ख्याग्रहणात् नियममिदमुपपद्यते चेदपि फूत्कारादौ सङ्ख्या तावत् त्वचा न गृह्यत अपि तु अनुमानेनैव इति वयुत्वजातीये फूत्कारे त्वचा सङ्ख्यायाः अग्रहात् वायोरपि त्वाचप्रत्यक्षत्वं न सिध्यति | द्वितीये तु पृष्ठलग्नवस्त्रादीनां ग्रहेऽपि तद्गतसङ्ख्यादेः त्वचा अग्रहात् करपरामर्शादिनैव ग्रहात् नियमोऽयं व्यभिचरित एव | तथा च वायुरप्रत्यक्ष एव | 


त्वगिन्द्रियेण स्पर्शातिरिक्तवस्तुनः अग्रहात् वायोरपि ग्रहणं न सम्भवति इति न्यायकन्दलीकाराणामाशयः | ननु 'एको घटः' इत्याद्याकारकघटादिविषयकत्वाचप्रत्यक्षं प्रसिद्धं खलु | तथा च स्पर्शातिरिक्तानां एकत्वघटादीनां त्वचा ग्रहणं सम्भवति इति चेन्न | तादृशस्थलेऽपि त्वगिन्द्रियव्यापारेण स्पर्श एव प्रतिभाति | 'एको घटः' इत्यादिप्रतीतयस्तु अभ्यासपाटवातिशयेन साक्षात्परामर्शोदयात् स्पर्शलिङ्गकानुमानमेव | 

न च वायुः स्पार्शनप्रत्यक्षविषयः उपलभ्यमानस्पर्शाश्रयत्वात् घटवत् इत्यनुमानेन वायोः स्पार्शनत्वं सिध्यति | तथा च यत्र उद्भूतस्पर्शः घटादौ तत्र त्वाचप्रत्यक्षत्वं यत्र तदभावः गगनादौ तत्र तदभावः इति अन्वयव्यतिरेकसहचारावपि त्वगिन्द्रियस्य द्रव्यग्राहकत्वं साधयतः | तथा च व्याप्तिज्ञानाद्यनपेक्षपरामर्शमपि न कल्पनीयमिति लाघवम् इति वाच्यम् | वायोः त्वाचप्रत्यक्षत्वसाधकमनुमानं बाधितमेव | तत्र यदि वायोः त्वाचप्रत्यक्षत्वं स्यात् तर्हि तद्गतसङ्ख्यापरिमाणादीनामपि ग्रहणं स्यात् , त्वचा तदग्रहणात् वायुरपि न गृह्यते इति सिद्धमेव |}

  
“विषयस्तूपलभ्यमानस्पर्शाधिष्ठानभूतः । स्पर्शशब्दधृतिकम्पलिङ्गस्तिर्यग्गमनस्वभावो मेघादिप्रेरणधारणादिसमर्थः” इति ।\footnote{प्र.भा.}

 
उपलभ्यमानस्पर्शाधिष्ठानभूत आश्रयो यः स विषय इति । किमस्यास्तित्वे प्रमाणम् ? प्रत्यक्षमेव , त्वगिन्द्रियव्यापारेण वायुर्वातीत्यपरोक्षज्ञानोत्पत्तेरिति कश्चित् , तन्न युक्तम्  । स्पर्शव्यतिरिक्तस्य वस्त्वन्तरस्यासंवेदनात् । अपरोक्षज्ञाने तु स्पर्श एव प्रतिभाति नान्यत् । यदपि वायुर्वातीति ज्ञानं तदभ्यासपाटवातिशयाद् व्याप्तिस्मरणाद्यनपेक्षं स्पर्शेनानुमानम् , चक्षुषेव वृक्षादिगतिक्रियोपलम्भात् । शीतोष्णस्पर्शभेदप्रतीतौ वयुप्रतिभिज्ञानमपि तदाश्रयोपनायकद्रव्यानुमानादेव । त्वगिन्द्रियेण तु शीतोष्णस्पर्शाभ्यामन्यस्य न प्रतिभासोऽस्ति । स्पार्शनप्रत्यक्षो वायुरुपलभ्यमानस्पर्शाधिष्ठानत्वाद् घटवदित्यनुमानं शशादिषु पशुत्वेन शृङ्गानुमानवदनुपलब्धिबाधितम् । द्रव्यस्य स्पार्शनत्वं चाक्षुषत्वेन व्याप्तमवगतं घटादिषु चाक्षुषत्वस्य च वायावभावस्तेनात्र शक्यं स्पार्शनत्वनिवृत्यनुमानमेतत्  , अतस्तस्याप्रत्यक्षस्य सद्भावेऽनुमानम् ।\footnote{न्या.कं.}

 
ननु वायुसत्वे किं मानम् ? अध्यक्षमनुमानम् वा । नाद्यः , अरूपिद्रव्यत्वात् । नेतरः , लिङ्गाभावात् इत्यादिना वायुद्भावे प्रामाणानि निराकृत्य मैवमित्यनेन प्रमाणं दर्शयति । तथाहि त्वगिन्द्रियस्य गन्धग्राहकत्वेन रसनवदपार्थिवत्वात् रूपरसाग्राहकत्वेनैवातैजसानाप्यत्वात् । शब्दाग्राहकत्वेनानाभासत्वात् । बहिरिन्द्रियत्वेन मनोव्यतिरिक्तत्वात् । अत एव ग्राह्यजातीयविशेषगुणवत्वेन स्पर्शवत्वाद् द्रव्यान्तरप्रकृतिकत्वेन वायुसिद्धिः । न च गन्धाग्राहकमप्यदृष्टविशेषसाहित्यात्पार्थिवं स्यादिति वाच्यम् । सर्वेन्द्रियाणामपलापप्रसङ्गात् । स चायं स्पर्शाद्याश्रयः सिद्ध इति । स एवान्यत्र कल्प्यते लाघवात् । पृथिवीत्वे तु रूपानुद्भवादिकल्पनागौरवमेव । व्यजनानिलश्च ग्राह्यजातीयासाधारणगुणोद्भवापेक्षः स्पर्शवञ्जकः सावयवेन्द्रियव्यववस्थापकगुणेष्वेकस्यैव व्यञ्जकत्वात् , आलोकवत् । न चेदेवं मेदस्विनां स्वेदिनामन्धकारे समस्तवस्त्ववभासापत्तिः । नुद्भूतस्पर्शानिलस्य च शैत्याव्यञ्जकतापत्तिः । स च स्पर्शो वायवीयः । स्पर्शसाधारणव्यञ्जकत्वात्  त्वक्स्पर्शवदिति वा अपार्थिवत्वसिद्धिः , रूपरहितद्वा वा ।\footnote{न्या.ली.१६६-१७६}
 
ननु वायुः प्रत्यक्ष एव किन्न स्यात् । त्वगिन्द्रियव्यापारानन्तरं वायुर्वातीति प्रतीतेः । तथा च प्रयोगः वायुः प्रत्यक्षः प्रत्यक्षस्पर्शाश्रयत्वात् यदेवं तदेवं यथा घटः इति चेन्न उद्भूतरूपवत्वस्योपाधित्वात् । आत्मनि साध्याव्यापकमिदमिति चेन्न साधनावच्छिन्नसाध्यव्यापकत्वात् । तदुक्तम् –\\
वाद्युक्तनियमच्युतोऽपि कथकैरुपाधिरुद्भाव्यः ।\\[-3mm]
पर्यवसितं नियमयन् दूषकताबीजसाम्राज्यात् ॥\\
चाक्षुषप्रत्यक्षतायामेव तत्तन्त्रं न तु स्पार्शनप्रत्यक्षतायामपि इति चेन्न सामान्ये बाधकाभावात् । किञ्च वायोः प्रत्यक्षत्वे तद्गतसंख्यापरिमाणादिग्रहणप्रसङ्गः । ननु यज्जातीयं प्रत्यक्षं तज्जातीयस्य सङ्ख्यादिकं प्रत्यक्षमिति व्याप्तिश्चेदभिमता तदा फूत्कारे सङ्ख्यादिग्रहोस्त्येव । अथ या व्यक्तिः प्रत्यक्षा तदीयं सङ्ख्यादिकं गृह्यत एवेति नियमस्तदा पृष्ठलग्नवस्त्रादौ व्यभिचार इति चेन्न फूत्कारादपि सङ्ख्याप्रतीतेरानुमानिकत्वात् पृष्ठलग्नवस्त्रादेरपि करपरामर्शेण सङ्ख्यापरिमाणादेर्ग्रहणात् । द्वितीयनियमेऽपि न व्यभिचारः , चक्षुषा वस्त्रसङ्ख्यादिग्रहणात् । न च ह्येतादृशो नियमो  यत् प्रत्यक्षं तत्सङ्ख्यादिविषयकमेव किं तर्हि या व्यक्तिः प्रत्यक्षा तदीयसङ्ख्यादिकं गृह्यत एवेति । वायौ च तथा नास्तीति वायुरप्रत्यक्ष एवेति । उद्भूतरूपवत्वम् उद्भूतस्पर्शवत्वञ्च मिलितं तन्त्रमिति तात्पर्याचार्याः । तन्मते चान्द्रं तेजो नयनगतपित्तद्रव्यं पद्मरागादिप्रभादिकं च किमपि न प्रत्यक्षम् । तथा चोद्भूतरूपवत्वस्योद्भूतस्पर्शवत्वसहकृतस्य वा तस्य प्रयोजकत्वे वायुरप्रत्यक्ष एवेति ।\footnote{क.र.}

मनोज्ञमत्तकाशिनीसंदर्शनेन संसारिरागानुमानम्, शत्रुदर्शनेन क्रोधानुमानम्, विषयभोगेन सुखानुमानम्, रोगादिना दुःखानुमानम्, अविकलेन्द्रियस्य योग्यविषयावधानेन ज्ञानानुमानम्, श्रद्धावतां यथाविधि यागाद्यनुष्ठानेन धर्मानुमानम्, निषिद्धाचरणेनाधर्मानुमानम्, पट्वभ्यासादरप्रत्ययैः संस्कारानुमानम्, मिथ्याप्रत्ययेन भविष्यत्संसारप्रवाहानुमानमित्यध्यात्मम् । बाह्ये च ज्वलन्तं तृणराशिमुपलभ्य भविष्यद्भस्मानुमानम्, तथाविधवृष्ट्या भविष्यन्नदीपूरादिज्ञानम्, प्रकारान्तरावगतेन वायुना त्वरावता वृक्षादिसंक्षोभानुमानमित्यादि नेयम् ॥\footnote{न्या.वा.ता.टी}

{\san नव्यनैयायिकेषु मणिकारास्तु वायोरप्रत्यक्षत्वमेवोररीचकार | प्रत्यक्षसामग्रीणां विचारणावसरे मणिकारैः बहिरिन्द्रियद्रव्यप्रत्यक्षं प्रति किं कारणमिति न्यरूपि | ननु चक्षुषा यद्यद्गृह्यते तत्र सर्वत्रापि रूपस्य सत्त्वात् रूपशून्यस्य गगनादीनामतीन्द्रियत्वात् योग्यरूपमेव तत्र तन्त्रमिति चेन्न | रूपशून्यस्य वायोः त्वचा प्रत्यक्षत्वसम्भवात् व्यभिचारः | तथा हि वायुः प्रत्यक्षः प्रत्यक्षस्पर्शाश्रयत्वात् महत्वे सत्युद्भूतस्पर्शवत्त्वाद्वा घटवत् इत्यनुमानाभ्यां तत्र प्रत्यक्षत्वं सिध्यति |  न च अचाक्षुषत्वं स्पार्शनत्वे बाधकम् | चाक्षुषद्रव्यस्यैव त्वचापि ग्रहणात् | वायोः अतथात्वात् न स्पार्शनत्वमिति वाच्यम् | चाक्षुषत्वस्य तत्राप्रयोजकत्वात् | उत्पन्नघटस्य कदाचित् चाक्षुषत्वशून्यस्यापि त्वचा ग्रहणात् | अन्यथा चक्षुरेव द्रव्यग्राहकमिन्द्रियं स्यात् , त्वगिन्द्रियेण तु स्पर्शमात्रस्य ग्रहणं स्यात् | ------ 	
 
ननु बहिरिन्द्रियजद्रव्यप्रत्यक्षं प्रति जन्येन्द्रियग्राह्यविशेषगुणवत्वं कार्यमहत्त्वसमानाधिकरणविशेषगुणवत्त्वं वा तन्त्रम् | अत्र जन्येन्द्रियकार्यमहत्त्वेत्युक्तत्वात् गगनमादाय न व्यभिचारः | अन्यथा शब्दरूपस्य इन्द्रियग्राह्यविशेषगुणस्य गगनवृत्तिपरममहत्त्वसमानाधिकरणविशेषगुणस्य च गगने सत्त्वात् तस्यापि बहिरिन्द्रियग्राह्यत्वापत्तिः | तथा च वायौ तादृशस्य स्पर्शस्य विद्यमानत्वात् वायुः प्रत्यक्षविषयः इति चेन्न | तादृषेण रसनेन्द्रियेण रसरूपविशेषगुणस्य ग्रहणात् तदाश्रयस्य प्रत्यक्षत्वापत्तिः | एवं घ्राणेन गन्धग्रहात् तदाश्रयीभूतद्रव्यस्य च प्रत्यक्षत्वं स्यात् | अतः बहिरिन्द्रियजन्यद्रव्यप्रत्यक्षं प्रति महत्त्वे सत्युद्भूतरूपं कारणमित्येव कार्यकारणभावः स्वीक्रियते | एवं च तत्र सर्वत्र उद्भूतरूपस्य अभावान्न दोषः |

अथ उद्भूतरूपवत्त्वस्य तथात्वे नयनगतपित्तद्रव्यस्य प्रत्यक्षत्वापत्तिः | तत्र उद्भूतपीतरूपस्य सत्त्वात् | तर्हि उद्भूतस्पर्शस्यैव तथात्वमस्तु, तदा वायौ तत्सत्त्वात् तस्य प्रत्यक्षत्वं सिध्यति | न च प्रभायामुद्भूतस्पर्शस्याभावात् तस्याः प्रत्यक्षत्वं न स्यादिति वाच्यम् | प्रभा हि तेजसो रूपम् , न तु द्रव्यम् | अतः‌ तत्र रूपमात्रप्रतीतिः, तदाश्रयद्रव्यप्रतीतिस्तु तल्लिङ्गकानुमानादेव | न च तत्र द्रव्यस्याप्रत्यक्षत्वात् तद्गतसंयोगकर्मादिप्रतीतिः न स्यादिति वाच्यम् | रूपस्य पूर्वदेशानुपलम्भोत्तरदेशोपलम्भाभ्यां तदाश्रयद्रव्यस्य संयोगविभागादीनां परोक्षज्ञानमुदेति | यथा वायोरप्रत्यक्षत्ववादिनये फूत्कारादौ सङ्ख्यादिग्रहणम् | तथा चानेन धूमबाष्पादीनामचाक्षुषत्वेऽपि तद्गतसङ्ख्यादीनां प्रतिपत्तिरप्युपपद्यते |

नन्वेवं बहिरिन्द्रियद्रव्यप्रत्यक्षं प्रति उद्भूतरूपस्य उद्भूतस्पर्शस्य च कारणत्वे युक्तीनां दर्शनात् विनिगमनाविरहेण उभयोरपि कारणत्वमस्तु | एवञ्च वायुरप्रत्यक्ष एव | तत्र उद्भूतस्पर्शसत्त्वेऽपि उद्भूतरूपस्याभावात् इति चेन्न | बहिरिन्द्रियद्रव्यसाक्षात्कारं प्रति उभयोः कारणत्वस्वीकारापेक्षया चाक्षुषसाक्षात्कारं प्रति उद्भूतरूपस्य स्पार्शनसाक्षात्कारं प्रति उद्भूतस्पर्शस्य च कारणत्वस्वीकारे लाघवम् | तथा च वायौ उद्भूतस्पर्शस्य सत्त्वात् त्वाचप्रत्यक्षत्वापलापः नैव शङ्क्यः | --- 

उच्यते | द्रव्यस्य स्पार्शनत्वे उद्भूतस्पर्शमात्रं न तन्त्रम् | अन्यथा निदाघोष्मणः वायूपनीतशीतोष्णद्रव्यस्य च प्रत्यक्षत्वेन तद्गतसङ्ख्यादीनां प्रत्यक्षत्वापत्तिः | न च द्रव्यसाक्षत्कारे तद्गतसङ्ख्यादीनां नियमेन ग्रहो न भवतीति वाच्यम् | योग्यव्यक्तिवृत्तित्वेन सङ्ख्यादीनां योग्यतया द्रव्यग्राहकेन्द्रयेण तेषामपि ग्रहे बाधकाभावात् | एवं द्रव्यस्य चाक्षुषत्वे उद्भूतरूपमात्रमपि न तन्त्रम् | अन्यथा चान्द्राद्युद्योतस्य नयनगतपित्तद्रव्यस्य च प्रत्यक्षत्वेन तद्गतसङ्ख्यादीनामपि प्रतीतिः स्यात् | तथा च बहिरिन्द्रियद्रव्यसाक्षात्कारे एकैकस्य व्यभिचारात् उभयोरपि हेतुत्वमवश्यमङ्गीकरणीयम् | तथा च वायावुभयोरभावात् तस्याप्रत्यक्षत्वमेव |}  	






उच्यते | द्रव्यस्य स्पार्शनत्वे उद्भूतस्पर्शमात्रं न तन्त्रम् | निदाघोष्मणि वायूपनीतशीतोष्णद्रव्ये च प्रत्यक्षत्वेन तद्गतसङ्ख्यापरिमाणसंयोगविभागकर्मणां प्रत्यक्षत्वप्रसङ्गात् | योग्यव्यक्तिवृत्तित्वेन तेषां योग्यतया द्रव्यग्राहकसामग्रीग्राह्यत्वावधारणात् | न चोष्मादिजातीये दोषाभावेऽपि घटादाविव करपरामर्शे कदाचित् केनापि सङ्ख्या गृह्यते | तथोद्भूतरूपवत्त्वमात्रस्य तथात्वे चान्द्राद्युद्योतस्य नयनगतपित्तद्रव्यस्य च प्रत्यक्षत्वे तद्गतसङ्ख्याग्रहोऽपि स्यात् | न च घटादाविव निपुणं निभालयन्तोऽपि तद्गतसङ्ख्याद्वित्वादि हस्तवितस्त्यादिपरिमाणं कर्म वा वीक्षामहे इत्येकैकव्यभिचाराद्विनिगमकाभावात् उभयमपि बहिरिन्द्रियद्रव्यप्रत्यक्षत्वे प्रयोजकमिति वायुरप्रत्यक्षः | न चैवमपसिद्धान्तः | पीतः शङ्खः इत्यादौ नयनपित्तपीतिमैव गृह्यते न तु पित्तद्रव्यम् | विभक्ता(विषक्ता)वयवत्वात् प्रभायामिव तेजस इति टीकाकृतामभिधानादिति नवीनाः \footnote{त.म.}



{\san दीधितिकारेति प्रसिद्धैः श्रीरघुनाथशिरोमणिभिः स्वकीये स्वतन्त्रग्रन्थे पदार्थतत्त्वनिरूपणाख्ये वायोः त्वाचप्रत्यक्षविषयत्वं न्यरूपि | तथा हि द्रव्यविषयकस्पार्शनप्रत्यक्षं प्रति उद्भूतस्पर्शमेव प्रयोजकम् | अत एव औष्णाधिक्ये सति यो वायुर्वति तदा शीतो वायुरिति प्रत्ययोऽपि त्वाचः सम्भवति | तस्यानुमित्याद्यात्मकत्वे तद्धेतुकल्पनापेक्षया त्वाचत्वकल्पनायामेव लाघवम् | ननु द्रव्यस्पार्शनं प्रति स्पर्शस्य हेतुत्वे त्रुटेः स्पार्शनापत्तिः | तत्रापि स्पर्शस्य विद्यमानत्वादिति वाच्यम् | इष्टापत्तिः | त्रुटेः स्पार्शनत्वस्वीकारे न किञ्चिद्बाधकमस्ति | 

\section{पदार्थतत्त्वनिरूपणे तु} द्रव्यस्पार्शनप्रत्यक्षे स्पर्शवत्वमेव प्रयोजकम् | अत एव शीतोऽवायुरित्यादिप्रत्ययोऽपि स्पार्शनः साधु सङ्गच्छते | त्रुटेरस्पार्शनत्वे तु प्रकृष्टतमं परिमाणमपि तथा गौरवान्मानाभावात् | त्वग्व्यापारानन्तरं वायुर्वातीति सार्वलौकिकप्रत्यक्षस्य अन्यथानुपपत्त्या च रूपं तत्र न निवेशनीयम् | फूत्कारादौ च स्फुटतरप्रत्यक्षाः सङ्ख्यादयः चाक्षुषद्रव्यप्रत्यक्षे रूपं तथा | गौरवान्मानाभावाच्च | निस्पर्शायामपि प्रभायां चलनादिप्रत्ययाच्च  स्पर्शोऽपि न तथा | द्रव्यस्य समवेतेन्द्रियप्रत्यक्षे तु अनात्मसमवेतशब्दरसगन्धजातीतरयोग्यधर्मसमवायित्वं तथा | सुखादिसमवायिकारणतावच्छेदकत्वेन सिद्धमात्मत्वं जातिर्नेश्वर इति तदीय ज्ञानादिपिशाचादिसंयोगवारणाय योग्येति | विषयिधर्मासामानाधिकरणेत्यभिधाने तु शब्दो नोपादेयः | अमदादिनयनसंसृष्टपित्तद्रव्यस्य परिमाणशून्यत्वं, परिमाणवत्वमते तु तादृशप्रत्यक्षे परिमाणवत्वमेव तथा | यद्वा नयनसंसृष्टपित्तद्रव्यं निरूपमेव | अन्यथा पुरुषान्तरेण तत्पीतिमोपलम्भापत्तिः | स्मर्यमाणस्तु पीतिमा दोशवषाच्छङ्खादावारोप्यत इति | द्रव्यप्रत्यक्षे च शब्दरसन्धजातीतरयोग्यधर्मसमवायित्वं तथा | रसनगतञ्च पित्तद्रव्यं न रूपवत् न वा स्पर्शवत् रसना च न द्रव्यग्राहिकेति न तत् प्रत्यक्षम् |\footnote{प.नि. ७७} 



\section{शास्त्रान्तरीयविषयः}

\subsection{तत्र मीमांसकाः} शीतादिषु स्पर्शविशेषोपलभ्यमानेषु शीतो वायुरुष्णो वायुरनुष्णाशीतो वायुरिति वायुद्रव्यस्यैकस्य प्रत्यभुज्ञायमानत्वात् कृष्णो घटः, पीतो घटः, श्वेतो घट इतिवत् सकलस्पर्शानुगतमेकमेव वायुद्रव्यं प्रत्यभिज्ञायतां भवतां स्पर्शमात्रमेव वयं प्रत्यभिजानीमो नान्यत् किञ्चिदिति वचनमनुभवविरुद्धमेव | प्रयोगश्च भवति - वायुः प्रत्यक्षः महत्त्ववत्त्वेऽनिन्द्रियत्वे सति च सति स्पर्शवत्त्वाद् भूतत्वाद्वा घटवदिति | यत्पुनः वायुरप्रत्यक्षः अनात्मत्वे सति नीरूपद्रव्यत्वान्मनोवदित्युक्तम् , तदरूपिद्रव्याणामपि दिक्कालादीनामस्मन्मते प्रत्यक्षत्वात् तेष्वनैकान्तिकम् |\footnote{मा.मे.}

