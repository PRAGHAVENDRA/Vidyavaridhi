\chapter{परमाणुः}

\section{कन्दली} परमाणु स्वभावायाः पृथिव्याः सत्त्वे किं प्रमाणम् ? अनुमानम् , अणुपरिमाणतारतम्यं क्वचिद्विश्रान्तं परिमाणतारतम्यत्वाद् महत्परिमाणतारतम्यवत् , यत्रेदं विश्रान्तं यतः परमाणुर्नास्ति स परमाणुः | अत एव नित्यो द्रव्यत्वे सत्यनवयवत्वादाकाशवत् | अथायं सावयवो न तर्हि परमाणुः, कार्यपरिमाणापेक्षया तदवयवपरिमाणस्य लोकेऽल्पीयस्त्वप्रतीतेः, यश्च तस्यावयवः स परमाणुर्भविष्यति | अथ सोऽपि न भवति, अवयवान्तरसद्भावात् ? एवं तर्ह्यनवस्था, ततश्चावयविनामल्पतरतमादिभावो न स्यात् , सर्वेषामनन्तकारणजन्यत्वाविशेषेण परिमाणप्रकर्षाप्रकर्षहेतोः कारणसङ्ख्याभूयस्त्वाभूयस्त्वयोरसम्भवात् | अस्ति तावदयं परिमाणभेदः, तस्मादणुपरिमाणं क्वचिन्निरतिशयमिति सिद्धो नित्यः परमाणुः |\footnote{न्या.कं. ७९}

\section{अत्र वल्लभाचार्यास्तु} अणुर्नाम द्रव्यान्तरमिति चेन्न , तदसिद्धेरित्येके । अणुपरिमाणतारतम्यं क्वचिद्विश्रान्तमिति चेन्न , तदसिद्धावणुपरिमाणस्यासिद्धेः । महत्वापकर्ष इत्थं व्यपदिश्यत इति चेन्न , तस्य त्रुटावेव विश्रामात् । अणुव्यवहारास्पदमात्रमनुपात्ताणुमहदादिवैचित्र्यं धर्मीति चेन्न , प्रमाणसिद्धस्य व्यवहारास्पदत्वेन महत्वभेदादन्यस्य व्यवहारास्पदताप्रतिक्षेपात् । त्रसरेणुर्भागवान् चाक्षुषद्रव्यत्वात् । अन्यथा तन्न स्यात् । महत्वप्रकर्षस्य चाक्षुषतानुविधानात् । दूरेकेशाद्यनुपलम्भेऽपि शाल(ताल)तमालोपलम्भात् । तथा चानेकद्रव्यवत्वमपि महत्वकारणत्वाद्धूम इवेन्धनमनुविधत्ते चाक्षुषत्वमिति तत्सिद्धिरिति चेन्न , महत्वनित्यत्वेनाप्युपपत्तेः । त्रसरेणुः कार्यः महत्वे सति क्रियावत्त्वादिति चेन्न , महत्वविशेषणाव्यावर्त्याप्रतीतावसामर्थ्यात् । तस्यान्यतः प्रतीतौ वैयर्थ्यापातात् । अतः प्रतीतावन्योन्याश्रयात् । नित्यश्च त्रसरेणुर्महत्वोत्कर्षापकर्षाभ्यां शून्यत्वात् गगनवत् । महत्वाधिकरणमनित्यमुत्कृष्टापकृष्टमहत्व सम्बन्धपतिबद्धम् । घटो हि कुतश्चिदल्प इति कुतश्चिन्महत्तरो भवति न त्वेवं गगनं , कुतोऽप्यल्पताया अभावात् । त्रसरेणोर्महत्वं (तु) महत्वान्तरेभ्योऽपकृष्टं न कुतोऽप्युत्कृष्टम् । न हि त्रसरेणोरल्पं महदस्तीति , दृश्यसमवायिहेतूनां च दृश्यत्वनियमात् , दृश्यत्वस्य च दृश्यानुपलब्धिप्रतिक्षिप्तत्वात् । अन्यथा दृश्यादृश्यव्यवस्थाविलयनियमात् । तदिदमसङ्गतम् । यथा तैजसानां कारणानां न दृश्यतानियमस्तथेहापि । अन्यथा नयनविलयात् । न च त्रसरेणुमहत्वं नित्यं महत्वे सति चाक्षुषत्वात् घटमहत्ववत् । न च नित्यस्त्रसरेणुर्महत्वापकर्षविरहमात्रस्य महत्वैकाधिकरणस्य नियमग्राहकमानगोचरत्वात् । अनुमानादेव परमाणुसिद्धेः ।

	तथापि कथं (तत्) पृथिव्यादिजातीयं , कार्यसमवायिनिदानयोस्तन्तुपटादौ वैजात्यस्यापि दर्शनात् । गन्धादिमत्वात्तदनुमानमिति चेन्न , तस्य दृश्यसंस्थानैकव्यङ्ग्यत्वनियमात् । अन्यथा स्फटिकलौहताम्रादिजातेरप्यनुपपत्तेः । ततः परममहद्गगनवदणवोऽपि द्रव्यान्तरमेव । अन्यथा तत्रापि का प्रत्याशेति स्थितम् । मैवम् । तदसिद्धेः । यदि हि दृश्यैकव्यङ्ग्येयं स्यात् स्यादेवम् । नत्वेवम् । केवलस्य दृश्यत्वस्य भूजलादिसाधारण्येन व्यभिचारित्वात् । गन्धादिना तु विशेषणे वैयर्थ्यात् गन्धसमवायिकारणतानिर्वाहकजातेरणावपि स्वीकारात् । अन्यथा गन्धसमवायिकारणतानुपपत्तेः । गम्धस्नेहभास्वरूपापाकजस्पर्शेन क्षितित्वादिजात्यनुमानात् , स्फटिकादिजातेश्च गुणविशेषस्य व्यवस्थापकस्याभावात् , दृश्यसंस्थानव्यङ्ग्यत्वेनाणुषु तदनुमानविरोधात् ।\footnote{न्या.ली.}





\section{न्यायमञ्जर्यान्तु} तदपि हि कार्यं स्वावयवाश्रितम् , सावयवत्वात् , परिदृश्यमानकार्यवत् | निरवयवत्वे तु तस्य परमाणुत्वमेव | परमाणुषु च सावयवत्वस्य च हेतोरसिद्धत्वान्नावयवान्तरकल्पना | तेषां हि सावयवत्वे तदवयवाः परमाणवो भवेयुः | न ते उत्पत्तिक्रमवत् विनाशक्रमेणापि परमाणवोऽनुमीयन्ते | लोष्टस्य प्रविभज्यमानस्य भागाः, तद्भागानां च भागान्तराणीत्येवं तावत् यावदशक्यभङ्गत्वमदर्शनविषयत्वं च भवति | तद्यतः परमवयवविभागो न सम्भवति, ते परमाणव उच्यन्ते | तेष्वपि हि विभज्यमानेषु तदवयवाः परमाणवो भवेयुर्न ते | तदेतदेवं उत्पत्तिक्रमवत् विनाशक्रमस्येदृशो दर्शनात् सन्ति परमाणवः |

अत्र हि त्रयी गतिः | अस्य घटादेः कार्यस्य निरवयवत्वमेव वा, अवयवानन्त्यं वा, प्रमाण्वन्तता वा ? तत्र निरवयवत्वमनुपपन्नम् , अवयवानां पटे तन्तूनां घटे च कपालानां प्रत्यक्षमुपलम्भात् | अनन्तावयवयोगित्वमपि न युक्तम् , मेरुसर्षपयोरनन्तावयवयोगित्वाविशेषेण तुल्यपरिमाणत्वप्रसङ्गात् | तस्मात्परमाण्वन्ततयैव युक्तिमती |\footnote{न्या.मं. ४२०}




\section{पदार्थतत्त्वनिरूपणम्} परमाणु द्व्यणुकयोश्च मानाभावः त्रुटावेव विश्रमात् | त्रुटिः समवेता चाक्षुषद्रव्यत्वात् घटवत् ते च समवायिनः समवेताः चाक्षुषद्रव्यसमवायित्वादिति चाप्रयोजकम् | अन्यथा तादृशसमवायिसमवायित्वादिभिरनवस्थिततत्समवायिपरम्परासिद्धिप्रसङ्गात् | अणुव्यवहारश्चापकृष्टपरिमाणनिबन्धनो महत्यपि महत्तमादणुव्यवहारात् |\footnote{प.त.नि २२}


\section{शास्त्रान्तरीयविषयः}

\subsection{तत्र वेदान्तिनः} अपि चाणवः प्रवृत्तिस्वभावा वा, निवृत्तिस्वभावा वा, उभयस्वभावा वा, अनुभयस्वभावा वा अभ्युपगम्यन्ते — गत्यन्तराभावात् ; चतुर्धापि नोपपद्यते — प्रवृत्तिस्वभावत्वे नित्यमेव प्रवृत्तेर्भावात्प्रलयाभावप्रसङ्गः ; निवृत्तिस्वभावत्वेऽपि नित्यमेव निवृत्तेर्भावात्सर्गाभावप्रसङ्गः ; उभयस्वभावत्वं च विरोधादसमञ्जसम् ; अनुभयस्वभावत्वे तु निमित्तवशात्प्रवृत्तिनिवृत्त्योरभ्युपगम्यमानयोरदृष्टादेर्निमित्तस्य नित्यसन्निधानान्नित्यप्रवृत्तिप्रसङ्गः, अतन्त्रत्वेऽप्यदृष्टादेर्नित्याप्रवृत्तिप्रसङ्गः । तस्मादप्यनुपपन्नः परमाणुकारणवादः ॥ १४ ॥\footnote{ब्र.शां.}


\subsection{तत्र मीमांसकाः} जालरन्ध्रविसरद्रवितेजोजालभासुरपदार्थविशेषान् |\\ अल्पकानिह पुनः परमाणून् कल्पयन्ति हि कुमारिलशिष्याः ||\footnote{मा.मे.}

