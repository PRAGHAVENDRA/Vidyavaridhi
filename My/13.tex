\chapter{आकृतेः पदशक्यत्वविचारः}

\section{न्यायमञ्जर्यान्तु} तत्राकृतिवादिनस्तावदाहुः - प्रयोगप्रतिपत्तिभ्यां किल शब्दार्थनिश्चयः | वृद्धाः स्वार्थे तद्व्यवहारवन्तः यस्मिन्नर्थे गोशब्दं प्रयुञ्जते, श्रोतारश्च यमर्थं ततः प्रतिपद्यन्ते, स तस्यार्थः | तत्र यदि गोशब्दः केसरादिमति न प्रयुज्यते, सास्नादिमति च प्रयुज्यते, तदसाधारणसन्निवेशविषय एवावगम्यते | प्रत्यक्षविषये गौरित्यादिपदं प्रयुज्यते | प्रत्यक्षं चाकृतिविषयम् | अश्वपिण्डसन्निवेशाद्विलक्षणो हि गोपिण्डसन्निवेश इन्द्रियेण प्रतीयते | तत्कृतमेव वस्तूनामितरेतरवैलक्षण्यम् | अतः प्रत्यक्षविषये पदं प्रवर्तमानं आकृतावेव वर्तितुमर्हति | प्रेषणादिक्रियायोगश्च व्यक्तिद्वारक आकृतेर्भविष्यतीति यदुक्तम् -\\ तदयुक्तं प्रतिव्यक्ति भिन्नसंस्थानदर्शनात् |\\ आनन्त्यव्यभिचाराभ्यां सम्बन्धज्ञप्त्यसम्भवात् ||

न नियतस्य शाबलेयसन्निवेशस्य गोशब्दो वाचकः, तदभावेऽपि बाहुलेयसन्निवेशदर्शनात् | न च त्रैलोक्यान्तर्गतसकलगोपिण्डगतसन्निवेशवचनत्वमवगन्तुं शक्यम् , आनन्त्यात् | ततश्च नाकृतिः शब्दार्थः, तस्यां क्रियानुपपत्तेः | न हि प्रेषणादिक्रियासाधनं सन्निवेशः, अपि तु व्यक्तिः ||\\ न च गामानयेत्युक्तः सत्यामपि तदाकृतौ |\\ चित्रप्रस्थमयं किञ्चित् गामानयति बुद्धिमान् ||

ननु जातिवाच्यत्वपक्षेऽपि गोत्वजातेः सर्वगतत्वात् किमिति मृद्गवानयनं नानुष्ठीयते ? उच्यते - हस्ती तर्हि कथं नानीयते ? सर्वगतत्वाज्जतेः | अथ सर्वत्रास्तित्वेऽपि व्यञ्जकव्यक्तिनियमेनापह्नूयते, हन्त ! तर्हि सास्नादिमत्प्राणी गोत्वजातेरभिव्यञ्जकः, न मृद्गव इति नातिप्रसङ्गः | सन्निवेशस्य च तत्र भावात् तद्वाच्यत्ववादिनः नैनमतिप्रसङ्गमतिक्रामन्ति | 

किञ्चाकृतिवचनत्वे गोशब्दस्य शुक्लादिगुणवाचिभिः पदान्तरैः सामानाधिकरण्यं न प्राप्नोति | न हि शुक्लादिगुणा आकृतिव्यक्तः, अपि तु व्यक्तिवृत्तयः | तत्सामानाधिकरण्यादिबलवत्तया वरं व्यक्तिः शब्दार्थ इष्यताम् |\\ व्यक्तौ तावत् क्रियायोगः जातौ सम्बन्धसौष्टवम् |\\ नाकृतौ द्वयमप्येतत् इति तद्वाच्यता कुतः ?\\ जातिव्यक्त्योरतः कार्या वाच्यत्वे सम्प्रधारणा | तत्र व्यक्त्यभिधेयत्ववादिभिस्तावदुच्यते |\footnote{न्या.मं. ४९,५०}









\section{आकृतेः पदशक्यत्वम्} 
	
		गोपदात् जात्याकृतिविशिष्टस्यैव अनुभवात् आकृतावपि शक्तिः स्वीक्रियते~। आकृतिश्च अवयवसंयोगः~।  तथा च गोपदस्य अवयवसंयोग - गोत्वजाति - एतदुभयविशिष्टव्यक्तौ शक्तिरिति फलितम्~।
		
		अत्र चिन्त्यते – व्यक्तिवत् अननुगतानामवयवसंयोगरूपाकृतीनां शक्यत्वे शक्यानन्त्यं दुर्वारम्~। न च जातिविशेषवान् यो अवयवसंयोगः संस्थानविशेषः तद्रूपाकृतीनां शक्यत्वम्~। तथा च जातिविशेषस्यैव अनुगमकतया नोक्तदोष इति वाच्यम्~। अन्यतरकर्मजत्वादिना सङ्करेण तादृशजात्यप्रसिद्धेः~। न च उपाधिरेवानुगतः सुलभः इति वाच्यम्~। तादृशोपाधेः अनिरुक्तेः~। न च ‘जातिविशेषवदवयवसंयोगः’ इति मणौ जातिविशेषवत्त्वं अवयवविशेषणम्~। तथा च कपालसंयोगादिकमेव तथेति वाच्यम्~। गवादिपदे सास्नाद्यवयवस्याप्यननुगतत्वेन दोषतादवस्थ्यात्~। 

		अत्रोच्यते – जातिविशेषवतः गवादेः अवयवसंयोगस्य आकृतेः शक्यत्वम्~। विशेषपदोपादानेन पृथिवीपदादौ नाकृतिः शक्या इति लब्धम्~। तथा च गवावयवसंयोगत्वादिकमेव अनुगतम् इति न शक्त्यानन्त्यम्~। न चैवं सति तत्प्रकारकधीः स्यादिति वाच्यम्~। इष्टापत्तेः~। गवादौ आकृतिवैशिष्ट्यञ्च परम्परासम्बन्धेन~। ‘पिष्टकमय्यो गावः’ इत्यत्र तु गोपदस्य गवाकृतिसदृशाकृतौ लक्षणा~। पिष्टकसंयोगविशेषस्याशक्यत्वात्~। समुदायशक्तस्य पदस्य प्रत्येके लाक्षणिकत्वात्~। न च ‘गौरुत्पन्नः’ इत्यत्र व्यक्तिमात्रपरत्वेऽपि न लक्षणेति वाच्यम्~। तत्र व्यक्तिमात्रपरत्वाभावात्~। तद्व्यक्तित्वादिना तदुपस्थितौ यत्र तात्पर्यं तत्र लक्षणा इष्यत एव~।  एवं जात्याकृतिव्यक्तीनां प्रत्येकमात्रस्य बोधनेऽपि लक्षणा~।  यथा ”गौर्नित्या” इत्यत्र जातिमात्रपरत्वं गोपदस्य~।  यथा च गुरुमते कार्यशक्तायाः लिङः लोके  कार्यत्वमात्रपरत्वे~।  समुदायशक्तस्य पदस्य प्रत्येकस्मिन् लाक्षणिकत्वमिति नियमात्~। 

		जात्याकृतिव्यक्तिषु मिलितासु एकैव शक्तिः सूत्रकृता ”व्यक्त्याकृतिजातयस्तु पदार्थः” इत्येकवचनप्रयोगेण सूच्यते~।  यत्पदात् नियमतो यत् प्रतीयते तस्य तत्पदशक्यत्वम् इति नियमात् गोपदात् जात्याकृतिव्यक्तीनां तिसृणामपि प्रतीतेः तत्त्रितयस्य गोपदशक्यत्वम् आवश्यकम्~। 


		\subparagraph{तत्त्वचिन्तामणिः - }

			जातिविशेषवदवयवसंयोगरूपाकृतिरपि पदशक्या गोपदात् जात्याकृतिविशिष्टस्यैवानुभवात्~। पिष्टकमय्यो गाव इत्यादौ गवाकृतिसदृशाकृतौ लक्षणा पिष्टकसंयोगविशेषस्याशक्यत्वात्~। जात्याकृतिव्यक्तीनां प्रत्येकमात्रपरत्वे लक्षणैव~। प्रत्येकस्य जात्याकृतिविशिष्टादन्यत्वात्~। यथा गुरूणां कार्यशक्ताया लिङो लोके कार्यत्वपरत्वे~। अत एव ’व्यक्त्याकृतिजातयस्तु पदार्थ’ इति पारमर्षसूत्रम्~। एकयैव शक्त्या एकवित्तिवेद्यत्वसूचनाय पदार्थ इत्येकवचनम्~। 



\section{शास्त्रान्तरीयविचारः}

\subsection{तत्र मीमांसकाः} 
