\chapter{रूपादीनामव्याप्यवृत्तित्वविचारः}

\section{पदार्थतत्त्वनिरूपणम्} रूपादिकं चाव्याप्यवृत्त्यपि, सविषयावृत्तेर्व्याप्यवृत्तिवृत्तिजातेरव्याप्यवृत्तिताविरोधस्तु निष्प्रामाणिकः | पक्वे च घटे रक्तप्रतीतिर्नाप्रमा बाधकाभावात् | उपलभ्यते च भग्ने तस्मिन्नन्तः श्यामत्वम् | स्मृतिरपि - “लोहितो यस्तु वर्णेन मुखे पुच्छे तु पाण्डुरः | श्वेतः खुरविषाणाभ्यां स नीलो वृष उच्यते || इति |

चित्रमपि नातिरिक्तं रूपं समानाधिकरणविजातीयरूपसमुदायादेव तथाविधव्यवहारोपपत्तेः नीलादेर्नीलाद्यतिरिक्तरूपाजनकत्वाच्च |

स्पर्शोऽपि चाव्याप्यवृत्तिः | अन्यथा सुकुमारकठिनाभ्यामवयवाभ्यामारब्धेऽवयविनि सुकुमारावच्छेदेन त्वक्संयोगे काठिन्यस्याप्युपलभ्यप्रसङ्गः | न च सुकुमारत्वकठिनत्वे संयोगविशेषौ चाक्षुषत्वप्रसङ्गात् |

रसोऽपि चाव्याप्यवृत्तिः | अन्यथा तिक्तमधुराभ्यामारब्धेऽवयविनि तिक्तावयवावच्छेदेन रसनायोगे माधुर्योऽलम्भप्रसङ्गः | नीरस एव वा तत्रावयवी | एतेन गन्धो व्याख्यातः |\footnote{प. त. नि.५७,६४}
