\chapter{तमसः अभावरूपत्वे मतभेदः}

\section{अत्र प्रशस्तपादः} तर्हि भासामभाव एवायं प्रतीयते ? न , तस्य नीलाकारेण प्रतिभासायोगात् । मध्यन्दिनेऽपि दूरगगनाभोगव्यापिनो नीलोम्नश्च प्रतीतेः । किञ्च गृह्यमाणे प्रतियोगिनि संयुक्तविशेषणतया तदन्यप्रतिषेधमुखेनाभावो गृह्यते , न स्वतन्त्रः । तमसि च गृह्यमाणे नान्यस्य ग्रहणमिति । न च प्रतिषेधमुखः प्रत्ययः । तस्मान्नाभावोऽयम् । न चालोकदर्शनमात्रेवैतत् , बहिर्मुखतया तम इति , छायेति च कृष्णाकारप्रतिभासनात् । तस्माद्रूपविशेषोऽयमत्यन्तं तेजोभावो सति सर्वतः समारोपितस्तमः इति प्रतीयते । दिवा चोर्ध्वं नयनगोलकस्य नीलिमाभास इति वक्ष्यामः । यदा तु नियतदेशाधिकरणो भासामभावस्तदा तद्देशसमारोपिते नीलिम्नि छायेत्यवगमः । अत एव दीर्घा , ह्रस्वा , महती , अल्पीयसी छायेत्यभिमानः , तद्देशव्यापिनो नीलिम्नः प्रतीतेः , अभावपक्षे च भावधर्माध्यारोपोऽपि दुरुपपादः ।  इति न्यायकन्दली ।

\section{न्यायलीलावत्याम्} तमस्तु भावान्तरं निषेधत्वेनानवभासमानत्वात् | बाधकाभावेन चारोपानुपपत्तेः आलोकाभावे चाक्षुषत्वं नास्तीति बाधकमिति चेन्न, तस्यालोकाभावव्यञ्जनीयत्वात् | अन्यथारोपानुपपत्तेः | भावत्वे यदि द्रव्यान्तरं नवैवेति व्याघातः अद्रव्यान्तरत्वं सर्ववादिनिषिद्धम् | अथ गुणान्तरं चतुर्विंशतित्वव्याघात इति मेयान्तरमेव तमः | अत्रैव सङ्ग्रहः श्लोकः -\\ नाभावोभाववैधर्म्यान्नारोपो बाधहानितः |\\[-5mm] द्रव्यादिषट्कवैधर्म्याज्ज्ञेयं मेयान्तरं तमः ||\footnote{न्या.ली.१८-२०}

	
\section{कणादरहस्ये तु} नीलरूपारोपोऽपि कथं तमसि बाह्यालोकमन्तरेणेति चेन्न । स्मर्यमाणनीलरूपारोपे कदपेक्षया अभावात् फलबलात् तथैव कल्पनात् । यद्वा पीतादिव्यावृत्तिनिबन्धन एव नीलव्यवहारः सषयोरिव स्व्यवहारे गौडानां चलतीति धीरपि भ्रान्ता आवरकद्रव्ये चलति यथा यथा तेजसोऽसान्निध्यं तथा तथा तत्प्रत्ययोपपत्तेः । आवरकद्रव्यगतेरेव तत्रारोपाद्वा । इति ।

\section{न्यायकन्दली} न चालोकादर्शनमात्रमेवैतत् , बहिर्मुखतया तम इति, छायेति च कृष्णाकारप्रतिभासनात् | तस्माद्रूपविशेषोऽयमत्यन्तं तेजोभावे सति सर्वतः समारोपितः तम इति प्रतीयते | दिवा चोर्ध्वं नयनगोलकस्य नीलिमाभास इति वक्ष्यामः | यदा तु नियतदेशाधिकरणो भासामभावस्तदा तद्देशसमारोपिते नीलिम्नि छायेत्यवगमः | अत एव दीर्घा, ह्रस्वा, महती, अल्पीयसी छायेत्यभिमानः, तद्देशव्यापिनो नीलोम्नः प्रतीतेः, अभावपक्षे च भावधर्माध्यारोपोऽपि दुरुपपादः | तदुक्तम् - “न च भासामभावस्य तमस्त्वं वृद्धसम्मतम् | छायायाः कार्ष्ण्यमित्येवं पुराणे भूगुणश्रुतेः || दूरासन्नप्रदेशादि  महदल्पचलाचला | देहानुवर्तिनी छाया न वस्तुत्वाद्विना भवेत् | इति || दुरुपपादश्च क्वचिच्छायायां कृष्णसर्पभ्रमः, चलतिप्रत्ययोऽपि गच्छत्यावरकद्रव्ये यत्र यत्र तेजसोऽभावस्तत्र तत्र रूपोपलब्धिकृतः | एवं परत्वादयोऽप्यन्यथासिद्धाः | तत्र चालोकाभावव्यञ्जनीयरूपविशेषे तमस्यालोकानपेक्षस्यैव चक्षुषः सामर्थ्यं, तद्भावभावित्वात् , यथालोकाभाव एव त्वन्मते |नन्वेवं तर्हि सूत्रविरोधः "द्रव्यगुणकर्मनिष्पत्तिवैधर्म्याद्भावस्तमः" इति न विरोधः, भाभावे सति तमसः प्रतीतेर्भाभावस्तम इत्युक्तम् |\footnote{न्या.कं. २४} 

\section{शास्त्रान्तरीयाणामाशयः}

\subsection{तत्र वेदान्तिनः} अत्रोच्यते - तमालश्यामलज्ञाने निर्बाधे जाग्रति स्फुटे |\\ [-3mm] द्रव्यान्तरं तमः कस्मादकस्मादपलप्यते ||\\ अस्ति हि तमस्तमालश्यामलमिति प्रतीतिः | न चाप्रतीतावेवायं प्रतीतिभ्रमः‌ तद्व्यवहारस्य तत्प्रतीतिमन्तरेणानुपपत्तेः | न चायमौपचारिक आलोकाभावे, शौक्ल्याभावे पटादौ नीलव्यावहार इवेति युक्तम् , मुख्ये बाधकाभावात् | रूपत्वे रूपवत्त्वे वा आलोकानपेक्षचक्षुर्जन्यज्ञानविषयत्वासम्भवो बाधक इति चेन्मैवम् | आलोकविरोधिनस्तमसश्चालोकाभावव्यञ्जनीयतया तन्निरपेक्षचक्षुर्विषयत्वोपपत्तेः, सामर्थ्यस्य कार्यगम्यत्वात् , न चेद् , आलोकाभावस्यापि रूपवदभावतया घाटाभाववदालोकनिरपेक्षचक्षुर्विषयत्वं न स्यात् , तद्विरोधितया तदनपेक्षत्वं तु प्रकृतेऽपि समानमन्यत्राभिनिवेशात् | न च निमीलितनयनस्य लोचनव्यापाराभावेऽपि तमस उपलम्भादचाक्षुषत्वम् , तत्रापि पक्ष्मपटलान्तर्वर्तितमसश्चक्षुर्वापारादेवोपलब्धेः पिहितकर्णपुटस्य श्रोतृव्यापारादेवान्तरशब्दोपलब्धिवत् |

यत्तु गन्धाभावे तद्व्याप्तं नीलरूपमपि न स्यादिति, तदपि न | अनुष्णाशीतस्पर्शस्य पृथिव्यां गन्धादिव्याप्तस्य तदभावेऽपि मरुति प्रतीतेरेवाङ्गीकारवदिहापि प्रतीतेरेव नीलरूपस्य गन्धाभावेऽप्यभ्युपेयत्वात् | अथ पाकजस्यैवानुष्णाशीतस्पर्शस्य गन्धादिना व्याप्तिस्तर्हीहापि पाकजस्यैव नीलिमगुणस्य गन्धादिना व्याप्तिरिति तुल्यम् | सद्भावग्राहिणः प्रत्यक्षस्योभयत्र समानत्वात् | न च स्पर्शाभावात्तमसो रूपवत्त्वाभावः रूपग्राहकप्रत्यक्षविरोधे हेतोः कालात्ययापदिष्टतया तदभावानुमानानुदयात् | अन्यथा रूपाभावेन पवनेऽपि स्पर्शाभावानुमानस्य दुर्निवारत्वादन्यत्र रूपवतामेव स्पर्शवत्त्वनियमात् | न चास्पर्शवत्त्वादेव तमःपरमाणूनां मनोवदनारम्भकत्वमनुमातुं युक्तम् | तत्र वैयर्थ्यस्योपाधेः सद्भावात् | न हि शरीरतया विषयतयेन्द्रियतया वा मनआरब्धद्रव्यस्य भोगसाधनता सम्भावनीया | पार्थिवादिचतुर्विधशरीरानन्तर्भूतस्येन्द्रियानाश्रयस्य शरीरतया भोगसाधनतानुपपत्तेः | न च विषयतया, रूपस्पर्शशून्यारब्धस्य  तद्रहितत्त्वेन विषयत्वायोगाच्छरीरेन्द्रियव्यतिरिक्तोऽवयवी विषय इत्यङ्गीकारात् | नापीन्द्रियतया मनस एवेन्द्रियत्वात्तदारब्धेन्द्रियान्तरस्वीकारवैयर्थ्यात् | न च शरीरेन्द्रियविषयानन्तर्भूतस्य कार्यद्रव्यस्य भोगसाधनत्वं परैरङ्गीक्रियते | आत्मादीनामपि पुरस्पर्शवतां शरीरेन्द्रियविषयारम्भवात्समानमेव द्रव्यान्तरारम्भवैयर्थ्यम् | इह पुनस्तमसो महतो रूपवतः सम्भवति भोगसाधनत्वमिति नारम्भवैयर्थ्यम् | किञ्चास्मन्मते न तमस्तमोवयवैरारब्धम् , तस्य मूलकारणान्मेघमण्डलान्महाविद्युदादिजन्मवज्जन्माभ्युपगमात् |

किञ्चेदमनुविधानम् , किमालोकावरकातपत्राद्यनुविधानम् , किं वा निवर्तकप्रदीपाद्यनुविधानम् | नाद्यः | निश्चलेऽप्यचलादौ तच्छायायाश्चलनोपलम्भात् | नापि द्वितीयः | वैपरीत्यादालोकागमने गच्छति छायेति प्रतीतेर्गमने चागच्छतीति प्रतीतेः, तद्विरोधित्वाच्च तदनुविधानानुपपत्तेः | 

किञ्च\\ 'चक्षुः प्रकाशनाजन्यरूपवद्वीक्षणक्षमम् |\\[-3mm] रूपिग्राहीन्द्रियत्वेन यथैव स्पर्शनेन्द्रियम् ||\\ चक्षुरालोकाजन्यरूपिद्रव्यसाक्षात्कारजनकं रूपिद्रव्यग्राहकेन्द्रियत्वात्त्वगिन्द्रियवत् | न च रूपरहितेन्द्रियत्वमुपाधिः | यद्रूपवदिन्द्रियं तदालोकजन्यरूपिसाक्षात्कारजनकं न भवति घ्राणवदिति व्यतिरेकव्याप्तौ रूपवदाग्राहकत्वस्यैवोपाधित्वात् | तत्सिद्धमेतत्तमो द्रव्यान्तरमिति |

\subsection{तत्र मीमांसकाः} अस्पर्शत्वे सति रूपवत्तमः | तच्च नेत्रेन्द्रियमात्रग्राह्यमालोकाभावप्राकाश्यं कृष्णरूपम् |\\ कालायकोमलच्छायं दर्शनीयं भृशं दृशाम् |\\ तमः कृष्णं विजानीयादागमप्रतिपादितम् ||\\ तत्पुनरन्धतमसादिरूपम् |\footnote{मा.मे.}
