\chapter{समवायः}

\section{प्रशस्तपादः} ननु यद्येकः समवायः ? द्रव्यगुणकर्मणां द्रव्यत्वगुणत्वकर्मत्वादिविशेषणैः सह सम्बन्धैकत्वात् पदार्थसङ्करप्रसङ्ग इति | न, आधाराधेयनियमात् | यद्यप्येकः समवायः सर्वत्र स्वतन्त्रः, तथाप्याधाराधेयनियमोऽस्ति | कथं द्रव्येष्वेव द्रव्यत्वम् , गुणेष्वेव गुणत्वम् , कर्मस्वेव कर्मत्वमिति | एवमादि कस्मात् ? अन्वयव्यतिरेकदर्शनात् | इहेति समवायनिमित्तस्य ज्ञानस्यान्वयदर्शनात् सर्वत्रैकः समवाय इति गम्यते | द्रव्यत्वादिनिमित्तानां व्यतिरेकदर्शनात् प्रतिनियमो ज्ञायते | यथा कुण्डदध्नोः संयोगैकत्वे भवत्याश्रयाश्रयिभावनियमः | तथा द्रव्यत्वादीनामपि समवायैकत्वेऽपि व्यङ्ग्यव्यञ्जकशक्तिभेदादाधाराधेयनियम इति | सम्बन्ध्यनित्यत्वेऽपि न संयोगवदनित्यत्वं भाववदकारणत्वात् | यथा प्रमाणत उपलभ्यत इति | कया पुनर्वृत्त्या द्रव्यादिषु समवायो वर्तते ? न संयोगः सम्भवति, तस्य गुणत्वेन द्रव्याश्रितत्वात् | नापि समवायः, तस्यैकत्वात् | न चान्या वृत्तिरस्ति ? न, तादात्म्यात् | यथा द्रव्यगुणकर्मणां सदात्मकस्य भावस्य नान्यः सत्तायोगोस्ति | एवमविभागिनो वृत्त्यात्मकस्य समवायस्य नान्या वृत्तिरस्ति, तस्मात् स्वात्मवृत्तिः | अत एवातीन्द्रियाः सत्तादीनामिव प्रत्यक्षेषु वृत्त्यभावात् , स्वात्मगतसंवेदनाभावाच्च | तस्मादिह बुध्यनुमेयः समवाय इति |\footnote{प्र.भा. ७७८-७८५}

\section{न्यायसारः} एतत्पञ्चविधसम्बन्धसम्बन्धिविशेषणविशेष्यभावाद् दृश्याभावसमवाययोर्ग्रहणम्। तद्यथा घटशून्यं भूतलम् , इह भूतले घटो नास्ति । एवं सर्वत्रोदाहरणीयम् । समवायस्य तु क्वचिदेव ग्रहणम् । यथा घटे रूपसमवायः, रूपसमवायवान् घटः इति ।\footnote{न्या.सा.१४,१५}



\section{न्यायमञ्जर्यान्तु} या च समवायात्मिका पिण्डेषु वृत्तिः सामान्यस्य औलूक्यैरुच्यते तामपि न बुध्यामहे वयम् |

अयुतसिद्धानामाधार्याधारभूतानां यः सम्बन्धः इह प्रत्ययहेतुः स समवाय इति यदुच्यते तद्विप्रतिषिद्धम् | अयुतसिद्धता च सम्बन्धश्चेति कथं सङ्गच्छते | पृथक्सिद्धे हि वस्तुनी कुण्डबदरवदन्योन्यं‌ सम्बध्येते स्त्रीपुंंसवद्वा | अयुतसिद्धे तु तदेकत्वात् किं केन सम्बन्ध्यते | द्रव्यगुणयोरपृथक्सिद्धयोरपि सम्बन्धो विद्यत एवेति चेत् , तदिदं मन्ये उन्मत्तस्योन्मत्तसंवर्णनम् | गुणिनोऽपि गुणव्यतिरिक्तस्यानुपलम्भादयं गुणी रूपादिभ्योऽर्थान्तरत्वे नात्मानमुपदर्शयति व्यतिरेकेञ्च तेभ्यो वाञ्छतीति चित्रम् |

यदपि नित्यानित्यविभागेन युतसिद्धेः स्वशास्त्रे परिभाषणं कृतम् , नित्यानां परमाणूनां पृथग्गतिमत्त्वं युतसिद्धिरनित्यानान्तु युताश्रयिसमवायित्वम् , विभूनान्तु परस्परमाकाशादीनां सम्बन्ध एव नास्तीति तदपि प्रक्रियामात्रम् | नानात्वेन सिद्धिर्निष्पत्तिर्ज्ञप्तिर्वा युतसिद्धिरित्युच्यते तद्विपर्ययादयुतसिद्धिरैक्येन सिद्धिरवतिष्ठते, तथा च सति सम्बन्धो दुर्वचः | अवयवावयविनोरपि समवायात्मा सम्बन्ध एवमेव परिहर्त्तव्यः | यथाह भट्टः 'नानिष्पन्नस्य सम्बन्धो निष्पत्तौ युतसिद्धिता' इति |

परमाण्वाकाशयोः परमाणुकालयोश्च सम्बन्ध इष्यते नाकाशकालयोरन्योन्यमिति प्रक्रियैवेति अलमवान्तरचिन्तनेन | तस्मान्न जातिव्यक्त्योः काचिद्व्यक्तिरुपपद्यते |\footnote{न्या.मं.}


\section{तत्त्वचिन्तमणिः} गुणक्रियाजातिविशिष्टबुद्धयो विशेषणसम्बन्धविषयाः विशिष्टबुद्धित्वात् , दण्डीति बुद्धिवत् | न च व्यभिचारः | अभावादिविशिष्टबुद्धेरपि स्वरूपसम्बन्धविषयत्वात् | न चैवमत्रापि तेनैवार्थान्तरम् , यतो गुणक्रियाजातिविशिष्टबुद्धीनां पक्षधर्मताबलेन विषयः सम्बन्धः सिध्यन् लाघवादेक एव सिध्यति | स एव समवायः, न तु स्वरूपसम्बन्धः, तत्स्वरूपाणामनन्तत्वात् सम्बन्धत्वेनाक्लृप्तत्वाच्च |

अथ वा विशेषणसम्बन्धनिमित्तिका इति साध्यम् | हेतौ तु सत्यत्वं विशेषणम् | विशेषणसम्बन्धश्च कारणत्वेनैक एव सिध्यति | लाघवात् , अनुगतकार्यस्य अनुगतकारणनियम्यत्वाच्च | न तु स्वरूपसम्बन्धः, तेषामननुगतत्वादनन्तत्वाच्च | न चोभयमप्यप्रयोजकम् | विशिष्टसाक्षात्कारस्य सम्बन्धाविषयत्वे तदजन्यत्वे वा गवाश्वादावपि विशिष्टबुद्धिप्रसङ्गात् |\footnote{त.म. ३२}

\section{शास्त्रान्तरीयविषयः} 
\subsection{तत्र वेदान्तिनः} अतो यथा समवायस्य स्वभावसामर्थ्यादेवात्मनि परत्र च सम्बन्धव्यवहारहेतुत्वमेवं गुणगुण्यादेरपि स्वभावसामर्थ्यादेव  सम्बन्धव्यावहाराहेतुत्वोपपत्तेर्न कल्पनागौरवानवस्थादुःस्थसमवायस्वीकारावकाशः | तदेवं लक्षणासम्भवात्प्रमाणाभावाच्च न समवायस्य द्रव्यादिभ्यो भेदसिद्धिः | अतो न क्वापि लक्षणभेदाद्भेदसिद्धिरिति सिद्धम् | 




नापि समवायलक्षणम् | तथा हि अयुतसिद्धानामाधार्याधारभूतानामिहप्रत्ययहेतुर्यः सम्बन्धः स समवाय इति समवायलक्षणपरं परेषां भाष्यम् | अस्य चायमर्थः | युतसिद्धिरुभयोरपि सम्बन्धिनोः परस्परपरिहारेण पृथगाश्रयाश्रयोत्वं सा ययोर्नास्ति तावयुतसिद्धौ तयोः सम्बन्धः समवायः यथा तन्तु पटयोः | तत्र यद्यपि तन्तवः स्वारंभकावयवाश्रिताः तथापि पटस्य तन्त्वाश्रितत्वान्न द्वयोः परस्परपरिहारेण पृथगाश्रयाश्रयित्वम् | तेनानित्यानां द्वयोः पृथगाश्रयाश्रयित्वं युतसिद्धिरयुतसिद्धिस्तु तद्विपरीता | नित्यानां तु पृथग्गमनयोगित्वं युतसिद्धिस्तद्विपरीतायुतसिद्धिः | यथाकाशद्रव्यत्वयोः तत्रायुतसिद्धयोः सम्बन्धः इत्येतावतापि लक्षणे सुखस्य धर्मस्य च कार्यकारणभावसम्बन्धोऽपि समवायः स्यात् | आत्मैकाश्रिततया तयोरयुतसिद्धत्वात्तदर्थमाधार्याधारभूतानामिति पदम् | एवमप्याकाशस्याकाशपदस्य च वाच्यवाचकसम्बन्धः समवायः स्यात्तन्निवृत्यर्थमिहप्रत्ययहेतुरिति पदम् | न हि तस्मात्सम्बन्धादिहेदानीमिति बुद्धिर्जायते | केवलाकशस्यैव प्रतीतेः | आधार्याधारभूतानामिहप्रत्ययहेतुरित्येतावतापि लक्षणे कुण्डबदरसम्बन्धेऽतिव्याप्तिस्तन्निवृत्यर्थमयुतसिद्धानामिति पदम् | न समस्तमेवेदं समवायलक्षणमिति कन्दलीकारः | तदसत् | 'न स्यादयुतसिद्ध्यादि समवायस्य लक्षणम् | विशेषणविशेष्यत्वसम्बन्धे व्यभिचारतः ||१४|| इह भूतले घटाभाव इति विशेषणविशेष्यभावलक्षणेऽपि सम्बन्धे लक्षणस्य अतिव्याप्तिः | अस्ति हि तत्रायुतसिद्धिर्भूतलघटाभावयोर्भूतलस्य स्वावयवाश्रितत्वेऽप्यभावस्य भूतलातिरिक्ताश्रयाभावात् | अस्ति चाधार्याधारभाव इह प्रत्ययश्च | अथ भावयोरीदृशः सम्बन्धः समवायः तथापीह पटे रूपसमवायो रूपसमवायवान्पट इति विशेषणविशेष्यभावे लक्षणस्यातिव्याप्तिः | उक्तलक्षणस्य तत्रापि सम्भवात् | अथ गुणगुणिनोः क्रियाकारकयोरवयवावयविनोर्जातिजातिमतोर्विशेषतद्वतोश्च यः सम्बन्ध उक्तरूपः समवाय इति चेन्न | तेषामेवान्योन्यविशेषणविशेष्यभावेऽतिव्याप्तिः |

तत्र समवायस्यैवेहप्रत्ययहेतुत्वं नेतरस्येति चेन्न | एवमपीहाकाशे शब्द इत्यत्रेहशब्दस्याकाशस्य च वाच्यवचकसम्बन्धे इहप्रत्ययहेतावतिव्याप्त्याप्तेस्तादवस्थ्यात् | तत्र हि गुणगुणिनोराधार्याधारभूतयोरयुतसिद्धयोरिहप्रत्ययहेतुरिहेतिशब्दाकाशयोरस्ति वाच्यवाचकलक्षणः सम्बन्धः | किञ्चाधार्याधारभूतानामिति व्यर्थं विशेषणम् | अयुतसिद्धयोरिहप्रत्ययहेतुः सम्बन्ध इत्येतावत्यपि लक्षणे भवतामव्याप्त्यतिव्याप्त्यभावात् , अनाधार्याधारभूतयोरिहप्रत्ययस्यैवानुदयात् | अस्तु तर्ह्येतावल्लक्षणमिति चेन्न | तत्रापि प्रागुक्तवाच्यवाचकसम्बन्धेऽतिव्याप्तेः | 

भवतु तर्ह्यर्वाचीनमतानुसारेण नित्यः सम्बन्धः समवाय इति लक्षमिति चेन्न | विशेषणविशेष्यभाव एव व्यभिचारात्सोऽनित्यः प्रतियोगिनोरनित्यत्वादिति चेत्तथापि नित्यप्रतियोगिकविशेषणविशेष्यभावे व्यभिचारस्य तुल्यत्वात् प्रतियोगिनोरनित्यत्वेन सम्बन्धानित्यत्वापादनस्य समवायेऽपि समानत्वाच्च | भावरूपत्वे सतीति विशेषणादयमदोष इति चेन्न | अजसंयोगे व्यभिचारात् | स एव नास्तीति चेन्मैवम् | प्रमाणसिद्धत्वात् आकाशमात्मना संयुज्यते संयोगित्वाद्घटवत् | न च क्रियावत्त्वमूर्तत्वादिरुपाधिः | तद्व्यतिरेकेऽसंयोगित्वस्यैवोपाधित्वात् | किञ्चाकाशमात्मना संयुज्यत इत्यात्मसम्बन्धाधार इति साध्ये क्रियावत्त्वमूर्तत्वयोरुपाधित्वमेव नास्त्यात्मनि साध्याव्याप्तेस्तस्मादुक्तयुक्त्या न समवायस्य लक्षणसम्भवः | प्रमाणमपि तत्र प्रत्यक्षमनुमानं वा | नाद्यः | तदभावात् |

इह तन्तुषु पट इति प्रत्यक्षमस्तीति चेन्न | विचारासहत्वात् | समवायोह्याधारबुद्धिं कुर्यादादेयबुद्धिं वोभयबुद्धिं वा |सर्वथापि नोपपद्यते |
