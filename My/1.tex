\setcounter{page}{0}

\titleformat {\chapter}[display]{\normalfont\Large} % format
{प्रथमोऽध्यायः\\[1mm]} % label
{-3.8ex}{ \rule{\textwidth}{1pt}\vspace{-5ex}
\centering
} % before-code
[
\vspace{-6ex}%
\rule{\textwidth}{1pt}
]
\titlespacing*{\chapter} {10pt}{-60pt}{50pt}



\chapter{उपमानप्रमाणविचारः}

तत्र आप्तेनारण्यकेन गोसदृशो गवय इति उपदिष्टः नागरिकः कदाचिदरण्ये गवयं दृष्ट्वा तत्र गोसादृश्यं पश्यन्नाप्तवाक्यं स्मरति | तेन गवयपदतदर्थयोः सम्बन्धज्ञानमुदेति | तदेवोपमितिरिति | किन्त्वत्र प्रत्यक्षानुमाने एव प्रमाणे इति कथयन्तः वैशेषिका उपमानमनुमाने एवान्तर्भावयन्ति |

तथा हि गवयपदेऽव्युत्पन्नं नागरिकं प्रत्यारण्यकेन 'गोसदृशो गवय' इति प्रयुक्ते वनं गतस्य गवयपिण्डं प्रत्यक्षयतः नागरिकस्यातिदेशवाक्यं स्मरतोऽनुमानमुदेति गवयपदं गवयत्वविशिष्टतत्पिण्डवाचकम् असति वृत्त्यन्तरे तद्भियुक्तेनात्र प्रयुज्यमानत्वादिति | अत एवात्र यो यत्रासति वृत्त्यन्तरे प्रयुज्यते स तस्य वाचक इतिव्याप्तिं गोशब्दो गोजातीयस्य इत्युदाहरणं च ग्रन्थकारेण प्रादर्शि | अत्र प्रतिभासमानजातीय इत्यस्य साक्षात्कारविषयीभूतपिण्डविशेष इत्यर्थः | ननु वृत्त्यन्तवं विना प्रयोगः शक्तिमज्ञात्वा ज्ञातुमशक्य इति विशेषणासिद्धिः, 'गोसदृशो गवय' इत्याप्तवाक्यात् सादृश्यविशिष्टे गवयशब्दवाच्यतोपगमात् गवयत्वविशिष्टे गवयशब्दप्रयोगोऽसिद्ध इति विशेष्यासिद्धिरिति चेन्न | वृत्त्यन्तरस्यानुपपत्तिगम्यत्वात् तदभावे तदभावस्य निश्चयान्न विशेषणासिद्धिः | गवयत्वोपस्थितौ तस्य जातित्वेनाखण्डत्वाद्गोसादृश्यस्योपाधित्वेन गुरुत्वात् गवयत्वस्य प्रवृत्तिनिमित्तत्वनिश्चयान्न विशेष्यासिद्धिः | गवयत्वे प्रयोगस्य वृत्त्यन्तरात् गोसादृश्यस्य प्रवृत्तिनिमित्तत्वान्मानान्तरादप्युपपत्तौ उपमानाद्गवयस्य वाच्यत्वप्रवृत्तिनिमित्तत्वासिद्धेः | तथा च सादृश्यस्य प्रवृत्तिनिमित्तत्वे यावद्गौरवाख्यः प्रतिकूलतर्को नावतरति तावन्नोपमानस्यावतार इति | तर्केणेतराप्रवृत्तिनिमित्तकत्वे निश्चिते गवयपदं गवयत्वप्रवृत्तिनिमित्तकम् इतराप्रवृत्तिनिमित्तकत्वे सति सप्रवृत्तिनिमित्तकत्वात् यन्नैवं तन्नैवं यथा घटपदमित्यनुमानात् गवयत्वस्य प्रवृत्तिनिमित्तत्वज्ञानमुदेति | अतिदेशवाक्यस्याप्तोक्तत्वेन सादृस्याप्रवृत्तिनिमित्तकत्वे निश्चिते तवयपदस्य च सप्रवृत्तिनिमित्तकत्वे ज्ञाते लक्षणया गोसदृशपदेन गवयत्वमुपस्थाप्यत इति शब्दादेव गवयशब्दवाच्यत्वज्ञानन्नोपमानं मानान्तरमित्यर्थः |

न च गोसादृश्यरूपलक्षणस्य स्वरूपतः अन्वयः गवयशब्दवाच्यतया सामानाधिकरण्यञ्च दृश्यते | तथा च यो गोसदृशः स गवयशब्दवाच्य इति सामानाधिकरण्यमात्रेणान्वयोपपत्तौ अन्वयानुपपत्तिरूपं लक्षणाबीजं नास्त्येव | अतः कुलो लक्षणास्वीकार इति वाच्यम् | यतः द्विविधं सामानाधिकरण्यम् | शब्दगतमर्थगतञ्चेति | तत्राद्यस्य कदाचिदनुपपत्तिः 'माता वन्ध्ये'ति दर्शनात् | द्वितीयस्य तु 'गङ्गायां घोष' इत्यत्र सप्तमीतः प्रतीयमानस्य आधाराधेयभावस्य अनुपपत्तिः | प्रवाहाधेयत्वस्य घोषेऽभावात् | गोसदृशो गवयशब्दवाच्य इत्यत्र द्वितीयमपि सामानाधिकरण्यं नोपपद्यते | गोसादृश्यस्य गवयपदप्रवृत्त्यनिमित्तत्वात् | गोसदृशपदस्य गङ्गापदस्य तीराद्यर्थवत् लक्षणया गवयत्वरूपार्थस्वीकारे अन्वयोपपत्तिसम्भवात् नोपमानं मानान्तरमित्यर्थः |

आरण्यकोक्तवचनादेव गवयपदतदर्थयोः सम्बन्धप्रतिपत्तौ तदर्थमतिरिक्तं प्रमाणं न कल्पनीयम् | ननु अप्तोक्तवाक्यादुत्पन्नायां प्रमितौ गवयत्वं गवयपदप्रवृत्तिनिमित्ततया नैव भासते | तदा नागरिकस्य गवयत्वरूपजातिविषयकज्ञानाभावात् | अपि तु यो गोसदृशः स गवयपदवाच्यः इति ज्ञानेन गोसादृश्यमेव प्रवृत्तिनिमित्ततया भासते | तस्मात् कालान्तरे जायमानज्ञाने गवयत्वेन रूपेण गवयस्य भानं यद्भवति तदुपमानसाध्यमिति चेन्न | यो गोसदृशः इत्यत्र गवयत्वेन रूपेणैव लक्षणया भानं सम्भवतीति नातिरिक्तप्रमाणस्यावश्यकता | एवं गोसादृश्यमपि गुरुत्वात् प्रवृनिमित्तं भवितुं नार्हति, सति गवयत्वरूपलघुधर्मस्य प्रवृत्तिनिमित्तत्वसम्भवे गोसादृश्यरूपगुरुधर्मस्य प्रवृत्तिनिमित्तत्वास्वीकारात् | तथा च गोसदृशो गवय इत्यस्य गवयत्वविशिष्टो धर्मी गवयपदवाच्य इत्यर्थः ।  

ननु गवयत्वे लक्षणास्वीकारे किं बीजम् ? न चान्वयानुपपत्तिः | गोसदृशाभिन्नः गवयपदवाच्यः इत्यन्वयसम्भवादिति चेन्न | तात्पर्यानुपपत्तिरेवात्र लक्षणाबीजम् | तथा हि गोसदृशो गवय इत्यत्र गवयत्वेन रूपेण कस्यचित्पिण्डस्य बोधने एव वक्तृतात्पर्यात् | तस्माल्लक्षणया गवयत्वं धीः सम्भवत्येव | ननु 'यष्टीः प्रवेशय' इत्यादौ लक्षणास्थले यष्टेः स्वयं प्रवेशासम्भवरूपान्वयानुपपत्तिः कथञ्चित्सम्भवतीति चेदत्रापि  लघुनि प्रवृत्तिनिमित्ते गुरुणा प्रवृत्तिनिमित्तेन व्युत्पादनमनुपपन्नम् । तस्मादतिदेशवाक्यादेवासौ व्युत्पन्नो यथा कम्बुग्रीववान् अर्थो घटो घटपदवाच्य इत्यतिदेशवाक्यात् लक्षणया घटत्वेन घटज्ञानं तथैव कीदृग्गवय इति किं लक्षणकोसाविति नागरिकेण पृष्ट आरण्यको गोसादृश्यं लक्षणत्वेनातिदिशति गोसदृशो गवय इति । ततः ज्ञातातिदेशवाक्यार्थः श्रोता तादृशं पिण्डमनुसन्दधानः प्रसिद्धेषु करितुरगमहिषादिषु कथञ्चिद्गोसादृश्यं पश्यन्नपि न गवयत्वेन गवयपदवाच्यत्वं जानाति । वैधर्म्येण तिरस्कृतत्वात्  | कदाचिद्वैधर्म्यं गोसदृशे गवयपिण्डे पश्यन्नपि नूनमयं गवयपदवाच्यः गोसदृशत्वात् । यन्न गवयपदवाच्यः नासौ गोसदृशो यथा महिषमातङ्गादिरिति व्यतिरेकेणैव वाच्यत्वं जानाति । 

अथ वा गवयपदेऽव्युत्पन्नं नागरिकं प्रत्यारण्यकेन 'गोसदृशो गवय' इति प्रयुक्ते वनं गतस्य गवयपिण्डं प्रत्यक्षयतः नागरिकस्यातिदेशवाक्यं स्मरतोऽनुमानमुदेति गवयपदं गवयत्वविशिष्टतत्पिण्डवाचकम् असति वृत्त्यन्तरे तदभियुक्तेनात्र प्रयुज्यमानत्वादिति | तस्मादनुमानेनैव गवयत्वेन गवयप्रतीतौ तदुपपादनार्थमुपमानाख्यमतिरिक्तं प्रमाणं न कल्पनीयमिति |

\section{अयं सारः} आरण्यकोक्ताद्वचनादेव गवयत्वेन गवयप्रतिपत्तौ तन्निर्वाहकं सादृश्यज्ञानरूपं प्रमाणान्तरं नापेक्षितम् , शब्दस्य च अनुमानरूपत्वात् अनुमानप्रमाणेनैव तत्सिद्धेः | 'गोसदृशो गवय' इत्यत्र शाब्दबोधे तावत् गोसादृश्येन रूपेणैव गवयभानं भवितुमर्हति, गवयत्वज्ञानाभावादिति चेन्न, लक्षणया गवयत्वभानसम्भवात् | अन्यथा गोसादृश्यरूपगुरुधर्म्स्य गवयपदप्रवृत्तिनिमित्तत्वं न सम्भवति | तस्मात् उपमानं न मानान्तरमिति |

अथ वा - गवयपदं गवयत्वविशिष्टतत्पिण्डवाचकम् असति वृत्त्यन्तरे तदभियुक्तेनात्र प्रयुज्यमानत्वात् | असतिवृत्त्यन्तरेऽभियुक्तेनात्र यदत्र प्रयुज्यते स तत्पदवाचक इति व्यप्तेः सत्त्वात् | तस्मादनुमानेनैव गवयत्वेनरूपेणैव गवयप्रतिपत्तौ किमतिरिक्तेनोपमानप्रमाणेन इति |


\section{अत्र प्रशस्तपादः} आप्तेनाप्रसिद्धस्य गवयस्य गवा गवयप्रतिपादनादुपमानमाप्तवचनमेव ।\footnote{प्र.भा.}

\section{अत्र कन्दली}	
आप्तिः साक्षादर्थस्य प्राप्तिः , यथार्थोपलम्भः , तया वर्तत इत्याप्तः , साक्षात्कृतधर्मा , यथार्थदृष्टस्यार्थस्य चिख्यापयिषया प्रयोक्तोपदेष्टा , तेनाप्तेन वनेचरेण विदितगवयेन अज्ञातगवयस्य नागरिकस्य कीदृग्गवय इति पृच्छतो गोसारूप्येण गवयस्य प्रतिपादनादुपमानं यथा गौर्गवयस्तथेति वाक्यमाप्तवचनमेव । वक्तृप्रामाण्यादेव तथा प्रतीतेः । आप्तवचनं चानुमानम् । तस्मादुपमानमप्यनुमानाव्यतिरिक्तमित्यभिप्रायः ।\footnote{न्या.कं.}

\section{लीलावत्यायान्तु} उपमानं च न मानान्तरम् , अनुमानादेव तदर्थसिद्धेः । यो यत्रासति वृत्यन्तरे प्रयुज्यते स तस्य वाचको यथा गोशब्दो गोजातीयस्य , प्रयुज्यते चायमसति प्रतिभासमानजातीय इति । न चायमसिद्धः ,  मुख्यानुपपत्तिं विनोपचारस्यासम्भवात् । सादृश्यवति प्रयोगस्य कल्पनागौरवापत्तिहृतत्त्वात् । व्यक्तिषु प्रयोगस्यानन्त्यदूषित्वात् । नूनमयमेतज्जातीयाभिधानाय प्रयुज्यते इति निश्चयोपपत्तेः । न चेदेवमुपमानेऽपि वृत्त्यन्तरनिमित्तान्तरविषयविशिष्टप्रयोगसम्भावनायामपेक्षितासिद्धिप्रसङ्गात् । तथा च तत्वकौमुद्यामाचार्य एवमेवोपमानं पचिक्षेपे । किरणावलीकारस्तु कल्पनालाघवात् सादृश्यविशीष्टव्यक्तिवाच्यताप्रतिपादकत्वेऽतिदेशवाक्यस्य प्रतिक्षिप्ते उपलक्षणीयेन गवयशब्दवाच्यतान्वयो गोसदृशो गवयशब्दवाच्य इत्यतो भासते गङ्गायां घोष इतिवत् । न च वाच्यं लक्षणस्वरूपान्वयेनैव पर्यवसानमत्र , तत्र तु श्रुतान्वयानुपपत्तेरुपलक्षणीयव्यवस्थितिरिति । यथो द्विविधं सामानाधिकरण्यम् । शब्दमार्थञ्च । तत्राद्यस्य न कदाचिदनुपपत्तिः माता वन्ध्येति दर्शनात् । द्वितीये तु  न क्रियाकारकान्वयमात्रस्य , गङ्गायां घोष इत्यत्रापि सप्तमीतः आधाराधेयभावस्य भासमानत्वात् । वास्तवन्तु नास्त्येव । न हि गोसदृशो गवयशब्दवाच्य इत्युपपद्यते सादृश्यानिमित्ततया त्वयैवोक्तत्वात् । उपलक्षणीयमादाय तूपपत्तिर्गङ्गायामिति वदत्रापि तुल्येति  नोपमानं मानान्तरम् – इत्याह । प्रलपितमेतत् विचारासहत्वात् ।तथा हि केन गवयपदवाच्यतासामानाधकरण्यमात्रमनुपपन्नम् । किं गोसादृश्येन ? उत तदीयगवयशब्दप्रवृत्तिनिमित्ततया ? तथापि किमतिदेशवाक्यव्यक्तया उत्तरकालीनविमर्शसम्भावितया वा ? नाद्यः , केसरो वृक्षपदवाच्य इतिवदुपपत्तेः । नेतरः , विनित्ततोपलक्षनतौदासीन्येऽपि वस्तुतोऽन्यतररूपिणोऽप्यस्य शब्दत्वात् । नान्त्यः , उदासीनस्वभावस्य शब्दावेदितत्वेन विरम्यव्यापारयोगात् । तत्रापि गङ्गार्थमात्रे घोष इति पर्यवसानं स्यादिति चेत् । तेन जलमयस्यैव गङ्गाशब्दवाच्यत्वोपलम्भात् , तत्र चानुपपत्तेः । अर्थान्तरशङ्कापि स्यादिति चेत् । न ।एवं सति स्थितेऽप्युपचारमात्रविलयात् ।

नन्वेवं सादृश्ये निमित्तत्वेनानुपलब्धेः कस्यचिद्भ्रान्त्यैवमपि स्यादिति चेत् । न ।  तं प्रति मानव्यवहाराभावात् ।\footnote{न्या.ली.५३१-५३६}

\section{कणादरहस्ये तु} तथाप्युपमानमस्तु प्रमाणान्तरमिति चेन्न शब्दादेव तत्र संज्ञासंज्ञिसम्बन्धपरिच्छेदात् । शब्दस्य चानुमानान्तर्भूतत्वात् । तथा हि गवयपदे व्युत्पित्सुं प्रति यदारण्यको ब्रूते गोसदृशो गवय इति तदा यो गोसदृशोधर्मी स गवयपदवाच्य इति वाक्यादेवासौ परिच्छिनत्ति । किमत्र प्रमाणान्तरेण । ननु गवयत्वेन प्रवृत्तिनिमित्तेन गवयपदवाच्योऽयं पिण्डः इति प्रमात्वप्रमितिमाचक्ष्महे । न चेयं वाक्यमात्रादुत्पत्तुमर्हति । तदानीं गवयत्वस्य जातेरनुपस्थितेरिति चेन्न । गोसादृश्यपदेन गवयत्वस्यैव लक्षणात् । न हि गोसादृश्यं प्रवृत्तिनिमित्तं सम्भवति अखण्डत्वेन गुरुत्वात् । तथा च गोसदृशो गवय इत्यस्य गवयत्वविशिष्टो धर्मी गवयपदवाच्य इत्यर्थः ।  तथा च किमवशिष्यते यदनुरोधादुपमानमिति ? ननु स्यादप्येवं यद्यन्वयानुपपत्तिः स्यान्न चैवं प्रकृते यो गोसदृशः स गवयपदवाच्य इत्यन्वयसम्भवादिति चेन्न । तात्पर्यानुपपत्तेर्लक्षणासम्भवात् । न हि यष्टीः प्रवेशय छत्रिणो यान्तीत्यादौ लक्षणास्थले कञ्चिदन्वयानुपपत्तिं पश्यामः । किमत्र तात्पर्यमनुपपन्नमिति चेत् लघुनि प्रवृत्तिनिमित्ते गुरुणा प्रवृत्तिनिमित्तेन व्युत्पादनमनुपपन्नम् । तस्मादतिदेशवाक्यादेवासौ व्युत्पन्नो यथा कम्बुग्रीववान् अर्थो घटो घटपदवाच्य इत्यतिदेशवाक्यात् यद्वा कीदृग्गवय इति किं लक्षणकोसाविति नागरिकेण पृष्ट आरण्यको गोसादृश्यं लक्षणत्वेनातिदिशति गोसदृशो गवय इति । ततः प्रतिपन्नातिदेशवाक्यार्थः श्रोता तादृशं पिण्डमनुसन्दधानः करितुरगमहिषादिष्वन्वयतमप्रसिद्धेषु कथञ्चिद्गोसादृश्यं पश्यन्नपि न व्युत्पद्यते । वैधर्म्यतिरस्कृतत्वात् सादृश्यात्तत्परिप्लुतमतिः कदाचिद्वैधर्म्यं गोसादृश्यं गवयपिण्डं पश्यन्नूनमयं गवयपदवाच्यः गोसदृशत्वात् । यन्न गवयपदवाच्यः नासौ गोसदृशो यथा महिषमातङ्गादिरित व्यतिरेकिणैव वाच्यत्वं परिच्छिन्नत्ति । यथा का पृथिवीति पृथिवीपदव्युत्पित्सुं प्रति गन्धवती पृथिवीपदवाच्यं लक्षणं परं शृण्वत इदं पृथिवीपदवाच्यं गन्धवत्वादिति । यद्वा श्रुतातिदेशवाक्यो वनङ्गतो गवयत्वविशिष्टं पिण्डमुपलभ्य स्मृतातिदेशवाक्यो अनुमिमीते अयमसौ गवयपदवाच्यः असति वृत्यन्तरे वृद्धैस्तत्र प्रयुज्यमानत्वात् यन्न यत्पदवाच्यं तदसति वृत्यन्तरे वृद्धैः न तत्र प्रयुज्यते यथा गोपदं महिष इति व्यतिरेकिणा तत्सिद्धिः । तत्तु वृत्यन्तरव्यतिरेको दुरधिगमः व्युत्पित्सुं प्रति लक्षणया प्रयोगानुपपत्तेस्तदधिगमात् । न च गोसादृश्यस्यैव प्रवृत्तिनिमित्तता अनुमानेन साध्यते न तु गवयत्वस्येति वाच्यम् । तस्य गौरवेण निरासात् अन्यथोपमानेप्यप्रतीकारात् गौरवावतारश्चेत्तदिहापि कथं न स्यात् क्लृप्तप्रमाणापेक्षया कल्पनीयस्य गुरुत्वाच्च ।\footnote{क.र.}


नैयायिकास्तु उपमानमतिरिक्तं प्रमाणमिति वदन्ति | तथा च सूत्रं प्रज्ञातेन सामान्यात् प्रज्ञापणीयस्य प्रज्ञापनमुपमानम्\footnote{न्या.सू. } | 


\section{तत्र न्ययभाष्यम्}
	प्रज्ञातेन सामान्यात् प्रज्ञापनीयस्य प्रज्ञापनमुपमानमिति । यथा गौरेवं गवय इति । किं पुनरत्रोपमानेन क्रियते ? यदा खल्वयं गवा समानधर्मं प्रतिपद्यते, तदा प्रत्यक्षतस्तमर्थं प्रतिपद्यत इति । समाख्यासम्बन्धप्रतिपत्तिः उपमानार्थ इत्याह । यथा गौरेवं गवय इत्युपमाने प्रयुक्ते गवा समानधर्माणम् अर्थमिन्द्रियार्थसन्निकर्षादुपलभमानोऽस्य गवयशब्दः संज्ञेति संज्ञासंज्ञिसम्बन्धं प्रतिपद्यते इति । यथा मुद्गस्तथा मुद्गपर्णी, यथा माषस्तथा माषपर्णीत्युपमाने प्रयुक्ते उपमानात्संज्ञासंज्ञिसम्बन्धं प्रतिपद्यमानस्तामोषधीं भैषज्याया आहरति । एवमन्योऽप्युपमानस्य लोके विषयो बुभुत्सितव्य इति ।\footnote{न्या.भा.}


\section{तत्र वार्तिकम्}
	अथोपमानम् । प्रसिद्धसाधर्म्यात् साध्यसाधनमुपमानम् । सूत्रार्थः पूर्ववत् । प्रसिद्धसाधर्म्यादिति । प्रसिद्धं साधर्म्यं यस्य, प्रसिद्धेन वा साधर्म्यं यस्य, सोऽयं प्रसिद्धसाधर्म्यो गवयः । तस्मात् साध्यसाधनमिति । समाख्यासम्बन्धप्रतिपत्तिरुपमानार्थः । किमुक्तं भवति ? आगमाहितसंस्कारस्मृत्यपेक्षं सारूप्यज्ञानमुपमानम् । यदा ह्यनेन श्रुतं भवति यथा गौरेवं गवय इति, प्रसिद्धेन गोगवयसाधर्म्ये  पुनर्गवयसाधर्म्यं पश्यति प्रत्यक्षम् । ततस्तस्य भवत्ययं गवय इति समाख्यासम्बन्धप्रतिपत्तिः ।
	प्रत्यक्षागमाभ्यां नोपमानं भिद्यते । कथमिति ? यदा तावुभौ गोगवयौ प्रत्यक्षेण पश्यति, तदायमनेन सरूप इति प्रत्यक्षतः प्रतिपद्यते । यदापि शृणोति यथा गौरेवं गवय इति तदास्य शृण्वतः एव बुद्धिरुपजायते । केचिद् गोधर्मा गवयेन्वयिन उपलभ्यन्ते, केचिद् व्यतिरेकेण इति । अन्यथा हि यथा तथेत्येतन्न स्यात् । भूयस्तु सारूप्यं गवा गवयस्य इत्येवं प्रतिपद्यते । तस्मात् नोपमानं प्रत्यक्षागमाभ्यां भिद्यते ।
	गवा गवयसारूप्यं प्रतिपद्यते गवयसत्तं वेत्यहो प्रमाणाभिज्ञा भदन्तस्य । गवा गवयसारूप्यप्रतिपत्तेस्तु संज्ञासंज्ञिसम्बन्धं प्रतिपद्यत इति सूत्रार्थः । तस्मादपरिज्ञाय सूत्रार्थं यत्किञ्चिदुच्यते ।\footnote{न्या.वा.}  
 
\section{न्यायसारः} संज्ञासंज्ञिसम्बन्धप्रतिपत्तिरप्याप्तवचनकार्या । तथा प्रश्नोत्तराभिधानादन्यप्रमाणानिर्देशाच्च । अस्य गवयशब्दः संज्ञेति प्रतिपत्तावुपमानसिद्धिः , तथा शब्दाश्रवणादिति चेत् । एवं तर्हि गौरयमित्यस्मिन् सङ्केते कृतेऽस्य गोशब्दः सज्ञेति प्रतिपत्तौ प्रमाणान्तरं वाच्यम् , समानन्यायत्वात् । गोपिण्डान्तरेऽपि सङ्केतग्रहणे प्रमाणान्तराभिधानप्रसङ्ग इति । तथा शब्दानभिधानेऽपि प्रतिपादकप्रतिपत्तेरेवमेवाभिप्रायः – ईदृशस्य सर्वस्य गोशब्दः संज्ञेति सामर्थ्यादेवं प्रतिपत्तिरिति चेत् । समानमेतदितरत्रापि । गोसदृशो गवय इति । शब्दादुभयोरेवमेवाभिप्रायः गोसदृशस्यार्थस्य गवयशब्दः संज्ञेति सामर्थ्यादेवं प्रतिपत्तिरिति । न च प्रत्यक्ष एवार्थे संज्ञासंज्ञिसम्बन्धप्रतिपत्तिः, अप्रत्यक्षेऽपि शक्रादौ संज्ञासंज्ञिसम्बन्धप्रतिपत्तिदर्शनात् । सूत्रविरोध इति चेत् । न । प्रमाणनिग्रहस्थानाभ्यां दृष्टान्तहेत्वाभासादीनामिव प्रयोजनवशेन पृथगभिधानात् । तर्हि प्रयोजनं वाच्यम् । उच्यते । शब्दप्रामाण्यसमर्थनं प्रयोजनम् । कथम् । केचिदाहुः – प्रत्यक्षानुमानविषयत्वे शब्दस्यानुवादकत्वम् । तदविषयत्वे सम्बन्धाग्रहणादवाचकत्वम् । पदार्थस्याप्रसिद्धत्वान्न पदेन सम्बन्धग्रहणमितरेतराश्रयत्वात् । वाक्यर्थस्तु प्रसिद्धानां पदानामन्वयमात्रमिति । तन्निराकरणार्थमुपमानं निदर्शनार्थत्वेन पृथगुक्तम् । यथा कार्यार्थिनोऽप्रसिद्धगवयस्य प्रसिद्धं गोसादृश्यमुपादायोपमानाख्येन वाक्येन संज्ञासंज्ञिसम्बन्धप्रतिपत्तिः क्रियते, तथा किञ्चिन्निमित्तमुपादाय शक्रादिपदपदार्थयोरपि । तस्मादन्यार्थत्वान्न सूत्रविरोधः ।\footnote{न्या.सा. १०५-११}



\section{शास्त्रान्तरीयविषयः}

\subsection{तत्र वेदान्तिनः} तत्र सादृश्यप्रमाकरणमुपमानम् | तथा हि प्राङ्गणेषु दृष्टगोपिण्डस्य पुरुषस्य वनं गतस्य गवेन्द्रियसन्निकर्षे सति भवति पिण्डो गोसदृश इति तदनन्तरं च भवति निश्चयोऽनेन सदृशी मदीया गौरिति | तत्रान्वयव्यतिरेकाभ्यां गवयनिष्ठगोसादृश्यज्ञानं करणं गोनिष्ठगवयसादृश्यज्ञानं फलम् | न चेदं प्रत्यक्षेण सम्भवति गोपिण्डस्य तदेन्द्रियासन्निकर्षात् , नाप्यनुमानेन गवयनिष्ठसादृश्यस्यातल्लिङ्गत्वात् |नापि मदीया गौरेतद्गवयसदृशी एतन्निष्ठसादृश्यप्रतियोगित्वात् | यो यद्गतसादृश्यप्रतियोगी स तत्सदृशः यथा मैत्रनिष्ठसादृश्यप्रतियोगी चैत्रः मैत्रसदृशः इत्यनुमानात्तत्संभव इति वाच्यम् | एवंविधानुमानानवतारेप्यनेन सदृशी मदीया गौरिति प्रतीतेरनुभवसिद्धत्वात् | उपमिनोमीत्यनुव्ययसायाच्च | तस्मादुपमानम्मानान्तरम् |\footnote{वे.प.}

\subsection{तत्र मीमांसकाः} दृश्यमानार्थसादृश्यात् स्मर्यमाणार्थगोचरम् |\\ असन्निकृष्टसादृश्यज्ञानं ह्युपमितिर्मता ||\\ यथा गवये गोसादृश्यदर्शनानन्तरं स्मर्यमाणे गवि गवयसादृश्यज्ञानम् |\footnote{मा.मे.}

\clearpage
