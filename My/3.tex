\chapter{परोक्षसंशयविचारः}

\section{प्रशस्तपादः}
अत्र प्रशस्तपादः – बहिःसंशयः द्विविधः । प्रत्यक्षविषये चाप्रत्यक्षविषये च । तत्राप्रत्यक्षविषये तावत् साधारणलिङ्गदर्शनादुभयविशेषानुस्मरणादधर्माच्च संशयो भवति । यथाऽटव्या विषाणमात्रदर्शनाद् गौर्गवयो वेति । प्रत्यक्षविषयेऽपि स्थाणुपुरुषयोरूर्ध्वतामात्रसादृश्यदर्शनाद् वक्रादिविशेषानुपलब्धितः स्थाणुत्वादिसामान्यविशेषानभिव्यक्तावुभयविशेषानुस्मरणादुभयत्राकृष्यमाणस्यात्मनः प्रत्ययो दोलायते -  किं नु खल्वयं स्थाणुः स्यात् पुरुषो वेति ?\footnote{प्र.भा.}

\section{कन्दली}
वाटान्तरितस्य  पिण्डस्याप्रत्यक्षस्य सामान्येन विषाणमात्रदर्शनानुमितस्य संशयविषयत्वादप्रत्यक्षविषयोऽयं संशयः ।\footnote{न्या.कं.}




\section{पदार्थतत्त्वनिरूपणम्}
पदजन्यधर्मिककोटिद्वयतदुभयविरोधिज्ञानसंशयात्मकयोग्यताज्ञानसहितात् शब्दादाहत्यैव संशयः | “समानानेके"त्यादिसूत्रं प्रमाणयतो महर्षेरपि सम्मतमिदम् | एकधर्मिकनानाविरुद्धधर्मप्रकारकज्ञानत्वरूपं कोटिद्वयविरोधिज्ञानसामग्रीसमाजाधीनं च संशयत्वं नीलघटत्ववन्न कार्यतावच्छेदकम् | एवं संशयत्वशून्यज्ञानत्वरूपनिश्चयत्वमपि नीलत्वशून्यघटत्ववत्तथेति |\footnote{प.त.नि.१४२}


\section{सत्प्रतिपक्षदीधितिः}
रत्नकोषकारस्तु सत्प्रतिपक्षाभ्यां प्रत्येकं स्वसाध्यानुमितिः संशयरूपा जायते, विरुद्धोभयज्ञानसामग्र्याः संशयजनकत्वात् | संशयद्वारास्य दूषकत्वम् | न च संशयरूपा नानुमितिः बाधस्येव विरोध्युपस्थितेरनुमितिसामग्रीविघटकत्वेनावधारणात् अन्यथा बाधेऽप्यनुमित्यापत्तेरिति वाच्यम् | अधिकबलतया बाधेन प्रतिबन्धात् , तुल्यबलत्वादनुमितिः स्यादेव सामग्रीसत्त्वात् | साध्याभावबोधस्य च तत्र प्रतिबन्धकत्वं न तु तद्बोधकस्य चक्षुरादेः | प्रत्येकं निर्णायकत्वेनावधारितात्कथं संशय इति चेन्न | प्रत्येकाद्धि ज्ञानमुत्पद्यमानं अर्थात् संशयो न तु प्रत्येकं संशयजनकत्वमिति मेने |\footnote{स.दी.६०}


\section{सत्प्रतिपक्षगादाधरी}
परे तु ग्राह्याभावेऽव्याप्यवृत्तित्वग्रहदशायां ग्राह्याभावनिश्चयविषये धर्मिणि ग्राह्यज्ञानोत्पत्त्या यथा तादृशाव्याप्यवृत्तित्वग्रहस्य तत्प्रतिबन्धकतायामुत्तेजकत्वं, तथा तद्धिर्मिकाव्याप्यवृत्तित्वग्रहदशायां तद्व्याप्यवत्तानिश्चयस्यापि तद्विपरीतज्ञानाप्रतिबन्धकतया तत्रापि तद्धर्मिकाव्याप्यवृत्तित्वग्रहस्योत्तेजकत्वमुपेयम् | एवं च कपिसंयोगतदभावयोर्दैशिकाव्याप्यवृत्तित्वग्रहदशायां पक्षतावच्छेदकशाखादिरूपैकावच्छेदेन तदुभयकोट्यवगाहिनी गन्धतदभावयोः कालिकाव्याप्यवृत्तित्वग्रहदशायां पक्षतावच्छेदकतत्क्षणावच्छेदेन तदुभयकोटिमवगाहमाना च संशयानुमितिः ग्राह्याभावज्ञानप्रतिबन्धकतया न शक्यते वारयितुम् |\footnote{स.गा.९३}
