\chapter{पाकविचारः}

\section{तत्र प्रशस्तपादः}
पार्थिवपरमाणुरूपादीनां पाकजोत्पत्तिविधानम् । घटादेः आमद्रव्यस्याग्निना सम्बद्धस्याग्न्यभिघातान्नोदनाद्वा तदारम्भकेष्वणुषु कर्माण्युत्पद्यन्ते । तेभ्यो विभागा , विभागेभ्यः संयोगविनाशाः , संयोगविनाशेभ्यश्च कार्यद्रव्यं विनश्यति । तस्मिन् विनष्टे स्वतन्त्रेषु परमाणुष्वाग्निसंयोगादौष्ण्यापेक्षाच्छ्यामादीनां विनाशः , पुनरन्यस्मादग्निसंयोगादौष्ण्यापेक्षात् पाकजा जायन्ते । तदनन्तरं भोगिनामदृष्टापेक्षादात्माणुसंयोगादुत्पन्नपाकजेष्वणुषु कर्मोत्पत्तौ तेषां परस्परसंयोगाद् द्व्यणुकादिक्रमेण कार्यद्रव्यमुत्पद्यते । तत्र च कारणगुणप्रक्रमेण रूपाद्युत्पत्तिः ।  न च कार्यद्रव्य एव रूपाद्युत्पत्तिर्विनाशो वा सम्भवति , सर्वावयवेष्वन्तर्बहिश्च वर्तमानस्याग्निना व्याप्त्यभावात् । अणुप्रवेशादपि च व्याप्तिर्न सम्भवति , कार्यद्रव्यविनाशादिति ।\footnote{प्र.भा.}

\section{कन्दली}
अथ कथं कार्यद्रव्ये एव रूपादीनामग्निसंयोगादुत्पादविनाशौ कल्प्येते ? प्रतीयन्ते हि पाकार्थमुपक्षिप्ता घटादयः सर्वावस्थासु प्रत्यक्षाश्छिद्रविनिवेशितदृशाः , प्रत्यभिज्ञायन्ते च पाकोत्तरकालमपि त एवामी घटादय इति चेत् । तत्रोच्यते अन्तर्बहिश्च सर्वेष्वयवेषु वर्तमानस्य समवेतस्यावयविनो बाह्ये वर्तमानेन  वह्निना व्याप्तेर्वापकस्य संयोगस्याभावात् कार्यरूपादीनामुत्पत्तिविनाशयोरक्लृप्तेरन्तर्वर्तिनामपाकप्रसङ्गादिति भावः । सच्छिद्राण्येवावयविद्रव्याणि । न तावत्परमाणवः सान्तराः , निर्भागत्वात् । द्व्यणुकस्य सान्तरत्वे चानुत्पत्तिरेव , तस्य परमाण्वोरसंयोगात् । संयुक्तौ चेदिमौ निरन्तरावेव । सभाग्ययोर्हि वस्तुनोः केनचिदंशेन संयोगात् केनचिदसंयोगात् सान्तरः संयोगः । निर्भागयोस्तु नायं विधिरवकल्पते । स्थूलद्रव्येषु प्रतीयमानेष्वन्तरं न प्रतिभात्येव , त्रणुकेष्वेवान्तरम् । तच्चानुपलब्धियोग्यत्वान्न प्रतीयत इति गुर्वीयं कल्पना । तस्मान्निरन्तरा एव घटादयः । तेषामन्तस्तावदग्निप्रवेशो नास्ति यावत्पार्थिवावयवानां व्याप्ति भेदो न स्यात् । स्पर्शवति द्रव्ये तथाभूतस्य द्रव्यान्तरस्य प्रतिघाताद् व्यतिभिद्यमानेषु चावयवेषु क्रियाविभागादिन्यायेन द्रव्यारम्भकसंयोगविनाशादवश्यं द्रव्यविनाश इति कुतस्तस्याणुप्रवेशादभिव्यक्तिः । तस्मात् परमाणुष्वेव पाकोत्पत्तिरिति ।\footnote{न्या.कं.}

\section{लीलावत्यान्तु}
कथं पुनः पक्वघटेऽपि कारणगुणपूर्वकता रूपादीनाम् ? असंभवद्विरुद्धधर्मसंसर्गो हि घटः प्रत्यभिज्ञायते , उपस्थितानामपाताच्च , तद्देशत्वतत्संख्यत्वतत्परिमाणत्वोपलब्धेश्च । अन्तःपाकस्यापि सच्छिद्रत्वादवयविनां स्फटिकान्तरिन्द्रियप्रवेशवत् घटे बहिःशैत्यवेदनवदुपपत्तेः । न च स्पर्शवद्वेगवद्द्रव्याभिघाताद्द्रव्यनाशः , द्रव्यविरोधिकर्मोत्पत्तौ मानाभावात् । कर्मत्वस्य च नभोभागविभागेनापि करकम्पे तत्परमाणुकम्पवदुपपत्तेः । न च संयोगमात्रं हेतुः । पवनसंमुखधृतघटपटादेरपि नाशापत्तेः । कर्मण एवोत्पत्यसिद्धेः । नियतहेतुकुलालादिव्यतिरिक्तेऽपि जन्मवत्तस्यापि संभावितत्वात् । न चैवं द्व्यणुकनाशासिद्धिः , मानाभावात् । रूपञ्चाश्रयनाशादेव विनश्यति , कार्यरूपत्वात् नष्टघटरूपादिवदित्यस्यापि प्रत्ययस्य प्रत्यभिज्ञाननिर्दलितत्वात् , विपक्षे बाधकाभावाच्च पाकजा एवैति इति युक्तमितिचेत् । न । 
	
सूचीसम्भेदनिर्दलितावयवारब्धपरम्परानष्टघटे घटान्तरोदयवत् सर्वस्योपपत्तेः । विरुद्धधर्माध्यसस्यापि काठिन्याकाठिन्यस्य दर्शनात् । न च स्पर्शभेदोऽसावन्य एव जायत इति वाच्यम् । चक्षुषापि तत्प्रतीतेस्ततः संयोगभेद एवायं , स च प्राक्तनो नास्त्यनुपलब्धेः । तन्नाशाय च क्रियैव दहनस्य कल्पनीया , दृष्टत्वात् । स्पर्शवद्वेगवत्पवनसंवलनबलेन च पिठरोदरकुहरसञ्चारिजलतण्दुलोन्मथनविक्लेदकारिणि ज्वलज्वनज्वालाकलापे का जीवितव्यवस्थितिरित्यनुमानमप्युदयमासादयति । तथाप्येक एव संयोगः किमिति नोदननाशहेतुः ? कल्पनालाघवात् । पूर्वध्वंससहकारिणा तज्जनने युगपदुत्पादस्यानुपपत्तेः ,  अन्त्यस्योपान्त्यनाश्यत्वाच्च , संस्कारस्यानुभवध्वंसिनः स्मृतिरूपज्ञानजननदर्शनाच्च । ’एवमेव किन्त्वन्यताव्यवहार उपचारादि’ति व्योमाचार्याः । निवर्तकजनकदहनयोर्भेदेन वेति तत्रैव कालगणना ग्राह्या ।\footnote{न्या.ली}


\section{कणादसिद्धान्तचन्द्रिकायां}
सत्यवयविनि पाकेन रूपनाशो न सम्भवति , समवेतनाशं प्रत्याश्रयनाशस्य हेतुतया विना कारणं कार्योत्पत्तेरसम्भवात् । अतः पाकेन द्व्यणुकाद्यन्तवयविपर्यन्तमवयवनाशक्रमेण नश्यति  । नष्टे च तस्मिन् केवलेषु परमाणुषु पाकतः श्यामनाशरूपान्तरोत्पत्ती सम्पद्येते ।ततो द्व्यणुकादिक्रमेण कार्यद्रव्याण्युत्पद्यन्ते । ततस्तेषु कारणगुणक्रमेण रूपान्तरोत्पादः । इत्थं च पाकात् परमाणावेव पूर्वरूपनाशरूपान्तरोत्पादाविति ।\footnote{क.सि.चं.}

\section{कणादरहस्ये तु}
अथ पाकजप्रक्रिया । तत्रादौ परमाणुष्वेव पाकजोत्पत्तिः पिठरेऽपि वेति विचारः । तत्र विप्रतिपत्तिः । पार्थिवा अवयविरूपरसगन्धस्पर्शा अग्निसंयोगजन्या न वा घटोऽग्निसंयोगासमवायिकारणकरूपवान्न वा अग्निसंयोगो घटसमवेतरूपादिजनको न वा घटत्वमग्निसंयोगजन्यरूपसमानधिकरणं न वा । तत्र पाकेनाग्निसंयोगेनावयविष्वपि रूपादिकं जन्यत इति नैयायिकाः ।\footnote{क.र.}




\section{न्यायमञ्जर्यान्तु} अपरे पुनः प्रत्यक्षबलवत्तया घटादेरविनाशमेव पच्यमानस्य मन्यन्ते | सुषिरद्रव्यारम्भाच्च अन्तर्बहिश्च पाकोऽप्युपपत्स्यते | दृश्यते च पक्वेऽपि कलशे निषिक्तानामपां बहिः शीतस्पर्शग्रहणम् | अतश्च पाककाले ज्वलदनलशिखाकलापानुप्रवेशकृतविनाशवत् तदपि शिशिरतरनीरकणनिकरानुप्रवेशकृतविनाशप्रसङ्गः | न चेदृशी प्रमाणदृष्टिः | अतः प्रकृतिशुषिरतयैव कार्यद्रव्यस्य घटादेरारम्भात् अन्तरान्तरा तेजः कणानुप्रवेशकृतपाकोपपत्तेरलं विनाशकल्पनया | पिठरपाकपक्ष एव पेशलः ||\\ यादृतेव हि निक्षिप्तः घटः पाकाय कन्दुके |\\ पाकेऽपि तादृगेवासौ उद्धतो दृश्यते ततः ||\\ 

क्वचित्तु सन्निवेशान्तरदर्शनं काष्ठाद्यभिघातकृतमुपपत्स्यते | पावकसम्पर्ककारित्वे तु सर्वत्र तथाभावः स्यात् | तस्मादविनष्टा एव घटादयः पच्यन्ते |\footnote{न्या.म. २८७,२८८}




 
