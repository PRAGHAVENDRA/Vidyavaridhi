\chapter{प्रतिज्ञापत्रम्}
\begin{small}
राष्ट्रियसंस्कृतसंस्थानस्य राजीवगान्धीपरिसरे सहायकाचार्यपदमलङ्कुर्वाणानां प्रातःस्मरणीयानां नवीवनन्यायपाठनरतानामस्माकं गुरूणां डा. नवीनहोळ्ळमहोदयानां दिङ्निर्देशेन अनुसन्धाताहं “प्राचीनन्याय-वैशेषिक-नव्यन्यायशास्त्रेषु तत्तद्व्याख्याकाराणां सैद्धान्तिकमतभेदानां सङ्कलनपूर्वकं विमर्शात्मकमध्ययनम्” इति शीर्षकालङ्कृतस्य शोधप्रबन्धस्य समर्पणावसरे समुद्घोषयामि यदेषः शोधप्रबन्धः ऐदं प्राथम्येन आरोल्लीत्युपाह्वेन राघवेन्द्रनाम्ना  मया गूरूणामाशिषा स्वप्रयत्नेन अपेक्षितसकलसामग्रीः सङ्कलय्य विश्वविद्यालयानुदानायोगस्य २००९ संवत्सरस्याधिनियमानुसारं व्यरचि~। तथा राष्ट्रियसंस्कृतसंस्थानस्य शृङ्गगिरिस्थराजीवगान्धीपरिसरे विद्यावारिधिः(Ph.D) इत्युपाधये समर्प्यत इति प्रतिजाने~। यदसौ विषयः न केनापि कुत्रचिद्विश्वविद्यालये शोधकार्यं कर्तुं स्वीकृतः इत्यपि प्रतिजाने~।
\bigskip

\noindent									
स्थानम् – शृङ्गेरी\hfill अनुसन्धाता\\ \\
दिनाङ्कः - 02/02/2015\hfill {\s राघवेन्द्र पी आरोल्ली}

\end{small}

\chapter{कार्तज्ञ्यवचनम्}

{\s वेदशास्त्रसुविचारकर्मणि प्रेषितौ मम मनीषया हि तौ ।\\
तोषयामि पितरौ निरन्तरं धन्यताञ्च कथयामि सन्ततम् ।।}

महदिदं प्रमोदस्थानं यत् शृङ्गाद्रिनिवासिन्याः भगवत्याः शारदाम्बायाः भगवतः चन्द्रमौळीशस्य च असीमानुकम्पया, जगद्गुरुश्रीश्रीभारतीतीर्थश्रीचरणानां तत्करकमलसञ्जातानां श्रीश्रीविधुशेखरभारतीमहास्वामिनां निरवग्रहानुग्रहेण च शङ्कराचार्याद्यनेकेषां यतिवरेण्यानां पादधूल्या पाविते राष्ट्रियसंस्कृतसंस्थानान्तर्गते शृङ्गेरीक्षेत्रस्थे राजीवगान्धीपरिसरे “विद्यावारिधिः” इत्युपाधये 	“प्राचीनन्यायवैशेषिकनव्यन्यायशास्त्रेषु तत्तद्व्याख्याकाराणां सैद्धान्तिकमतभेदानां सङ्कलनपूर्वकं विमर्शात्मकमध्ययनम्” इति शीर्षकाङ्कितशोधप्रबन्धः समाप्तिमगात् इति~। नव्यन्यायशास्त्रं पाठयित्वा प्राचीनादितर्कशास्त्रेषु विद्यमानसैद्धान्तिकमतभेदान्वेषणे प्रचोद्य तत्र विमर्शात्मकाध्यनरूपेऽतिगभीरेऽस्मिन्  अनुसन्धानकर्मणि  मां नियोज्य इदानीं मया समर्प्यमाणे शोधप्रबन्धे प्रतिपदं कूलङ्कषावलोकनेन मार्गदर्शनं कृतवन्तो हि मे गुरवः सदा सदाचरणसञ्चरणशीलाः विद्यातपोनुष्ठानतत्पराः न्याय-वेदान्तादिशास्त्रेषु अनितरसाधारणमतयः  विभूषितविद्यावारिधयः विद्यापीठस्य न्यायविभागे सहायकाचार्याः {\s श्रीमन्तः नवीनहोळ्ळाः}~। अतस्तेषां चरणकमलाभ्यां नतिसहस्रं समर्पयामि~।

शोधकार्यार्थम् अवसरं प्रकल्पितवद्भ्यः राष्ट्रियसंस्कृतसंस्थानस्य सम्माननीयकुलपतिभ्यः आचार्येभ्यः परमेश्वरनारायणशास्त्रिवर्येभ्यः सादरं प्रणामान् समर्पयामि~। माननीयकुलसचिवेभ्यः आचार्येभ्यः सुब्रह्मण्यशर्मामहाभागेभ्यः सगौरवं हार्दान् धन्यवादान् समर्पयामि~। तथैव परीक्षानियन्तृभ्यः आचार्येभ्यः मनोजकुमारमिश्रमहाभागेभ्यः, शोधविभागप्रमुखेभ्यश्च  हृत्पूर्वकं धन्यवादान् वितनोमि~।

शोधकार्ये साहाय्यम् आचरितवद्भ्यः राजीवगान्धीपरिसरप्राचार्येभ्यः ज्यौतिषादिशास्त्रविशारदेभ्यः अस्मच्छ्रेयोभिलाषिभ्यः आचार्येभ्यः सच्चिदानन्द उडुपमहोदयेभ्यः सश्रद्धं प्रणामान् समर्पयामि~। तथैव न्यायविभागाध्यक्षेभ्यः प्रो. का. ई. मधुसूदनवर्येभ्यः  कार्तज्ञ्यवचनानि समर्प्यन्ते~। एवमेव शोधप्रबन्धकर्मणि उपकृतवद्भ्यः परिसरस्य अध्यापकेभ्यः कार्यालयप्रमुखेभ्यश्च धन्यतामाविष्करोमि ।

विशेषतः अस्मिन् शोधकर्मणि गणकयन्त्रे ग्रन्थविन्यासार्थं मह्यं \Latex नामकं सरलं मार्गं समुद्बोधितवन्तस्तथैव अस्मिन् कर्मणि अवकाशमपि दत्तवन्तः Shriranga Digitals प्रमुखाः गणितशास्त्रे अनितरसाधारणमतयः आचार्याः योगानन्दवर्याः । अतस्तेभ्यः अन्येभ्यश्च सविनयं कृतज्ञतां समर्पयामि । 

गृहसम्बन्धीनि कार्याणि कुर्वती शोधकार्येऽस्मिन् मां तदा तदा प्रेरितवती मम सहधर्मिणी श्रीमती वीणा । अतस्तस्यै साधुवादाः वितीर्यन्ते । शोधप्रबन्धस्य सुन्दरमुखपुटविन्यासेन साहाय्यं कृतवन्तं श्री ईश्वरमूर्तिवर्यं सुहृदं संस्मरामि । तथैव तदा तदा अपेक्षितं साहाय्यं कृतवतः सहशोधकान् मन्मित्रान् नितरां स्मरामि । एवमेव प्रत्यक्षरूपेण परोक्षरूपेण च साहाय्यमाचरितवद्भ्यः सर्वेभ्योऽपि मयाऽयं कार्तज्ञ्यकुसुमाञ्जलिः समर्प्यते~।






\chapter{उपोद्घातः}



अथैतस्मिन् जगतीतले आब्रह्मस्तम्बपर्यन्तं सर्वेषां जन्तूनां दुःखापगमनार्था सुखसम्पादनार्था च निसर्गत एव जायमाना प्रवृत्तिः दृश्यते । तत्रापि दुःखनिवृत्त्यर्थं जायमाना प्रवृत्तिर्बलवती,  प्रत्यहं दुःखनिचयैः प्रतिहन्यमानत्वात् । सुखार्थं प्रवृत्तोऽपि कश्चित् तादृशसुखमनुभवन् ततोऽधिकसुखावाप्तये प्रवर्तत एव । न तु प्रायः कोऽप्येकत्र विरमति । यथा लब्धशतरूपकः‌ सहस्रमिच्छति । सहस्राधिपो लक्षमिच्छति इति ।  एतादृशसुखावाप्तिदुःखनिवृत्त्योः साधनीभूतं द्रव्यमिति विज्ञाय तदर्जने प्रवर्तन्ते लोका । एवं लोकेऽस्मिन्ननुभूयमानानि शब्दस्पर्शादिविषयसम्बन्धजन्यानि  नानाविधानि वैषयिकसुखानि दुःखनिवृत्तिश्च 

\section{वैशेषिकदर्शनम्}

\section{प्राचीनन्यायशास्त्रम्} 

\section{नव्यन्यायः} 

\section{वैशेषिकदर्शनम्} 

\section{वैशेषिकदर्शनम्} 

\section{वैशेषिकदर्शनम्} 

\begin{tabu}{|[2pt]p{3cm} | p{6cm}|[2pt]}
\tabucline[2pt]{-}
	वै.सू. & वैशेषिकसूत्रम् \\ \hline
	प्र.भा. & प्रशस्तपादभाष्यम् \\ \hline
	न्या.कं. & न्यायकन्दली \\ \hline
	न्या.ली. & न्यायलीलावती \\ \hline
	क.र. & कणादरहस्यम् \\ \hline
	कि. & किरणावलिः \\ \hline
	व्यो. & व्योमवती \\ \hline
	न्या.सू. & न्यायसूत्रम् \\ \hline
	न्या.भा. & न्यायभाष्यम् \\ \hline
	न्या.वा. & न्यायवार्तिकम् \\ \hline
	न्या.वा.ता.टी. & न्यायवार्तिकतात्पर्यटीका \\ \hline
	न्या.वा.ता.प. & न्यायवार्तिकतात्पर्यपरिशुद्धिः \\ \hline
	न्या.मं. & न्यायमञ्जरी \\ \hline
	न्या.कु. & न्यायकुसुमाञ्जलिः \\ \hline
	न्या.सा. & न्यायसारः \\
\tabucline[2pt]{-}
\end{tabu}

\begin{tabu}{|[2pt]p{3cm} | p{6cm}|[2pt]}
\tabucline[2pt]{-}
	त.चि. & तत्त्वचिन्तामणिः \\ \hline
	स.त.म. & सत्प्रतिपक्षतत्त्वचिन्तामणिः \\ \hline
	स.गा. & सत्प्रतिपक्षगादाधरिः \\ \hline
	श.वा. & शक्तिवादः \\ \hline
	श.प्र. & शब्दशक्तिप्रकाशिका \\ \hline
	प.नि. & पदार्थतत्त्वनिरूपणम् \\ \hline
	श्लो.वा. & श्लोकवार्तिकम् \\ \hline
	शा.दी. & शास्त्रदीपिका \\ \hline
	मा.मे. & मानमेयोदयः \\ \hline
	कृ.य.ब्रा. & कृष्णयजुर्वेदब्राह्मणम् \\ \hline
	वे.प. & वेदान्तपरिभाषा \\ \hline
	ब्र.शां.भा. & ब्रह्मसूत्रशाङ्करभाष्यम् \\ \hline
	तै.शां. & तैत्तिरीयशाङ्करभाष्यम् \\ \hline
	पं.पा. & पञ्चपादिका \\ \hline
	व्या. म. & व्याकरणमहाभाष्यम् \\ \hline
	अमर. & अमरकोशः \\ \hline
	स.श. & सप्तशती \\
\tabucline[2pt]{-}
\end{tabu}






\begin{center}
    \begin{tabular}{ | p{0.5cm} | p{3cm} | p{3cm} | p{4.5cm} | p{2cm} |}
    \hline
	क्र. सं. & ग्रन्थाः  & ग्रन्थकर्तारः & प्रकाशनम् & प्रकाशनसमयः \\ \hline \hline    
	१ &  &  &  & \\ \hline
	२ &  &  & & \\ \hline
	३  &  &  &  & \\ \hline
	४  &  &  &  & \\ \hline
	५  &  &  & & \\ \hline
	६  &  &  & & \\ \hline
	७  &  &  & & \\ \hline
	८  &  &  & & \\ \hline
	९  &  &  & & \\ \hline
	१०  &  &  & & \\ \hline
	११  &  &  & & \\ \hline
	१२  &  &  & & \\ \hline
	१३  &  &  & & \\ \hline
	१४  &  &  & & \\ \hline
	१५  &  &  & & \\ \hline
	१६  &  &  & & \\ \hline
	१७  &  &  & & \\ \hline
	१८  &  &  & & \\ \hline
	१९  &  &  & & \\ \hline
	२०  &  &  & & \\ \hline
	२१  &  &  & & \\ \hline
	२२  &  &  & & \\ \hline
	२३  &  &  & & \\ \hline
	२४  &  &  & & \\ \hline
	२५  &  &  & & \\ \hline
	२६  &  &  & & \\ \hline
	२७  &  &  & & \\ \hline
	२८  &  &  & & \\ \hline
	२९  &  &  & & \\ \hline
	३०  &  &  & & \\ \hline
	३१  &  &  & & \\ \hline
	३२  &  &  & & \\ \hline
	३३  &  &  & & \\ \hline
	३४  &  &  & & \\ \hline
	३५  &  &  & & \\ \hline
	३६ &  &  & & \\ \hline
	३७  &  &  & & \\ \hline
	३८  &  &  & & \\ \hline
	३९  &  &  & & \\ \hline
	४०  &  &  & & \\ \hline
	४१  &  &  & & \\ \hline
	४२  &  &  & & \\ \hline
	४३  &  &  & & \\ \hline
	४४  &  &  & & \\ \hline
	४५  &  &  & & \\ \hline
	४६  &  &  & & \\ \hline
	४७  &  &  & & \\ \hline
	४८  &  &  & & \\ \hline
	४९  &  &  & & \\ \hline
	५०  &  &  & & \\ \hline
    \hline
    \end{tabular}
\end{center}

\chapter{उपसंहारः}

	श्रीमद्गङ्गेशोपाध्यायविरचिते ”तत्वचिन्तामणौ शक्तिवादस्य विमर्शात्मकमध्ययनम्” इति विषयमधिकृत्य श्रीमदाचार्याणां डा। नवीनहोळ्ळमहाभागानां मार्गदर्शनेन शोधकार्यं मया व्यधायि~। विद्यावारिध्युपाधये विषयमिमं स्वीकृत्य पञ्चाध्यायात्मकः शोधप्रबन्धः सज्जीकृतः~। तत्रादौ शोधप्रबन्धस्य भूमिकारूपेण शोधविषयः प्रस्तुतः~। प्रथमाध्याये जातिशक्तिमाश्रित्यपूर्वपक्षः मया न्यरूपि~। अनन्तरं व्यक्तिशक्तिवादे प्राभाकरोक्तदूषणमवलोकितम्~। एवं जातिशक्तिवादे व्यक्तिभानानुपपत्तेः समाधानमत्र विवेचितम्~। तथा च व्यक्तिशक्तिवादसमर्थनम्~, द्वितीयाध्याये पङ्कजपदस्य रूढिं विना पद्मबोधक्त्वमिति पूर्वपक्षोपस्थापनं, पङ्कजपदस्य रूढेः समर्थनमित्यादिविषयाः निरूपिताः~। तृतीयाध्याये पदसाधुत्वविचारः,अपभ्रंशविचारश्च विशेषेण चर्चितः~। चतुर्थाध्याये लक्षणायाः वृत्त्यन्तरत्वादि अनेके विषयाः विचारार्थं स्वीकृताः~। पञ्चमाध्याये  तु अकृतेरपि पदशक्यताविचारः,योगरूढ्योः बलाबलविचारश्च प्रस्तुतः~।

	श्रीमद्भिः गङ्गेशोपाध्यायैः विरचितोऽयं तत्त्वचिन्तामणिनामकः ग्रन्थः, सकलन्यायतर्कग्रन्थेभ्यः सारमुद्धृत्य नव्यपरिष्कारपरिष्कृतः, अनितरसाधारणः, बह्वर्थगभीरश्च विराजते~। ग्रन्थस्यास्य नैकाः टीकाः वर्तन्त एव~। तथाऽपि तत्र  क्रमेण रघुनाथशिरोमणि-गदाधरभट्टाचार्य-जगदीशतर्कालङ्कारविरचितदीधिति-गादाधरी-जागादीशीतिव्याख्याः प्रसिद्धाः इति निश्चप्रचमेव~। तत्त्वचिन्तामणौ जातिशक्तिवादे मीमांसकानामेव पूर्वपक्षः विशिष्य अवलोकितः~। तत्रस्थविचाराः शोधप्रबन्धेऽस्मिन् यथाशक्ति मया विवेचिताः~। 

	तत्त्वचिन्तामणेः बहूनि व्याख्यानानि सन्त्येव, तथाऽपि जागदीश्याः अनुसन्धानं बहुषु भागेषु कृतं, समादृतञ्च~। तत्रापि इतोऽपि अधिकविषये अनुसन्धानं शोधकर्तृभिः कर्तुं शक्यते~। तद्विषये अध्ययनार्थं शोधप्रबन्धोऽयं प्रेरकश्चेत् आत्मानं धन्यम्मन्ये~। अनेन विमर्शात्मकाध्ययनेन व्याख्यानानां स्पष्टप्रतिपत्तिः, मूलग्रन्थस्य वैशिष्ट्यं, गभीरता च करतलामलकायते~। 

	शास्त्राणि दुर्ग्राह्याणि तथापि त्रिकालसत्यानि बहुभिः बहुधा व्याख्यातानि च, अतः इतः परमपि विमर्शात्मकाध्ययनस्य प्रसक्तिः वरीवर्ति~। कनीयांसः मे अनुजकल्पाः अनेन प्रेरिताः स्युः, विद्वज्जनाश्च माम् अनुगृह्णीयुरिति आशास्य जगदम्बां शारदाम्बां जगद्गुरुचरणाम्बुजञ्च प्रणिपत्य विस्तरात् विरमामि~।
	
	इत्थं शोधप्रबन्धोऽयम् उपोद्घातेन, ग्रन्थसम्पादनेन, अध्ययनेन, उपसंहारेण, परिशिष्टेन, परिशीलितग्रन्थसूच्या च समलङ्कृतोऽस्ति~।
\bigskip

\noindent									
माघ-पूर्णिमा\hfill अनुसन्धाता\\ \\
भौमवासरः\hfill {\s राघवेन्द्र पी. आरोल्ली}
