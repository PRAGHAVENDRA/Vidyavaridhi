
\begin{center} \begin{huge}शोधसंक्षिप्तिका\\\end{huge}  
\begin{Large}॥ प्राचीनन्याय-वैशेषिक-नव्यन्यायशास्त्रेषु\\[-5mm] तत्तद्व्याख्याकाराणां सैद्धान्तिकमतभेदानां\\[-5mm] सङ्कलनपूर्वकं विमर्शात्मकमध्ययनम्~॥\\[1mm]\end{Large}
\begin{large}(A critical study of differant theoratical\\[-5mm] veiw-points of commentatores of\\[-5mm] prachinanyaya-vaisheshika-navyanyaya)\end{large}
\end{center}

\section{उपोद्घातः}
इह किल सकलप्राणिसमभिलषितायामात्यन्तिक-दुःखनिवृत्तावात्म-तत्त्वसाक्षात्कारस्यैव प्रयोजकत्वं व्यवातिष्ठिपन् शास्त्रनदीष्णवः~। तत्रापि श्रुतिविहितेषु श्रवण-मनन-निदिध्यासनेष्वसम्भावनादिसकलधीदोषनिरासफलकं मननमेवाभ्यर्हिततममिति निरदीधरन् दार्शनिकाः~। मननञ्चेदं सपरिकरैर्हेतुभिरनुमानमेव~। तच्चेदं चिरन्तनैः कणादगौतमादिभिः महर्षिभिः नव्यैश्च गङ्गेशोपाध्यायप्रभृतिभिः विद्वद्भिः पदार्थविमर्शपुरस्सरं व्युत्पादितमित्यत्र प्रेक्षावतामन्तरङ्गमेव प्रमाणम्~।
	
धर्मविशेषप्रसूतात् द्रव्यगुणकर्मसामान्यविशेषसमवायानां पदार्थानां साधर्म्यवैधर्म्याभ्यां तत्वज्ञानान्निःश्रेयसम् इति शास्त्रस्यास्यापवर्गप्रयोजनताप्रतिपादकं काणादसूत्रम्~। द्रव्यादिषड्भावपादार्थविचारमेव प्रधानतयाकार्षुः वैशेषिकाः~।तेषां पदार्थनिरूपणविषयेऽभियुक्तोक्तिरियम्~-\\
         \begin{center} द्वित्वे च पाकजोत्पत्तौ विभागे च विभागजे ।\\[-1mm]
         यस्य न स्खलिता बुद्धिः तं वै वैशेषिकं विदुः~॥\end{center} इति ।
	
प्रमाणादिषोडषपदार्थपरीक्षणपरमक्षपादप्रणीतं न्यायशास्त्रम्~। अनु ईक्षा-अन्वीक्षा, तद्व्युत्पादकं शास्त्रम्-आन्वीक्षिकीविद्या इति~। सा च न्यायशास्त्रमेव~। गौतममुनिप्रणीतमिदमष्टादशविद्यास्थानेषु  प्रधानतया  गणितम्~।
	
एतयोर्माणवकत्वेनोत्पन्नमेव श्रीमद्गङ्गेशोपाध्यायप्रणीतं परिष्कारजटिलं नव्यन्यायशास्त्रम् ।


\section{न्यायवैशेषिकदर्शनयोर्विकासः}
भगवता गौतमेन आदौ न्यायसूत्राणि प्रणीतानि~। तेषां सूत्राणां वात्स्यायनमहर्षिणः भाष्यमकार्षुः~। तस्योपरि वार्तिकमरचयत् भारद्वाजोद्योतकरः~। वाचस्पतिमिश्राः वार्तिकतात्पर्यटीकां प्रचक्रुः~। तस्योपरि वार्तिकतात्पर्यपरिशुद्धिनामकं ग्रन्थमुदयनाचार्याः कृतवन्तः~। एवं न्यायकुसुमाञ्जलिः  आत्मतत्वविवेकः  इत्यादिग्रन्थाः उदयनाचार्यैः विकसिताः~। ततः परं न्यायसारमिति स्वतन्त्रग्रन्थं विलिख्य तस्य व्याख्यानमप्यकार्षुः भासर्वज्ञाः~। एवं क्रमेण प्राचीनन्यायप्रपञ्चस्योदयः ।

कणादमहर्षिणा वैशेषिकसूत्राणि प्रणीतानि~। तेषां भाष्यं प्रशस्तपादाः प्रचक्रुः~। तस्योपरि अष्टौ व्याख्यानानि सन्ति~। तत्र श्रीधराचार्याणां न्यायकन्दली, उदयनाचार्याणां किरणावली, व्योमशिवाचार्याणां व्योमवती, पद्मनाभाचार्याणां सेतुटीका जगदीशतर्कालङ्काराणां सूक्तिटीका इत्येतान्येव प्रसिद्धानि व्याख्यानानि~। न्यायलीलावती इत्यादयः वल्लभाचार्यादिभिः कृताः स्वतन्त्रग्रन्था अपि दृश्यन्ते ।
श्रीमद्गङ्गेशोपाध्यायैः तत्वचिन्तामणिनामकोग्रन्थः प्रणीतः~। ततः नव्यन्यायप्रपञ्चस्योदयः~। लिखनसंक्षेपनिर्बन्धिनः श्रीरघुनाथशिरोमणिवर्याः तस्य व्याख्यानं दीधितिनामकग्रन्थद्वारा अकार्षुः~। तस्य अनेकानि व्याख्यानानि सन्ति~। तत्र श्रीगदाधरभट्टाचार्याणां गादाधरी, श्रीजगदीशतर्कालङ्काराणां जागदीशी, श्रीमथुरानाथानां माथुरी इत्यादयः प्रमुखाः ग्रन्थाः वर्तन्ते~। एवं जगदीशस्य शब्दशक्तिप्रकाशिका, गदाधरस्य व्युत्पत्तिवादशक्तिवादादयः स्वतन्त्रग्रन्थाः अपि विद्यन्ते ।

\section{अनुसन्धानस्योद्धेशः}
प्राचीनन्याय-वैशेषिक-नव्यन्यायशास्त्रेषु पदार्थतत्वविवेचकाः प्रमाणतत्वनिरूपणपराः शास्त्रव्युत्पत्तिसम्पादकाः वात्स्यायन-उद्योतकर-प्रशस्तपाद-श्रीधराचार्य-रघुनाथ-गदाधरादिन्यायवैशेषिक-चक्रवर्तिभिः सङ्ग्रथिताः नैके ग्रन्थाः विलसन्ति~। तेषु विदुषां स्वसंस्कारवैचित्र्यात् प्रतिभाविशेषाच्च तत्र तत्र सैद्धान्तिकमतभेदः परिदृश्यते~। श्रुति-स्मृत्यादिप्रधानेभ्यः वेदान्तसाङ्ख्यादिभ्यः युक्तिप्रधानस्यास्य मतभेदस्य स्वाभाविकत्वेऽपि युक्तिप्रधानेषु प्राचीनन्याय-वैशेषिक-नव्यन्यायशास्त्रेषु तत्तद्वाख्यातॄणां सैद्धान्तिकमतभेदः नूनं विचारार्हो वर्तते~। यद्यपि तत्तच्छास्त्रार्थस्य प्रतिपादनसरण्यां प्रक्रियायां व्याख्याने वा मतभेदो न क्षतिमावहेत्~। तथापि सैद्धान्तिकमतभेदः नूनमेवाध्येतॄणां शास्त्रेष्वप्रामाण्यशङ्कामुत्पादयेत्~, उत्कटां निष्कर्शजिज्ञासाञ्च जनयत्येव~। अत एव विषयोऽयं संशोधने प्राधान्यमावहतीति साम्प्रतं प्राचीनन्याय-वैशेषिक-नव्यन्यायशास्त्रेषु तत्तद्व्याख्यानकाराणां सैद्धान्तिकमतभेदसङ्कलनपूर्वकं विमर्शात्मकमध्ययनम् इति विषयः संशोधनाय स्वीकृतः ।



\section{अनुसन्धानस्य सीमा}
\begin{itemize}
\item तत्त्वचिन्तामणौ शक्तिवादस्य तद्व्याख्यानस्य च अध्ययनम्~।
\item गदाधरीयशक्तिवादस्य शब्दशक्तिप्रकाशिकायाश्च अवलोकनम्~।
\item न्यायसूत्र-वार्तिक-तात्पर्यटीका-परिशुद्धीनां परिशीलनम्~।
\item महाभाष्य-वाक्यपदीय-लघुमञ्जूषादीनां परिशीलनम्~।
\item शाबरभाष्य-भाट्टदीपिकयोः परिशीलनम्~।
\item न्यायमीमांसासिद्धान्तयोः तुलनम्~।
\end{itemize}



\section{प्रथमोऽध्यायः}
 प्रमाणविषयकमतभेदपरीक्षणम्
\begin{itemize}
\item उपमानप्रमाणविषये वैशेषिकन्यायसरण्योः वैलक्षण्यम्
\renewcommand{\labelenumii}{\Roman{enumii}}	
	\begin{enumerate}	
	\item वैशेषिकानामाशयः
	\item नैयायिकानामाशयः
	\item शास्त्रान्तरीयविषयाः
	\item विमर्शः
	\end{enumerate}

\item आकृतेः पदशक्यत्वविषये प्राच्यनव्यन्यायसरण्योः वैलक्षण्यम्
	\begin{enumerate}
	\item प्राचीनानामाशयः
	\item नव्यानामाशयः
	\item शास्त्रान्तरीयविषयाः
	\item विमर्शः
	\end{enumerate}
इमे विषयाः अध्यायेऽस्मिन् निरूप्यन्ते~।


\section{द्वितीयोऽध्यायः}
द्रव्यविषयकमतभेदपरीक्षणम्
\item वायोः त्वाचप्रत्यक्षत्वे वैशेषिकनव्यन्याययोः वैलक्षण्यम्
	\begin{enumerate}
	\item वैशेषिकानामाशयः
	\item नव्यनैयायिकानामाशयः
	\item शास्त्रान्तरीयविषयाः
	\item विमर्शः
	\end{enumerate}

\item परमाणु विषये वैशेषिकनव्यन्याययोः वैलक्षण्यम्
	\begin{enumerate}
	\item वैशेषिकानामाशयः
	\item नव्यनैयायिकानामाशयः
	\item शास्त्रान्तरीयविषयाः
	\item विमर्शः
	\end{enumerate}

इमे विषयाः अध्यायेऽस्मिन् निरूप्यन्ते~।



\section{तृतीयोऽध्यायः}
गुणविषयकमतभेदपरीक्षणम्
\item रूपादीनामव्याप्यवृत्तित्वविषये वैशेषिकनव्यन्याययोः वैलक्षण्यम्
	\begin{enumerate}	
	\item वैशेषिकानामाशयः
	\item नव्यनैयायिकानामाशयः
	\item शास्त्रान्तरीयविषयाः
	\item विमर्शः
	\end{enumerate}

\item पीलुपाकपिठरपाकयोः विचारः
	\begin{enumerate}
	\item वैशेषिकानामाशयः
	\item नैयायिकानामाशयः
	\item विमर्शः
	\end{enumerate}

\item परोक्षसंशयसत्त्वे वैशेषिकनव्यन्यायसरण्योः‌ वैलक्षण्यम्
	\begin{enumerate}
	\item वैशेषिकानामाशयः
	\item नव्यनैयायिकानामाशयः
	\item शास्त्रान्तरीयविषयाः
	\item विमर्शः
	\end{enumerate}

इमे विषयाः अध्यायेऽस्मिन् निरूप्यन्ते~।


\section{चतुर्थोऽध्यायः}
समवायविषयकमतभेदपरीक्षणम्
\item समवायस्य एकत्वे ऐन्द्रियकत्वे च वैशेषिक-प्राच्य-नव्यन्यायसरणीनां वैलक्षण्यम्
	\begin{enumerate}
	\item वैशेषिकानामाशयः
	\item नैयायिकानामाशयः
	\item नव्यनैयायिकानामाशयः
	\item शास्त्रान्तरीयविषयाः
	\item विमर्शः
	\end{enumerate}
इमे विषयाः अध्यायेऽस्मिन् निरूप्यन्ते~।

 \section{पञ्चमोऽध्यायः}
अभावविषयकमतभेदपरीक्षणम्
\item तमस्स्वरूपविषये कन्दलीकाराणामितरेषां वैशेषिकानाञ्च मतवैलक्षण्यम्
	\begin{enumerate}
	\item कन्दलीकाराणामाशयः
	\item इतरेषां वैशेषिकानामाशयः
	\item शास्त्रान्तरीयाणामाशयः
	\item विमर्शः
	\end{enumerate}
 
इमे विषयाः अध्यायेऽस्मिन् निरूप्यन्ते~।

\section{षष्ठोऽध्यायः}
अपवर्गविषयकमतभेदपरीक्षणम्
\item मोक्षकारणविषये न्यायसारकृतामितरेषां च मतवैलक्षण्यम्
	\begin{enumerate}
	\item न्यायसारकृतामाशयः
	\item इतरेषां नैयायिकानामाशयः
	\item शास्त्रान्तरीयाणामाशयः
	\item विमर्शः
	\end{enumerate}

\item मोक्षस्वरूपविषये वैशेषिकानां नैयायिकानां च मतवैलक्षण्यम्
	\begin{enumerate}	
	\item वैशेषिकानामाशयः
	\item नैयायिकानामाशयः
	\item शास्त्रान्तरीयविषयाः
	\item विमर्शः
	\end{enumerate}
\end{itemize}

इमे विषयाः अध्यायेऽस्मिन् निरूप्यन्ते~।      
 

\section{उपसंहारः}
अनेन प्रकारेण सुपरिश्रम्य यथाज्ञानं यथाकालं च सज्जीकृतोऽसौ शोधप्रबन्धः~। शोधप्रबन्धोऽयं न्यायशास्त्रमधीयानानां विमर्शकानाञ्च उपयोगाय कल्पत इति मे द्रढीयान् विश्वासः ।
इत्थं  “प्राचीनन्याय-वैशेषिक-नव्यन्यायशास्त्रेषु तत्तद्व्याख्याकाराणां सैद्धान्तिकमतभेदानां सङ्कलनपूर्वकं विमर्शात्मकमध्ययनम्” इत्यमुं विषयमधिकृत्य राष्ट्रिय-संस्कृतसंस्थानस्य राजीवगान्धीपरिसरस्य न्यायविभागीयप्राध्यापकानाम् आचार्याणां  नवीनहोळ्ळवर्याणां मार्गदर्शने  विद्यावारिधिः इत्युपाधये मयैव शोधप्रबन्धः सज्जीकृतो वर्तते~। अन्यत्र च असमर्पितोऽयं शोधप्रबन्धः~। शोधप्रबन्धसमर्पणात् प्राक्  “शोधसंक्षिप्तिका” संस्थायै समर्पणीया इति नियमानुसारेण शोधसारः समर्प्यते ।\\[10mm]

\setlength\parindent{0pt}

मार्गनिर्देशकः\hfill भवतां विधेयः\\[7mm]
\setlength\parindent{0pt}								
आचार्यः  डा. वि. नवीनहोळ्ळः\hfill राघवेन्द्र पी. आरोल्ली\\
प्राध्यापकः, न्यायविभागः\\
राजीवगान्धीपरिसरः





\section{आकरग्रन्थाः}
\begin{itemize}
\item न्यायसूत्रम्~।
\item न्यायभाष्यम्~।
\item न्यायवार्तिकम्~।
\item न्यायवार्तिकतात्पर्यटीका – चौखाम्भा संस्कृत संस्थान~, वारणासी~।
\item न्यायवार्तिकतात्पर्यटीकापरिशुद्धिः–भारतीयदार्शनिकानुसन्धानपरिषत्प्रकाशिता~।
\item न्यायमञ्जरी 
\item प्रशस्तपादभाष्यम् – सम्पूर्णानन्द संस्कृत विश्वविद्यालय~, वारणासी~।
\item न्यायलीलावती - चौखाम्बा संस्कृत सीरीज आफिस~, वारणासी~।
\item न्यायकन्दली
\item कणादरहस्यम्
\item तत्त्वचिन्तामणिः - चौखाम्बा संस्कृत प्रतिष्ठान दिल्ली~।
\item दीधितिः - चौखाम्बा संस्कृत सीरीज आफिस~, वारणासी~।
\item गादाधरी- चौखाम्बा संस्कृत सीरीज आफिस~, वारणासी~।
\item जागदीशी - चौखाम्बा संस्कृत सीरीज आफिस~, वारणासी~।
\item मुक्तावळी
\item दिनकरी
\item रामरुद्री
\item मानमेयोदयः – चौखाम्बा विद्याभवन~,वारणासी~।
\item शास्त्रदीपिका
\item वेदान्तपरिभाषा
\item तत्त्वदीपिका - चौखाम्बा संस्कृत प्रतिष्ठान दिल्ली~।
\item शाब्दबोधमीमांसा – राष्ट्रियसंस्कृतसंस्थानम्~, दिल्लि~।
\item  शब्दशक्तिप्रकाशिका – चौखाम्बा संस्कृत प्रतिष्ठान दिल्ली~।
\end{itemize}
