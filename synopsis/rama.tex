\usepackage{fontspec,graphicx}
\usepackage{graphicx}
\graphicspath{ {images/} }
\usepackage{indentfirst}
\parindent 1.5cm
\usepackage{setspace}
\tolerance=1
\emergencystretch=\maxdimen
\hyphenpenalty=10000
\hbadness=10000
\hyphenchar\font=-1
\sloppy
\usepackage{polyglossia}
\usepackage{hyperref}
\hypersetup{
    colorlinks=true,
    linkcolor=blue,      
    urlcolor=cyan,
    pdftitle={mainfile},
    pdfpagemode=FullScreen,
}

\usepackage{titlesec}
%\usepackage{thispagestyle}

\titleformat{\section}
  {\normalfont\fontsize{12}{15}\bfseries}{\thesection}{1em}{}
\usepackage{metalogo}
\usepackage{perpage}
\MakePerPage{footnote}
 





%\setmainfont{Sanskrit2003}
\setmainfont[Scale=1.8]{Sanskrit2003}
\setdefaultlanguage{sanskrit}
\newfontfamily\AA[Script=Devanagari]{Sanskrit2003}

\usepackage{blindtext}
\usepackage{multicol}
\setlength{\columnsep}{2cm}
\usepackage{fancyhdr}
\usepackage{xltxtra}
\usepackage{lipsum}
\usepackage{titlesec}
\titlespacing*{\section}{0pt}{0.5\baselineskip}{0.5\baselineskip}

\renewcommand*\contentsname {विषयानुक्रमणिका}

\usepackage[papersize={210 mm,297mm},textwidth=100mm,
textheight=200mm,headheight=14.5mm,headsep=12.7mm,topmargin=1.5cm,botmargin=3cm,
leftmargin=4cm,rightmargin=2.5cm,footskip=1.0cm]{zwpagelayout}


\linespread{2.6}
\renewcommand{\headrulewidth}{0.1pt}
\makeatletter
\def\@makechapterhead#1{%
  %\vspace*{10\p@}%
  {\parindent \z@ \raggedright \normalfont
    \ifnum \c@secnumdepth >\m@ne
      \if@mainmatter
        \Large \bfseries \center \space 
        \par\nobreak
        \vskip 6\p@
      \fi
    \fi
    \interlinepenalty\@M
    \Large \bfseries #1\par\nobreak
    \vskip 1\p@
  }}
\makeatother

\fancypagestyle{}{%
\chead[]{}
\lhead[]{}
\rhead[]{}
\cfoot[]{}
}

 
\pagestyle{fancy}
\fancyhf{}
\fancyhead[LE,LO]{\leftmark}
\fancyhead[LE,RO]{}
\fancyfoot[CE,CO]{\bfseries \thepage}

\renewcommand\chaptermark[1]{\markboth{\thechapter. #1}{}}
\renewcommand\sectionmark[1]{}
\renewcommand\subsectionmark[1]{}
\renewcommand\subsubsectionmark[1]{}


\pretolerance=9990
