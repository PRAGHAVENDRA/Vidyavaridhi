\section{समवायः}


'घटपटौ संयुक्तौ' इति विशिष्टप्रतीतौ संसर्गस्य संयोगस्यापि विषयत्वमनुभवसिद्धमेव | तथा च सर्वत्रापि विशिष्टप्रतीतौ संसर्गः भासत एव | 'फलं पतति' इत्यपि विशिष्टप्रतीतिरेव | तत्र पतनस्य फलेन सह कः संसर्गः इति जिज्ञासायां न संयोगः‌ 'द्रव्ययोरेव संयोगः' इति नियमः लोके दृष्टः‌, न निरूपकत्वादिकं वृत्त्यनियामकसंसर्गत्वात् , तस्मादतिरिक्तः‌ एव संसर्गः‌ कल्प्यते समवाय इति | तथा चानुमानं 'फलं पतति इति विशिष्टप्रतीतिः‌ विशेषणविशेष्यसंभन्धविषया विशिष्टप्रतीतित्वात् भूतलं घटवदिति विशिष्टप्रतीतिवत्' इति | अत्रानेनानुमानेन लाघवादेकः‌ नित्यश्च संसर्गः कल्प्यते | स च समवाय इति | अत्र अस्य नित्यत्वे एकत्वे ऐन्द्रियकत्वे च ग्रन्थकर्तृषु विप्रतिपत्तिः दृश्यते | तदत्र निरूप्यते |

\subsection{समवायसद्भावे प्रमाणपदर्शनम्} 

'किं पुनः समवायसिद्धौ मानम् ?' इत्यादिना श्री वल्लभाचार्याः समवायसद्भावे प्रत्यक्षं प्रमाणं निराकृत्य  अनुमानप्रमाणं प्रदर्शयामसुः | तथा हि - समवायसद्भावे तस्यातीन्द्रियत्वात् प्रत्यक्षं प्रमाणं भवितुं नार्हति | न च 'इह तन्तुषु पटः' इत्यादिप्रत्यक्षप्रतीतयः तत्र प्रमाणमिति वाच्यम् | द्रव्यगुणकर्मसामान्यविशेषेषु विद्यमानस्य समवायस्य एकत्वात् एकसम्बन्धविशिष्टानां तेषां कस्य केन समवायः इति व्यवस्थानुपपत्तेः | जलादिषु रूपादिसमवायस्यापि ग्रहणप्रसङ्गात् | तर्हि तत्सत्त्वे किं मानमिति चेदुच्यते - 'जात्यादि\footnote{अत्रादिशब्देन द्रव्यगुणकर्मविशेषाणां ग्रहणम् |}विषयकः  विशिष्टव्यवहारः विशेषणविशेष्यसम्बन्धविषयः भावमात्रविषयाबाधितविशिष्टव्यवहारत्वात् सघटं भूतलमिति विशिष्टव्यवहारवत्' इत्यनुमानं प्रमाणम् | अत्र विशिष्टव्यवहाराणां सर्वेषां विशेषणविशेष्यसम्बन्धविषयकत्वनियमात् जात्यादिविषयकविशिष्टव्यवहाराणामपि विशिष्टव्यवहारत्वात् सम्बन्धविषयकत्वं सिध्यति | न च तत्र संयोगः सम्बन्धः, द्रव्ययोरेव संयोगसम्भवात् | नापि स्वरूपसम्बन्धः, अभावस्यैव तन्नियमात् | एतत्सूचनायैव हेतुकुक्षौ भावमात्रविषयत्वं योजितम् इति |

{\fontsize{11.7}{0}\selectfont\s अत्रोच्यते - जात्यादिगोचरो विशिष्टव्यवहारः सम्बन्धनियतः भावमात्रविषयाबाधितविशिष्टव्यवहारत्वात् सघटं भूतलमिति विशिष्टव्यवहारवत् |\footnote{न्या.ली. ७०९}}



\subsection{समवायस्यैकत्वं नित्यत्वमतीन्द्रियत्वञ्च} 

एकः नित्यश्च समवायः इति वैशेषिकभाष्यकाराणां प्रशस्तपादाचार्याणामाशयः | तथा हि - 'इह तन्तुषु पटः', 'इह घटे रूपम्' इत्यादिज्ञानेषु समवायनिमित्तस्य इहेत्यंशस्य दर्शनात् , तस्य नानात्वे प्रत्यक्षादेः बाधकस्य असत्वात् लाघवाच्च एकः समवायः इति सिध्यति | न च द्रव्यगुणकर्मसु विद्यमानानां द्रव्यत्वगुणत्वकर्मत्वादीनां सम्बन्धैकत्वात् तेषां सङ्करप्रसङ्गः | द्रव्यत्वेनैव द्रव्यस्य भानात् गुणत्वादिना द्रव्यत्वाभानात् अन्वयव्यतिरेकाभ्यां समवायैक्येऽपि द्रव्यत्वादीनां द्रव्येष्वेव सम्बन्धः इति प्रतिनियमो सिध्यति | यथा कुण्डदध्नोः संयोगस्य एकत्वेऽपि तयोः अधाराधेयभावः, तद्वत् | 

न च संयोगस्य यथा सम्बन्ध्यनित्यत्वात् अनित्यत्वं तथैव समवायस्यापि अनित्यत्वं भवतु इति वाच्यम् | भावस्य तस्योत्पत्तौ उत्पादकाभावात् क्वचित् सम्बन्धि\footnote{परमाणुतद्गतपरिमाणादयः}नित्यत्वात् उत्पादकादिकल्पनाभावप्रयुक्तलाघवाच्च समवायः नित्यसम्बन्ध एव | 

ननु एतादृशः समवायः केन सम्बन्धेन तिष्ठति ? न संयोगेन, द्रव्ययोरेव संयोगनियमात् | न समवायेन, तस्यैकत्वकथनात् | नान्यः कोऽपि सम्बन्धो विद्यते इति चेन्न | तादात्म्यमेव समवायस्य सम्बन्धः | अत एव तादात्म्यघटितेन्द्रियसन्निकर्षाभावात् तस्यातीन्द्रियत्वमिति |

{\fontsize{11.7}{0}\selectfont\s ननु यद्येकः समवायः ? द्रव्यगुणकर्मणां द्रव्यत्वगुणत्वकर्मत्वादिविशेषणैः सह सम्बन्धैकत्वात् पदार्थसङ्करप्रसङ्ग इति | न, आधाराधेयनियमात् | यद्यप्येकः समवायः सर्वत्र स्वतन्त्रः, तथाप्याधाराधेयनियमोऽस्ति | कथं द्रव्येष्वेव द्रव्यत्वम् , गुणेष्वेव गुणत्वम् , कर्मस्वेव कर्मत्वमिति | एवमादि कस्मात् ? अन्वयव्यतिरेकदर्शनात् | इहेति समवायनिमित्तस्य ज्ञानस्यान्वयदर्शनात् सर्वत्रैकः समवाय इति गम्यते | द्रव्यत्वादिनिमित्तानां व्यतिरेकदर्शनात् प्रतिनियमो ज्ञायते | यथा कुण्डदध्नोः संयोगैकत्वे भवत्याश्रयाश्रयिभावनियमः | तथा द्रव्यत्वादीनामपि समवायैकत्वेऽपि व्यङ्ग्यव्यञ्जकशक्तिभेदादाधाराधेयनियम इति | सम्बन्ध्यनित्यत्वेऽपि न संयोगवदनित्यत्वं भाववदकारणत्वात् | यथा प्रमाणत उपलभ्यत इति | कया पुनर्वृत्त्या द्रव्यादिषु समवायो वर्तते ? न संयोगः सम्भवति, तस्य गुणत्वेन द्रव्याश्रितत्वात् | नापि समवायः, तस्यैकत्वात् | न चान्या वृत्तिरस्ति ? न, तादात्म्यात् | यथा द्रव्यगुणकर्मणां सदात्मकस्य भावस्य नान्यः सत्तायोगोस्ति | एवमविभागिनो वृत्त्यात्मकस्य समवायस्य नान्या वृत्तिरस्ति, तस्मात् स्वात्मवृत्तिः | अत एवातीन्द्रियाः सत्तादीनामिव प्रत्यक्षेषु वृत्त्यभावात् , स्वात्मगतसंवेदनाभावाच्च | तस्मादिह बुध्यनुमेयः समवाय इति |\footnote{प्र.भा. ७७८-७८५}}

\subsection{स्वरूपापेक्षया समवायः भिन्न एव} 

श्रीमद्भिः गङ्गेशोपाध्यायैः तत्त्वचिन्तामणौ समवायसम्बन्धविचारः विशदतया निरूपितः | तथा हि - समवायसद्भावे  गुणक्रियाजातिविशिष्टबुद्धयो विशेषणसम्बन्धविषयाः विशिष्टबुद्धित्वात् दण्डीति बुद्धिवत् , गुणक्रियाजातिविशिष्टबुद्धयो विशेषणसम्बन्धनिमित्तिकाः सत्यत्वे सति विशिष्टबुद्धित्वात् , दण्डीति बुद्धिवत् इति वा अनुमानं  प्रमाणम् | तत्रादौ प्रथमानुमाने 'नीलो घटः' इत्यादिविशिष्टबुद्धौ विशेषणसम्बन्धविषयकत्वं सिध्यति, तस्याः विशिष्टबुद्धित्वात् | न च 'अघटं भूतलम्' इत्यादौ व्यभिचारः इति वाच्यम् | तत्रापि विशिष्टबुद्धौ स्वरूपसम्बन्धस्य विषयत्वात् | न च स्वरूपसम्बन्धमादाय अनाकांक्षितार्थाभिधानादर्थान्तरम् | स्वरूपसम्बन्धेनैव प्रकृतपक्षेऽपि विशिष्टव्यवहारसम्भवादिति वाच्यम् | गुणक्रियाजातिविशिष्टबुद्धीनां विशिष्टबुद्धित्वात् सम्बन्धविषयकत्वन्तु कल्पनीयमेव | कल्प्यमानश्च सः लाघवज्ञानसहकृतपक्षधर्मताज्ञानेन एक एव सिद्ध्यति | स च स्वरूपादतिरिक्तः | अन्यथा अनन्तानां स्वरूपाणां सम्बन्धकल्पने आनन्त्यरूपगौरवप्रसङ्गः |

द्वितीयानुमानेऽपि विशेषणसम्बन्धः विशिष्टव्यवहारनिमित्तत्वेन सिध्यन् लाघवादेक एव सिध्यति | तेन अनुगतानां तादृशविशिष्टबुद्धीनां अनुगतसम्बन्धजन्यत्वमपि उपपद्यते | अस्मिन्ननुमाने हेतुकुक्षौ सत्यत्वदलं निवेशनीयम् | अन्यथा भ्रमात्मकविशिष्टज्ञाने विशेषणसम्बन्धनिमित्तकत्वाभावात् व्यभिचारः स्यात् | 

न चेदमुभयमप्यप्रयोजकम् | विशिष्टसाक्षात्कारस्य सम्बन्धविषयकत्वतज्जन्यत्वनियमात् | अन्यथा गवाश्वादावपि विशिष्टबुद्धिप्रसङ्गः इति |

{\fontsize{11.7}{0}\selectfont\s  गुणक्रियाजातिविशिष्टबुद्धयो विशेषणसम्बन्धविषयाः विशिष्टबुद्धित्वात् , दण्डीति बुद्धिवत् | न च व्यभिचारः | अभावादिविशिष्टबुद्धेरपि स्वरूपसम्बन्धविषयत्वात् | न चैवमत्रापि तेनैवार्थान्तरम् , यतो गुणक्रियाजातिविशिष्टबुद्धीनां पक्षधर्मताबलेन विषयः सम्बन्धः सिध्यन् लाघवादेक एव सिध्यति | स एव समवायः, न तु स्वरूपसम्बन्धः, तत्स्वरूपाणामनन्तत्वात् सम्बन्धत्वेनाक्लृप्तत्वाच्च |}

{\fontsize{11.7}{0}\selectfont\s   अथ वा विशेषणसम्बन्धनिमित्तिका इति साध्यम् | हेतौ तु सत्यत्वं विशेषणम् | विशेषणसम्बन्धश्च कारणत्वेनैक एव सिध्यति | लाघवात् , अनुगतकार्यस्य अनुगतकारणनियम्यत्वाच्च | न तु स्वरूपसम्बन्धः, तेषामननुगतत्वादनन्तत्वाच्च | न चोभयमप्यप्रयोजकम् | विशिष्टसाक्षात्कारस्य सम्बन्धाविषयत्वे तदजन्यत्वे वा गवाश्वादावपि विशिष्टबुद्धिप्रसङ्गात् |\footnote{त.म. ३२}}


\subsection{समवायोऽपि नैकः} 

दीधितिकारेति प्रसिद्धाः श्रीरघुनाथशिरोमणयः स्वकीये पदार्थतत्त्वनिरूपणाख्ये ग्रन्थे समवायस्य नानात्वं प्रतिपादयामासुः | तथा हि - समवायः यदि एकः स्यात् तदा पृथिव्यां विद्यमानस्य गन्धस्य जलादावुपलब्धिप्रसङ्गः | तत्रापि अनुयोगितया स्नेहसमवायस्य प्रसिद्धत्वात् | एवं वाय्वादौ रूपाद्युपलब्धिप्रसङ्गः | तस्मात् नानैव समवायः | समवायत्वं तु न जातिः असम्बन्धात् | अपि तु सकलसमवायानुगतः अखण्डोपाधिरेव इति |

{\fontsize{11.7}{0}\selectfont\s समवायोऽपि च नैको जलादेर्गन्धादिमत्त्वप्रसङ्गात् | परन्तु नानैव, समवायत्वं तु पुनरनुगतमखण्डोपाधिरिति |\footnote{प.नि. १४९}}


\subsection{समवायः प्रत्यक्षः} 

न्यायसारे तु समवायस्य प्रत्यक्षत्वमुपपादितम् | तथा हि - समवायस्य प्रत्यक्षत्वे कः सन्निकर्षः ? इति जिज्ञासायां वदति अन्तिमः विशेषणविशेष्यभावः सन्निकर्षः इति | यथा 'इह भूतले घटो नास्ति' इति प्रतीतिः तथैव 'इह घटे रूपसमवायः' इति प्रतीतिरपि क्वचित् सम्भवति | तदुपपादनाय कश्चन सन्निकर्षः अवश्यं ग्राह्यः | संयोगस्य समवायस्य च प्रतियोगितया अनुयोगितया वा समवाये असत्त्वात् क्लप्तः विशेषणविशेष्यभाव एव सन्निकर्ष उच्यते | तथा च घटे रूपसमवायः दैशिकविशेषणतासम्बन्धेन वर्तते इति एतेषामाशयः |

{\fontsize{11.7}{0}\selectfont\s एतत्पञ्चविधसम्बन्धसम्बन्धिविशेषणविशेष्यभावाद् दृश्याभावसमवाययोर्ग्रहणम् । तद्यथा घटशून्यं भूतलम् , इह भूतले घटो नास्ति । एवं सर्वत्रोदाहरणीयम् । समवायस्य तु क्वचिदेव ग्रहणम् । यथा घटे रूपसमवायः, रूपसमवायवान् घटः इति ।\footnote{न्या.सा.१४,१५}}


\subsection{विमर्शः}

अयुतसिद्धानां सम्बन्धेतिप्रसिद्धस्य समवायस्य सत्त्वं तस्य नित्यत्वमेकत्वमैन्द्रियकत्वञ्चात्र विमृश्यते |

\subsubsection{समवायसत्त्वे प्रमाणविमर्शः}

'नीलो घटः', 'शीतं जलम्' , 'फलं पतति' इत्यादिप्रतीतयः लोके प्रसिद्धाः | एतेषां संसर्गविषयकत्वन्तु सन्दिग्धमेव, विशेषणविशेष्ययोः मिथः भानासम्भवात् , संसर्गस्य तत्र अप्रत्यक्षत्वाच्च | न च 'दण्डी  पुरुषः' इत्यादिविशिष्टप्रतीतीनां संयोगादिसंसर्गविषयकत्वात् विशिष्टप्रतीतीनां सर्वेषां संसर्गविषयकत्वमभ्युपगन्तव्यम् | तथा च 'नीलो घटः' इत्यादिप्रतीतीनामपि विशिष्टप्रतीतित्वात् संसर्गविषयकत्वं सिध्यति | तथा चानुमानं 'नीलो घटः इति विशिष्टप्रतीतिः विशेषणविशेष्यसम्बन्धविषया विशिष्टप्रतीतित्वात् दण्डी पुरुष इति विशिष्टप्रतीतिवत्' इति | अनेन यः संसर्गः सिध्यति सः संयोगातिरिक्तः, द्रव्ययोरेव संयोगदर्शनात् | न च स्वरूपसम्बन्धः, गौरवात् | नापि विशेषणविशेष्यभावादिः, तस्योभयनिरूपितत्वाभावात् वृत्त्यनियामकत्वात् | तस्मादतिरिक्तः कश्चन सम्बन्धः सिध्यति | सैव समवायः इत्युच्यते | कल्प्यमानश्चः सः लाघवादेकः नित्यश्च कल्प्यते इति चेन्न | 

'अघटं भूतलम्' इत्यादौ व्यभिचारात् स्वरूपेण अर्थान्तराच्च | तत्र हि विशिष्टप्रतीतौ विशेषणविशेष्यापेक्षया अतिरिक्तः कोऽपि सम्बन्धः‌ न भासते | भूतलस्वरूपस्यैव तत्र विशिष्टप्रतीतिनिर्वाहकसम्बन्धत्वात् | तथा च विशेषणविशेष्यसम्बन्धविषयकत्वाभाववति 'अघटं भूतलम्' इत्यादौ विशिष्टप्रतीतित्वसत्त्वाद्व्यभिचारः | न च तत्रापि भूतलस्वरूपं सम्बन्धः विशिष्टप्रतीतिनिर्वाहकः अस्त्येव | तथा च केवल विशेषणविशेष्यसम्बन्धविषयकत्वमेव साध्यते | न तु तत्र विशेषणविशेष्यातिरिक्तसम्बन्धविषयकत्वम् | तथा च न व्यभिचार इति वाच्यम् | तत्रापि तत्तत्स्वरूपेणैव विशिष्टप्रतीतिनिर्वाहसम्भवे अतिरिक्तसम्बन्धकल्पनमर्थान्तरम् | न च हेतु कुक्षौ बाधकाभावे सत्यतिरिक्तसम्बन्धविषयकत्वरूपं विशेषणं दीयते | 'अघटं भूतलम्' त्यादिप्रतीतौ अधिकरणापेक्षयातिरिक्तस्यासम्बन्धत्वात् हेत्वभावः | तथा च न व्यभिचार इति वाच्यम् | भ्रमात्मकप्रतीतिमादाय व्यभिचारवारणाय बाधकाभावरूपदलस्यैव आवश्यकत्वात् अतिरिक्तसम्बन्धविषयकत्वदलं व्यर्थमेव | 

किञ्च बाधकाबावदळनिवेशे तु अप्रयोजकत्वात् साधकाभावदळेनापि विशेषणविशेष्यसम्बन्धविषयकत्वाभावसाधनसम्भवात् सत्प्रतिपक्षितमेवेदमनुमानम् | न च साध्यकुक्षौ सम्बन्धिभिन्नत्वरूपं विशेषणं दीयते | तथा च स्वरुपादतिरिक्तः सम्बन्धः सिध्यति इति वाच्यम् | 'अघटं भूतलम्' इत्यादौ व्यभिचारात् | तत्र विशिष्टप्रतीतित्वसत्त्वेऽपि सम्बन्धिभिन्नविशेषणविशेष्यसम्बन्धविषयकत्वाभावात् व्यभिचारः |

किञ्च अनेनानुमानेन यः सम्बन्धः सिध्यति स किं विशेषणतया भासते ? उत विशेष्यतया भासते ? स्वरूपेणभासते वा ? नाद्यः, विशिष्टप्रतीतेः पूर्वं तस्याज्ञानात् , विशिष्टबुद्धौ विशेषणज्ञानस्य हेतुत्वात् | न द्वितीयः, 'अनयोः समवायः' इत्यादिना समवायस्य विशेष्यतया अननुभवात् | न तृतीयः 'समवायं जानामि' इति प्रतीतेरभावात् | तस्मात् अतिरिक्तः सम्बन्धः गुणक्रियादिविशिष्टबुद्धौ नैव भासते | 

न च अवयविगुणक्रियाजातितद्वतां 'इह तन्तुषु पटः', 'इह पटे शौक्ल्यम्' इत्यादौ इहप्रत्ययः विशेषणविशेष्यसम्बन्धनिमित्तकः यथार्थेहप्रत्ययत्वात् 'इह कुण्डे बदरम्' इति इहप्रत्ययवदित्यनुमानं समवायसाधकम् | तत्रेह प्रत्ययस्य यथा कुण्डबधरयोराधाराधेयभावनिमित्तकसंयोगाख्यसम्बन्धविषयकत्वं तथैव 'इह तन्तुषु पटः' इत्यादिप्रतीतौ इहप्रत्ययस्य विशेषणविशेष्यसम्बन्धनिमित्तकत्वमस्ति | स च संयोगाद्यरितिक्तः समवाय एव इति वाच्यम् | 'इह भूतले घटाभावः' इत्यादौ व्यभिचारात् | तत्रापि स्वरूपसम्बन्धनिमित्तकस्य इहप्रत्ययस्य सत्त्वात् विशेषणविशेष्यापेक्षयातिरिक्तसम्बन्धस्याभावात् | 

न च 'शब्दजातिरूपादिः इन्द्रियसम्बद्धः प्रत्यक्षत्वात् घटवत्'  इत्यनुमानेन इन्द्रियसम्बन्धघटकतया समवायः सिध्यति | तथा हि - 'अयं घटः' इत्याद्याकारकद्रव्यप्रत्यक्षं यज्जायते तत्तु संयोगरूपेन्द्रियसन्निकर्षेणैव जायते | इन्द्रियघटयोः द्रव्यत्वात् संयोगः तत्र उपपद्यते | तद्वत् 'इदं नीलरूपम्' इत्यादयः शब्दजातिगुणादिविषयकाः अपि साक्षात्काराः अनुभूयन्ते | तेऽपि प्रत्यक्षत्वात् इन्द्रियसम्बन्द्धाः एव | रूपादिना सह इन्द्रियस्य संयोगासम्भवात् अन्यः एव सन्निकर्षः तत्र वक्तव्यः | द्रव्येन सह सन्निकर्षानन्तरं तादृशप्रतीतेरुदयात् द्रव्यस्य गुणस्य च सम्बन्धः कश्चन सन्निकर्षघटकः इत्यभुपगन्तव्यम् | स च संयुक्तसमवायः | तथा च तद्घटकतया समवायः सिध्यति इति वाच्यम् | अनया रीत्या समवायाभ्युपगमेऽपि 'भूतलं घटाभाववत्' इति प्रत्यक्षप्रतीतिनिर्वाहाय स्वरूपसम्बन्धस्य सन्निकर्षत्वेनापि अवश्यमभ्युपगम्यमानत्वात् तेनैव गुणादिव्यवहाराणां सिद्धौ किमतिरिक्तसम्बन्धकल्पनेन ? तथा हि रूपादीनां इन्द्रियसम्बद्धद्रव्यविशेषणतया, तत्रत्यरूपत्वादिजातीनामिन्द्रियसम्बन्द्धविशेषणविशेषणतया, शब्दस्य इन्द्रियविशेषणतया शब्दत्वस्य इन्द्रियविशेषणविशेषणतया प्रत्यक्षत्वमुपपादयितुं शक्यते | तथा च तदर्थमतिरिक्तसम्बन्धकल्पनमनुचितमित्यर्थान्तरम् |

ननु 'तन्तुषु पटः', 'नीलो घटः' इत्याद्याः प्रतीतयः याः अनुभूयन्ते ताः एव अवयवावयविनोः‌ गुणगुणिनोः क्रियाक्रियावतोः जातिव्यक्त्योः समवायाख्यसम्बन्धसाधकाः इति चेन्न | अन्योन्याश्रयात् | तादृशप्रतीत्या समवायसिद्धिः, समवायसिद्धौ च तादृशविशिष्टप्रतीतयः इति अन्योन्याश्रयः | किञ्च समवायाख्यस्य सम्बन्धस्य विशेषणविशेष्यापेक्षया अतिरिक्तत्वे तस्यापि कश्चन सम्बन्धः कल्पनीयः | तस्यापि समवायत्वे अनवस्थाप्रसङ्गः | तत्र अनवस्थाभिया तादात्म्यमेव सम्बन्धः कल्प्यते इति चेत् समवायस्थाने तादात्म्यसम्बन्धकल्पने एव लाघवात् समवायाख्यातिरिक्तसम्बन्धकल्पनं गुरुभूतम् | तदुक्तम् - {\fontsize{11.7}{0}\selectfont\s अवयवावयविनोः गुणगुणिनोः जातिजातिमतोः क्रियाक्रियावतोश्च परस्परं तादात्म्यमेव सम्बन्धः\footnote{मा.मे. २७५}} इति | मैवम् |

'अवयविगुणक्रियाजातिविशिष्टबुद्धयः विशेषणविशेष्यसम्बन्धविषयाः विशिष्टबुद्धित्वात् दण्डीतिबुद्धिवत्' ,  गुणक्रियाजातिविशिष्टबुद्धयो विशेषणसम्बन्धनिमित्तिकाः सत्यत्वे सति विशिष्टबुद्धित्वात् , दण्डीति बुद्धिवत् इति वा अनुमानं  प्रमाणम् | तत्रादौ प्रथमानुमाने 'नीलो घटः' इत्यादिविशिष्टबुद्धौ विशेषणसम्बन्धविषयकत्वं सिध्यति, तस्याः विशिष्टबुद्धित्वात् | न च 'अघटं भूतलम्' इत्यादौ व्यभिचारः इति वाच्यम् | तत्रापि विशिष्टबुद्धौ स्वरूपसम्बन्धस्य विषयत्वात् | न च स्वरूपसम्बन्धमादाय अनाकांक्षितार्थाभिधानादर्थान्तरम् | स्वरूपसम्बन्धेनैव प्रकृतपक्षेऽपि विशिष्टव्यवहारसम्भवादिति वाच्यम् | गुणक्रियाजातिविशिष्टबुद्धीनां विशिष्टबुद्धित्वात् विषयतया कश्चित्सम्बन्धः सिध्यति | कल्प्यमानश्च सः लाघवज्ञानसहकृतपक्षधर्मताज्ञानेन एक एव सिद्ध्यति | स च स्वरूपादतिरिक्तः | अन्यथा अनन्तानां स्वरूपाणां सम्बन्धकल्पने आनन्त्यरूपगौरवप्रसङ्गः |

द्वितीयानुमानेऽपि विशेषणसम्बन्धः विशिष्टव्यवहारनिमित्तत्वेन सिध्यन् लाघवादेक एव सिध्यति | तेन अनुगतानां तादृशविशिष्टबुद्धीनां अनुगतसम्बन्धजन्यत्वमपि उपपद्यते | अस्मिन्ननुमाने हेतुकुक्षौ सत्यत्वदलं निवेशनीयम् | अन्यथा भ्रमात्मकविशिष्टज्ञाने विशेषणसम्बन्धनिमित्तकत्वाभावात् व्यभिचारः स्यात् | 

न चेदमुभयमप्यप्रयोजकम् | विशिष्टसाक्षात्कारस्य सम्बन्धविषयकत्वतज्जन्यत्वनियमात् | अन्यथा गवाश्वादावपि विशिष्टबुद्धिप्रसङ्गः इति |

नव्यास्तु 'गुणक्रियादिविशिष्टबुद्धिः सम्बन्धिभिन्नविशेषणसम्बन्धविषया निर्विषयकभावविशेषणकविशिष्टबुद्धित्वात् दण्डीतिबुद्धिवत्' , 'गुणक्रियादिविशिष्टबुद्धिः सम्बन्धिभिन्नविशेषणसम्बन्धविषया इतरनिरूपणानिरूप्यविशेषणविशिष्टबुद्धित्वात् दण्डीतिबुद्धिवत्' इति वा अनुमानं समवायं साधयति | अत्र सम्बन्धिभिन्नविशेषणसम्बन्धविषयकत्वाभाववति 'भूतलं घटाभाववत्' , 'भूतलं ज्ञातम्' इत्यादिज्ञानेषु विशिष्टबुद्धित्वसत्त्वाद्व्यभिचारः | अतस्तद्वारणाय प्रथमानुमाने हेतुकुक्षौ भावविशेषणकत्वस्य निर्विषयकत्वस्य च निवेशः | 'भूतलं घटाभाववत्' इति बुद्धेः‌ अभावविशेषणकत्वात् 'भूतलं ज्ञातम्' इति बुद्धेश्च सविषयकपदार्थविशेषणकत्वाच्च हेत्वभावान्न व्यभिचारः | द्वितीयानुमाने तु घटाभावस्य घटरूपप्रतियोगिनिरूपितत्वात् ज्ञानस्य तु विषयनिरूपितत्वात् इतरनिरूपणनिरूप्यविशेषणकत्वमेव इति न व्यभिचारः | न चेदमप्रयोजकम् | उपाधिसत्त्वे एव अस्य अप्रयोजकत्वसिद्धिः | प्रकृते उपाधेरेवाभावात् नेदमप्रयोजकम् इति |

अथवा 'गुणक्रियाजातिविषयकविशिष्टसाक्षात्कारः इन्द्रियसम्बन्धसाध्यः जन्यप्रत्यक्षत्वात् दण्डिज्ञानवत्' इत्यनुमानेन इन्द्रियसम्बन्धघटकतया समवायः सिध्यति | न चात्रापि स्वरूपेणार्थान्तरम् दोषः | 'सिध्यतः सम्बन्धस्य एकत्वे नित्यत्वे च लाघवम्' इति लाघवज्ञानसहकृतपरामर्शात् कल्पनीयश्च सम्बन्धः एकः नित्यश्च कल्प्यते | स्वरूपसम्बन्धस्य तु तत्तत्स्वरूपाणामनन्तत्वात् अनित्यत्वाच्च न तेन अर्थान्तरम् | एवं जात्यादिसाक्षात्कारे इन्द्रियसम्बद्धविशेषणविशेषणतायाः सन्निकर्षत्वस्वीकारे महद्गौरवं तदपेक्षया समवायघटितसन्निकर्षे एव लाघवमित्यप्युक्तम् |

नन्वत्र लाघवज्ञानसहकृतपरामर्शजन्यानुमानेन समवायः सिध्यति इति सर्वत्र प्रतिपादितम् | किन्तु अभावस्थले विशिष्टबुद्धिनिर्वाहाय अवश्यं स्वरूपसम्बन्धः अभ्युपेयः | तथा च स्वरूपसम्बन्धस्य क्लृप्तत्वात् समवायस्य तु कल्पनीयत्वात् समवायकल्पने एव गौरवम् इति चेन्न | गुणक्रियाजातिविषयकविशिष्टबुद्धिनियामकस्य स्वरूपसम्बन्धत्वे गुणजात्यादिशून्यस्य द्रव्यस्य कदाचित्प्रत्यक्षत्वप्रसङ्गः | तथा हि - अभावविशिष्टस्य ज्ञानविशिष्टस्य च भूतलादेः यथा 'नीलवद्भूतलम्' इत्यादौ तदविशेषणतया प्रतीतिः सम्भवति तद्वत् गुणादिशून्यस्यापि द्रव्यादीनां कदाचित्ग्रहणप्रसङ्गः | कदाचिदपि तादृशप्रतीतेरभावात् तत्र स्वरूपादतिरिक्त एव सम्बन्धः तादृशप्रतीतिनियामकः इत्युच्यते |

ननु गुणादीनां द्रव्याणाञ्च पृथक्तया प्रतीतेरभावात् तत्र तादात्म्यसम्बन्ध एव भवतु | अन्यथा समवायस्यापि सम्बन्धिभेदसाधनार्थमन्यः तादात्म्यसम्बन्धः वक्तव्यः इति चेन्न | गुणादीनां द्रव्यादिना सह तादात्म्यसम्बन्धाभ्युपगमे द्रव्याद्यपेक्षया गुणादीनामतिरिक्तत्वं न सिध्यति | तथा च आत्मनि ज्ञानादीनां क्षणिकत्वानुभवात् तदा आत्मनः अपि नाशप्रसङ्गः | तस्माद्द्रव्याद्यपेक्षया गुणादीनामतिरिक्तत्वमवश्यमभ्युपगन्तव्यमेव | अतः तन्निर्वाहाय तादात्म्यादतिरिक्त एव सम्बन्धः स्वीकार्यः | स च समवाय इत्युच्यते | तस्योत्पत्तिविनाशाकल्पनाप्रयुक्तलाघवात्तस्य नित्यत्वमभ्युपगम्यते |


\subsubsection{समवायस्यैकत्वविमर्शः}

आनुमानप्रमाणेन यः सम्बन्धः गुणादीनां कल्प्यते तस्य नानात्वे प्रत्यक्षादेः बाधकस्य असत्वात् लाघवज्ञानसहकृतपरामर्शकार्यत्वाच्च एकः समवायः इति सिध्यति | ननु तस्य एकत्वे तेन सम्बन्धेन द्रव्यत्वस्य द्रव्यवृत्तित्वम् , न तु गुणवृत्तित्वमिति कथं सिध्यति | द्रव्यत्वसमवायस्य गुणेऽपि सत्त्वात् , गुणस्य समवायानुयोगित्वात् , समवायस्य च एकत्वात् | तथा च द्रव्यत्वादीनां सङ्करप्रसङ्गः | किञ्च वायौ स्पर्शसमवायस्य सत्त्वात् पृथिव्यादिषु रूपसमवायस्य प्रसिद्धत्वात् वायोरपि रूपाधिकरणत्वप्रसङ्गः | तथा च वायोरपि चाक्षुषत्वापत्तिः इति चेन्न | अधाराधेयभावनियमात् द्रव्यत्वसमवायस्य द्रव्यवृत्तित्वमेवाभ्युपगम्यते, न तु गुणादिवृत्तित्वम् | एवं गुणत्वादीनां तत्तदधिकरणवृत्तित्वमेव तेन सम्बन्धेन उच्यते, न त्वन्यत्र | तथा च वायौ समवायसत्त्वेऽरूपाधिकरणत्वाभावात् न तत्र रूपोपलब्धिः, नापि तस्य चाक्षुषत्वम् | ननु आधाराधेयभावनियामकः कः ? इति चेत् अन्वयव्यतिरेकसहचारावेव | सर्वदा सर्वत्र द्रव्येष्वेव द्रव्यत्वदर्शनात् गुणादौ तददर्शनात् द्रव्यस्यैव द्रव्यत्वाधारत्वमभ्युपगन्तव्यम् | एवमेव अन्यत्रापि | अन्यथा 'भूतलं घटवत्' इत्यत्रापि भूतलघतयोर्मध्ये विद्यमानसंयोगस्य एकत्वात् भूतलस्यैव आधारत्वं घटस्यैव आधेयत्वमिति न सिध्येत् | तथा च 'घटः भूतलवान्' इत्यपि तत्र प्रतीतिः स्यात् | किन्तु तत्र भूतलस्य घटाधेयत्वं यथा अन्वयव्यतिरेकाभ्यां नाभ्युपगम्यते तद्वदत्रापि वक्तव्यम् इति चेत् |

अत्रोच्यते - यद्यपि भूतलघटयोः आधाराधेयभावः अन्वयव्यतिरेकाभ्यां सिध्येत तथापि तन्न समवायैकत्वं साधयति | अपि तु द्रव्येष्वेव द्रव्यत्वम् , न तु द्रव्यत्वे द्रव्यमिति विपरीताधाराधेयभावं निवारयति | विपरीताधाराधेयभावनिवृत्त्या सम्बन्धैकत्वं नैव सिध्यति | अन्यथा संयोगस्यापि एकत्वगप्रसङ्गः | न च तत्र प्रत्यक्षं बाधकम् | अत्र तु समवायस्यानित्यत्वान्न प्रत्यक्षं बाधकमिति वाच्यम् | बाधकान्तरसत्त्वात् | न हि एककालावच्छेदेन सम्बन्धप्रतियोग्यनुयोगिनोः सत्त्वे सम्बन्धस्य च एकत्वे तादृशानुयोगिनि प्रतियोगिप्रकारकबुध्युत्पत्तौ बाधकमस्ति | अन्वयव्यतिरेकसहचारस्तु न बाधकम् | तस्य आधाराधेयभावनियामकत्वात् , न अन्यप्रतियोगिनः अन्यानुयोगिवृत्तित्वनिषेधकत्वम् | तथा च एकस्मिन् क्षणे वायुसत्त्वे पृथिव्यादिगतरूपादीनाञ्च तस्मिन्नेव क्षणे सत्त्वे तयोः इन्द्रियसन्निकर्षे जाते तदुत्तरक्षणे वायौ रूपवत्ताबुद्धिप्रसङ्गः | किञ्च आधाराधेयभावनियामकसम्बन्धाः संयोगादयः द्विनिष्ठाः एव, न तु तदधिकवृत्तयः इति लोके प्रसिद्धाः | यत्र घटपटभूतलानां संयोगः तत्र स च संयोगः नैकः, घटस्य भूतलेन सह संयोगः पटेन सह संयोगश्च अन्य एव इत्यनुभवः | न च 'भूतलं घटपटोभयाभाववत्' इत्यत्र भूतलस्य घटाभावपटाभावयोः साकमेकमेव सम्बन्धः | सैव तत्राधाराधेयभावनियामकः इति वाच्यम् | अभावस्थले तथा दर्शनेऽपि भावस्थले तथा अदर्शनात् | एवञ्च भावयोः आधाराधेयभावसम्बन्धः द्विनिष्ठ एव | तथा च एतादृशनियमानुपपत्तिः वाय्वादौ रूपवत्तप्रतीत्यापत्तिश्च समवायस्यैकत्वे बाधिका | एवञ्च बाधकस्यैव सत्त्वात् समवायस्य नानात्वमभ्युपगन्तव्यमेव |

\subsubsection{समवायस्यैन्द्रियकत्वविमर्शः}

केचित्तु - विशेषणविशेष्यभावसन्निकर्षेण समवायाभावयोर्ग्रहणमिति वदन्ति | तथा चानुमानम् - 'समवायः प्रत्यक्षः प्रत्यक्षप्रयोजकसम्बन्धत्वात् संयोगवत्' इति | प्रत्यक्षप्रयोजकेन्द्रियसम्बन्धानां संयोगादीनां प्रत्यक्षत्वात् समवायस्यापि प्रत्यक्षप्रयोजकत्वात् तस्यापि प्रत्यक्षत्वं सिध्यति इति | तन्न | 

अनुमानस्य व्यभिचारदोषकबळितत्वात् | तथा हि - इन्द्रियद्रव्यसंयोगः प्रत्यक्षप्रयोजकस्तावदप्रत्यक्ष एव, सम्बन्धप्रत्यक्षं प्रति यावत्सम्बन्धिप्रत्यक्षस्य कारणत्वात् , इन्द्रियस्य चाप्रत्यक्षत्वात् | तथा च प्रत्यक्षत्वाभाववति तादृशसंयोगे प्रत्यक्षकारणत्वस्य सत्त्वात् व्यभिचारः | न च समवायेन साकम् इन्द्रियसन्निकर्षस्य कारणस्य सत्त्वात् तस्य प्रत्यक्षत्वमिति वाच्यम् | समवायो हि स्वानुयोगिनि तादात्म्येन वर्तते | तादात्म्यञ्च स्वापेक्षया नातिरिक्तः सम्बन्धः | तस्मात् प्रतियोगिना सह इन्द्रियसम्बन्धाभावात् न समवायस्य इन्द्रियैः ग्रहणमिति | अभावस्थले तु  स्वरूपसम्बन्धः अनुयोग्यात्मकः, न तु प्रतियोग्यात्मकः | अनुयोगिना सह इन्द्रियसंयोगादेः सम्भवात् तत्र विशेषणतया अभावादेर्भानं सम्भवति | किञ्च 'समवायं साक्षात्करोमि' इत्यननुभवात् तस्य अप्रत्यक्षत्वं सिध्यति |

इदन्तु चिन्त्यम् - सम्बन्धप्रत्यक्षं प्रति अन्वयव्यतिरेकाभ्यां यावतां सम्बन्धिनां प्रत्यक्षस्य कारणत्वं सिद्धम् | अत एव घटाकाशयोः संयोगः आकाशस्याप्रत्यक्षत्वात् नैव इन्द्रियग्राह्यः | तथा च समवायस्थले तस्यैकत्वे परमाण्वाकाशादीनां समवायसम्बन्धिनामप्रत्यक्षत्वात् तस्याप्रत्यक्षत्वमेव | तस्य नानात्वे तु तन्तुपटादीनां रूपघटादीनां पतनफलादीनां गोत्वगवादीनाञ्च समवायसम्बन्धिनां प्रत्यक्षत्वात् तस्यापि प्रत्यक्षत्वमभ्युपगन्तव्यम् | न च तत्र इन्द्रियसन्निकर्षरूपकारणान्तरविरहः | समवायस्यापि स्वरूपेण स्वाधिकरणे अवस्थित्यभ्युपगमात् | न च 'समवायं साक्षात्करोमि' इत्यननुभवः | क्वचित् रूपघटयोः सम्बन्धविषयकज्ञानस्याप्यनुभवात् इति |
